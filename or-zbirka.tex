\documentclass[a4paper, 11pt, titlepage]{article}
\usepackage[slovene]{babel}
\usepackage[utf8]{inputenc}
\usepackage[slovene]{babelbib}
\usepackage{lmodern}
\usepackage[T1]{fontenc}
\usepackage{microtype}
\usepackage{chngcntr}
\usepackage{environ}
\usepackage{needspace}
\usepackage{fourier}
\usepackage[small]{caption}
%%%%%%%%%%%%%%%%%%%%%%%%%%%%%%%%%%%%%%%%%%%%%%%%%%%%%%%%
\usepackage{amsmath,amssymb,amstext,wasysym,mathtools,mathdots}
\usepackage{algorithmicx,algpseudocode}
%%%%%%%%%%%%%%%%%%%%%%%%%%%%%%%%%%%%%%%%%%%%%%%%%%%%%%%%
\usepackage{tabularx}
\usepackage{enumerate}
\usepackage[colorlinks=true]{hyperref}
\usepackage{eurosym}
\usepackage{tikz}
\usepackage{pgflibraryshapes}
\usetikzlibrary{calc}
\usetikzlibrary{matrix}
\usetikzlibrary{backgrounds}
\usetikzlibrary{decorations.pathmorphing}
\pgfrealjobname{or-zbirka}
%%%%%%%%%%%%%%%%%%%%%%%%%%%%%%%%%%%%%%%%%%%%%%%%%%%%%%%%
\newcommand{\A}{\ensuremath{\mathcal{A}}}
\newcommand{\N}{\ensuremath{\mathbb{N}}}
\newcommand{\R}{\ensuremath{\mathbb{R}}}
\newcommand{\Z}{\ensuremath{\mathbb{Z}}}
\newcommand{\p}{\ensuremath{\mathcal{P}}}
\newcommand{\set}[2]{\ensuremath{\left\{ #1 \; \middle| \; #2 \right\}}}
\newcommand{\mm}{\ensuremath{\phantom{|}-\phantom{|}}}
\newcommand{\pp}{\ensuremath{\phantom{|}+\phantom{|}}}
\newcommand{\EVPI}{\ensuremath{\operatorname{EVPI}}}
\newcommand{\EVE}{\ensuremath{\operatorname{EVE}}}
\newcommand{\Adj}{\ensuremath{\text{\sc Adj}}}
\newcommand{\length}{\ensuremath{\operatorname{length}}}
\newcommand{\reverse}{\ensuremath{\operatorname{reverse}}}
\newcommand{\reverseStart}{\ensuremath{\operatorname{reverseStart}}}
\newcommand{\append}{\ensuremath{\operatorname{append}}}
\newcommand{\pop}{\ensuremath{\operatorname{pop}}}
\newcommand{\isEmpty}{\ensuremath{\operatorname{isEmpty}}}
\renewcommand{\algorithmicrequire}{{\bf Vhod:}}
%%%%%%%%%%%%%%%%%%%%%%%%%%%%%%%%%%%%%%%%%%%%%%%%%%%%%%%%
\newcommand{\razdelek}[1]{\novastran\subsection{#1}\renewcommand{\novastran}{\clearpage}}
\newcommand{\opis}[2][2]{\shortintertext{\small\needspace{#1\baselineskip}#2:}}
\newcommand{\namig}[1]{\\ {\small {\bf Namig:} #1}}
\newcommand{\nadaljevanje}[1]{{\em Nadaljevanje sledi v nalogi~\nal{#1}.}}
%%%%%%%%%%%%%%%%%%%%%%%%%%%%%%%%%%%%%%%%%%%%%%%%%%%%%%%%
\DeclareUnicodeCharacter{20AC}{\euro}
\DeclareRobustCommand{\officialeuro}{%
  \ifmmode\expandafter\text\fi
  {\fontencoding{U}\fontfamily{eurosym}\selectfont e}}
\newcommand{\ME}{\ensuremath{\operatorname{M€}}}
%%%%%%%%%%%%%%%%%%%%%%%%%%%%%%%%%%%%%%%%%%%%%%%%%%%%%%%%
%% https://tex.stackexchange.com/questions/51733/global-renewcommand-equivalent-of-global-def/51751#51751
\makeatletter
\def\gnewcommand{\g@star@or@long\new@command}
\def\grenewcommand{\g@star@or@long\renew@command}
\def\g@star@or@long#1{%
  \@ifstar{\let\l@ngrel@x\global#1}{\def\l@ngrel@x{\long\global}#1}}
\makeatother

\newcounter{naloga}
\counterwithin*{naloga}{subsection}
\renewcommand{\thenaloga}{\arabic{subsection}.\arabic{naloga}}
\newenvironment{naloga}[3][]{\refstepcounter{naloga}\avtor{#2}\naslov{#1}\vir{#3}\needspace{3\baselineskip}\addcontentsline{toc}{subsubsection}{Naloga \thenaloga}\noindent {\bf Naloga}}{\bigskip}
\newenvironment{slika}[1][t]{\begin{figure}[#1]\centering}{\label{fig:\curlabel}\end{figure}}
\newenvironment{tabela}[1][t]{\begin{table}[#1]\centering}{\label{tab:\curlabel}\end{table}}
\newcommand{\pgfslika}[1][]{\grenewcommand{\curlabel}{\taoznaka{#1}}\leavevmode\beginpgfgraphicnamed{fig-\taoznaka{#1}}\input{slike/\taoznaka{#1}.tikz}\endpgfgraphicnamed}
\newcommand{\oznaka}[1]{\if\relax\detokenize{#1}\relax\grenewcommand{\thislabel}{\thenaloga}\else\grenewcommand{\thislabel}{#1}\fi\grenewcommand{\curlabel}{\thislabel}}
\newcommand{\taoznaka}[1]{\if\relax\detokenize{#1}\relax\thislabel\else#1\fi}
\newcommand{\izpisiNalogo}[1]{\oznaka{#1}\input{naloge/#1.tex}}

\newcommand{\taavtor}{}
\newcommand{\tanaslov}{}
\newcommand{\tavir}{}
\newcommand{\thislabel}{?}
\newcommand{\curlabel}{\thislabel}
\newcommand{\nal}[1]{\ref{nal:\taoznaka{#1}}}
\newcommand{\fig}[1]{\ref{fig:\taoznaka{#1}}}
\newcommand{\tab}[1]{\ref{tab:\taoznaka{#1}}}
\newcommand{\res}[1]{\ref{res:\taoznaka{#1}}}
\newcommand{\podnaslov}[2][]{\caption{#2 za nalogo~\if\relax\detokenize{#1}\relax\del{}\else#1\fi.}}

\allowdisplaybreaks

\begin{document}
\title{Zbirka nalog iz Operacijskih raziskav}
\author{Janoš Vidali}
\maketitle

\section{Naloge}
{
\newcommand{\del}{\nal}
\newcommand{\avtor}[1]{\renewcommand{\taavtor}{\hfill {\small Avtor: #1}\par}}
\newcommand{\naslov}[1]{\renewcommand{\tanaslov}{\if\relax\detokenize{#1}\relax\else\noindent {\small\bf (#1)}\fi}}
\newcommand{\vir}[1]{\renewcommand{\tavir}{\hfill {\small Vir: #1}\par}}
\newcommand{\novastran}{}
\newenvironment{vprasanje}{\label{nal:\thislabel}{\bf \res{}.}\taavtor\tanaslov\tavir\smallskip\noindent\ignorespaces}{}
\NewEnviron{odgovor}{}
\razdelek{Zahtevnost algoritmov}

\izpisiNalogo{zrcaljenje}
\izpisiNalogo{stevec}
\izpisiNalogo{bubblesort}
\izpisiNalogo{mergesort}
\izpisiNalogo{razcep}
\izpisiNalogo{korenski}
\izpisiNalogo{alg}
\izpisiNalogo{onotacija}
\izpisiNalogo{usmerjentrikotnik}
\izpisiNalogo{reverse}
\izpisiNalogo{hamilton}
\izpisiNalogo{reversestart}
\izpisiNalogo{kosi}
\izpisiNalogo{zlivanje}
\izpisiNalogo{sistem}
\izpisiNalogo{preurejanje}

\razdelek{Celoštevilsko linearno programiranje}

\izpisiNalogo{proj}
\izpisiNalogo{proj2}
\izpisiNalogo{skladisca}
\izpisiNalogo{dominacija}
\izpisiNalogo{maxprirejanje}
\izpisiNalogo{najkrajsapot}
\izpisiNalogo{kukavica}
\izpisiNalogo{vinar}
\izpisiNalogo{razdaljeclp}
\izpisiNalogo{kromaticno}
\izpisiNalogo{tsp}
\izpisiNalogo{vpetodrevo}
\izpisiNalogo{asistenti}
\izpisiNalogo{opravila}
\izpisiNalogo{polnjenje}
\izpisiNalogo{youtube}
\izpisiNalogo{sadje}
\izpisiNalogo{dodelitev}
\izpisiNalogo{linija}
\izpisiNalogo{nogomet}
\izpisiNalogo{ceki}
\izpisiNalogo{kosarka}
\izpisiNalogo{mostovi}
\izpisiNalogo{vinar2}
\izpisiNalogo{minolovec}
\izpisiNalogo{skatle}
\izpisiNalogo{drazba}
\izpisiNalogo{vsotakvadratov}
\izpisiNalogo{tovor}
\izpisiNalogo{stevila}
\izpisiNalogo{nijke}
\izpisiNalogo{crnko}
\izpisiNalogo{kopalec}
\izpisiNalogo{maxminprirejanje}
\izpisiNalogo{pakiranje}
\izpisiNalogo{uravnotezeno}
\izpisiNalogo{fgh}
\izpisiNalogo{f1f2}
\izpisiNalogo{sudoku}
\izpisiNalogo{turnir}

\razdelek{Teorija odločanja}

\izpisiNalogo{kovanec}
\izpisiNalogo{zemljice}
\izpisiNalogo{operacija}
\izpisiNalogo{produkt}
\izpisiNalogo{koncert}
\izpisiNalogo{rexhep}
\izpisiNalogo{dectree}
\izpisiNalogo{avtobus}
\izpisiNalogo{konkurenca}
\izpisiNalogo{poker}
\izpisiNalogo{dectree2}
\izpisiNalogo{podjetnik}
\izpisiNalogo{vrac}
\izpisiNalogo{dectree3}
\izpisiNalogo{ajkalaj}
\izpisiNalogo{bp}
\izpisiNalogo{rabljenoletalo}
\izpisiNalogo{drugastopnja}
\izpisiNalogo{direktorica}
\izpisiNalogo{vezava}
\izpisiNalogo{letalskekarte}
\izpisiNalogo{fenko}
\izpisiNalogo{ljubo}
\izpisiNalogo{tovarna}
\izpisiNalogo{zlatnik}
\izpisiNalogo{lunaairways}
\izpisiNalogo{maphija}
\izpisiNalogo{judo}

\razdelek{Dinamično programiranje}

\izpisiNalogo{plakati}
\izpisiNalogo{nahrbtnik}
\izpisiNalogo{matrika}
\izpisiNalogo{strnjenpodniz}
\izpisiNalogo{strnjenapodmatrika}
\izpisiNalogo{kovanci}
\izpisiNalogo{tat}
\izpisiNalogo{hlod}
\izpisiNalogo{jagode}
\izpisiNalogo{marketing}
\izpisiNalogo{boltezar}
\izpisiNalogo{ucbeniki}
\izpisiNalogo{nelinearni}
\izpisiNalogo{hazarder}
\izpisiNalogo{blago}
\izpisiNalogo{blago2}
\izpisiNalogo{strnjenprodukt}
\izpisiNalogo{oscilirajoce}
\izpisiNalogo{aplikacija}
\izpisiNalogo{lobisti}
\izpisiNalogo{domine}
\izpisiNalogo{vlagatelj}
\izpisiNalogo{pocivalisca}
\izpisiNalogo{neznan}
\izpisiNalogo{pohlepar}
\izpisiNalogo{jatifi}
\izpisiNalogo{rekurzija}
\izpisiNalogo{prodajalec}
\izpisiNalogo{stiskanje}
\izpisiNalogo{kocke}
\izpisiNalogo{dnk}
\izpisiNalogo{simboli}
\izpisiNalogo{stekelce}
\izpisiNalogo{spust}
\izpisiNalogo{antonio}
\izpisiNalogo{razdelitev}
\izpisiNalogo{plezanje}
\izpisiNalogo{igra}
\izpisiNalogo{ovce}
\izpisiNalogo{delavci}
\izpisiNalogo{turkmenscina}
\izpisiNalogo{hierarhija}
\izpisiNalogo{maxvsota}
\izpisiNalogo{skoki}
\izpisiNalogo{enice}
\izpisiNalogo{investicije}

\razdelek{Algoritmi na grafih}

\izpisiNalogo{matgraf}
\izpisiNalogo{sosgraf}
\izpisiNalogo{digraf}
\izpisiNalogo{trikotnik}
\izpisiNalogo{zvezda}
\izpisiNalogo{bfs}
\izpisiNalogo{premer}
\izpisiNalogo{dijkstra}
\izpisiNalogo{negdijkstra}
\izpisiNalogo{avto}
\izpisiNalogo{dijkstra2}
\izpisiNalogo{dfs}
\izpisiNalogo{bf}
\izpisiNalogo{topo}
\izpisiNalogo{oviratlon}
\izpisiNalogo{topoclp}
\izpisiNalogo{vectopo}
\izpisiNalogo{zaklad}
\izpisiNalogo{razdalje}
\izpisiNalogo{pocitnice}
\izpisiNalogo{prerezna}
\izpisiNalogo{dag}
\izpisiNalogo{pot}
\izpisiNalogo{trditve}
\izpisiNalogo{nesrece}
\izpisiNalogo{poslovnez}
\izpisiNalogo{trditve2}
\izpisiNalogo{lenoritas}
\izpisiNalogo{letalisce}
\izpisiNalogo{stpoti}
\izpisiNalogo{otoki}
\izpisiNalogo{cesta}
\izpisiNalogo{delegati}
\izpisiNalogo{cmrlji}
\izpisiNalogo{zabe}
\izpisiNalogo{optika}
\izpisiNalogo{pog}
\izpisiNalogo{omrezje}
\izpisiNalogo{marco}
\izpisiNalogo{obnova}
\izpisiNalogo{cikel}
\izpisiNalogo{studij}
\izpisiNalogo{alternirajoca}
\izpisiNalogo{neodvisna}
\izpisiNalogo{operaterji}

\razdelek{CPM/PERT}

\izpisiNalogo{brezzobec}
\izpisiNalogo{brezzobec-lp}
\izpisiNalogo{palacinke}
\izpisiNalogo{palacinke2}
\izpisiNalogo{viadukt}
\izpisiNalogo{splet}
\izpisiNalogo{studij}
\izpisiNalogo{konferenca}
\izpisiNalogo{letalo}
\izpisiNalogo{objekt}

\subsection{Upravljanje zalog}

\izpisiNalogo{toaletni}
\izpisiNalogo{mitsuzuki}
\izpisiNalogo{akumulatorji}
\izpisiNalogo{marta}
\izpisiNalogo{vulkanizer}
\izpisiNalogo{sedezne}

\razdelek{Pretoki in prerezi}

\izpisiNalogo{minmaxpretok}
\izpisiNalogo{minkapaciteta}
\izpisiNalogo{lenority}
\izpisiNalogo{pretokx}
\izpisiNalogo{terminali}
\izpisiNalogo{greedyflow}
\izpisiNalogo{minprerezi}
\izpisiNalogo{kri}
\izpisiNalogo{kri2}
\izpisiNalogo{zajetje}
\izpisiNalogo{divjizur}
\izpisiNalogo{mafija}
\izpisiNalogo{rovi}
\izpisiNalogo{strezniki}
\izpisiNalogo{pretok1}
\izpisiNalogo{pretok2}
\izpisiNalogo{pretok3}
\izpisiNalogo{pretok4}
\izpisiNalogo{pretok5}

\razdelek{Razdelitve}

\izpisiNalogo{austin}
\izpisiNalogo{nadomestila}
\izpisiNalogo{zobcek}
\izpisiNalogo{zobcek2}


}

\clearpage

\section{Rešitve}
{
\newcommand{\del}{\res}
\newcommand{\avtor}[1]{}
\newcommand{\naslov}[1]{}
\newcommand{\vir}[1]{}
\newcommand{\novastran}{}
\NewEnviron{vprasanje}{\label{res:\thislabel}{\bf \nal{}.}}
\newenvironment{odgovor}{\noindent\ignorespaces}{}
\razdelek{Zahtevnost algoritmov}

\izpisiNalogo{zrcaljenje}
\izpisiNalogo{stevec}
\izpisiNalogo{bubblesort}
\izpisiNalogo{mergesort}
\izpisiNalogo{razcep}
\izpisiNalogo{korenski}
\izpisiNalogo{alg}
\izpisiNalogo{onotacija}
\izpisiNalogo{usmerjentrikotnik}
\izpisiNalogo{reverse}
\izpisiNalogo{hamilton}
\izpisiNalogo{reversestart}
\izpisiNalogo{kosi}
\izpisiNalogo{zlivanje}
\izpisiNalogo{sistem}
\izpisiNalogo{preurejanje}

\razdelek{Celoštevilsko linearno programiranje}

\izpisiNalogo{proj}
\izpisiNalogo{proj2}
\izpisiNalogo{skladisca}
\izpisiNalogo{dominacija}
\izpisiNalogo{maxprirejanje}
\izpisiNalogo{najkrajsapot}
\izpisiNalogo{kukavica}
\izpisiNalogo{vinar}
\izpisiNalogo{razdaljeclp}
\izpisiNalogo{kromaticno}
\izpisiNalogo{tsp}
\izpisiNalogo{vpetodrevo}
\izpisiNalogo{asistenti}
\izpisiNalogo{opravila}
\izpisiNalogo{polnjenje}
\izpisiNalogo{youtube}
\izpisiNalogo{sadje}
\izpisiNalogo{dodelitev}
\izpisiNalogo{linija}
\izpisiNalogo{nogomet}
\izpisiNalogo{ceki}
\izpisiNalogo{kosarka}
\izpisiNalogo{mostovi}
\izpisiNalogo{vinar2}
\izpisiNalogo{minolovec}
\izpisiNalogo{skatle}
\izpisiNalogo{drazba}
\izpisiNalogo{vsotakvadratov}
\izpisiNalogo{tovor}
\izpisiNalogo{stevila}
\izpisiNalogo{nijke}
\izpisiNalogo{crnko}
\izpisiNalogo{kopalec}
\izpisiNalogo{maxminprirejanje}
\izpisiNalogo{pakiranje}
\izpisiNalogo{uravnotezeno}
\izpisiNalogo{fgh}
\izpisiNalogo{f1f2}
\izpisiNalogo{sudoku}
\izpisiNalogo{turnir}

\razdelek{Teorija odločanja}

\izpisiNalogo{kovanec}
\izpisiNalogo{zemljice}
\izpisiNalogo{operacija}
\izpisiNalogo{produkt}
\izpisiNalogo{koncert}
\izpisiNalogo{rexhep}
\izpisiNalogo{dectree}
\izpisiNalogo{avtobus}
\izpisiNalogo{konkurenca}
\izpisiNalogo{poker}
\izpisiNalogo{dectree2}
\izpisiNalogo{podjetnik}
\izpisiNalogo{vrac}
\izpisiNalogo{dectree3}
\izpisiNalogo{ajkalaj}
\izpisiNalogo{bp}
\izpisiNalogo{rabljenoletalo}
\izpisiNalogo{drugastopnja}
\izpisiNalogo{direktorica}
\izpisiNalogo{vezava}
\izpisiNalogo{letalskekarte}
\izpisiNalogo{fenko}
\izpisiNalogo{ljubo}
\izpisiNalogo{tovarna}
\izpisiNalogo{zlatnik}
\izpisiNalogo{lunaairways}
\izpisiNalogo{maphija}
\izpisiNalogo{judo}

\razdelek{Dinamično programiranje}

\izpisiNalogo{plakati}
\izpisiNalogo{nahrbtnik}
\izpisiNalogo{matrika}
\izpisiNalogo{strnjenpodniz}
\izpisiNalogo{strnjenapodmatrika}
\izpisiNalogo{kovanci}
\izpisiNalogo{tat}
\izpisiNalogo{hlod}
\izpisiNalogo{jagode}
\izpisiNalogo{marketing}
\izpisiNalogo{boltezar}
\izpisiNalogo{ucbeniki}
\izpisiNalogo{nelinearni}
\izpisiNalogo{hazarder}
\izpisiNalogo{blago}
\izpisiNalogo{blago2}
\izpisiNalogo{strnjenprodukt}
\izpisiNalogo{oscilirajoce}
\izpisiNalogo{aplikacija}
\izpisiNalogo{lobisti}
\izpisiNalogo{domine}
\izpisiNalogo{vlagatelj}
\izpisiNalogo{pocivalisca}
\izpisiNalogo{neznan}
\izpisiNalogo{pohlepar}
\izpisiNalogo{jatifi}
\izpisiNalogo{rekurzija}
\izpisiNalogo{prodajalec}
\izpisiNalogo{stiskanje}
\izpisiNalogo{kocke}
\izpisiNalogo{dnk}
\izpisiNalogo{simboli}
\izpisiNalogo{stekelce}
\izpisiNalogo{spust}
\izpisiNalogo{antonio}
\izpisiNalogo{razdelitev}
\izpisiNalogo{plezanje}
\izpisiNalogo{igra}
\izpisiNalogo{ovce}
\izpisiNalogo{delavci}
\izpisiNalogo{turkmenscina}
\izpisiNalogo{hierarhija}
\izpisiNalogo{maxvsota}
\izpisiNalogo{skoki}
\izpisiNalogo{enice}
\izpisiNalogo{investicije}

\razdelek{Algoritmi na grafih}

\izpisiNalogo{matgraf}
\izpisiNalogo{sosgraf}
\izpisiNalogo{digraf}
\izpisiNalogo{trikotnik}
\izpisiNalogo{zvezda}
\izpisiNalogo{bfs}
\izpisiNalogo{premer}
\izpisiNalogo{dijkstra}
\izpisiNalogo{negdijkstra}
\izpisiNalogo{avto}
\izpisiNalogo{dijkstra2}
\izpisiNalogo{dfs}
\izpisiNalogo{bf}
\izpisiNalogo{topo}
\izpisiNalogo{oviratlon}
\izpisiNalogo{topoclp}
\izpisiNalogo{vectopo}
\izpisiNalogo{zaklad}
\izpisiNalogo{razdalje}
\izpisiNalogo{pocitnice}
\izpisiNalogo{prerezna}
\izpisiNalogo{dag}
\izpisiNalogo{pot}
\izpisiNalogo{trditve}
\izpisiNalogo{nesrece}
\izpisiNalogo{poslovnez}
\izpisiNalogo{trditve2}
\izpisiNalogo{lenoritas}
\izpisiNalogo{letalisce}
\izpisiNalogo{stpoti}
\izpisiNalogo{otoki}
\izpisiNalogo{cesta}
\izpisiNalogo{delegati}
\izpisiNalogo{cmrlji}
\izpisiNalogo{zabe}
\izpisiNalogo{optika}
\izpisiNalogo{pog}
\izpisiNalogo{omrezje}
\izpisiNalogo{marco}
\izpisiNalogo{obnova}
\izpisiNalogo{cikel}
\izpisiNalogo{studij}
\izpisiNalogo{alternirajoca}
\izpisiNalogo{neodvisna}
\izpisiNalogo{operaterji}

\razdelek{CPM/PERT}

\izpisiNalogo{brezzobec}
\izpisiNalogo{brezzobec-lp}
\izpisiNalogo{palacinke}
\izpisiNalogo{palacinke2}
\izpisiNalogo{viadukt}
\izpisiNalogo{splet}
\izpisiNalogo{studij}
\izpisiNalogo{konferenca}
\izpisiNalogo{letalo}
\izpisiNalogo{objekt}

\subsection{Upravljanje zalog}

\izpisiNalogo{toaletni}
\izpisiNalogo{mitsuzuki}
\izpisiNalogo{akumulatorji}
\izpisiNalogo{marta}
\izpisiNalogo{vulkanizer}
\izpisiNalogo{sedezne}

\razdelek{Pretoki in prerezi}

\izpisiNalogo{minmaxpretok}
\izpisiNalogo{minkapaciteta}
\izpisiNalogo{lenority}
\izpisiNalogo{pretokx}
\izpisiNalogo{terminali}
\izpisiNalogo{greedyflow}
\izpisiNalogo{minprerezi}
\izpisiNalogo{kri}
\izpisiNalogo{kri2}
\izpisiNalogo{zajetje}
\izpisiNalogo{divjizur}
\izpisiNalogo{mafija}
\izpisiNalogo{rovi}
\izpisiNalogo{strezniki}
\izpisiNalogo{pretok1}
\izpisiNalogo{pretok2}
\izpisiNalogo{pretok3}
\izpisiNalogo{pretok4}
\izpisiNalogo{pretok5}

\razdelek{Razdelitve}

\izpisiNalogo{austin}
\izpisiNalogo{nadomestila}
\izpisiNalogo{zobcek}
\izpisiNalogo{zobcek2}


}

\clearpage

\bibliographystyle{bababbrv-fl}
\bibliography{reference}

\end{document}
