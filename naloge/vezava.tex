\begin{naloga}{Povh, Pustavrh}{\cite[Naloga~6.9]{pp}}
\begin{vprasanje}
Janez želi naložiti vsoto $1\,000 €$ v banko za dobo petih let.
Odloča se med tem, da bi jo vezal za pet let (obrestna mera $5\%$)
ali pa petkrat zaporedoma po eno leto (obrestna mera $4\%$).
Če denar veže za pet let, vmes pa varčevanje prekine,
mu pripada le obrestna mera $3\%$.
Ocenjuje, da lahko v naslednjih petih letih pride do naslednjih situacij:
\begin{itemize}
\item varčeval bo pet let, pri tem se obrestna mera ne spremeni,
\item varčeval bo pet let, obrestna mera se po treh letih poveča za $30\%$,
\item varčeval bo pet let, obrestna mera se po treh letih zmanjša za $20\%$,
\item varčeval bo tri leta.
\end{itemize}
Opiši, kako naj se odloči glede na posamezne kriterije
(optimist, pesimist, Laplace, Savage).
Določi vrednosti parametera $\alpha$,
pri katerih je po Hurwiczevem kriteriju možnih več enakovrednih odločitev.
\end{vprasanje}

\begin{odgovor}
Zaradi enostavnosti predpostavimo,
da se obrestuje samo glavnica (enkrat na leto),
pri čemer pri petletnem varčevanju vsakič velja začetna obrestna mera.
Izračunajmo obresti po petih letih pri različnih shemah vezave
glede na dogajanje po treh letih:
\begin{center}
\begin{tabular}{c|cccc}
shema & nespremenjena & povečana & zmanjšana & prekinitev \\ \hline
$5$ let              & $250$ & $250$ & $250$ &   $-$ \\
$3 + 2 \cdot 1$ leto & $170$ & $194$ & $154$ &   $-$ \\
$3$ leta             &  $90$ &  $90$ &  $90$ &  $90$ \\ \hline
$5 \cdot 1$ leto     & $200$ & $224$ & $184$ &   $-$ \\
$3 \cdot 1$ leto     & $120$ & $120$ & $120$ & $120$ \\
\end{tabular}
\end{center}
Drugih možnosti ne bomo obravnavali,
saj se očitno $5$-letna vezava ne izplača, če jo bomo predčasno prekinili.
Iz zgornje tabele je torej jasno,
da se bomo odločili bodisi za $5$-letno vezavo ali pa vsakoletno vezavo,
z morebitno prekinitvijo oziroma prenehanjem po $3$ letih
brez nadaljnje vezave.
Dobimo torej sledečo odločitveno tabelo:
\begin{center}
\begin{tabular}{c|cccc}
začetna shema & nespremenjena & povečana & zmanjšana & prekinitev \\ \hline
$5$ let          & $250$ & $250$ & $250$ &  $90$ \\
$5 \cdot 1$ leto & $200$ & $224$ & $184$ & $120$ \\
\end{tabular}
\end{center}

Pri optimistovem kriteriju se ravnamo glede na najboljši možen izid.
Pri $5$-letni vezavi je to $250 €$, pri letni vezavi pa $224 €$,
zato se odločimo za $5$-letno vezavo.

Pri pesimistovem (Waldovem) kriteriju se ravnamo glede na najslabši možen izid.
Pri $5$-letni vezavi je to $90 €$, pri letni vezavi pa $120 €$,
zato se odločimo za letno vezavo.

Pri Laplaceovem kriteriju so vse možnosti enako verjetne.
Pri $5$-letni vezavi je torej pričakovani dobiček enak
$(250 € + 250 € + 250 € + 90 €)/4 = 210 €$,
pri letni vezavi pa $(200 € + 224 € + 184 € + 120 €)/4 = 182 €$,
zato se odločimo za $5$-letno vezavo.

Pri Savageovem kriteriju želimo minimizirati največje možno obžalovanje.
Pri $5$-letni vezavi je največje obžalovanje enako $120 € - 90 € = 30 €$,
pri letni vezavi pa $250 € - \min\{200€, 224€, 184€\} = 66 €$,
zato se odločimo za $5$-letno vezavo.

Pri Hurwiczevem kriteriju upoštevamo,
da se najboljši izid zgodi z ve\-rjet\-nost\-jo $\alpha$,
najslabši pa z verjetnostjo $1 - \alpha$.
Pri $5$-letni vezavi je torej pričakovan dobiček enak
$\alpha \cdot 250 € + (1 - \alpha) \cdot 90 € = \alpha \cdot 160 € + 90 €$,
pri letni vezavi pa
$\alpha \cdot 224 € + (1 - \alpha) \cdot 120 € = \alpha \cdot 104 € + 120 €$.
Pri vrednosti $\alpha = 15/28$ sta obe možnosti enakovredni.
Pri večjih vrednostih $\alpha$ se odločimo za $5$-letno vezavo,
pri manjših pa za letno vezavo.
\end{odgovor}
\end{naloga}
