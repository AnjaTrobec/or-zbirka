\begin{naloga}{Janoš Vidali}{Izpit OR 5.7.2018}
\begin{vprasanje}
Nogometni klub pred novo sezono prenavlja svoj igralski kader.
Naj bo $A$ množica trenutnih igralcev kluba,
$B$ pa množica igralcev, za katere se v klubu zanimajo
($A$ in $B$ sta disjunktni množici).
Za vsakega igralca $i \in A$ imajo s strani kluba $j$ ($1 \le j \le n$)
ponudbo v višini $p_{ij} \in \N$
(če se klub $j$ ne zanima za igralca $i$, lahko predpostaviš $p_{ij} = 0$),
za vsakega igralca $i \in B$ pa bodo morali plačati odkupnino $r_i \in \N$,
če ga hočejo pripeljati v klub.
Vsakega igralca $i \in A$ lahko seveda prodamo kvečjemu enemu klubu.
Skupni stroški trgovanja
(tj., razlika stroškov nakupov in dobička od odprodaj)
ne smejo preseči $S$,
število igralcev v klubu pa mora ostati enako kot pred trgovanjem.

Za vsakega igralca $i \in A \cup B$ poznamo njegov količnik kvalitete $q_i$,
vsak pa pripada natanko eni izmed množic
$G$ (vratarji), $D$ (branilci), $M$ (vezni igralci) in $F$ (napadalci).
Število igralcev kluba v vsaki izmed teh množic se lahko spremeni
(poveča ali zmanjša) za največ $1$.
Poleg tega lahko igralca $a, b \in A$ prodamo le istemu klubu
-- ali pa oba igralca ostaneta v klubu.
Doseči želimo, da bo kvaliteta kluba čim večja
-- maksimizirati želimo torej vsoto količnikov $q_i$ za igralce v klubu.

Zapiši celoštevilski linearni program, ki modelira zgoraj opisani problem.
\end{vprasanje}
\begin{odgovor}
\end{odgovor}
\end{naloga}
