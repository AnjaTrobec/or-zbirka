\begin{naloga}{Janoš Vidali}{Izpit OR 5.7.2018}
\begin{vprasanje}
Nogometni klub pred novo sezono prenavlja svoj igralski kader.
Naj bo $A$ množica trenutnih igralcev kluba,
$B$ pa množica igralcev, za katere se v klubu zanimajo
($A$ in $B$ sta disjunktni množici).
Za vsakega igralca $i \in A$ imajo s strani kluba $j$ ($1 \le j \le n$)
ponudbo v višini $p_{ij} \in \N$
(če se klub $j$ ne zanima za igralca $i$, lahko predpostaviš $p_{ij} = 0$),
za vsakega igralca $i \in B$ pa bodo morali plačati odkupnino $r_i \in \N$,
če ga hočejo pripeljati v klub.
Vsakega igralca $i \in A$ lahko seveda prodamo kvečjemu enemu klubu.
Skupni stroški trgovanja
(tj., razlika stroškov nakupov in dobička od odprodaj)
ne smejo preseči $S$,
število igralcev v klubu pa mora ostati enako kot pred trgovanjem.

Za vsakega igralca $i \in A \cup B$ poznamo njegov količnik kvalitete $q_i$,
vsak pa pripada natanko eni izmed množic
$G$ (vratarji), $D$ (branilci), $M$ (vezni igralci) in $F$ (napadalci).
Število igralcev kluba v vsaki izmed teh množic se lahko spremeni
(poveča ali zmanjša) za največ $1$.
Poleg tega lahko igralca $a, b \in A$ prodamo le istemu klubu
-- ali pa oba igralca ostaneta v klubu.
Doseči želimo, da bo kvaliteta kluba čim večja
-- maksimizirati želimo torej vsoto količnikov $q_i$ za igralce v klubu.

Zapiši celoštevilski linearni program, ki modelira zgoraj opisani problem.
\end{vprasanje}

\begin{odgovor}
Za igralca $i \in A$ in $j$-ti klub ($1 \le j \le n$)
bomo uvedli spremenljivko $x_{ij}$,
za igralca $i \in B$ pa spremenljivko $y_i$.
Njihove vrednosti interpretiramo kot
\begin{align*}
x_{ij} &= \begin{cases}
1; & \text{igralca $i$ prodamo $j$-temu klubu, in} \\
0  & \text{sicer;}
\end{cases} \\
\text{ter} \quad
y_i &= \begin{cases}
1; & \text{odkupimo ogralca $i$} \\
0  & \text{sicer.}
\end{cases}
\end{align*}
Zapišimo celoštevilski linearni program.
\begin{alignat*}{3}
\max &\ & \sum_{i \in B} q_i y_i - \sum_{i \in A} \sum_{j=1}^n q_i x_{ij}
&& \text{p.p.} \\
\forall i \in A \ \forall j \in \{1, \dots, n\}: &\ &
0 \le x_{ij} &\le 1, &\quad x_{ij} &\in \Z \\
\forall i \in B: &\ & 0 \le y_i &\le 1, &\quad y &\in \Z
\opis{Vsakega igralca prodamo največ enemu klubu}
\forall i \in A: &\ & \sum_{j=1}^n x_{ij} &\le 1
\opis{Omejitev stroškov}
&& \sum_{i \in B} r_i y_i - \sum_{i \in A} \sum_{j=1}^n p_{ij} x_{ij} &\le S
\opis{Skupno število igralcev}
&& \sum_{i \in B} y_i - \sum_{i \in A} \sum_{j=1}^n x_{ij} &= 0
\opis{Število igralcev za vsako pozicijo}
-1 &\le& \sum_{i \in B \cap G} y_i
- \sum_{i \in A \cap G} \sum_{j=1}^n x_{ij} &\le 1 \\
-1 &\le& \sum_{i \in B \cap D} y_i
- \sum_{i \in A \cap D} \sum_{j=1}^n x_{ij} &\le 1 \\
-1 &\le& \sum_{i \in B \cap M} y_i
- \sum_{i \in A \cap M} \sum_{j=1}^n x_{ij} &\le 1 \\
-1 &\le& \sum_{i \in B \cap F} y_i
- \sum_{i \in A \cap F} \sum_{j=1}^n x_{ij} &\le 1
\opis{Igralca $a$ in $b$ gresta v isti klub}
\forall j \in \{1, \dots, n\}: &\ & x_{aj} &= x_{bj}
\end{alignat*}
\end{odgovor}
\end{naloga}
