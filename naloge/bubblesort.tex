\begin{naloga}{Janoš Vidali}{Vaje OR 21.2.2018}
\begin{vprasanje}
Algoritem {\sc BubbleSort} uredi vhodni seznam $\ell[1 \dots n]$ tako,
da zamenjuje so\-sed\-nje elemente v nepravem vrstnem redu:
\begin{small}
\begin{algorithmic}
\Function{BubbleSort}{$\ell[1 \dots n]$}
    \State $z \gets n$
    \While{$z > 1$}
        \State $y \gets 1$
        \For{$i = 2, \dots, z$}
            \If{$\ell[i-1] > \ell[i]$}
                \State $\ell[i-1], \ell[i] \gets \ell[i], \ell[i-1]$
                \State $y \gets i$
            \EndIf
        \EndFor
        \State $z \gets y-1$
    \EndWhile
\EndFunction
\end{algorithmic}
\end{small}

\begin{enumerate}[(a)]
\item Izvedi algoritem na seznamu $[7, 11, 16, 7, 5]$.
\item Določi časovno zahtevnost algoritma.
\end{enumerate}
\end{vprasanje}

\begin{odgovor}
\begin{enumerate}[(a)]
\item Izpišimo vrednosti spremenljivk ob koncu vsakega obhoda zanke {\bf for}
oziroma {\bf while}, ko se prejšnja konča.

$$
\begin{array}{c|c|c|c|c}
\text{obhod {\bf while}} & i & y & z & \ell[1 \dots 5] \\ \hline
1 & 2 & 1 & 5 & [7, 11, 16, 7, 5] \\
1 & 3 & 1 & 5 & [7, 11, 16, 7, 5] \\
1 & 4 & 4 & 5 & [7, 11, 7, 16, 5] \\
1 & 5 & 5 & 5 & [7, 11, 7, 5, 16] \\
1 &   & 5 & 4 & [7, 11, 7, 5, 16] \\
2 & 2 & 1 & 4 & [7, 11, 7, 5, 16] \\
2 & 3 & 3 & 4 & [7, 7, 11, 5, 16] \\
2 & 4 & 4 & 4 & [7, 7, 5, 11, 16] \\
2 &   & 4 & 3 & [7, 7, 5, 11, 16] \\
3 & 2 & 1 & 3 & [7, 7, 5, 11, 16] \\
3 & 3 & 3 & 3 & [7, 5, 7, 11, 16] \\
3 &   & 3 & 2 & [7, 5, 7, 11, 16] \\
4 & 2 & 2 & 2 & [5, 7, 7, 11, 16] \\
4 &   & 2 & 1 & [5, 7, 7, 11, 16] \\
\end{array}
$$

\item Naj bodo $z_1, z_2, \dots, z_k$ vrednosti,
ki jih zavzame spremenljivka $z$ ob vsakem vstopu v zanko {\bf while}.
Očitno velja $z_i - 1 \ge z_{i+1}$ za vsak $i$,
tako da velja $k \le n-1$.
Največje število korakov je torej
$$
\sum_{z=2}^n (z-1) = {n(n-1) \over 2} .
$$
Tako število korakov dosežemo,
če je seznam $\ell$ na začetku urejen padajoče
-- tako vsakič pride do zamenjave v zadnjem koraku zanke {\bf for},
zato se $z$ vsakič zmanjša za $1$.
Časovna zahtevnost algoritma je torej $O(n^2)$.
\end{enumerate}
\end{odgovor}
\end{naloga}
