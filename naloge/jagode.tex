\begin{naloga}{Hillier, Lieberman}{\cite[Problem~11.3-1]{hl}}
\begin{vprasanje}
Lastnik verige $n$ trgovin z živili je kupil $m$ zabojev svežih jagod.
Naj bo $p_{ij}$ pričakovan dobiček v trgovini $j$,
če tja dostavimo $i$ zabojev.
Zanima nas, koliko zabojev naj gre v vsako trgovino,
da bomo imeli čim večji zaslužek.
Zaradi logističnih razlogov zabojev ne želimo deliti.
\begin{enumerate}[(a)]
\item Z dinamičnim programiranjem reši problem za podatke $m = 5$, $n = 3$
in $p_{ij}$ iz sledeče tabele:
$$
\begin{array}{c|ccc}
p_{ij} & 1 & 2 & 3 \\
\hline
0 &  0 &  0 &  0 \\
1 &  5 &  6 &  4 \\
2 &  9 & 11 &  9 \\
3 & 14 & 15 & 13 \\
4 & 17 & 19 & 18 \\
5 & 21 & 22 & 20 \\
\end{array}
$$
\item Napiši algoritem, ki reši opisani problem v splošnem.
\end{enumerate}

\end{vprasanje}
\begin{odgovor}
\end{odgovor}
\end{naloga}
