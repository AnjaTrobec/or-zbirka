\begin{naloga}{Hillier, Lieberman}{\cite[Problem~11.3-1]{hl}}
\begin{vprasanje}
Lastnik verige $n$ trgovin z živili je kupil $m$ zabojev svežih jagod.
Naj bo $p_{ij}$ pričakovan dobiček v trgovini $j$,
če tja dostavimo $i$ zabojev.
Zanima nas, koliko zabojev naj gre v vsako trgovino,
da bomo imeli čim večji zaslužek.
Zaradi logističnih razlogov zabojev ne želimo deliti.
\begin{enumerate}[(a)]
\item Z dinamičnim programiranjem reši problem za podatke $m = 5$, $n = 3$
in $p_{ij}$ iz sledeče tabele:
$$
\begin{array}{c|ccc}
p_{ij} & 1 & 2 & 3 \\
\hline
0 &  0 &  0 &  0 \\
1 &  5 &  6 &  4 \\
2 &  9 & 11 &  9 \\
3 & 14 & 15 & 13 \\
4 & 17 & 19 & 18 \\
5 & 21 & 22 & 20 \\
\end{array}
$$
\item Napiši algoritem, ki reši opisani problem v splošnem.
\end{enumerate}
\end{vprasanje}

\begin{odgovor}
Naj bo $z_{ij}$ največji zaslužek, ki ga lahko imamo,
če $i$ zabojev razporedimo v prvih $j$ trgovin.
Določimo začetne pogoje in rekurzivne enačbe.
\begin{align*}
z_{i1} &= p_{i1} && (0 \le i \le m) \\
z_{ij} &= \max\set{z_{h,j-1} + p_{i-h,j}}{0 \le h \le i}
&& (0 \le i \le m, \ 2 \le j \le n)
\end{align*}
Vrednosti $z_{ij}$ računamo v leksikografskem vrstnem redu glede na $(j, i)$.
Največji skupni zaslužek dobimo kot $z^* = z_{mn}$.
\begin{enumerate}[(a)]
\item Izračunajmo vrednosti $z_{ij}$ ($0 \le i \le m$, $2 \le j \le n$).
\begin{alignat*}{2}
z_{02} &= 0+0 &&= 0 \\
z_{12} &= \max\{\underline{0+6}, 5+0\} &&= 6 \\
z_{22} &= \max\{\underline{0+11}, 5+6, 9+0\} &&= 11 \\
z_{32} &= \max\{0+15, \underline{5+11}, 9+6, 14+0\} &&= 16 \\
z_{42} &= \max\{0+19, \underline{5+15}, 9+11, 14+6, 17+0\} &&= 20 \\
z_{52} &= \max\{0+22, 5+19, 9+15, \underline{14+11}, 17+6, 21+0\} &&= 25 \\
z_{03} &= 0+0 &&= 0 \\
z_{13} &= \max\{0+4, \underline{6+0}\} &&= 6 \\
z_{23} &= \max\{0+9, 6+4, \underline{11+0}\} &&= 11 \\
z_{33} &= \max\{0+13, 6+9, 11+4, \underline{16+0}\} &&= 16 \\
z_{43} &= \max\{0+18, 6+13, \underline{11+9}, 16+4, 20+0\} &&= 20 \\
z_{53} &= \max\{0+20, 6+18, 11+13, \underline{16+9}, 20+4, 25+0\} &&= 25
\end{alignat*}
Največji zaslužek je torej $z^* = z_{53} = 25$.
Poiščimo še optimalno razporeditev zabojev.
\begin{align*}
z_{53} &= z_{32} + p_{23} && \text{$2$ zaboja v tretjo trgovino} \\
z_{32} &= z_{11} + p_{22} && \text{$2$ zaboja v drugo trgovino} \\
z_{11} &= p_{11}          && \text{$1$ zaboj v prvo trgovino}
\end{align*}
Druga optimalna razporeditev je,
da gredo $3$ zaboji v prvo trgovino, $2$ v drugo trgovino in noben v tretjo.

\item Da bomo lahko rekonstruirali rešitev,
bomo hranili še vrednosti $k_{ij}$ ($0 \le i \le m$, $2 \le j \le n$),
tako da bo veljalo $z_{ij} = z_{k_{ij},j-1} + p_{i-k_{ij},j}$.
\begin{small}
\begin{algorithmic}
\Function{Jagode}{$(p_{ij})_{i,j=0,1}^{m,n}$}
    \For{$i = 0, \dots, m$}
        \State $z_{i1} \gets p_{i1}$
        \State $k_{i1} \gets 0$
    \EndFor
    \For{$j = 2, \dots, n$}
        \For{$i = 0, \dots, m$}
            \State $z_{ij}, k_{ij} \gets
                \max\set{(z_{h,j-1} + p_{i-h,j}, h)}{0 \le h \le i}$
        \EndFor
    \EndFor
    \State {\sl resitev} $\gets$ tabela z $n$ polji
    \State $j \gets m$
    \For{$i = n, \dots, 1$}
        \State {\sl resitev}$[i] \gets j - k_{ij}$
        \State $j \gets k_{ij}$
    \EndFor
    \State \Return ($z_{mn}$, {\sl resitev})
\EndFunction
\end{algorithmic}
\end{small}
Algoritem teče v času $O(m^2 n)$.
\end{enumerate}
\end{odgovor}
\end{naloga}
