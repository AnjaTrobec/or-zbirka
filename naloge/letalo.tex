\begin{naloga}{Janoš Vidali}{Izpit OR 28.8.2018}
\begin{vprasanje}
Pri izdelavi letala imamo faze, opisane v tabeli~\tab.

\begin{enumerate}[(a)]
\item S pomočjo topološke ureditve ustreznega grafa
določi kritična opravila in čas izdelave.
Uporabi podatke iz stolpca ``trajanje''.

\item Katero opravilo je najmanj kritično?
Najmanj kritično jo opravilo,
katerega trajanje lahko najbolj podaljšamo,
ne da bi vplivali na trajanje izdelave.

\item Naročnik bi rad, da letalo izdelamo v $75$ dneh.
Posamezna opravila lahko skrajšamo tako, da zanje zadolžimo več delavcev.
Seveda prinese tako krajšanje dodatne stroške,
poleg tega pa za vsako opravilo poznamo najmanjše možno trajanje
(glej zadnja dva stolpca v zgornji tabeli).
Zapiši linearni program,
s katerim modeliramo iskanje razporeda opravil,
ki nam prinese čim manjše stroške.
Linearnega programa ne rešuj.
\end{enumerate}

\begin{tabela}
\makebox[\textwidth][c]{
\begin{tabular}{c|l|c|c||c|c}
opravilo & opis & trajanje & pogoji & najmanjše trajanje & cena za dan manj \\
\hline
A & izgradnja nosu              & 40 dni & -    & 36 dni & 1\,000 \\
B & izgradnja trupa s krili     & 50 dni & -    & 48 dni & 1\,500 \\
C & izgradnja repa              & 35 dni & -    & 31 dni &    800 \\
D & vgraditev pilotske kabine   & 30 dni & A    & 28 dni & 1\,200 \\
E & opremljanje potniške kabine & 18 dni & F, G & 16 dni &    500 \\
F & povezovanje nosu s trupom   & 10 dni & A, B &  9 dni & 1\,100 \\
G & povezovanje repa s trupom   & 12 dni & B, C & 10 dni & 1\,300 \\
H & vgraditev motorjev          & 20 dni & B    & 18 dni & 1\,400 \\
\end{tabular}
}
\podnaslov{Podatki}
\end{tabela}
\end{vprasanje}
\begin{odgovor}
\end{odgovor}
\end{naloga}
