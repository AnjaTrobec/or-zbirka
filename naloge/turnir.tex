\begin{naloga}{Janoš Vidali}{Kolokvij OR 19.4.2019}
\begin{vprasanje}
Na turnirju poleg tebe sodeluje še $n$ igralcev,
pri čemer se bo vsak igralec soočil z vsakim drugim igralcem v enem dvoboju.
Vsak igralec na začetku dobi $M$ žetonov,
za katere se (po začetni fazi zbiranja informacij) vnaprej odloči,
kako jih bo razdelil med dvoboje.
V dvoboju zmaga igralec z večjim številom vloženih žetonov.
Če oba igralca vložita enako število žetonov, je izid dvoboja izenačen.
Zmagovalec dvoboja dobi $3$ točke, poraženec $0$ točk,
v primeru izenačenega izida pa vsak dvobojevalec dobi $1$ točko.
Cilj vsakega igralca je zbrati čim večje število točk.

\begin{enumerate}[(a)]
\item Denimo, da za vsakega igralca izveš, kako namerava igrati.
Naj bo torej $c_i$ število žetonov,
ki jih bo $i$-ti nasprotnik ($1 \le i \le n$) vložil v dvoboj proti tebi.
Svojo strategijo želiš načrtovati tako, da bi dosegel čim večje število točk.
Zapiši celoštevilski linearni program, ki modelira ta problem.

\item Celoštevilskemu linearnemu programu iz prejšnje točke dodaj omejitve,
ki modelirajo sledeče pogoje:
    \begin{itemize}
    \item Proti nobenemu od nasprotnikov iz množice $A$
    ne smeš doseči izenačenega izida.
    \item Izgubiš lahko kvečjemu proti dvema nasprotnikoma iz množice $B$.
    \item Če izgubiš proti nasprotniku $u$,
    moraš proti nasprotnikoma $v$ in $w$ doseči vsaj $4$ točke.
    \item Več kot $t$ žetonov lahko vložiš v največ $k$ dvobojev.
    \item Izgubiti moraš vsaj $\ell$ dvobojev.
    \end{itemize}
\end{enumerate}
\end{vprasanje}

\begin{odgovor}
\begin{enumerate}[(a)]
\item Naj bo $x_i$ število žetonov,
ki jih vložimo v dvoboj z $i$-tim nasprotnikom ($1 \le i \le n$).
Poleg tega bomo uvedli še spremenljivke $y_i$ in $z_i$ ($1 \le i \le n$),
katerih vrednosti interpretiramo kot
\begin{align*}
y_i &= \begin{cases}
1, & \text{če ne izgubimo proti $i$-temu nasprotniku, in} \\
0  & \text{sicer;}
\end{cases}
\shortintertext{ter}
z_i &= \begin{cases}
1, & \text{če premagamo $i$-tega nasprotnika, in} \\
0  & \text{sicer.}
\end{cases}
\end{align*}
Zapišimo celoštevilski linearni program.
\begin{alignat*}{2}
&& \max &\ \sum_{i=1}^n (y_i + 2z_i) \quad \text{p.p.} \\
\forall i \in \{1, \dots, n\}: &\ & x_i &\ge 0, \quad x_i \in \Z \\
\forall i \in \{1, \dots, n\}: &\ & 0 \le y_i &\le 1, \quad y_i \in \Z \\
\forall i \in \{1, \dots, n\}: &\ & 0 \le z_i &\le 1, \quad z_i \in \Z
\opis{Porabimo $M$ žetonov}
&& \sum_{i=1}^n x_i &= M
\opis{$y_i = 1$ natanko tedaj, ko $x_i \ge c_i$}
\forall i \in \{1, \dots, n\}: &\ & c_i y_i &\le x_i \le c_i - 1 + M y_i
\opis{$z_i = 1$ natanko tedaj, ko $x_i > c_i$}
\forall i \in \{1, \dots, n\}: &\ & (c_i + 1) z_i &\le x_i \le c_i + M z_i
\end{alignat*}

\item Zapišimo še dodatne omejitve.
\odstraniprostor
\begin{alignat*}{2}
\opis{Proti nasprotnikom iz $A$ ne igramo izenačeno}
\forall i \in A: &&{} y_i &\le z_i
\opis{Največ dva poraza proti nasprotnikom iz $B$}
&& \sum_{i \in B} y_i &\ge |B| - 2
\opis{Če izgubimo proti $u$, dosežemo vsaj $4$ točke proti $v$ in $w$}
4 y_u + y_v + y_w &&{}+ 2 z_v + 2 z_w &\ge 4
\opis{Največ $k$ dvobojev z več kot $t$ žetoni}
\forall i \in \{1, \dots, n\}: &&{} 0 \le r_i &\le 1, \quad r_i \in \Z \\
\forall i \in \{1, \dots, n\}: &&{} x_i &\le t + M r_i \\
&& \sum_{i=1}^n r_i &\le k
\opis{Vsaj $\ell$ porazov}
&& \sum_{i=1}^n y_i &\le n - \ell
\end{alignat*}
\end{enumerate}
\end{odgovor}
\end{naloga}
