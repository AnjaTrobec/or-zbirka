\begin{naloga}{Boštjan Gabrovšek}{Izpit OR 9.2.2011}
\begin{vprasanje}
Selektor košarkaške reprezentance bi rad sestavil petčlansko začetno postavo, ki bo imela povprečno
višino kar se da visoko. Na voljo ima sledeče igralce:
\begin{center}
\begin{tabular}{lll}
Igralec & Višina & Pozicija \\ \hline
Anže    & 213    & Center   \\
Borut   & 208    & Center   \\
Ciril   & 203    & Krilo    \\
David   & 192    & Krilo    \\
Emil    & 196    & Krilo    \\
Filip   & 197    & Obramba  \\
Gorazd  & 200    & Obramba  \\
Hugo    & 195    & Obramba  \\
\end{tabular}
\end{center}
Pri izbiri mora selektor upoštevati naslednje pogoje:
\begin{itemize}
\item v začetni postavi morajo biti zastopane vse tri pozicije,
\item v rezervi mora biti bodisi Ciril bodisi Filip, ne pa oba,
\item v začetni postavi je lahko največ en center,
\item če začne Borut ali David, mora Hugo ostati v rezervi.
\end{itemize}
Formuliraj problem kot celoštevilski linearni program.
\end{vprasanje}

\begin{odgovor}
Za vsakega igralca želimo vedeti, ali bo član začetne postave. Igralcem po vrsti
dodelimo števila od 1 do 8.
Za $i$-tega igralca ($1 \le i \le 8$) bomo uvedli spremenljivko $x_{i}$,
katere vrednost interpretiramo kot
$$
x_{i} = \begin{cases}
1, & \text{če bo na $i$-ti igralec v začetni postavi, in} \\
0  & \text{sicer.}
\end{cases}
$$
Zapišimo celoštevilski linearni program.
\begin{alignat*}{2}
\max \ (213x_1 + 208x_2 + 203x_3 + 192x_4 + 196x_5 + 197x_6 + 
200x_7 + 195x_8 && \text{p.p.} \\
\forall i \in \{1, \dots, 8\}: &\ &
0 \le x_i &\le 1, \quad x_i \in \Z \\
\opis{V postavi so zastopane vse pozicije}
x_1 + x_2 > 0 \\
x_3 + x_4 + x_5 > 0 \\
x_6 + x_7 + x_8 > 0
\opis{V postavi je ali Ciril ali Filip}
x_3 + x_6 = 1
\opis{V postavi je največ en center}
x_1 + x_2 \leq 1
\opis{Če začne Borut ali David, Hugo ostane v rezervi}
y_8 = 1 - x_8 \\
x_2 + x_4  \leq 2y_8
\end{alignat*}
\end{odgovor}
\end{naloga}
