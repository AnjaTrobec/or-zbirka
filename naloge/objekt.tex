\begin{naloga}{David Gajser}{Izpit OR 28.8.2015}
\begin{vprasanje}
Gradimo objekt in želimo narediti opravila iz tabele~\tab,
za katera vemo tudi, katera druga opravila morajo biti predhodno izvedena.
\begin{enumerate}[(a)]
\item Opravila A--J določajo usmerjen acikličen graf,
kjer so povezave določene s predhodnimi opravili.
Nariši ta graf in ga topološko uredi
(lahko ga že narišeš topološko urejenega).

\item Določi kritična opravila, kritično pot in trajanje gradnje.

\item Katero opravilo je najmanj kritično?
\end{enumerate}

\begin{tabela}
\begin{tabular}{c|l|c|c}
& aktivnost & trajanje & predhodna opravila \\
\hline
A & postavitev stebra 1 & 4 dni & / \\
B & postavitev stebra 2 & 3 dni & D \\
C & postavitev stebra 3 & 5 dni & H \\
D & postavitev stebra 4 & 6 dni & A, E \\
E & postavitev stene 1  & 4 dni & A, G \\
F & postavitev stene 2  & 2 dni & H, I \\
G & postavitev stene 3  & 2 dni & / \\
H & postavitev stene 4  & 1 dan & / \\
I & postavitev stene 5  & 6 dni & C \\
J & betoniranje plošče  & 2 dni & C, E, F \\
\end{tabular}
\podnaslov{Podatki}
\end{tabela}
\end{vprasanje}
\begin{odgovor}
\end{odgovor}
\end{naloga}
