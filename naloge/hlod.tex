\begin{naloga}{?}{Kolokvij OR 31.5.2012}
\begin{vprasanje}
Imamo hlod dolžine $\ell$,
ki bi ga radi razžagali na $n$ označenih mestih
$0 < x_1 < x_2 < \dots < x_n < \ell$.
Eno rezanje stane toliko, kolikor je dolžina hloda, ki ga režemo.
Ko hlod prerežemo, dobimo dva manjša hloda, ki ju režemo naprej.
Poiskati želimo zaporedje rezanj z najmanjšo ceno.
\begin{enumerate}[(a)]
\item S pomočjo dinamičnega programiranja reši problem v splošnem.
Oceni tudi njegovo časovno zahtevnost.
\item Reši problem pri podatkih $\ell = 10$ in $(x)_{i=1}^4 = (3, 5, 7, 8)$.
\end{enumerate}
\end{vprasanje}

\begin{odgovor}
\begin{enumerate}[(a)]
\item Naj bo $x_0 = 0$ in $x_{n+1} = \ell$,
ter naj bo $p_{ij}$ cena rezanja dela hloda od $x_i$ do $x_{j+1}$.
Določimo začetne pogoje in rekurzivne enačbe.
\begin{align*}
p_{ii} &= 0 && (0 \le i \le n) \\
p_{ij} &= x_{j+1} - x_i + \min\set{p_{ih} + p_{h+1,j}}{i \le h < j}
&& (0 \le i < j \le n)
\end{align*}
Vrednosti $p_{ij}$ ($0 \le i \le j \le n$)
računamo v leksikografskem vrstnem redu glede na $(j-i, i)$.
Najmanjšo ceno rezanja dobimo kot $p^* = p_{0n}$.

Za izračun spremenljivke $p_{ij}$ potrebujemo $j-i$ korakov.
Skupno število korakov algoritma je torej
\begin{multline*}
\sum_{i=0}^n \sum_{j=i}^n (j-i) = \sum_{i=0}^n \sum_{j=0}^{n-i} j =
\sum_{i=0}^n {(n-i)(n-i+1) \over 2} = \\
\sum_{i=0}^n {i(i+1) \over 2} = {n(n+1)(n+2) \over 6} = O(n^3) .
\end{multline*}

\item Izračunajmo vrednosti $p_{ij}$ ($0 \le i < j \le n$).
\begin{alignat*}{3}
p_{01} &=&  5 - 0 &+ \min\{\underline{0+0}\} &&= 5 \\
p_{12} &=&  7 - 3 &+ \min\{\underline{0+0}\} &&= 4 \\
p_{23} &=&  8 - 5 &+ \min\{\underline{0+0}\} &&= 3 \\
p_{34} &=& 10 - 7 &+ \min\{\underline{0+0}\} &&= 3 \\
p_{02} &=&  7 - 0 &+ \min\{\underline{0+4}, 5+0\} &&= 11 \\
p_{13} &=&  8 - 3 &+ \min\{\underline{0+3}, 4+0\} &&= 8 \\
p_{24} &=& 10 - 5 &+ \min\{\underline{0+3}, 3+0\} &&= 8 \\
p_{03} &=&  8 - 0 &+ \min\{\underline{0+8}, 5+3, 11+0\} &&= 16 \\
p_{14} &=& 10 - 3 &+ \min\{0+8, \underline{4+3}, 8+0\} &&= 14 \\
p_{04} &=& 10 - 0 &+ \min\{0+14, \underline{5+8}, 11+3, 16+0\} &&= 23
\end{alignat*}
Najmanjša cena rezanja je torej $23$, dosežemo pa jo tako,
da najprej razrežemo na mestu $x_2 = 5$,
nato pa en del še na mestu $x_1 = 3$,
drugi del pa še na mestih $x_3 = 7$ in $x_4 = 8$
(pri slednjih dveh vrstni red ni pomemben).
\end{enumerate}
\end{odgovor}
\end{naloga}
