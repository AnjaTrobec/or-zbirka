\begin{naloga}{?}{Vaje OR 20.4.2016}
\begin{vprasanje}
Zasnuj podatkovno strukturo za grafe,
ki temelji na matrični predstavitvi.
Podatkovna struktura naj ima sledeče metode:
\begin{itemize}
\item {\tt \_\_init\_\_(G)}: ustvarjanje praznega grafa
\item {\tt dodajVozlisce(G, u)}: dodajanje novega vozlišča
\item {\tt dodajPovezavo(G, u, v)}: dodajanje nove povezave
\item {\tt brisiPovezavo(G, u, v)}: brisanje povezave
\item {\tt brisiVozlisce(G, u)}: brisanje vozlišča
\item {\tt sosedi(G, u)}: seznam sosedov danega vozlišča
\end{itemize}
Za vsako od naštetih metod podaj tudi njeno časovno zahtevnost
v odvisnosti od števila vozlišč, števila povezav in stopenj vhodnih vozlišč.
Oceni tudi prostorsko zahtevnost celotne strukture.
\end{vprasanje}
\begin{odgovor}
\end{odgovor}
\end{naloga}
