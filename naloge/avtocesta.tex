\begin{naloga}{Janoš Vidali}{Izpit OR 19.6.2019}
\begin{vprasanje}
Gradbeno podjetje gradi avtocestni odsek na težavnem terenu.
Identificirali so faze grad\-nje, ki so zbrane v tabeli~\tab.

\begin{enumerate}[(a)]
\item S pomočjo topološke ureditve ustreznega grafa določi kritična opravila
in čas izdelave.
Uporabi podatke iz stolpca ``trajanje''.

\item Katero opravilo je najmanj kritično?
Najmanj kritično jo opravilo,
katerega trajanje lahko najbolj podaljšamo,
ne da bi vplivali na trajanje izdelave.

\item Naročnik bi rad, da podjetje odsek zgradi v $200$ dneh.
Posamezna opravila lahko skrajšajo tako, da zanje zadolžijo več delavcev.
Seveda prinese tako krajšanje dodatne stroške,
poleg tega pa za vsako opravilo poznamo najmanjše možno trajanje
(glej zadnja dva stolpca v tabeli~\tab).
Zapiši linearni program,
s katerim modeliramo iskanje razporeda opravil,
ki nam prinese čim manjše stroške.
Linearnega programa ne rešuj.
\end{enumerate}
%
\begin{tabela}
\makebox[\textwidth][c]{
\begin{tabular}{c|l|c|c||c|c}
opravilo & opis & trajanje & pogoji & najmanjše trajanje & cena za dan manj \\
\hline
$a$ & izgradnja južnega priključka         & 45 dni & /      & 40 dni &  300 \\
$b$ & izgradnja severnega priključka       & 20 dni & /      & 15 dni &  200 \\
$c$ & vrtanje tunelske cevi                & 90 dni & $a, b$ & 70 dni &  700 \\
$d$ & postavitev zahodnega stebra viadukta & 25 dni & $a$    & 22 dni & 1100 \\
$e$ & postavitev vzhodnega stebra viadukta & 30 dni & $c$    & 26 dni & 1500 \\
$f$ & gradnja cestišča na viaduktu         & 35 dni & $d, e$ & 28 dni &  900 \\
$g$ & ureditev napeljave v tunelu          & 80 dni & $c$    & 65 dni &  400 \\
$h$ & asfaltiranje viadukta                & 10 dni & $f$    &  8 dni &  150
\end{tabular}
}
\podnaslov{Podatki}
\end{tabela}
\end{vprasanje}

\begin{odgovor}
\begin{enumerate}[(a)]
\item Projekt lahko predstavimo z uteženim grafom s slike~\fig,
iz katerega je raz\-vid\-na topološka ureditev $s, a, b, d, c, e, g, f, h, t$.
V tabeli~\tab[avtocesta-resitev]
so podani naj\-zgod\-nej\-ši začetki opravil in najpoznejši začetki,
da se celotno trajanje gradnje ne podaljša.
Gradnja bo trajala najmanj $215$ dni,
edina kritična pot pa je $s - a - c - g - t$,
ki tako vsebuje vsa kritična opravila.

\item Iz tabele~\tab[avtocesta-resitev] je razvidno,
da je najmanj kritično opravilo $d$,
saj lahko njegovo trajanje podaljšamo za $100$ dni,
ne da bi vplivali na trajanje celotnega projekta.

\item Zapišimo linearni program,
ki bo za vsako opravilo imel spremenljivki $x_u$ in $y_u$,
ki določata število dni, za katerega skrajšamo trajanje opravila $u$,
oziroma čas začetka izvajanja opravila $u$.
\begin{alignat*}{2}
\min &&&{} 300 x_a + 200 x_b + 700 x_c + 1100 x_d \\[-1mm]
&&+\,&{} 1500 x_e + 900 x_f + 400 x_g + 150 x_h
\opis{Opravila brez pogojev}
y_a, &&{} y_b &\ge 0
\opis{Odvisnosti med opravili}
y_c - y_a &\,+& x_a &\ge 45 \\
y_c - y_b &\,+& x_b &\ge 20 \\
y_d - y_a &\,+& x_a &\ge 45 \\
y_e - y_c &\,+& x_c &\ge 90 \\
y_f - y_d &\,+& x_d &\ge 25 \\
y_f - y_e &\,+& x_e &\ge 30 \\
y_g - y_c &\,+& x_c &\ge 90 \\
y_h - y_f &\,+& x_f &\ge 35
\opis{Želeni čas trajanja}
y_g &\,-& x_g &\le 120 \\
y_h &\,-& x_h &\le 190
\opis{Minimalno trajanje opravil}
0 &\le&\, x_a &\le 5 \\
0 &\le&\, x_b &\le 5 \\
0 &\le&\, x_c &\le 20 \\
0 &\le&\, x_d &\le 3 \\
0 &\le&\, x_e &\le 4 \\
0 &\le&\, x_f &\le 7 \\
0 &\le&\, x_g &\le 15 \\
0 &\le&\, x_h &\le 2
\end{alignat*}
\end{enumerate}
%
\begin{slika}
\pgfslika
\podnaslov{Graf odvisnosti med opravili in kritična pot}
\end{slika}
%
\begin{tabela}
\setlabel{avtocesta-resitev}
$$
\begin{array}{c|cccccccccc}
& s & a & b & d & c & e & g & f & h & t \\ \hline
\text{najzgodnejši začetek} &
0 & 0_s & 0_s & 45_a & 45_a & 135_c & 135_c & 165_e & 200_f & 215_g \\
\text{najpoznejši začetek} &
0 & 0_c & 25_c & 145_f & 45_g & 140_f & 135_t & 170_h & 205_t & 215 \\
\text{razlika} &
0 & 0^* & 25 & 100 & 0^* & 5 & 0^* & 5 & 5 & 0
\end{array}
$$
\podnaslov{Razporejanje opravil}
\end{tabela}\end{odgovor}
\end{naloga}
