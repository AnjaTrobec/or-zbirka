\begin{naloga}{David Gajser}{Kolokvij OR 9.5.2013}
\begin{vprasanje}
Oviratlon je tekalna preizkušnja
na 8 do 10 kilometrov dolgi poti z različnimi ovirami.
Zanima nas, na koliko različnih načinov lahko pridemo od štarta do cilja.
Dan je utežen usmerjen acikličen graf $G$ ter vozlišči $s$ in $t$,
ki predstavljata štart oziroma cilj.
Uteži na povezavah nam predstavljajo,
na koliko načinov jih lahko prečkamo.

\begin{enumerate}[(a)]
\item Zapiši algoritem, ki reši dani problem.
Kakšna je njegova časovna zahtevnost?

\item Reši nalogo za graf s slike~\fig.
\end{enumerate}

\begin{slika}
\pgfslika
\podnaslov{Graf}
\end{slika}
\end{vprasanje}

\begin{odgovor}
\begin{enumerate}[(a)]
\item Algoritem deluje podobno kot algoritem {\sc NajkrajšaPot}
iz naloge~\res[topo]{}(b),
le da vsakič prišteje število možnih poti iz trenutnega vozlišča.
Pri tem sledi topološki ureditvi,
ki jo izračuna z algoritmom {\sc Topo} iz naloge~\res[topo]{}(a).
\begin{small}
\begin{algorithmic}
\Function{ŠteviloPoti}{$G = (V, E), s, t, L : E \to \R$}
\State $C \gets$ slovar z vrednostjo $0$ za vsako vozlišče $v \in V$
\State $C[s] \gets 1$
\For{$u \in$ {\sc Topo}$(G)$}
    \For{$v \in \Adj(G, u)$}
        \State $C[v] \gets C[v] + C[u] \cdot L_{uv}$
    \EndFor
\EndFor
\State \Return $C[t]$
\EndFunction
\end{algorithmic}
\end{small}
Časovna zahtevnost algoritma je $O(m + n)$,
kjer je $n$ število vozlišč in $m$ število povezav grafa.

\item Potek izvajanja algoritma na grafu s slike~\fig
je prikazan v tabeli~\tab.
Pri tem je uporabljena topološka ureditev vozlišč
$[s, a, b, d, c, e, f, g, t]$.
Od vozlišča $s$ do vozlišča $t$ je mogoče priti na $58\,000$ načinov.
\end{enumerate}

\begin{tabela}
$$
\begin{array}{c|c|c|c}
u & v & L_{uv} & C[s, a, b, c, d, e, f, g, t] \\ \hline
  &   &        & [1, 0, 0, 0, 0, 0, 0, 0, 0] \\
s & a &     10 & [1, 10, 0, 0, 0, 0, 0, 0, 0] \\
s & b &     10 & [1, 10, 10, 0, 0, 0, 0, 0, 0] \\
s & c &     10 & [1, 10, 10, 10, 0, 0, 0, 0, 0] \\
s & d &     10 & [1, 10, 10, 10, 10, 0, 0, 0, 0] \\
a & b &      5 & [1, 10, 60, 10, 10, 0, 0, 0, 0] \\
a & e &      5 & [1, 10, 60, 10, 10, 50, 0, 0, 0] \\
b & e &      2 & [1, 10, 60, 10, 10, 170, 0, 0, 0] \\
d & c &      5 & [1, 10, 60, 60, 10, 170, 0, 0, 0] \\
c & f &      2 & [1, 10, 60, 60, 10, 170, 120, 0, 0] \\
e & g &     10 & [1, 10, 60, 60, 10, 170, 120, 1\,700, 0] \\
f & g &     10 & [1, 10, 60, 60, 10, 170, 120, 2\,900, 0] \\
g & t &     10 & [1, 10, 60, 60, 10, 170, 120, 2\,900, 58\,000]
\end{array}
$$
\podnaslov{Potek izvajanja algoritma}
\end{tabela}
\end{odgovor}
\end{naloga}
