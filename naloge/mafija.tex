\begin{naloga}{Bašić, Gajser}{Izpit OR 26.6.2012}
\begin{vprasanje}
V nekem velikem mehiškem mestu je center mesta sestavljen iz ulic,
ki so pravokotne ena na drugo
in iz ptičje perspektive izgledajo kot pravokotna karirasta mreža.
Vse banke se nahajajo v centru,
in sicer tik ob križiščih (v vsakem križišču kvečjemu ena banka).
Križišča so označena s svojimi koordinatami v mreži.
V tem mestu so ropi bank zelo pogosti.
Tako pogosti, da se včasih roparji na begu zaletijo med seboj
(saj bežijo pred policijskimi vozili pri veliki hitrosti).
Če ropar uspe pripeljati do ulice, ki vodi ven iz centra, je ``rešen'',
saj center obdajajo slumi, kjer policija nima moči,
saj jih nadzirajo mafijski botri.
Roparji so se sestali na Veliki mafijski konferenci
in sklenili temu kaosu narediti konec.
Odločili so se, da bodo naročili informacijski sistem,
kamor bodo vsak večer vnesli lokacije bank, ki jih nameravajo oropati,
program pa jim bo sestavil načrt ropov za naslednji dan.
Rope je mogoče izvesti, če:
\begin{itemize}
\item so vse lokacije bank različne
(dve ekipi ne smeta na isti dan oropati iste banke), in
\item za vsako lokacijo obstaja pot, ki pelje ven iz centra,
tako da nobeni dve poti ne gresta vzdolž iste ulice ali čez isto križišče
(s tem poskušamo preprečiti, da bi se roparji medsebojno zaleteli;
smer vožnje pri tem ni pomembna).
\end{itemize}
Opiši algoritem, ki bo kot vhod dobil celi števili $n$ in $m$,
ki predstavljata število vodoravnih in navpičnih ulic,
ter seznam lokacij bank $L$.
Algoritem naj vrne seznam poti,
po katerih bodo roparji bežali (za vsako lokacijo po eno pot).
Če vseh ropov ni mogoče izvesti, naj algoritem javi napako.

Za $n = 7$, $m = 6$ in
$$
L = \{(4, 5), (4, 4), (4, 3), (6, 4), (7, 4),
      (2, 5), (6, 5), (2, 4), (2, 3), (6, 3)\}
$$
lahko dobimo na primer poti, podane na sliki~\fig.

\begin{slika}
\pgfslika
\podnaslov{Shema ulic in bank ter primer izhoda}
\end{slika}
\end{vprasanje}

\begin{odgovor}
Zapišimo želeni algoritem.
\begin{small}
\begin{algorithmic}
\Function{NačrtRopov}{$n, m, L$}
    \If{kateri od vnosov v $L$ se ponovi}
        \State \Return {\em Izvedba ni možna -- lokacije bank niso vse različne}
    \EndIf
    \State $V \gets \{s, t\}
              \cup \set{u_{ij}, v_{ij}}{1 \le i \le n, 1 \le j \le m}$
    \State $\begin{aligned}
        E \gets&{} \set{s \to u_{ij}}{(i, j) \in L} \\
           \cup&{} \set{u_{ij} \to v_{ij}}{1 \le i \le n, 1 \le j \le m} \\
           \cup&{} \set{v_{ij} \to u_{i-1, j}, v_{i-1, j} \to u_{ij}}{2 \le i \le n, 1 \le j \le m} \\
           \cup&{} \set{v_{ij} \to u_{i, j-1}, v_{i, j-1} \to u_{ij}}{1 \le i \le n, 2 \le j \le m} \\
           \cup&{} \set{v_{i1} \to t, v_{im} \to t}{2 \le i \le n-1} \\
           \cup&{} \set{v_{1j} \to t, v_{nj} \to t}{2 \le j \le m-1} \\
           \cup&{} \set{v_{ij} \to t}{i \in \{1, n\}, j \in \{1, m\}}
        \end{aligned}$
    \State $C \gets$ slovar kapacitet z vrednostjo $1$ za vsako povezavo iz $E$
    \State $F \gets$ {\sc FordFulkerson}$(G = (E, V), C, s, t)$
        \hfill vrne slovar pretokov za vsak $e \in E$
    \If{$\exists (i, j) \in L: F[s \to u_{ij}] < 1$}
        \State \Return {\em Izvedba ni možna --
            ni mogoče zagotoviti disjunktnih izhodnih poti}
    \EndIf
    \State $S \gets$ prazen seznam
    \For{$(i, j) \in L$} \hfill sestavimo poti
        \State $P \gets$ prazen seznam
        \State $x \gets u_{ij}$
        \While{$x \ne t$}
            \State $P.\append((i, j))$
            \State $i', j' \gets$ indeksa, da je $x = u_{i'j'}$
            \State $x' \gets$ vozlišče, da je $F[v_{i'j'} \to x'] = 1$
            \State $x, i, j \gets x', i', j'$
        \EndWhile
        \State $S.\append(P)$
    \EndFor
    \State \Return $S$
\EndFunction
\end{algorithmic}
\end{small}
Algoritem sestavi usmerjen graf,
v katerem je križišče na lokaciji $(i, j)$
pred\-stav\-lje\-no s povezavo $u_{ij} \to v_{ij}$,
med sosednjima križiščema $(i, j)$ in $(i', j')$
pa sta povezavi $v_{ij} \to u_{i'j'}$ in $v_{i'j'} \to u_{ij}$.
Poleg tega so v grafu še povezave od izvora $s$ do vozlišča $u_{ij}$
za vsako banko na križišču $(i, j)$,
ki jo želijo oropati,
in povezave $v_{ij}$ do ponora $t$ za vsako križišče $(i, j)$ na robu centra.
Ker imajo vse povezave kapaciteto $1$,
bo mogoče skozi vsako križišče potovati največ enkrat
-- tako bo mogoče tudi vsako ulico uporabiti največ enkrat.
Maksimalen pretok bo $|L|$
(ali ekvivalentno,
pretok po povezavi $s \to u_{ij}$ bo $1$ za vsak $(i, j) \in L$)
natanko tedaj,
ko je mogoče za vsako banko najti pot iz centra,
ne da bi se te poti srečale v kakšnem križišču.
\end{odgovor}
\end{naloga}
