\begin{naloga}[problem kombinatorične dražbe]{?}{Izpit OR 14.6.2013}
\begin{vprasanje}
Pohorski škratje so se v osrednji sobani svojega podzemnega kraljestva
zbrali na dražbi.
Vodja dražbe prodaja predmete $A = \{1, 2, \dots, k\}$,
ki so jih škratje uspeli ``pridobiti'' od ljudi.
Ponudbe so pari $(B_i, c_i)$,
kjer je $B_i \subseteq A$ in je $c_i > 0$ cena ponudbe.
Škrat, ki je ponudil $j$-to ponudbo,
je torej pripravljen za predmete $B_i$ plačati $c_i$
(če bi dobil le nekaj od predmetov iz $B_i$, ne bi plačal nič).
Vodja dražbe prejme $k$ ponudb $\{(B_i, c_i)\}_{i=1}^k$,
od vsakega škrata največ eno.
Predmete želi razdeliti med ponudnike, da bo iztržil kar največ.

\begin{enumerate}[(a)]
\item Predstavi problem kot celoštevilski linearni program!

\item Kako se celoštevilski linearni program spremeni,
če dodamo naslednja pogoja?
    \begin{itemize}
    \item Če je ponudba 1 uspešna,
    potem mora biti uspešna tudi ponudba 2 ali ponudba 3.
    \item Če so uspešne vse ponudbe s sodim indeksom,
    potem morajo biti ne\-uspeš\-ne vsaj tri ponudbe z lihim indeksom.
    \end{itemize}
\end{enumerate}
\end{vprasanje}

\begin{odgovor}
\begin{enumerate}[(a)]
\item Za $i$-to ponudbo ($1 \le i \le k$) bomo uvedli spremenljivko $x_i$,
katere vrednost interpretiramo kot
$$
x_i = \begin{cases}
1, & \text{če naj dražitelj sprejme $i$-to ponudbo, in} \\
0  & \text{sicer.}
\end{cases}
$$
Zapišimo celoštevilski linearni program.
\begin{alignat*}{2}
\max \ \sum_{i=1}^n c_i x_i && \text{p.p.} \\
\forall i \in \{1, \dots, k\}: &\ & 0 \le x_i &\le 1, \quad x_i \in \Z \\
\forall p \in A: &\ & \sum_{\substack{i=1 \\ p \in B_i}}^k x_i &\le 1
\end{alignat*}

\item Celoštevilskemu linearnemu programu dodamo še dva pogoja.
    \begin{itemize}
    \item Če je ponudba 1 uspešna,
    potem mora biti uspešna tudi ponudba 2 ali ponudba 3:
    $$
    x_1 \le x_2 + x_3
    $$
    \item Če so uspešne vse ponudbe s sodim indeksom,
    potem morajo biti ne\-uspeš\-ne vsaj tri ponudbe z lihim indeksom:
    $$
    \sum_{i=1}^{k} \left(2 + (-1)^i\right) x_{i}
    \le k + 2 \left\lfloor {k \over 2} \right\rfloor - 3
    $$
    Razlaga: na levi strani ponudbe z lihim indeksom štejemo enkrat,
    ponudbe s sodim indeksom
    (teh je $\lfloor {k \over 2} \rfloor$)
    pa trikrat.
    Če sprejmemo vse ponudbe s sodim indeksom,
    potem desna stran omejuje število sprejetih ponudb na $k - 3$
    -- vsaj treh ponudb z lihim indeksom torej nismo sprejeli.
    Če pa ne sprejmemo vseh ponudb s sodim indeksom,
    je vsota na levi za vsaj $3$ manjša od maksimalne,
    tako da lahko sprejmemo poljubno število ponudb z lihim indeksom.
    \end{itemize}
\end{enumerate}
\end{odgovor}
\end{naloga}
