\begin{naloga}{?}{Vaje OR 16.3.2016}
\begin{vprasanje}
Obravnavajmo posplošen scenarij iz naloge~\nal[proj].
\begin{enumerate}[(a)]
\item Denimo, da so projekti lahko med seboj odvisni.
Imejmo množico $V \subseteq \p^2$, ki določa,
da za vsak par projektov $(p_i, p_j) \in V$ velja,
da lahko projekt $p_i$ sponzoriramo le,
če sponzoriramo tudi projekt $p_j$.

\item Nekateri izmed projektov so lahko v konfliktu.
Naj bo $S \subseteq 2^\p$ družina množic, ki določa,
da so za vsako množico $H \in S$ projekti iz $H$ med seboj v konfliktu
(tj., hkrati lahko sponzoriramo le enega izmed njih.)
\end{enumerate}
Opiši, kako bi modelirali opisane omejitve.
\end{vprasanje}

\begin{odgovor}
Celoštevilskemu linearnemu programu iz rešitve naloge~\res[proj]
bomo dodali nove omejitve.
\begin{enumerate}[(a)]
\item $\displaystyle \forall (p_i, p_j) \in V : x_i \le x_j$
\item $\displaystyle \forall H \in S : \sum_{p_i \in H} x_i \le 1$
\end{enumerate}
\end{odgovor}
\end{naloga}
