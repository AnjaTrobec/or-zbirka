\begin{naloga}{Janoš Vidali}{Kolokvij OR 3.6.2019}
\begin{vprasanje}
{\em Neodvisna množica} vozlišč grafa $G = (V, E)$
je taka množica $S \subseteq V$,
da sta poljubni vozlišči iz množice $S$ nesosedni v $G$,
torej $uv \not\in E$ za vsaka $u, v \in S$.

Dano je drevo $T = (V, E)$ in uteži vozlišč $c_v$ ($v \in V$).
V drevesu $T$ želimo najti {\em najtežjo neodvisno množico}
-- torej tako množico vozlišč $S \subseteq V$,
ki maksimizira vsoto njihovih uteži, torej vrednost $\sum_{u \in S} c_u$.

\begin{enumerate}[(a)]
\item Denimo, da je drevo $T$ podano kot neusmerjen graf,
predstavljen s seznami sosedov.
Razloži, kako lahko sestaviš slovar $\pred$,
ki za vsako vozlišče $v \in V$ določa njegovega prednika,
če za koren izbereš vozlišče $r \in V$.
Koren $r$ je lahko izbran poljubno, zanj pa velja $\pred[r] = \Null$.
Kako iz seznamov sosedov in slovarja $\pred$ ugotovimo,
katera vozlišča so neposredni nasledniki danega vozlišča v drevesu?

\item Napiši rekurzivne enačbe za reševanje problema
najtežje neodvisne množice v drevesu $T$.
Razloži, kaj pred\-stav\-lja\-jo spremenljivke,
v kakšnem vrstnem redu jih računamo, ter kako dobimo optimalno rešitev.
Predpostaviš lahko,
da imaš poleg seznamov sosedov in uteži vozlišč drevesa $T$
na voljo tudi slovar $\pred$ kot v točki (a).
\namig{za vsako vozlišče uporabi dve spremenljivki
-- eno za primer, ko je vozlišče izbrano, in eno za primer, ko ni.}

\item Natančno opiši postopek (z besedami ali psevdokodo),
ki iz zgoraj izračunanih vred\-no\-sti
sestavi najtežjo neodvisno množico v $T$.

\item Oceni časovno zahtevnost algoritma iz točk (b) in (c).

\item S svojim algoritmom poišči najtežjo neodvisno množico
na drevesu s slike~\fig.
Iz re\-šit\-ve naj bo jasno, kako poteka izvajanje algoritma.
\end{enumerate}
%
\begin{slika}
\makebox[\textwidth][c]{
\pgfslika
}
\podnaslov[\nal{}(e)]{Drevo}
\end{slika}
\end{vprasanje}

\begin{odgovor}
\begin{enumerate}[(a)]
\item Algoritem {\sc bfs} iz naloge~\res[bfs] vrne želeni slovar.
Ker lahko koren poljubno izberemo,
lahko slovar $\pred$ dobimo s klicem {\sc bfs}$(T, V, \NOp)$.
Neposredni nasledniki vozlišča $u$ v drevesu $T$
so tisti njegovi sosedi $v \in \Adj(T, u)$,
za katere velja $v \ne \pred[u]$.

\item Naj bosta $x_u$ in $y_u$ teži najtežje neodvisne množice $S$
v poddrevesu s korenom $u$ (glede na slovar $\pred$),
za katero velja $u \in S$ oziroma $u \not\in S$.
Potem velja
\begin{align*}
x_u &= c_u + \!\!\!\!\sum_{\substack{v \sim u \\ v \ne \pred[u]}}\!\!\!\! y_v
\qquad \text{in} \\
y_u &= \sum_{\substack{v \sim u \\ v \ne \pred[u]}}\!\!\!\! \max\{x_x, y_v\} .
\end{align*}
Vrednosti $x_u$ in $y_u$ računamo od listov v smeri proti korenu $r$
(tj., v topološki ureditvi glede na $\pred$).
Težo najtežje neodvisne množice dobimo kot $c^* = \max\{x_r, y_r\}$.

\item Da iz vrednosti $x_u$ in $y_u$ ($u \in V$)
izluščimo najtežjo neodvisno množico $S$ v drevesu $T$,
se sprehodimo od korena proti listom (npr. z iskanjem v širino)
in posamezno vozlišče $u$ dodamo v množico $S$,
če njegov prednik ni v $S$ in velja $x_u \ge y_u$.
\begin{small}
\begin{algorithmic}
\Function{NajtežjaNeodvisnaMnožica}%
        {$T = (V, E), r, (x_u)_{u \in V}, (y_u)_{u \in V}$}
    \State $S \gets$ prazna množica
    \Function{Visit}{$u, v$}
        \If{$v \not\in S \land x_u \ge y_u$}
            \State $S.\add(u)$
        \EndIf
    \EndFunction
    \State {\sc bfs}$(T, \{r\}, \visit)$
    \State \Return $S$
\EndFunction
\end{algorithmic}
\end{small}

\item Naj bo $n$ število vozlišč drevesa $T$.
Potem ima $T$ natanko $n-1$ povezav.
Za izračun najtežje neodvisne množice naredimo tri preglede v širino,
pri čemer v vsakem vozlišču porabimo konstantno mnogo časa.
Časovna zahtevnost celotnega algoritma je torej $O(n)$.

\item Izračunajmo vrednosti $x_u$ in $y_u$ ($u \in V$),
če za koren vzamemo vozlišče $a$.
\begin{alignat*}{4}
x_i &&&= 9 &\qquad y_i &&&= 0 \\
x_j &&&= 4 &\qquad y_j &&&= 0 \\
x_k &&&= 2 &\qquad y_k &&&= 0 \\
x_d &&&= 1 &\qquad y_d &&&= 0 \\
x_e &= 1+0+0 &&= 1 &\qquad
y_e &= \max\{\underline{9}, 0\} + \max\{\underline{4}, 0\} &&= 13 \\
x_f &&&= 4 &\qquad y_f &&&= 0 \\
x_g &= 8+0 &&= 8 &\qquad
y_g &= \max\{\underline{2}, 0\} &&= 2 \\
x_h &&&= 4 &\qquad y_h &&&= 0 \\
x_b &= 1+0+13+0 &&= 14 &\qquad y_b &= \max\{\underline{1}, 0\}
+ \max\{1, \underline{13}\} + \max\{\underline{4}, 0\} &&= 18 \\
x_c &= 9+2+0 &&= 11 &\qquad
y_c &= \max\{\underline{8}, 0\} + \max\{\underline{4}, 0\} &&= 12 \\
x_a &= 8+18+12 &&= 38 &\qquad
y_a &= \max\{14, \underline{18}\} + \max\{11, \underline{12}\} &&= 30
\end{alignat*}
Teža najtežje neodvisne množice $S$ je torej $c^* = \max\{x_a, y_a\} = 38$.
Poglejmo, katera vozlišča so v množici $S$.
\begin{align*}
c^* &= x_a \\
x_a &= c_a + y_b + y_c && a \in S \\
y_b &= x_d + y_e + x_f && b \not\in S \\
y_c &= x_g + x_h && c \not\in S \\
x_d &= c_d && d \in S \\
y_e &= x_i + x_j && e \not\in S \\
x_f &= c_f && f \in S \\
x_g &= c_g + y_k && g \in S \\
x_h &= c_h && h \in S \\
x_i &= c_i && i \in S \\
x_j &= c_j && j \in S \\
y_k &= 0 && k \not\in S
\end{align*}
Najtežja neodvisna množica je torej $S = \{a, d, f, g, h, i, j\}$.
\end{enumerate}
\end{odgovor}
\end{naloga}
