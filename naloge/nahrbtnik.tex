\begin{naloga}{?}{Vaje OR 6.4.2016}
\begin{vprasanje}
Imamo nahrbtnik nosilnosti $M$ kilogramov.
Danih je $n$ objektov z vrednostmi $v_i$ in težami $t_i$ ($1 \le i \le n$).
Problem nahrbtnika sprašuje po izbiri predmetov
$I \subseteq \{1, 2, \dots, n\}$,
ki maksimizira njihovo skupno vrednost pri omejitvi $\sum_{i \in I} t_i \le M$.
\begin{enumerate}[(a)]
\item Napiši rekurzivne enačbe za opisani problem.
\item Z uporabo rekurzivnih enačb reši problem za parametre $M = 8$, $n = 8$,
\begin{align*}
(v_i)_{i=1}^n &= (9, 9, 8, 11, 10, 15, 3, 12) \quad \text{in} \\
(t_i)_{i=1}^n &= (3, 5, 1, 4, 3, 8, 2, 7).
\end{align*}
\end{enumerate}
\end{vprasanje}

\begin{odgovor}
\begin{enumerate}[(a)]
\item Naj bosta $I_j$ in $c_j$
optimalna izbira predmetov ter njihova skupna vrednost,
če imamo nahrbtnik z nosilnostjo $j$ kilogramov.
Določimo začetni pogoj in rekurzivno enačbo.
\begin{align*}
(c_0, I_0) &= (0, \emptyset) \\
(c_j, I_j) &= \max\left(\{(c_{j-1}, I_{j-1})\} \cup
    \set{(c_{j-t_i} + v_i, I_{j-t_i} \cup \{i\})}%
        {1 \le i \le n, t_i \le j, i \not\in I_{j-t_i}}\right)
\end{align*}
Vrednosti $c_j$ in $I_j$ računamo naraščajoče po indeksu $j$
($1 \le j \le M$).
Optimalno skupno vrednost dobimo kot $c^* = c_M$,
optimalno izbiro predmetov pa kot $I^* = I_M$.

\item Potek algoritma je prikazan v tabeli~\tab.
V vsaki vrstici je podčrtana vrednost pri tistem predmetu,
ki ga dodamo v ustreznem koraku.
Če nobena vred\-nost ni podčrtana,
do izboljšave ni prišlo in zato uporabimo predhodno rešitev.
Preberemo lahko,
da je optimalna izbira predmetov $3$, $4$ in $5$ s skupno vrednostjo $29$.
\end{enumerate}

\begin{tabela}
\makebox[\textwidth][c]{
\begin{tabular}{c|cc||cccccccc}
\multicolumn{2}{c}{} & $v_i$ & 9 & 9 & 8 & 11 & 10 & 15 & 3 & 12 \\
\multicolumn{2}{c}{} & $t_i$ & 3 & 5 & 1 &  4 &  3 &  8 & 2 &  7 \\
\cline{3-11}
$j$ & $c_j$ & $I_j$ & 1 & 2 & 3 &  4 &  5 &  6 & 7 &  8 \\
\hline
0 &  0 & $\emptyset$ & \\
1 &  8 & $\{3\}$ &&& \underline{$0+8$} \\
2 &  8 & $\{3\}$ &&&&&&& $0+3$ \\
3 & 11 & $\{3, 7\}$ & $0+9$ &&&& $0+10$ && \underline{$8+3$} \\
4 & 18 & $\{3, 5\}$ & $8+9$ &&& $0+11$ & \underline{$8+10$} && $8+3$ \\
5 & 19 & $\{3, 4\}$ & $8+9$ & $0+9$ && \underline{$8+11$} & $8+10$ \\
6 & 21 & $\{3, 5, 7\}$ &
    $11+9$ & $8+9$ && $8+11$ & \underline{$11+10$} && $18+3$ \\
7 & 27 & $\{1, 3, 5\}$ &
    $\underline{18+9}$ & $8+9$ && $11+11$ &&& $19+3$ & $0+12$ \\
8 & 29 & $\{3, 4, 5\}$ &
    $19+9$ & $11+9$ && \underline{$18+11$} & $19+10$ & $0+15$ && $8+12$
\end{tabular}
}
\podnaslov[\res{}(b)]{Potek reševanja}
\end{tabela}
\end{odgovor}
\end{naloga}
