\begin{naloga}{?}{Izpit OR 5.6.2014}
\begin{vprasanje}
Dano je zaporedje števil $c_1, c_2, \dots, c_n$.
Igralca $A$ in $B$ igrata igro,
pri kateri $A$ bodisi z začetka bodisi s konca danega zaporedja
izbere eno (prej še neizbrano) število,
$B$ pa tako izbira dvakrat zapored.
Igrata, dokler števil ne zmanjka.
Zmaga igralec z največjo skupno vsoto izbranih števil.
Denimo, da začne igralec $A$.
S pomočjo dinamičnega programiranja določi maksimalni znesek,
ki ga $A$ lahko doseže, če tudi drugi igralec igra optimalno.
\end{vprasanje}
\begin{odgovor}
\end{odgovor}
\end{naloga}
