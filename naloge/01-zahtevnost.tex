\subsection{Zahtevnost algoritmov}

\begin{naloga}{Janoš Vidali}{Vaje OR 21.2.2018}
\begin{vprasanje}
Naj bo $A[1 \dots n][1 \dots n]$ matrika (tj., seznam seznamov)
dimenzij $n \times n$.
Dan je spodnji program:
\begin{small}
\begin{algorithmic}
\For{$i = 1, \dots, n$}
    \For{$j = i+1, \dots, n$}
        \State $A[i][j] \gets A[j][i]$
    \EndFor
\EndFor
\end{algorithmic}
\end{small}


\begin{enumerate}[(a)]
\item Kaj počne zgornji program?
\item Oceni število korakov, ki jih opravi zgornji program,
v odvisnosti od parametra $n$.
\end{enumerate}
\end{vprasanje}
\begin{odgovor}
\end{odgovor}
\end{naloga}


\begin{naloga}{Janoš Vidali}{Vaje OR 21.2.2018}
\begin{vprasanje}
Naj bo $\ell[1 \dots n]$ seznam,
ki ima na začetku vse vrednosti nastavljene na $0$.
Dan je spodnji program:

\begin{small}
\begin{algorithmic}
\State $i \gets 1$
\While{$i \le n$}
    \State $\ell[i] \gets 1 - \ell[i]$
    \If{$\ell[i] = 1$}
        \State $i \gets 1$
    \Else
        \State $i \gets i+1$
    \EndIf
\EndWhile
\end{algorithmic}
\end{small}

\begin{enumerate}[(a)]
\item Kaj se dogaja, ko teče zgornji program?
\item Oceni število korakov, ki jih opravi zgornji program,
v odvisnosti od parametra $n$.
\end{enumerate}

\end{vprasanje}
\begin{odgovor}
\end{odgovor}
\end{naloga}


\begin{naloga}{Janoš Vidali}{Vaje OR 21.2.2018}
\begin{vprasanje}
Algoritem {\sc BubbleSort} uredi vhodni seznam $\ell[1 \dots n]$ tako,
da zamenjuje sosednje elemente v nepravem vrstnem redu:
\begin{small}
\begin{algorithmic}
\Function{BubbleSort}{$\ell[1 \dots n]$}
    \State $z \gets n$
    \While{$z > 1$}
        \State $y \gets 1$
        \For{$i = 2, \dots, z$}
            \If{$\ell[i-1] > \ell[i]$}
                \State $\ell[i-1], \ell[i] \gets \ell[i], \ell[i-1]$
                \State $y \gets i$
            \EndIf
        \EndFor
        \State $z \gets y-1$
    \EndWhile
\EndFunction
\end{algorithmic}
\end{small}

\begin{enumerate}[(a)]
\item Izvedi algoritem na seznamu $[7, 11, 16, 7, 5]$.
\item Določi časovno zahtevnost algoritma.
\end{enumerate}

\end{vprasanje}
\begin{odgovor}
\end{odgovor}
\end{naloga}


\begin{naloga}{Janoš Vidali}{Vaje OR 12.10.2016}
\begin{vprasanje}
Algoritem {\sc MergeSort} uredi vhodni seznam tako,
da ga najprej razdeli na dva dela,
nato vsakega rekurzivno uredi,
nazadnje pa zlije dobljena urejena seznama.
\begin{enumerate}[(a)]
\item S psevdokodo zapiši algoritem {\sc MergeSort}.
\item Izvedi algoritem na seznamu $[7, 11, 16, 7, 5, 0, 14, 1, 19, 13]$.
\item Določi časovno zahtevnost algoritma.
\end{enumerate}

\end{vprasanje}
\begin{odgovor}
\end{odgovor}
\end{naloga}


\begin{naloga}{Janoš Vidali}{Vaje OR 21.2.2018}
\begin{vprasanje}
Število $n$ želimo razcepiti
na dva netrivialna celoštevilska faktorja,
kar storimo s sledečim algoritmom:
\begin{small}
\begin{algorithmic}
\Function{Razcep}{$n$}
    \For{$i = 2, \dots, \lfloor \sqrt{n} \rfloor$}
        \If{$n/i$ je celo število}
            \State \Return $(i, n/i)$
        \EndIf
    \EndFor
    \State \Return $n$ je praštevilo
\EndFunction
\end{algorithmic}
\end{small}
Določi časovno zahtevnost algoritma.
Ali je ta algoritem polinomski?

\end{vprasanje}
\begin{odgovor}
\end{odgovor}
\end{naloga}


\begin{naloga}{Janoš Vidali}{Vaje OR 21.2.2018}
\begin{vprasanje}
Zapiši rekurziven algoritem,
ki na vhod dobi celo število $n$ in teče v času $O(\sqrt{n})$.
Uporaba korenjenja ni dovoljena.

\end{vprasanje}
\begin{odgovor}
\end{odgovor}
\end{naloga}


\begin{naloga}{?}{Kolokvij OR 17.3.2013}
\begin{vprasanje}
Dani so končna neprazna množica $S \subset \N$ moči $n$,
število $k \in \{1, 2, \dots, n\}$ ter algoritem {\sc Alg}:
\begin{small}
\begin{algorithmic}
\Function{Alg}{$S, k$}
    \State $x \gets$ naključen element $S$
    \If{$|S| = 1$}
        \State \Return $x$
    \Else
        \State $S^+ \gets \set{y \in S}{y \ge x}$
        \State $S^- \gets \set{y \in S}{y < x}$
        \If{$|S^-| < k$}
            \State \Return {\sc Alg}$(S^+, k - |S^-|)$
        \Else
            \State \Return {\sc Alg}$(S^-, k)$
        \EndIf
    \EndIf
\EndFunction
\end{algorithmic}
\end{small}
Ugotovi, kaj je izhod algoritma pri danih vhodnih podatkih $S$ in $k$.
Oceni časovno zahtevnost algoritma v najslabšem in v povprečnem primeru.
\end{vprasanje}
\begin{odgovor}
\end{odgovor}
\end{naloga}
