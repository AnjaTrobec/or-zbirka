\begin{naloga}{Bašić, Gajser}{Kolokvij OR 5.4.2012}
\begin{vprasanje}
Naj bo $G = (V, E)$ enostaven usmerjen graf
(tj., brez zank in vzporednih povezav),
podan z matriko sosednosti $A = (a_{ij})_{i,j=1}^n$ ($n = |V|$),
kjer je $a_{ij} = 1$, če je $ij \in E$, in $a_{ij} = 0$ sicer.
Želimo ugotoviti, ali v $G$ obstaja kak usmerjen trikotnik
(tj., usmerjen cikel dolžine $3$).
\begin{enumerate}[(a)]
\item Konrad je napisal naslednji algoritem, ki naj bi rešil ta problem:
\begin{small}
\begin{algorithmic}
\For{$i = 1, \dots, n$}
    \For{$j = 1, \dots, n$}
        \For{$k = 1, \dots, n$}
            \For{$\ell = 1, \dots, n$}
                \If{$a_{ij} = a_{jk} = a_{k\ell} = 1 \land \ell = i$}
                    \State \Return \True
                \EndIf
            \EndFor
        \EndFor
    \EndFor
\EndFor
\State \Return \False
\end{algorithmic}
\end{small}
Ali algoritem deluje pravilno?
Kakšna je njegova časovna zahtevnost?

\item V psevdokodi napiši algoritem, ki reši problem v času $O(n^3)$.
Koliko korakov naredi tvoj algoritem v najboljšem
in koliko v najslabšem primeru?
\end{enumerate}
\end{vprasanje}

\begin{odgovor}
\begin{enumerate}[(a)]
\item Algoritem deluje pravilno,
saj za vsako urejeno četverico vozlišč preveri,
ali so ta zaporedoma povezana ter ali je začetno vozlišče enako končnemu,
in se ustavi, ko naleti na četverico, ki ustreza tem pogojem
-- torej na trikotnik.
Časovna zahtevnost je $O(n^4)$.

\item Zapišimo učinkovitejši algoritem.
\begin{small}
\begin{algorithmic}
\For{$i = 1, \dots, n$}
    \For{$j = 1, \dots, n$}
        \If{$a_{ij} = 1$}
            \For{$k = 1, \dots, n$}
                \If{$a_{jk} = a_{ki} = 1$}
                    \State \Return \True
                \EndIf
            \EndFor
        \EndIf
    \EndFor
\EndFor
\State \Return \False
\end{algorithmic}
\end{small}
Zgornji algoritem reši problem v času $O(n^3)$.

V najboljšem primeru, torej ko obstaja usmerjen trikotnik $(1, 2, 3)$
(ignorirali bomo primere, ko ima graf največ dve vozlišči),
algoritem enkrat vstopi v zunanjo zanko in dvakrat v srednjo
-- pri drugem vstopu v srednjo zanko nato še trikrat vstopi v notranjo zanko.
Ocenimo torej, da v tem primeru algoritem opravi $6$ korakov.

V najslabšem primeru usmerjenega trikotnika ni in se tako vse zanke iztečejo.
Algoritem bo $n$-krat vstopil v zunanjo zanko, $n^2$-krat v srednjo zanko,
za vsako povezavo pa bo še $n$-krat vstopil v notranjo zanko.
V tem primeru torej algoritem opravi $n + n^2 + mn$ korakov,
kjer je $m$ število povezav v grafu.
\end{enumerate}
\end{odgovor}
\end{naloga}
