\begin{naloga}{David Gajser}{Izpit OR 14.6.2013}
\begin{vprasanje}
Pohorski škratje že 644.~leto zapored
organizirajo tradicionalni tek po rovih od $A$ do $\check{Z}$,
ki letos poteka na trasi, prikazani na sliki~\fig{}.

Udeleženci začnejo v točki $A$ in končajo v točki $\check{Z}$,
zmeraj pa tečejo le v smeri povezav.
Vsakemu škratu posebej pot določijo organizatorji,
zaradi dobre pripravljenosti pa lahko pričakujemo,
da bodo vsi prišli do cilja.
Ker so škratje zelo težki,
je med tekom po rovu nevarnost, da se le-ta poruši.
Na povezavah je napisano število udeležencev, ki jih vsak rov prenese.
Organizatorji takoj za tem, ko rov preči en tekmovalec manj,
kot je nosilnost rova, rov zaprejo
(npr. rov $x$ se zapre, ko ga preči $(x-1)$-ti tekmovalec
-- $x$-ti tekmovalec ne stopi v rov).

\begin{enumerate}[(a)]
\item Trenutno stanje trase je takšno, da je $x = 1$ in $y = 2$.
Največ koliko udeležencev lahko preteče od $A$ do $\check{Z}$?
Odgovor utemelji.
Napiši, kako naj organizatorji pripravijo poti za škrate
(tj., koliko škratov so pripravljeni spustiti po kateri poti).

\item Rova $x$ in $y$ lahko škratje dodatno utrdijo,
ostalih ni moč preurejati.
Da lahko povečajo nosilnost rova za enega udeleženca, morajo delati $10$ dni.
Najmanj koliko dni morajo delati in na katerem rovu,
da bo na tradicionalnem teku od $A$ do $\check{Z}$
lahko sodelovalo kar največ udeležencev?
Odgovor utemelji.
Napiši, kako naj organizatorji v tem primeru pripravijo poti za škrate
(tj., koliko škratov so pripravljeni spustiti po kateri poti).
\end{enumerate}

\begin{slika}
\pgfslika
\podnaslov{Graf}
\end{slika}
\end{vprasanje}
\begin{odgovor}
\end{odgovor}
\end{naloga}
