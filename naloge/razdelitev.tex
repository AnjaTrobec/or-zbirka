\begin{naloga}[pravična delitev]{Blaž Jelenc}{Kolokvij OR 17.4.2014}
\begin{vprasanje}
Dana je množica naravnih števil $\{a_1, a_2, \dots, a_n\}$.
Naš cilj je razdeliti to množico na dva dela $S_1$ in $S_2$ tako,
da bo absolutna vrednost razlike
med vsoto elementov v $S_1$ in $S_2$ čim manjša.
S pomočjo dinamičnega programiranja opiši postopek,
ki izračuna minimalno vrednost te razlike.
\end{vprasanje}

\begin{odgovor}
Naj bo $v_{ij}$ največja vsota števil izmed $a_1, a_2, \dots, a_i$,
ki ne preseže vrednosti $j$.
Definirajmo še $s = \left\lfloor {1 \over 2} \sum_{i=1}^n a_i \right\rfloor$
(tj., največja možna vrednost za $\min\{\sum S_1, \sum S_2\}$).
Določimo začetne pogoje in rekurzivne enačbe.
\begin{align*}
v_{00} &= v_{i0} = v_{0j} = 0 && (1 \le i \le n, 1 \le j \le s) \\
v_{ij} &= \max\left(\{v_{i-1, j}\}
                    \cup \set{a_i + v_{i-1, j-a_i}}{a_i \le j}\right)
&& (1 \le i \le n, 1 \le j \le s)
\end{align*}
Vrednosti $v_{ij}$ računamo v leksikografskem vrstnem redu indeksov
(npr.~najprej naraščajoče po $i$, nato pa naraščajoče po $j$).
Optimalna vrednost $v^* = v_{ns}$ nam pove največjo vrednost za $\sum S_1$
(brez škode za splošnost lahko predpostavimo, da ta ne preseže $\sum S_2$);
razliko dobimo kot $\sum_{i=1}^n a_i - 2v^*$.
\end{odgovor}
\end{naloga}
