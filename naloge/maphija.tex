\begin{naloga}{Jernej Azarija}{Izpit OR 24.5.2016}
\begin{vprasanje}
Kavarna {\em ma$\phi$ja} je razvila recept za novo torto.
Slaščičarna {\em sladkeoperacijskeraziskave}
je za eksluzivno pravico do recepta pripravljena plačati $15\,002 €$.
Če se kavarna {\em ma$\phi$ja} odloči za samostojno prodajo torte,
jih začetni vložek stane $10\,000 €$,
za vsako prodano torto pa zaslužijo $25 €$.
Po njihovi presoji
je verjetnost uspeha recepture (prodajo $100\,000$ tort) enaka $0.4$,
verjetnost propada (prodali bi zgolj $10\,000$ tort) pa $0.6$.
Kavarna se lahko odloči, da zaprosi za pomoč tudi pod\-jet\-je,
ki na osnovi degustacij torte sestavi mnenje o uspehu recepta.
Svetovanje jih stane $35\,000 €$,
natančnost svetovanja pa opisuje spodnja tabela.
\begin{center}
\begin{tabular}{c|cc}
$P(\text{mnenje svetovalca} \;|\;\ \text{uspeh recepta})$
& Recept uspe & Recept ne uspe \\ \hline
Ugodno   & ${3 \over 4}$ & ${1 \over 2}$ \\
Neugodno & ${1 \over 4}$ & ${1 \over 2}$
\end{tabular}
\end{center}
Kako naj se kavarna {\em ma$\phi$ja} odloči?
Zakaj?
\end{vprasanje}
\begin{odgovor}
\end{odgovor}
\end{naloga}
