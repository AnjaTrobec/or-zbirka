\begin{naloga}{Jernej Azarija}{Izpit OR 24.5.2016}
\begin{vprasanje}
Kavarna {\em ma$\phi$ja} je razvila recept za novo torto.
Slaščičarna {\em sladkeoperacijskeraziskave}
je za eksluzivno pravico do recepta pripravljena plačati $15\,002 €$.
Če se kavarna {\em ma$\phi$ja} odloči za samostojno prodajo torte,
jih začetni vložek stane $10\,000 €$,
za vsako prodano torto pa zaslužijo $25 €$.
Po njihovi presoji
je verjetnost uspeha recepture (prodajo $100\,000$ tort) enaka $0.4$,
verjetnost propada (prodali bi zgolj $10\,000$ tort) pa $0.6$.
Kavarna se lahko odloči, da zaprosi za pomoč tudi pod\-jet\-je,
ki na osnovi degustacij torte sestavi mnenje o uspehu recepta.
Svetovanje jih stane $35\,000 €$,
natančnost svetovanja pa opisuje spodnja tabela.
\begin{center}
\begin{tabular}{c|cc}
$P(\text{mnenje svetovalca} \;|\;\ \text{uspeh recepta})$
& Recept uspe & Recept ne uspe \\ \hline
Ugodno   & ${3 \over 4}$ & ${1 \over 2}$ \\
Neugodno & ${1 \over 4}$ & ${1 \over 2}$
\end{tabular}
\end{center}
Kako naj se kavarna {\em ma$\phi$ja} odloči?
Zakaj?
\end{vprasanje}

\begin{odgovor}
Definirajmo spremenljivke, ki jih bomo potrebovali,
in izpišimo znane podatke.
\begin{align*}
X &\dots \text{zaslužek}\\
A &\dots \text{recept uspe}\\
B &\dots \text{prosijo za nasvet}\\
C &\dots \text{ugodno mnenje}\\
D &\dots \text{sami proizvajajo}
\end{align*}
\begin{align*}
P(A) &= \frac{2}{5} &
P(\overline{A}) &= \frac{3}{5} \\
P(C \mid A)& = \frac{3}{4} &
P(C \mid \overline{A}) &= \frac{1}{2} \\
P(\overline{C} \mid A) &= \frac{1}{4} &
P(\overline{C} \mid \overline{A}) &= \frac{1}{2}
\end{align*}
Izračunajmo verjetnosti in pričakovane vrednosti,
ki jih nato vpišemo v od\-lo\-čit\-ve\-no drevo.
\begin{alignat*}{3}
P(C) &= P(C \mid A) P(A) + P(C \mid \overline{A}) P(\overline{A})
&&= \frac{3}{4}\cdot \frac{2}{5} + \frac{1}{2}\cdot \frac{3}{5}
&&= \frac{3}{5} \\
P(\overline{C}) &= P(\overline{C} \mid A) P(A) + P(\overline{C} \mid \overline{A}) P(\overline{A})
&&= \frac{1}{4}\cdot \frac{2}{5} + \frac{1}{2}\cdot \frac{3}{5}
&&= \frac{2}{5} \\
P(A \mid C) &= \frac{P(C \mid A) P(A)}{P(C)}
&&= {3/4 \cdot 2/5 \over 3/5} &&= \frac{1}{2} \\
P(\overline{A} \mid C) &= \frac{P(C \mid \overline{A})P(\overline{A})}{P(C)}
&&= {1/2 \cdot 3/5 \over 3/5} &&= \frac{1}{2} \\
P(A \mid \overline{C}) &= \frac{P(\overline{C} \mid A) P(A)}{P(\overline{C})}
&&= {1/4 \cdot 2/5 \over 2/5} &&= \frac{1}{4} \\
P(\overline{A} \mid \overline{C}) &= \frac{P(\overline{C} \mid \overline{A})P(\overline{A})}{P(\overline{C})}
&&= {1/2 \cdot 3/5 \over 2/5} &&= \frac{3}{4}
\end{alignat*}

\begin{align*}
E(X \mid D \cap \overline{B} \cap A) &= 2\,490\,000 € &
P(A \mid \overline{B}) &= 0.4 \\
E(X \mid D \cap \overline{B} \cap \overline{A}) &= 240\,000 € &
P(\overline{A} \mid \overline{B})& = 0.6 \\
E(X \mid \overline{D} \cap \overline{B}) &= 15\,002 € \\[1ex]
E(X \mid B \cap C \cap A) &=2\,455\,000 € &
P(A \mid B \cap C) &= 0.5 \\
E(X \mid B \cap C \cap \overline{A}) &=205\,000 € &
P(\overline{A} \mid B \cap C) &= 0.5 \\
E(X \mid \overline{D} \cap B \cap C) &= -19\,998 € \\[1ex]
E(X \mid B \cap \overline{C} \cap A) &=2\,455\,000 € &
P(A \mid B \cap \overline{C}) &= 0.25 \\
E(X \mid B \cap \overline{C} \cap \overline{A}) &=205\,000 € &
P(\overline{A} \mid B \cap \overline{C}) &= 0.75 \\
E(X \mid \overline{D} \cap B \cap \overline{C}) &= -19\,998 €
\end{align*}


Ko s pomočjo odločitvenega drevesa s slike~\fig
izračunamo preostale pričakovane vrednosti,
lahko vidimo, da velja
$$
E(X \mid B) < E(X \mid \overline{B}) ,
$$
zato je za kavarno ugodneje,
da ne prosi za nasvet in se odloči za proizvodnjo.

\begin{slika}
\makebox[\textwidth][c]{
\pgfslika
}
\podnaslov{Odločitveno drevo}
\end{slika}
\end{odgovor}
\end{naloga}
