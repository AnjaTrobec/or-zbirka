\begin{naloga}{?}{Kolokvij OR 26.1.2010}
\begin{vprasanje}
V ugledni banki nameravajo okrepiti upravni odbor
z direktorjem za informacijske tehnologije.
Za svetovanje in pridobivanje kandidatov za to mesto
so prosili znano agencijo za kadrovsko svetovanje {\em Kadria d.o.o.}
Ti so jim priporočili prodornega g.~Miho Ajkalaja.
Po intervjuju z g.~Ajkalajem je direktor ocenil,
da bo g.~Ajkalaj z verjetnostjo $p = 3/5$ v enem letu uspešen pri svojem delu.
Če bo uspešen, bo banka dodatno prislužila $2.100.000 €$,
sicer pa bo pridelala dodatno izgubo $800.000 €$
(izobraževanje, vpeljava, odpravnina, \dots).

\begin{enumerate}[(a)]
\item Modeliraj problem v okviru teorije odločanja (stanja, odločitve).
Kakšno odločitev svetuješ direktorju glede na pričakovni dobiček?
Kako je odločitev odvisna od verjetnosti $p$?
Nariši graf, ki prikazuje optimalni pričakovani dobiček v odvisnosti od $p$.

\item Za dodatnih $80.000 €$ lahko agencija
z g.~Ajkalajem opravi dodatne inteligenčne in psihološke teste,
s katerimi lahko ugotovi, ali bo g.~Ajkalaj uspešen.
Statistični povzetek uspešnosti metode v preteklosti je naslednji:
\begin{center}
\begin{tabular}{l|cc}
$P(\text{izid testa} \;|\; \text{posl.~rezultat})$ &
\multicolumn{2}{c}{Izid testa} \\
Poslovni rezultat & pozitiven & negativen \\ \hline
uspešen   &  $7/8$ & $1/8$ \\
neuspešen &  $1/4$ & $3/4$
\end{tabular}
\end{center}
Izračunaj $\EVPI$ in $\EVE$ ter komentiraj,
ali je smiselno izvesti dodatna testiranja.

\item Nariši drevo odločitev in ugotovi,
ali naj direktor izvede testiranje in če ga,
kako naj se odloči glede zaposlitve g.~Ajkalaja.
\end{enumerate}
\end{vprasanje}

\begin{odgovor}
\begin{enumerate}[(a)]
\item Imamo eno odločitev -- ali naj zaposlimo g.~Ajkalaja.
Če ga ne zaposlimo, je dobiček $0 €$,
če pa ga zaposlimo, pa pridemo v stanje,
v katerem imamo z verjetnostjo $p$ dobiček $2.100.000 €$,
z verjetnostjo $1-p$ pa $-800.000 €$.
V tem stanju je torej pričakovan dobiček enak
$$
p \cdot 2.100.000 € - (1-p) \cdot 800.000 € =
p \cdot 2.900.000 € - 800.000 € .
$$
Za $p < 8/29$ torej pričakujemo izgubo in se nam ga ne izplača zaposliti,
sicer pa imamo dobiček in se ga nam tako izplača zaposliti.
Graf pričakovanega dobička je prikazan na sliki~\fig{ajkalaj-graf}.
Pri podani vrednosti $p = 3/5$ se nam torej izplača zaposliti g.~Ajkalaja,
saj tedaj pričakujemo dobiček $940.000 €$.

\item $\EVPI$ izračunamo tako,
da od pričakovanega dobička pri znanem izidu
odštejemo pričakovani dobiček pri neznanem izidu:
$$
\EVPI = 3/5 \cdot 2.100.000 € + 2/5 \cdot 0€ - 940.000 € = 320.000 €
$$
Ker je $\EVPI$ večji od cene dodatnega testiranja,
izračunajmo še $\EVE$.
Najprej določimo verjetnosti:
\begin{align*}
P(\text{test pozitiven}) = 7/8 \cdot 3/5 + 1/4 \cdot 2/5 &= 5/8 \\
P(\text{test negativen}) = 1/8 \cdot 3/5 + 3/4 \cdot 2/5 &= 3/8 \\
P(\text{uspešen rezultat} \;|\; \text{test pozitiven}) =
{7/8 \cdot 3/5 \over 5/8} &= 21/25 \\
P(\text{neuspešen rezultat} \;|\; \text{test pozitiven}) =
{1/4 \cdot 2/5 \over 5/8} &= 4/25 \\
P(\text{uspešen rezultat} \;|\; \text{test negativen}) =
{1/8 \cdot 3/5 \over 3/8} &= 1/5 \\
P(\text{neuspešen rezultat} \;|\; \text{test negativen}) =
{3/4 \cdot 2/5 \over 3/8} &= 4/5
\end{align*}
Sedaj lahko izračunamo $\EVE$:
\begin{multline*}
\EVE = (21/25 \cdot 2.100.000 € - 4/25 \cdot 800.000 €) \cdot 5/8 \, + \\
(1/5 \cdot 0 € + 4/5 \cdot 0 €) \cdot 3/8 - 940.000 € = 82.500 €
\end{multline*}
Ker je $\EVE$ večji od cene testiranja, se torej odločimo za testiranje.

\item Odločitveno drevo je prikazano na sliki~\fig{ajkalaj-drevo}.
Odločimo se za test -- če je ta pozitiven, zaposlimo g.~Ajkalaja, sicer pa ne.
\end{enumerate}

\begin{slika}
\pgfslika[ajkalaj-graf]
\podnaslov[\res{}(a)]{Graf pričakovanega dobička}
\end{slika}

\begin{slika}
\makebox[\textwidth][c]{
\pgfslika[ajkalaj-drevo]
}
\podnaslov[\res{}(c)]{Odločitveno drevo}
\end{slika}
\end{odgovor}
\end{naloga}
