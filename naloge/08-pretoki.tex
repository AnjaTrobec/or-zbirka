\razdelek{Pretoki in prerezi}

\begin{naloga}{?}{Kolokvij OR 19.11.2009}
\begin{vprasanje}[minmaxpretok]
Poišči minimalno vrednost parametra $a$,
da bo pretok skozi omrežje s slike~\fig{} maksimalen.
Utemelji izbiro.

\begin{slika}
\pgfslika
\podnaslov{Omrežje}
\end{slika}
\end{vprasanje}
\begin{odgovor}
\end{odgovor}
\end{naloga}


\begin{naloga}{?}{Kolokvij OR 26.1.2010}
\begin{vprasanje}
Imamo podatke za problem maksimalnega pretoka.
Ali je povezava z najmanjšo kapaciteto
(predpostavimo, da je natanko ena taka)
vedno vsebovana v kakem minimalnem prerezu?
Ali je morda vsebovana celo v vsakem minimalnem prerezu?
\end{vprasanje}
\begin{odgovor}
\end{odgovor}
\end{naloga}


\begin{naloga}{?}{Kolokvij OR 5.2.2010}
\begin{vprasanje}[lenority]
Na univerzi v mestu Lenority so asistenti res nekoliko razvajeni
in imajo močen sindikat.
(To ni nič nenavadnega, če pa univerza zahteva,
da se izpiti začnejo že ob sedmih zjutraj!)
Bliža se izpitno obdobje in potrebno je paziti izpite.
Za vsak dan so izpiti dani vnaprej.
Znano je, da asistenti ne pazijo izpitov, če so vmes ``luknje'',
saj si je to pravico izboril njihov sindikat.
Vodstvu univerze tako ne preostane nič drugega, kot da to upošteva.
Asistenti torej pridejo na faks ob neki uri,
pazijo poljubno dolgo zaporedje izpitov brez odmorov, in potem gredo domov.
Izpiti so podani v obliki časovnih intervalov,
ki se začnejo in končajo ob polni uri.
Upoštevajoč te pogoje mora vodstvo univerze venomer prositi asistente,
naj pridejo pazit izpite.
Toda tudi vodstvo si želi minimizirati ``bolečino'' ob moledovanju
in hoče vsak dan prositi kar se da malo asistentov,
naj vendarle pridejo pazit izpite.

\begin{enumerate}[(a)]
\item Modeliraj problem s pomočjo usmerjenih acikličnih grafov,
pri katerih so intervali predstavljeni z vozlišči.
Identificiraj, kaj je tvoja optimizacijska naloga na takem grafu.

\item Opiši, kako problem prevedeš na problem maksimalnih pretokov.
Natančno utemelji,
zakaj so maksimalni pretoki v tvojem modelu v bijektivni korespondenci
z optimalnimi rešitvami tvoje optimizacijske naloge.

\item S pomočjo algoritma iz prejšnje točke reši problem za petkove izpite,
ki so podani z intervali $(7, 8)$, $(8, 9)$, $(9, 10)$, $(8, 12)$,
$(10, 13)$, $(12, 15)$, $(13, 15)$, $(9, 12)$, $(7, 10)$.
\end{enumerate}
\end{vprasanje}
\begin{odgovor}
\end{odgovor}
\end{naloga}

\begin{naloga}{?}{Izpit OR 28.6.2010}
\begin{vprasanje}[pretokx]
Na sliki~\fig{} je podano enoparametrično omrežje
za problem največjega pretoka.
Parameter $x$ je poljubno nenegativno število.

\begin{enumerate}[(a)]
\item Kakšno je pridruženo omrežje,
če začnemo z ničelnim $(s, t)$-tokom
in ga najprej povečamo vzdolž povečujoče poti $s-a-c-d-t$
ter nato še vzdolž povečujoče poti $s-c-d-b-t$?

\item Poišči maksimalni $(s, t)$-tok in minimalni $(s, t)$-prerez.

\item Kapaciteto ene povezave vhodnega omrežja lahko povečamo za $1$.
Kateri povezavi naj povečamo kapaciteto,
da bo vrednost maksimalnega $(s, t)$-toka v spremenjenem omrežju čim večja?
Odgovor je načeloma odvisen od vrednosti parametra $x$.
\end{enumerate}

\begin{slika}
\pgfslika
\podnaslov{Omrežje}
\end{slika}
\end{vprasanje}
\begin{odgovor}
\end{odgovor}
\end{naloga}


\begin{naloga}{?}{Izpit OR 15.9.2010}
\begin{vprasanje}[terminali]
Na sliki~\fig{} je podano omrežje enosmernih prehodov
med letališkimi terminali skupaj s prepustnostmi (v $1000$ ljudeh na uro).

\begin{enumerate}[(a)]
\item Načrtovalce zanima,
koliko največ potnikov na uro lahko preide med terminaloma $F$ in $B$.
Predlagaj ustrezen algoritem in ga izvedi na grafu.

\item V izrednih razmerah dva terminala proglasijo za vstopna
(na njih letala le pristajajo),
ostale terminale pa za izstopne (letala le vzletajo).
Tako morajo potniki nujno prehajati od vstopnih k izstopnim terminalom.
Strateška odločitev je, da je vstopni terminal $F$,
zaradi redundance pa želijo, da bi bila vstopna terminala dva.
Ostali terminali so torej izstopni.
Katerega izmed terminalov (poleg $F$) se splača še proglasiti za vstopnega,
da bo pretočnost med vhodnimi in izhodnimi terminali kar največja?
\end{enumerate}

\begin{slika}
\pgfslika
\podnaslov{Omrežje}
\end{slika}
\end{vprasanje}
\begin{odgovor}
\end{odgovor}
\end{naloga}


\begin{naloga}{?}{Kolokvij OR 25.11.2010}
\begin{vprasanje}[greedyflow]
Nov in neizkušen predavatelj Operacijskih raziskav se odloči,
da bo poenostavil Ford-Fulkersonov algoritem.
Študentom pove,
naj za iskanje maksimalnega pretoka na omrežju s končnimi kapacitetami
uporabijo naslednji algoritem:
\begin{small}
\begin{algorithmic}
\Function{Greedyflow}{$G = (V, E), u, s, t$}
    \For{$e \in E$}
        \State $f(e) \gets 0$
    \EndFor
    \While{obstaja pot od $s$ do $t$}
        \State $P \gets$ poljubna pot od $s$ do $t$
        \State $\gamma \gets$ minimalna kapaciteta povezav poti $P$
        \For{$e \in P$}
            \State $f(e) \gets f(e) + \gamma$
            \If{$u(e) = \gamma$}
                \State odstrani $e$ iz $G$
            \Else
                \State $u(e) \gets u(e) - \gamma$
            \EndIf
        \EndFor
    \EndWhile
    \State \Return $f$
\EndFunction
\end{algorithmic}
\end{small}

\begin{enumerate}[(a)]
\item Dokaži,
da lahko na grafu s slike~\fig{}
z uporabo algoritma {\sc Greedyflow} dobimo pravilno rešitev.

\item Poišči primer omrežja in izbire poti,
tako da algoritem {\sc Greedyflow} ne da pravilnega rezultata.

\item Ali za vsako omrežje obstaja taka izbira poti,
da bo algoritem {\sc Greedyflow} dal maksimalni pretok?

\item Dokaži, da je razlika med maksimalnim pretokom in pretokom,
ki ga vrne algoritem {\sc Greedyflow}, lahko poljubno velika.
\end{enumerate}

\begin{slika}
\pgfslika
\podnaslov{Graf}
\end{slika}
\end{vprasanje}
\begin{odgovor}
\end{odgovor}
\end{naloga}


\begin{naloga}{?}{Izpit OR 9.2.2011}
\begin{vprasanje}
\begin{enumerate}[(a)]
\item Poišči usmerjen graf $G$, vozlišči $s$, $t$ in kapacitete povezav,
da bo imel $G$ natanko en minimalen $(s, t)$-prerez.

\item Poišči usmerjen graf $G$, vozlišči $s$, $t$ in kapacitete povezav,
da bo imel $G$ več minimalnih $(s, t)$-prerezov.

\item Poišči usmerjen graf $G_n$, vozlišči $s$, $t$ in kapacitete povezav,
da bo imel $G_n$ natanko $n$ minimalnih $(s, t)$-prerezov.
\end{enumerate}
\end{vprasanje}
\begin{odgovor}
\end{odgovor}
\end{naloga}


\begin{naloga}{?}{Izpit OR 28.6.2011}
\begin{vprasanje}
Navdušenje nad koncem študijskega leta na ljubljanski univerzi
je botrovalo temu,
da v ambulanti Univerzitetne klinike
$169$ študentov išče nujno zdravniško pomoč.
Vsak od študentov potrebuje eno enoto krvi,
klinika pa ima na zalogi $170$ enot krvi.
V spodnji tabeli je prikazano število enot krvi posamezne krvne skupine,
ki so na zalogi,
in število študentov z dano krvno skupino.
\begin{center}
\begin{tabular}{l|cccc}
Krvna skupina &  A &  B &  O & AB \\ \hline
Zaloga        & 46 & 34 & 45 & 45 \\
Povpraševanje & 39 & 38 & 42 & 50
\end{tabular}
\end{center}
Osebe s krvno skupino A lahko prejmejo kri tipa A ali O.
Osebe tipa B lahko prejmejo kri tipa B ali O.
Osebe tipa O lahko prejmejo samo kri tipa O.
Osebe tipa AB lahko prejmejo katerokoli krvno skupino.
Naša naloga je razporediti enote krvi tako,
da bomo z njo oskrbeli kar se da veliko število študentov.
Iz podatkov sestavi omrežje
in modeliraj nalogo kot problem maksimalnega pretoka.
Največ koliko pacientov lahko oskrbimo?
Utemelji.
\end{vprasanje}
\begin{odgovor}
\end{odgovor}
\end{naloga}
