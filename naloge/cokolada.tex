\begin{naloga}{Janoš Vidali}{Izpit OR 3.6.2019}
\begin{vprasanje}
Podjetje je od konkurenta v finančnih težavah odkupilo tovarno čokolade,
ki že nekaj časa ni bila v uporabi.
Ker že prejemajo nova naročila,
želijo v naslednjih $8$ tednih proizvesti čim večjo količino čokolade.
Ocenjujejo, da so z verjetnostjo $0.4$ stroji v dobrem stanju
in lahko proizvedejo $10\,000$ kg čokolade na teden,
z verjetnostjo $0.6$ pa so v slabem stanju
in lahko proizvedejo le $1\,000$ kg čokolade na teden.

Odločijo se lahko, da pred zagonom proizvodnje na strojih izvedejo servis.
Izvedba servisa lahko poteka gladko in konča v enem tednu,
ali pa se pojavijo težave in tako servis traja tri tedne
(za proizvodnjo tako ostane le še $7$ oziroma $5$ tednov).
Če so stroji v dobrem stanju, bo servis z verjetnostjo $0.9$ potekal gladko,
z verjetnostjo $0.1$ pa se bodo pojavile težave
-- v obeh primerih pa bodo stroji po servisu ostali v dobrem stanju.
Če so stroji v slabem stanju,
bo servis z verjetnostjo $0.2$ potekal gladko
in jih tedaj z verjetnostjo $0.7$ spravil v dobro stanje,
z verjetnostjo $0.8$ pa se bodo pojavile težave
-- tedaj je verjetnost,
da bodo stroji po servisu v dobrem stanju, enaka $0.5$.
Po zagonu proizvodnje te zaradi previsokih stroškov
ni mogoče predčasno prekiniti;
stroški servisa pa niso pomembni,
saj ga bodo morali izvesti kasneje, če se zanj ne odločijo takoj.

Kako naj se v podjetju odločijo?
Nariši odločitveno drevo in ga uporabi pri sprejemanju odločitev.
Izračunaj tudi pričakovano količino proizvedene čokolade.
\end{vprasanje}

\begin{odgovor}
\end{odgovor}
\end{naloga}
