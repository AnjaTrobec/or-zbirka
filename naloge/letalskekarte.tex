\begin{naloga}{?}{Kolokvij OR 17.4.2014}
\begin{vprasanje}
Na letalu je $100$ sedežev.
Dane so naslednje verjetnosti za število potnikov,
ki ne pridejo na vkrcanje,
pri čemer je $P(i)$ verjetnost, da natanko $i$ potnikov ne pride na vkrcanje:
$$
P(0) = 0.25, \quad P(1) = 0.5, \quad P(2) = 0.25 .
$$
\begin{enumerate}[(a)]
\item Koliko letalskih kart naj proda letalska družba,
če vsak prazen sedež na letalu prinese $100 €$ izgube,
vsak nezadovoljen potnik, ki ne dobi sedeža, pa $200 €$ izgube?

\item Za $200 €$ lahko izvedemo predhodno analizo,
ki nam napove število potnikov, ki jih ne bo na vkrcanju.
Zanesljivost analize je podana z verjetnostmi
$$
(p_{ij})_{i,j=0}^2 = \begin{bmatrix}
0.7 & 0.1 & 0 \\
0.2 & 0.8 & 0.1 \\
0.1 & 0.1 & 0.9 \\
\end{bmatrix} ,
$$
kjer je $p_{ij}$ verjetnost,
da analiza napove odpoved $i$ potnikov v primeru,
ko imamo dejansko $j$ odpovedi.
Ali naj letalska družba izvede analizo pred prodajo kart?
\end{enumerate}
\end{vprasanje}
\begin{odgovor}
\end{odgovor}
\end{naloga}
