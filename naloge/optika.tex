\begin{naloga}{David Gajser}{Izpit OR 14.6.2013}
\begin{vprasanje}
Družina škrata Bolfenka živi v rovih nekje na Pohorju.
Odločili so se, da v sobane napeljejo optični internet.
Ker je družina velika, potrebujejo algoritem, s katerim bi določili,
po katerih rovih naj poteka optični kabel
in kje naj se optični kabel priključi
na zunanje (``človeško'') optično omrežje.
Dan je povezan graf $G$ s pozitivnimi utežmi na povezavah.
Vozlišča grafa so sobane, kjer želimo imeti dostop do optičnega interneta,
povezave nam povedo, kje lahko potegnemo kabel,
uteži na povezavah pa nam povedo ceno polaganja kabla (v guldih).
Za vsako sobano je znano,
kolikšna je cena priklopa na zunanje optično omrežje iz te sobane.
Ker se škratje skrivajo,
bodo naredili priklop na zunanji internet iz natanko ene sobane.
Pomagaj škratom najti rove, po katerih naj napeljejo optični kabel,
in sobano, iz katere naj se priključijo na zunanji internet,
da bo vsaka sobana imela dostop do optičnega interneta in bo cena čim manjša!

\begin{enumerate}[(a)]
\item Opiši algoritem, ki efektivno reši zgoraj opisani problem.

\item Predlagani algoritem izvedi na grafu s slike~\fig{}.
Številke pri vsakem vozlišču
pomenijo ceno priklopa na zunanje omrežje iz sobane.
\end{enumerate}

\begin{slika}
\pgfslika
\podnaslov{Graf}
\end{slika}
\end{vprasanje}
\begin{odgovor}
\end{odgovor}
\end{naloga}
