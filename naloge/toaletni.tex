\begin{naloga}{Hillier, Lieberman}{\cite[Problem~19.3-2]{hl}}
\begin{vprasanje}
V trgovini vsak teden prodajo $600$ škatel toaletnega papirja.
Vsako naročilo stane $25 €$, za posamezno škatlo pa trgovina plača $3 €$.
Cena skladiščenja posamezne škatle je $0.05 €$ na teden.
\begin{enumerate}[(a)]
\item Denimo, da primanjkljaj ni dovoljen.
Kako pogosto naj trgovina naroča toaletni papir?
Kako velika naj bodo naročila?
\item Kaj pa, če dovolimo primanjkljaj,
ki nas stane $2 €$ na teden za vsako škatlo?
\end{enumerate}
\end{vprasanje}

\begin{odgovor}
Imamo sledeče podatke (pri časovni enoti $1$ teden).
\begin{align*}
\nu &= 600 &
K &= 25 € \\
c &= 3 € &
s &= 0.05 €
\end{align*}
Ker artiklov ne proizvajamo
(tj., ob dostavi imamo na voljo celotno količino naročila),
vzamemo $\lambda = \infty$ in torej $\alpha = \nu$.

\begin{enumerate}[(a)]
\item Ker primanjkljaj ni dovoljen, vzamemo $p = \infty$ in torej $\beta = 1$.
S temi podatki poračunamo optimalni interval naročanja $\tau^*$
in optimalno velikost naročila $M^*$.
\begin{alignat*}{3}
\tau^* &= \sqrt{2K \beta \over s \alpha}
&&= \sqrt{50 \over 30} &&\approx 1.291 \\
M^* &= \sqrt{2K \alpha \over s \beta}
&&= \sqrt{30\,000 \over 0.05} &&\approx 774.597
\end{alignat*}
Optimalni interval naročanja je torej $1.291$ tedna oziroma $9.037$ dni,
vsakič naj pa trgovina naroči okoli $775$ škatel toaletnega papirja.

\item Vzamemo $p = 2 €$ in torej $\beta = 1 + s/p = 1.025$.
Izračunajmo optimalni interval naročanja in optimalno vrednost naročila
še v tem primeru.
\begin{alignat*}{3}
\tau^* &= \sqrt{2K \beta \over s \alpha}
&&= \sqrt{51.25 \over 30} &&\approx 1.307 \\
M^* &= \sqrt{2K \alpha \over s \beta}
&&= \sqrt{30\,000 \over 0.05125} &&\approx 765.092
\end{alignat*}
Optimalni interval naročanja je v tem primeru torej $1.307$ tedna
oziroma $9.149$ dni,
vsakič naj pa trgovina naroči okoli $765$ škatel toaletnega papirja.
\end{enumerate}
\end{odgovor}
\end{naloga}
