\begin{naloga}{Alen Orbanić}{Izpit OR 15.9.2010}
\begin{vprasanje}
Na sliki~\fig je podano omrežje enosmernih prehodov
med letališkimi terminali skupaj s prepustnostmi (v $1\,000$ ljudeh na uro).

\begin{enumerate}[(a)]
\item Načrtovalce zanima,
koliko največ potnikov na uro lahko preide med terminaloma $f$ in $b$.
Predlagaj ustrezen algoritem in ga izvedi na grafu.

\item V izrednih razmerah dva terminala proglasijo za vstopna
(na njih letala le pristajajo),
ostale terminale pa za izstopne (letala le vzletajo).
Tako morajo potniki nujno prehajati od vstopnih k izstopnim terminalom.
Strateška odločitev je, da je vstopni terminal $f$,
zaradi redundance pa želijo, da bi bila vstopna terminala dva.
Ostali terminali so torej izstopni.
Katerega izmed terminalov (poleg $f$) se splača še proglasiti za vstopnega,
da bo pretočnost med vhodnimi in izhodnimi terminali kar največja?
\end{enumerate}

\begin{slika}
\pgfslika
\podnaslov{Omrežje}
\end{slika}
\end{vprasanje}
\begin{odgovor}
\end{odgovor}
\end{naloga}
