\begin{naloga}{Janoš Vidali}{Izpit OR 10.7.2017}
\begin{vprasanje}
V podjetju imajo na voljo $m$ milijonov evrov sredstev,
ki jih bodo vložili v razvoj nove aplikacije.
Denar bodo porazdelili med tri skupine.
Naj bodo $x_1$, $x_2$ in $x_3$ količine denarja (v milijonih evrov),
ki jih bodo dodelili razvijalcem, oblikovalcem in marketingu.
Vrednosti $x_1, x_2, x_3$ niso nujno cela števila.
Razvijalci morajo dobiti vsaj $a_1$ milijonov evrov,
potencial, ki ga ustvarijo, pa je $p_1 = n_1 + k_1 x_1$.
Oblikovalci morajo dobiti vsaj $a_2$ milijonov evrov,
potencial, ki ga ustvarijo, pa je $p_2 = n_2 + k_2 x_2$.
Marketing mora dobiti vsaj $a_3$ milijonov evrov,
ustvari pa faktor $p_3 = n_3 + k_3 x_3$.
Pričakovani dobiček v milijonih evrov
se izračuna po formuli $d = (p_1 + p_2) p_3$.
V podjetju bi radi sredstva porazdelili med skupine tako,
da bo pričakovani dobiček čim večji.

\begin{enumerate}[(a)]
\item Zapiši rekurzivne enačbe za reševanje danega problema.
\item Z zgoraj zapisanimi enačbami reši problem
pri podatkih $m = 15$, $a_1 = 4$, $n_1 = 3$, $k_1 = 1.5$,
$a_2 = 3$, $n_2 = 4$, $k_2 = 2$, $a_3 = 2$, $n_3 = 0.4$ in $k_3 = 0.3$.
\end{enumerate}
\end{vprasanje}

\begin{odgovor}
\begin{enumerate}[(a)]
\item Za izračun pričakovanega dobička
bomo definirali funkcije $v_i(z)$ ($i = 1, 2, 3$).
Zapišimo njihove definicije.
\begin{align*}
v_1(z) &= n_1 + k_1 z \\
v_2(z) &= \max\set{v_1(z-y) + n_2 + k_2 y}{a_2 \le y \le z - a_1} \\
v_3(z) &= \max\set{v_2(z-y) \cdot (n_3 + k_3 y)}{a_3 \le y \le z - a_1 - a_2}
\end{align*}
Pričakovani dobiček dobimo tako, da izračunamo $v_3(m)$.

\item Vstavimo podatke v zgornje formule.
\begin{align*}
v_1(z)  &= 3 + 1.5 z \\[2mm]
v_2(z)  &= \max\set{3 + 1.5 (z-y) + 4 + 2y}{3 \le y \le z - 4} \\
        &= \max\set{7 + 1.5 z + 0.5 y}{3 \le y \le z - 4} \\
        &= 5 + 2z \qquad (z \ge 7) \\[2mm]
v_3(15) &= \max\set{(5 + 2(15-y)) \cdot (0.4 + 0.3 y)}{2 \le y \le 8} \\
        &= \max\set{-0.6 y^2 + 9.7 y + 14}{2 \le y \le 8}
\end{align*}
Naj bo $f(y) = -0.6 y^2 + 9.7 y + 14$
-- gre za kvadraten polinom z negativnim vodilnim členom,
tako da ima maksimum tam, kjer je odvod ničeln.
\begin{align*}
0 = f'(y) &= -1.2 y + 9.7 \\
y &= 9.7/1.2 > 8
\end{align*}
Opazimo torej, da maksimum leži desno od intervala $[2, 8]$,
kar pomeni, da velja $v_3(15) = f(8) = 53.6$.
Podjetje bo torej maksimiziralo pričakovani dobiček na $53.8$ milijonov evrov,
če dodeli $8$ milijonov evrov marketingu,
$3$ milijone evrov oblikovalcem in $4$ milijone evrov razvijalcem.
\end{enumerate}
\end{odgovor}
\end{naloga}
