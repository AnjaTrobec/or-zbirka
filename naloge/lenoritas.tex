\begin{naloga}{?}{Izpit OR 28.6.2010}
\begin{vprasanje}
Na univerzi na grškem otoku Lenoritas
so zaradi gospodarske krize uvedli varčevalne ukrepe,
ki vključujejo tudi nov režim plačevanja za izpite,
ki jih pazijo asistenti.
Asistentom sta ostali le dve boniteti:
nadomestilo za izvedbo strnjenega zaporedja izpitov
ter nadomestilo za vsako ``luknjo'', daljšo od pol ure,
med strnjenimi zaporedji izpitov.
Ostale bonitete,
kot so trinajsta in štirinajsta plača
ter dodatek za točnost na delovnem mestu,
je univerza pred kratkim ukinila.
Tako se univerzi splača, da uporabi čim manj asistentov,
ki pazijo strnjena zaporedja izpitov brez ``lukenj''.
Za vsak dan so termini izpitov podani v naprej.
Podatki za ponedeljek so $(7, 8)$, $(8, 9)$, $(9, 10)$, $(8, 12)$,
$(10, 13)$, $(12, 15)$, $(13, 15)$, $(9, 12)$, $(7, 10)$.

Ne glede na ekonomičnost se je dekan odločil,
da bo v ponedeljek vodji sindikata asistentov dr.~Dusanikusu Semolitakisu
zagodel in mu dal kar najdaljše možno strnjeno zaporedje izpitov.
Izberi ustrezen algoritem za izvedbo te naloge
s kar se da najboljšo časovno zahtevnostjo (in jo tudi navedi),
ter s pomočjo le-tega reši dekanov ``problem'' za ponedeljek.

\nadaljevanje{lenority}
\end{vprasanje}
\begin{odgovor}
\end{odgovor}
\end{naloga}
