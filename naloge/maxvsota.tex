\begin{naloga}{?}{Izpit OR 12.5.2016}
\begin{vprasanje}
Dan je seznam števil $A = [a_1, a_2, \dots, a_n]$.
Želimo poiskati maksimalno vsoto takega podzaporedja v $A$,
ki ne vsebuje dveh zaporednih členov seznama.

\begin{enumerate}[(a)]
\item Napiši rekurzivne enačbe za reševanje opisanega problema.

\item S pomočjo dinamičnega programiranja napiši algoritem,
ki rešuje dani problem.
Kakšna je njegova časovna zahtevnost?
\end{enumerate}
\end{vprasanje}
\begin{odgovor}

\begin{enumerate}[(a)]

\item Naj bo $V_i$ največja vsota nesosednjih členov zaporedja, ki jo lahko dobimo,
če gledamo samo elemente od 1 do $i$, in vključuje $a_i$.
\begin{align*}
V_{-1} &= 0 \\
V_0 &= 0 \\
V_i &= \max_{j < i - 1}\left(V_j\right) + a_i
\qquad (1 \le i \le n)
\end{align*}
Vidimo, da na $i$-tem koraku iščemo največjo prejšnjo nesosednjo vsoto $V_j$,
ki ji bomo priključili element $a_i$, torej izračunamo največjo vsoto,
ki zadošča pogoju nesosednosti in vključuje člen $a_i$.

Za izračun vrednosti $V_i$ moramo imeti podane že vse prejšnje $V_j$,
torej vse $V_j$ z $j < i$.
To dosežemo tako, da te vrednosti računamo naraščajoče z začetkom v $1$ in koncem v $n$.
Po izračunu vseh vrednosti $V_i$ bo maksimalna vrednost iskanega problema
enaka $V^* = \max_{1 \leq i \leq n}(V_i)$.

Podamo lahko še alternativno rekurzivno zvezo za ta problem, in sicer
definirajmo $U_i$ kot največjo vsoto nesosednjih členov zaporedja,
ki jo dobimo, če gledamo samo elemente od $1$ do $i$.
Ta definicija se razlikuje od zgornje po tem, da dobljena vsota ne vključuje nujno elementa $a_i$.
\begin{align*}
U_{-1} &= 0 \\
U_0 &= 0 \\
U_i &= \max\left(U_{i - 2} + a_i, U_{i - 1}\right)
\qquad (1 \le i \le n)
\end{align*}
Ideja tega rekurzivnega zapisa je podobna zgornji,
le da se namesto iska\-nja maksimuma odločamo,
ali bomo $i$-ti člen vključili v vsoto oziroma ga bomo izpustili.

Vrstni red iskanja vrednosti $U_i$ določimo z naraščajočim zaporedjem indeksov $i$, ko $1\leq i \leq n$,
saj imamo tako pred reševanjem problema $U_i$
definirane vse predhodne vrednosti.
Ko izračunamo vse vrednosti $U_i$,
je maksimum vsote zaporedja enak $U^* = U_n$,
kar sledi neposredno iz definicije vrednosti $U_n$.

\item Algoritem bo sledil drugi rekurzivni zvezi.

\begin{small}
\begin{algorithmic}
\Function{MaxVsota}{$(a_i)_{i=1}^n$}
	\State $U_{-1} \gets 0$
	\State $U_0 \gets 0$
	\For{$i = 1, 2, \dots, n$}
		\State $U_i \gets \max(U_{i-2} + a_i, U_{i-1})$
	\EndFor
    \State $\ell \gets$ prazen seznam
    \State $i \gets n$
    \While{$i > 0$}
        \If{$U_i = U_{i-1}$}
            \State $i \gets i-1$
        \Else
            \State $\ell.\append(i)$
            \State $i \gets i-2$
        \EndIf
    \EndWhile
    \State $\ell.\reverse()$
	\State \Return $(U_n, \ell)$
\EndFunction
\end{algorithmic}
\end{small}

Algoritem enostavno upošteva drugo rekurzivno zvezo in iterativno,
naraščajoče po $i$,
računa vrednosti $U_i$, dokler ne pride do $U_n$,
kar vrne kot optimalno vrednost.
Časovna zahtevnost je očitno $O(n)$, saj zanko izvedemo enkrat,
v vsakem obhodu pa porabimo konstantno časa.

Izvajanje algoritma na primeru $a = [-2, 3, 10, -4, 7, 21, 5, 0, -10]$
si lahko ogledamo v tabeli~\tab.
Iz zadnje vrstice lahko ugotovimo,
da je podzaporedje z največjo vsoto $(a_3, a_6) = (10, 21)$.

Algoritem vrne tudi indekse elementov podzaporedja, ki maksimizira vsoto.
To doseže tako, da primerja sosednje elemente zaporedja $(U_i)_{i=0}^n$
-- če se $U_{i-1}$ in $U_i$ razlikujeta,
potem mora očitno veljati $U_i = U_{i-2} + a_i$,
sicer pa element $a_i$ ni bil uporabljen v vsoti.
Po koncu izvajanja zanke tako dobimo seznam indeksov elementov zaporedja,
ki smo jih dejansko prišteli v vsoto.
\end{enumerate}

\begin{tabela}
$$
\begin{array}{c|ccccccccccc}
i & U_{-1} & U_0 & U_1 & U_2 & U_3 & U_4 & U_5 & U_6 & U_7 & U_8 & U_9 \\ \hline
0 & 0 & 0 & 0 & 0 & 0 & 0 & 0 & 0 & 0 & 0 & 0 \\
1 & 0 & 0 & 0 & 0 & 0 & 0 & 0 & 0 & 0 & 0 & 0 \\
2 & 0 & 0 & 0 & 3 & 0 & 0 & 0 & 0 & 0 & 0 & 0 \\
3 & 0 & 0 & 0 & 3 & 10 & 0 & 0 & 0 & 0 & 0 & 0 \\
4 & 0 & 0 & 0 & 3 & 10 & 10 & 0 & 0 & 0 & 0 & 0 \\
5 & 0 & 0 & 0 & 3 & 10 & 10 & 17 & 0 & 0 & 0 & 0 \\
6 & 0 & 0 & 0 & 3 & 10 & 10 & 17 & 31 & 0 & 0 & 0 \\
7 & 0 & 0 & 0 & 3 & 10 & 10 & 17 & 31 & 31 & 0 & 0 \\
8 & 0 & 0 & 0 & 3 & 10 & 10 & 17 & 31 & 31 & 31 & 0 \\
9 & 0 & 0 & 0 & 3 & 10 & 10 & 17 & 31 & 31 & 31 & 31
\end{array}
$$
\podnaslov{Potek izvajanja algoritma}
\end{tabela}

\end{odgovor}
\end{naloga}
