\begin{naloga}{David Gajser}{Izpit OR 19.5.2015}
\begin{vprasanje}
Dana je zgornje trikotna matrika $A \in \Z^{n \times n}$,
ki ima na diagonali neničelne vred\-no\-sti,
in vektor $b \in \Z^n$.
V psevdokodi zapiši algoritem, ki sprejme $A$ in $b$
ter reši sistem enačb $Ax = b$.
V psevdokodi lahko uporabiš le aritmetične operacije na številih,
ne pa recimo operacije invertiranja matrike $A$
(razen, če posebej napišeš psevdokodo za tako operacijo).

Kakšna je časovna zahtevnost tvojega algoritma?
\end{vprasanje}

\begin{odgovor}
Napišemo si matriko $A$, jo pomnožimo z $x$,
in v $i$-ti vrstici izrazimo $x_i$ iz enačb $Ax = b$.
Opazimo, da so vse vrednosti, ki jih rabimo za izračun $x_i$, že poračunane.
To je natanko reševanje sistema enačb z Gaussovo eliminacijo,
ki je bila za nas že vnaprej narejena,
ko smo kot vhod dobili zgornje trikotno matriko.
\begin{small}
\begin{algorithmic}
\Require{$A \in \Z^{n \times n}$ zgornje trikotna matrika
z neničelnimi vrednostmi na diagonali, $b \in \Z^n$}
\For{$i = n, n-1, \dots, 1$}
    \State $x_i \gets (b_i - \sum_{j=i+1}^n A_{ij} x_j) / A_{ii}$
\EndFor
\State \Return $(x_i)_{i=1}^n$
\end{algorithmic}
\end{small}
Časovna zahtevnost algoritma je $O(n^2)$.
\end{odgovor}
\end{naloga}
