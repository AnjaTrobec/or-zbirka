\begin{naloga}[problem trgovskega potnika]{?}{Vaje OR 16.3.2016}
\begin{vprasanje}
Danih je $n$ mest na zemljevidu.
Strošek potovanja iz mesta $i$ v mesto $j$ je $c_{ij}$ ($1 \le i, j \le n$).
Trgovski potnik želi obiskati vseh $n$ mest,
pri tem pa minimizirati strošek potovanja.
Na smiseln način modeliraj opisani problem z linearnim programom.
\end{vprasanje}

\begin{odgovor}
Pot bomo opisali z množico povezav v polnem grafu z $n$ vozlišči (mesti),
ki tvorijo cikel.
Za vsak par mest $(i, j)$ ($1 \le i, j \le n$)
bomo uvedli spremenljivko $x_{ij}$,
katere vrednost interpretiramo kot
$$
x_{ij} = \begin{cases}
1, & \text{iz $i$-tega mesta odpotujemo v $j$-to mesto} \\
0  & \text{sicer.}
\end{cases}
$$
Poskrbeti bomo morali še,
da bo naša rešitev res sestavljena iz enega samega cikla.
V ta namen bomo za $i$-to mesto ($1 \le i \le n$) uvedli spremenljivko $y_i$,
ki pove, katero po vrsti smo to mesto obiskali.
Ker je rešitev cikel, si lahko poljubno izberemo,
katero je prvo obiskano mesto (npr.~mesto z indeksom $1$).

Zapišimo celoštevilski linearni program.
\begin{alignat*}{2}
\min \ \sum_{i=1}^n \sum_{j=1}^n x_{ij} c_{ij} && \text{p.p.}\\
\forall i, j \in \{1, \dots, n\}: &\ &
0 \le x_{ij} &\le 1, \quad x_{ij} \in \Z
\opis{V vsako mesto prispemo natanko enkrat}
\forall j \in \{1, \dots, n\}: &\ & \sum_{i=1}^n x_{ij} &= 1
\opis{Vsako mesto zapustimo natanko enkrat}
\forall i \in \{1, \dots, n\}: &\ & \sum_{i=1}^n x_{ij} &= 1
\opis{Če iz mesta $i$ potujemo v mesto $j > 1$, mu to sledi v vrstnem redu}
\forall i  \in \{1, \dots, n\} \ \forall j \in \{2, \dots, n\}: &\ &
y_i + n x_{ij} &\le y_j + n - 1
\end{alignat*}
\end{odgovor}
Zadnji pogoj poskrbi, da rešitev nima cikla brez mesta z indeksom $1$.
Skupaj s pogojema, da vsako mesto obiščemo in zapustimo natanko enkrat,
si tako zagotovimo, da bo rešitev obsegala natanko en cikel.
Vrednost spremenljivk $y_i$ ($1 \le i \le n$) ni omejena
-- zadnji pogoj poskrbi le, da je največja razlika med njimi enaka $n-1$.
\end{naloga}
