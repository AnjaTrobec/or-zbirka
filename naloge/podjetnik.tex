\begin{naloga}{Janoš Vidali}{Kolokvij OR 23.4.2018}
\begin{vprasanje}
Mladi podjetnik je razvil inovativen izdelek
in se odloča za nadaljnje korake pri njegovem trženju.
Naenkrat lahko naroči izdelavo $500$ izdelkov po ceni $10\,000 €$
ali $1\,000$ izdelkov po ceni $18\,000 €$,
lahko se pa odloči tudi, da izdelave ne naroči.
Podjetnik ocenjuje, da je izdelek tržno zanimiv z verjetnostjo $0.8$.
Odloča se, ali naj posamezen izdelek prodaja po ceni $50 €$ ali $60 €$,
pri čemer so v spodnji tabeli zbrana pričakovana števila prodanih kosov
v odvisnosti od teh pogojev.

\begin{center}
\begin{tabular}{c|cc}
& cena $50 €$ & cena $60 €$ \\
\hline
tržno zanimiv & $650$ & $550$ \\
tržno nezanimiv & $250$ & $100$ \\
\end{tabular}
\end{center}

Predpostavi, da se bodo prodali vsi izdelani kosi,
če je število teh manjše od pričakovane prodaje pri danih pogojih.

\begin{enumerate}[(a)]
\item Kako naj se podjetnik odloči, da bo pričakovani zaslužek čim večji?
Nariši od\-lo\-čit\-ve\-no drevo in ga uporabi pri sprejemanju odločitev.

\item V zadnjem trenutku je podjetnik izvedel,
da bo Podjetniški pospeševalnik objavil
razpis za nagrado za najboljši izdelek.
Razpisni pogoji zahtevajo,
da se za potrebo kontrole kvalitete izdela vsaj $1\,000$ izdelkov,
od katerih komisija izbere $20$ za dejansko kontrolo,
z ostalimi pa lahko prijavitelj prosto razpolaga.
Če podjetnik zmaga, bo dobil nagrado v višini $k$ evrov,
pri čemer $k \in [1\,000, 5\,000]$ še ni znan,
poleg tega pa si obeta tudi $20\%$ povečanje
pričakovanega števila prodanih kosov (v vseh zgoraj omenjenih pogojih).
Če ne zmaga, se pričakovanja ne spremenijo.
Podjetnik ocenjuje, da je verjetnost zmage enaka $0.6$,
če je izdelek tržno zanimiv,
in $0.1$, če izdelek ni tržno zanimiv.

Naj se podjetnik prijavi na razpis?
Nariši odločitveno drevo in odločitve sprejmi v odvisnosti od parametra $k$!
\end{enumerate}
\end{vprasanje}
\begin{odgovor}
\end{odgovor}
\end{naloga}
