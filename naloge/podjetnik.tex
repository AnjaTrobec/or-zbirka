\begin{naloga}{Janoš Vidali}{Kolokvij OR 23.4.2018}
\begin{vprasanje}
Mladi podjetnik je razvil inovativen izdelek
in se odloča za nadaljnje korake pri njegovem trženju.
Naenkrat lahko naroči izdelavo $500$ izdelkov po ceni $10\,000 €$
ali $1\,000$ izdelkov po ceni $18\,000 €$,
lahko se pa odloči tudi, da izdelave ne naroči.
Podjetnik ocenjuje, da je izdelek tržno zanimiv z verjetnostjo $0.8$.
Odloča se, ali naj posamezen izdelek prodaja po ceni $50 €$ ali $60 €$,
pri čemer so v spodnji tabeli zbrana pričakovana števila prodanih kosov
v odvisnosti od teh pogojev.

\begin{center}
\begin{tabular}{c|cc}
& cena $50 €$ & cena $60 €$ \\
\hline
tržno zanimiv & $650$ & $550$ \\
tržno nezanimiv & $250$ & $100$ \\
\end{tabular}
\end{center}

Predpostavi, da se bodo prodali vsi izdelani kosi,
če je število teh manjše od pričakovane prodaje pri danih pogojih.

\begin{enumerate}[(a)]
\item Kako naj se podjetnik odloči, da bo pričakovani zaslužek čim večji?
Nariši od\-lo\-čit\-ve\-no drevo in ga uporabi pri sprejemanju odločitev.

\item V zadnjem trenutku je podjetnik izvedel,
da bo Podjetniški pospeševalnik objavil
razpis za nagrado za najboljši izdelek.
Razpisni pogoji zahtevajo,
da se za potrebo kontrole kvalitete izdela vsaj $1\,000$ izdelkov,
od katerih komisija izbere $20$ za dejansko kontrolo,
z ostalimi pa lahko prijavitelj prosto razpolaga.
Če podjetnik zmaga, bo dobil nagrado v višini $k$,
pri čemer $k \in [1\,000 €, 5\,000 €]$ še ni znan,
poleg tega pa si obeta tudi $20\%$ povečanje
pričakovanega števila prodanih kosov (v vseh zgoraj omenjenih pogojih).
Če ne zmaga, se pričakovanja ne spremenijo.
Podjetnik ocenjuje, da je verjetnost zmage enaka $0.6$,
če je izdelek tržno zanimiv,
in $0.1$, če izdelek ni tržno zanimiv.

Naj se podjetnik prijavi na razpis?
Nariši odločitveno drevo in odločitve sprejmi v odvisnosti od parametra $k$!
\end{enumerate}
\end{vprasanje}

\begin{odgovor}
\begin{enumerate}[(a)]
\item Izračunajmo pričakovane dobičke pri različnih pogojih:
\begin{align*}
% BUG: \opis doda prazno vrstico
\shortintertext{\small\needspace{2\baselineskip}%
%\opis{
Izdela $500$, prodaja po $50€$, tržno zanimiv%
:%
}
-10\,000 € + 500 \cdot 50 € &= 15\,000 €
\opis{Izdela $500$, prodaja po $50€$, tržno nezanimiv}
-10\,000 € + 250 \cdot 50 € &= 2\,500 €
\opis{Izdela $500$, prodaja po $60€$, tržno zanimiv}
-10\,000 € + 500 \cdot 60 € &= 20\,000 €
\opis{Izdela $500$, prodaja po $60€$, tržno nezanimiv}
-10\,000 € + 100 \cdot 60 € &= -4\,000 €
\opis{Izdela $1\,000$, prodaja po $50€$, tržno zanimiv}
-18\,000 € + 650 \cdot 50 € &= 14\,500 €
\opis{Izdela $1\,000$, prodaja po $50€$, tržno nezanimiv}
-18\,000 € + 250 \cdot 50 € &= -5\,500 €
\opis{Izdela $1\,000$, prodaja po $60€$, tržno zanimiv}
-18\,000 € + 550 \cdot 60 € &= 15\,000 €
\opis{Izdela $1\,000$, prodaja po $60€$, tržno nezanimiv}
-18\,000 € + 100 \cdot 60 € &= -12\,000 €
\end{align*}
Odločitveno drevo je prikazano na sliki~\fig[podjetnik-a].
Podjetnik naj se odloči, da naroči izdelavo $500$ izdelkov,
te pa prodaja po ceni $60 €$.
Pričakovani dobiček je tedaj $15\,200 €$.

\item Najprej določimo verjetnosti, ki jih bomo potrebovali za odločitev.
\begin{align*}
P(\text{zmaga}) = 0.6 \cdot 0.8 + 0.1 \cdot 0.2 &= 0.5 \\
P(\text{ne zmaga}) = 0.4 \cdot 0.8 + 0.9 \cdot 0.2 &= 0.5 \\
P(\text{zanimiv} \mid \text{zmaga}) = {0.6 \cdot 0.8 \over 0.5} &= 0.96 \\
P(\text{ni zanimiv} \mid \text{zmaga}) = {0.1 \cdot 0.2 \over 0.5} &= 0.04 \\
P(\text{zanimiv} \mid \text{ne zmaga}) = {0.4 \cdot 0.8 \over 0.5} &= 0.64 \\
P(\text{ni zanimiv} \mid \text{ne zmaga}) = {0.9 \cdot 0.2 \over 0.5} &= 0.36
\end{align*}
Sedaj izračunajmo pričakovane dobičke
pri različnih pogojih po zmagi na razpisu.
Opazimo, da v nobenem pogoju pričakovano število prodanih izdelkov
ne bo preseglo količine $980$ izdelkov,
kolikor jih ostane po kontroli kvalitete.
Če ne zmaga na razpisu,
so pričakovani dobički enaki tistim iz točke (a) pri izdelavi $1\,000$ kosov.
\begin{align*}
% BUG: \opis doda prazno vrstico
\shortintertext{\small\needspace{2\baselineskip}%
%\opis{
Zmaga, prodaja po $50€$, tržno zanimiv%
:%
}
-18\,000 € + 650 \cdot 50 € \cdot 1.2 + k &= 21\,000 € + k
\opis{Zmaga, prodaja po $50€$, tržno nezanimiv}
-18\,000 € + 250 \cdot 50 € \cdot 1.2 + k &= -3\,000 € + k
\opis{Zmaga, prodaja po $60€$, tržno zanimiv}
-18\,000 € + 550 \cdot 60 € \cdot 1.2 + k &= 21\,600 € + k
\opis{Zmaga, prodaja po $60€$, tržno nezanimiv}
-18\,000 € + 100 \cdot 60 € \cdot 1.2 + k &= -10\,800 € + k
\end{align*}
Z zgornjimi podatki lahko narišemo
odločitveno drevo s slike~\fig[podjetnik-b].
Izračunajmo pričakovane dobičke v vozliščih $E, F, G, H$.
\begin{alignat*}{3}
E &=&\, 0.96 \cdot (21\,000 € + k) &+ 0.04 \cdot (-3\,000 € + k)
&&= 20\,040 € + k \\
F &=&\, 0.96 \cdot (21\,600 € + k) &+ 0.04 \cdot (-10\,800 € + k)
&&= 20\,304 € + k \\
G &=&\, 0.64 \cdot 14\,500 € &- 0.36 \cdot 5\,500 € &&= 7\,300 € \\
H &=&\, 0.64 \cdot 15\,000 € &- 0.36 \cdot 12\,000 € &&= 5\,280 €
\intertext{%
Očitno je v vozlišču $C$ ugodnejša pot k vozlišču $F$,
v vozlišču $D$ pa je ugodnejša pot k vozlišču $G$
-- velja torej $C = 20\,304 € + k$ in $D = 7\,300 €$.
Sedaj lahko izračunamo še pričakovano vrednost v vozlišču $B$.
}
B &=&\, 0.5 \cdot (20\,304 € + k) &+ 0.5 \cdot 7\,300 €
&&= 13\,802 € + {k \over 2}
\end{alignat*}
Podjetnik naj se torej prijav na razpis,
če velja $13\,802 € + k/2 \ge 15\,200 €$ oziroma $k \ge 2\,796 €$.
Če nato zmaga na razpisu, naj izdelke prodaja po ceni $60 €$,
v nasprotnem primeru pa naj jih prodaja po ceni $50 €$.
Pričakovani dobiček je tedaj
$13\,802 € + k/2 \in [15\,200 €, 16\,302 €]$.

Če velja $k < 2\,796 €$, naj se podjetnik ne prijavi na razpis.
Tako kot v točki (a) naj naroči izdelavo $500$ izdelkov,
te pa prodaja po ceni $60 €$.
Pričakovani dobiček je tedaj $15\,200 €$.
\end{enumerate}

\begin{slika}
\pgfslika[podjetnik-a]
\podnaslov[\res{}(a)]{Odločitveno drevo}
\end{slika}

\begin{slika}
\makebox[\textwidth][c]{
\pgfslika[podjetnik-b]
}
\podnaslov[\res{}(b) v odvisnosti od vrednosti parametra $k$]%
{Odločitveno drevo}
\end{slika}
\end{odgovor}
\end{naloga}
