\subsection{Celoštevilsko linearno programiranje}

\begin{naloga}{}{?}{Vaje OR 16.3.2016}
\begin{vprasanje}
\label{nal:proj}
Občina Ljubljana želi projekte iz množice
$\p = \{p_1, p_2, \dots, p_n\}$,
pri čemer ima na voljo $M$ evrov kapitala.
V želji po razvoju regije želijo,
da se v sklopu sponzoriranih projektov ustvari vsaj $N$ delovnih mest.
Projekt $p_i$ ($1 \le i \le n)$ potrebuje $d_i$ evrov finančne pomoči
in zaposli $a_i$ ljudi.
Na občini so ocenili,
da ima projekt $p_i$ ob uspešnem dokončanju donos $c_i$ evrov.
Katere projekte naj sponzorira, da bo donos čim večji?
Na smiseln način modeliraj opisani problem z linearnim programom.
\end{vprasanje}
\begin{odgovor}
\end{odgovor}
\end{naloga}


\begin{naloga}{}{?}{Vaje OR 16.3.2016}
\begin{vprasanje}
Obravnavajmo posplošen scenarij iz naloge~\ref{nal:proj}.
\begin{enumerate}[(a)]
\item Denimo, da so projekti lahko med seboj odvisni.
Imejmo množico $V \subseteq \p^2$, ki določa,
da za vsak par projektov $(p_i, p_j) \in V$ velja,
da lahko projekt $p_i$ sponzoriramo le,
če sponzoriramo tudi projekt $p_j$.

\item Nekateri izmed projektov so lahko v konfliktu.
Naj bo $S \subseteq 2^\p$ družina množic, ki določa,
da so za vsako množico $H \in S$ projekti iz $H$ med seboj v konfliktu
(tj., hkrati lahko sponzoriramo le enega izmed njih.)
\end{enumerate}
Opiši, kako bi modelirali opisane omejitve.
\end{vprasanje}
\begin{odgovor}
\end{odgovor}
\end{naloga}


\begin{naloga}{problem prevoza in skladiščenja dobrin}{?}{Vaje OR 16.3.2016}
\begin{vprasanje}
V Evropski uniji je na voljo $n$ skladišč,
pri čemer znašajo stroški najema $i$-tega skladišča $f_i$
(ne glede na zasedenost),
vsako skladišče pa lahko hrani enoto dobrine.
Imamo $m$ strank, ki jim dostavljamo dobrine,
pri čemer $c_{ij}$ ($1 \le i \le n$, $1 \le j \le m$)
predstavlja strošek dostave dobrine stranki $j$ iz skladišča $i$.
Predpostavimo tudi, da ima vsaka stranka določeno potrebo $d_j$,
ki ponazarja število enot dobrine, ki jo potrebuje.
V katerih skladiščih naj hranimo dobrine,
da bodo skupni stroški najema in dostave čim manjši?
Na smiseln način modeliraj opisani problem z linearnim programom.
\end{vprasanje}
\begin{odgovor}
\end{odgovor}
\end{naloga}


\begin{naloga}{problem kombinatorične dražbe}{Janoš Vidali}{Vaje OR 5.3.2018}
\begin{vprasanje}
Dražitelj ponuja predmete iz množice $A$
in prejme ponudbe $\{(B_i, c_i)\}_{i=1}^k$,
pri čemer je $c_i$ cena,
ki jo udeleženec dražbe ponudi za predmete v množici $B_i \subseteq A$.
Katere ponudbe naj dražitelj sprejme,
da maksimizira dobiček,
če lahko vsak predmet proda največ enkrat?
Modeliraj opisani problem z linearnim programom.
\end{vprasanje}
\begin{odgovor}
\end{odgovor}
\end{naloga}


\begin{naloga}{}{?}{Vaje OR 16.3.2016}
\begin{vprasanje}
Definiraj problem dominacijske množice v grafu
in zapiši celoštevilski linearni program,
ki rešuje opisani problem.
\end{vprasanje}
\begin{odgovor}
\end{odgovor}
\end{naloga}


\begin{naloga}{}{?}{Vaje OR 16.3.2016}
\begin{vprasanje}
Napiši linearni program,
ki modelira iskanje najkrajše poti
med danima vozliščema $u$ in $v$ v usmerjenem grafu $G$.
\end{vprasanje}
\begin{odgovor}
\end{odgovor}
\end{naloga}
