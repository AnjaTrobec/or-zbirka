\subsection{Celoštevilsko linearno programiranje}

\begin{naloga}{?}{Vaje OR 16.3.2016}
\begin{vprasanje}[nal:proj]
Občina Ljubljana želi projekte iz množice
$\p = \{p_1, p_2, \dots, p_n\}$,
pri čemer ima na voljo $M$ evrov kapitala.
V želji po razvoju regije želijo,
da se v sklopu sponzoriranih projektov ustvari vsaj $N$ delovnih mest.
Projekt $p_i$ ($1 \le i \le n)$ potrebuje $d_i$ evrov finančne pomoči
in zaposli $a_i$ ljudi.
Na občini so ocenili,
da ima projekt $p_i$ ob uspešnem dokončanju donos $c_i$ evrov.
Katere projekte naj sponzorira, da bo donos čim večji?
Na smiseln način modeliraj opisani problem z linearnim programom.
\end{vprasanje}
\begin{odgovor}
\end{odgovor}
\end{naloga}


\begin{naloga}{?}{Vaje OR 16.3.2016}
\begin{vprasanje}
Obravnavajmo posplošen scenarij iz naloge~\ref{nal:proj}.
\begin{enumerate}[(a)]
\item Denimo, da so projekti lahko med seboj odvisni.
Imejmo množico $V \subseteq \p^2$, ki določa,
da za vsak par projektov $(p_i, p_j) \in V$ velja,
da lahko projekt $p_i$ sponzoriramo le,
če sponzoriramo tudi projekt $p_j$.

\item Nekateri izmed projektov so lahko v konfliktu.
Naj bo $S \subseteq 2^\p$ družina množic, ki določa,
da so za vsako množico $H \in S$ projekti iz $H$ med seboj v konfliktu
(tj., hkrati lahko sponzoriramo le enega izmed njih.)
\end{enumerate}
Opiši, kako bi modelirali opisane omejitve.
\end{vprasanje}
\begin{odgovor}
\end{odgovor}
\end{naloga}


\begin{naloga}[problem prevoza in skladiščenja dobrin]{?}{Vaje OR 16.3.2016}
\begin{vprasanje}
V Evropski uniji je na voljo $n$ skladišč,
pri čemer znašajo stroški najema $i$-tega skladišča $f_i$
(ne glede na zasedenost),
vsako skladišče pa lahko hrani enoto dobrine.
Imamo $m$ strank, ki jim dostavljamo dobrine,
pri čemer $c_{ij}$ ($1 \le i \le n$, $1 \le j \le m$)
predstavlja strošek dostave dobrine stranki $j$ iz skladišča $i$.
Predpostavimo tudi, da ima vsaka stranka določeno potrebo $d_j$,
ki ponazarja število enot dobrine, ki jo potrebuje.
V katerih skladiščih naj hranimo dobrine,
da bodo skupni stroški najema in dostave čim manjši?
Na smiseln način modeliraj opisani problem z linearnim programom.
\end{vprasanje}
\begin{odgovor}
\end{odgovor}
\end{naloga}


\begin{naloga}[problem kombinatorične dražbe]%
{Sergio Cabello}{Vaje OR 5.3.2018}
\begin{vprasanje}
Dražitelj ponuja predmete iz množice $A$
in prejme ponudbe $\{(B_i, c_i)\}_{i=1}^k$,
pri čemer je $c_i$ cena,
ki jo udeleženec dražbe ponudi za predmete v množici $B_i \subseteq A$.
Katere ponudbe naj dražitelj sprejme,
da maksimizira dobiček,
če lahko vsak predmet proda največ enkrat?
Modeliraj opisani problem z linearnim programom.
\end{vprasanje}
\begin{odgovor}
\end{odgovor}
\end{naloga}


\begin{naloga}{?}{Vaje OR 16.3.2016}
\begin{vprasanje}
Definiraj problem dominacijske množice v grafu
in zapiši celoštevilski linearni program,
ki rešuje opisani problem.
\end{vprasanje}
\begin{odgovor}
\end{odgovor}
\end{naloga}


\begin{naloga}{?}{Vaje OR 16.3.2016}
\begin{vprasanje}
Napiši linearni program,
ki modelira iskanje najkrajše poti
med danima vozliščema $u$ in $v$ v usmerjenem grafu $G$.
\end{vprasanje}
\begin{odgovor}
\end{odgovor}
\end{naloga}


\begin{naloga}{?}{Izpit OR 28.8.2013}
\begin{vprasanje}
Kukavica bo izlegla $16$ jajc in jih podtaknila v $12$ gnezd,
ki pripadajo dvema taščicama, štirim vrtnim penicam, trem travniškim cipam,
dvema belima pastiricama in eni sovi.
V vsako gnezdo lahko izleže največ tri jajca,
pri čemer je verjetnost, da mladiči v gnezdu $i$ preživijo,
enaka $p_{ij}$, kjer je $j$ število podtaknjenih jajc v gnezdu $i$
(preživijo bodisi vsi ali noben mladič v posameznem gnezdu).
Pri vsaki od petih vrst ptic želi izleči vsaj eno jajce,
pri taščicah pa želi izleči strogo več jajc kot pri belih pastiricah.
Poleg tega pri drugi beli pastirici ne bo odložila jajca,
če bo pri prvi taščici odložila dve jajci ali več.
Kukavica želi maksimizirati pričakovano število preživelih mladičev.

Zapiši problem kot celoštevilski linearni program.
\end{vprasanje}
\begin{odgovor}
\end{odgovor}
\end{naloga}


\begin{naloga}{?}{Kolokvij OR 31.5.2012}
\begin{vprasanje}
Vinar Janez je pridelal $2000$ litrov rumenega muškata,
$10000$ litrov laškega rizlinga in $5000$ litrov renskega rizlinga.
Njegovi kupci so bara Kocka in Luka ter župnišče Sv.~Martin in občina Duplek.
Vsak od njih je pripravljen kupiti največ določeno količino vina
po fiksni ceni, ne glede na sorto:

\begin{center}
\begin{tabular}{r|cccc}
kupec & Kocka & Luka & župnišče & občina \\ \hline
cena za liter & $1.0$ & $1.1$ & $1.5$ & $1.8$ \\
največja količina v litrih & $15000$ & $5000$ & $500$ & $1000$ \\
\end{tabular}
\end{center}

Janez se je odločil, da bo vsako sorto prodal največ enemu kupcu,
in sicer v maksimalni količini
(če kupec ne kupi vsega vina iste sorte, ga Janez ohrani zase).
Župan pravi, da občina drugega vina kot renskega rizlinga ne bo kupila.
Bar Luka želi rumeni muškat, če bar Kocka dobi laški rizling.
Pri Kocki so se dogovorili, da če občina in župnišče ne kupijo nič,
tudi oni ne bodo kupili ničesar.
Janezova žena pa vztraja,
da če kupec $A$ kupi sorto $s_A$ in kupec $B$ kupi sorto $s_B$,
potem naj sorta $s_C$ ostane doma ali jo kupi kupec $C$ (za neke $A, B, C$).
Kako naj Janez proda vino, da bo čim več zaslužil?

Zapiši problem kot celoštevilski linearni program.
\end{vprasanje}
\begin{odgovor}
\end{odgovor}
\end{naloga}

\begin{naloga}{?}{Vaje OR 16.3.2016}
\begin{vprasanje}
Napiši linearni program,
ki poišče razdalje od danega vozlišča $u$ v grafu $G$.
\end{vprasanje}
\begin{odgovor}
\end{odgovor}
\end{naloga}


\begin{naloga}{?}{Vaje OR 16.3.2016}
\begin{vprasanje}
Napiši linearni program,
ki modelira določanje kromatičnega števila grafa.

\end{vprasanje}
\begin{odgovor}
\end{odgovor}
\end{naloga}


\begin{naloga}[problem trgovskega potnika]{?}{Vaje OR 16.3.2016}
\begin{vprasanje}
Danih je $n$ mest na zemljevidu.
Strošek potovanja iz mesta $i$ v mesto $j$ je $c_{ij}$ ($1 \le i, j \le n$).
Trgovski potnik želi obiskati vseh $n$ mest,
pri tem pa minimizirati strošek potovanja.
Na smiseln način modeliraj opisani problem z linearnim programom.
\end{vprasanje}
\begin{odgovor}
\end{odgovor}
\end{naloga}


\begin{naloga}{?}{Vaje OR 23.3.2016}
\begin{vprasanje}
S celoštevilskim linearnim programom
modeliraj problem iskanja minimalnega vpetega drevesa v grafu.
\end{vprasanje}
\begin{odgovor}
\end{odgovor}
\end{naloga}
