\subsection{Celoštevilsko linearno programiranje}

\begin{naloga}{?}{Vaje OR 16.3.2016}
\begin{vprasanje}[proj]
Občina Ljubljana želi projekte iz množice
$\p = \{p_1, p_2, \dots, p_n\}$,
pri čemer ima na voljo $M$ evrov kapitala.
V želji po razvoju regije želijo,
da se v sklopu sponzoriranih projektov ustvari vsaj $N$ delovnih mest.
Projekt $p_i$ ($1 \le i \le n)$ potrebuje $d_i$ evrov finančne pomoči
in zaposli $a_i$ ljudi.
Na občini so ocenili,
da ima projekt $p_i$ ob uspešnem dokončanju donos $c_i$ evrov.
Katere projekte naj sponzorira, da bo donos čim večji?
Na smiseln način modeliraj opisani problem z linearnim programom.
\end{vprasanje}
\begin{odgovor}
\end{odgovor}
\end{naloga}


\begin{naloga}{?}{Vaje OR 16.3.2016}
\begin{vprasanje}
Obravnavajmo posplošen scenarij iz naloge~\nal{proj}.
\begin{enumerate}[(a)]
\item Denimo, da so projekti lahko med seboj odvisni.
Imejmo množico $V \subseteq \p^2$, ki določa,
da za vsak par projektov $(p_i, p_j) \in V$ velja,
da lahko projekt $p_i$ sponzoriramo le,
če sponzoriramo tudi projekt $p_j$.

\item Nekateri izmed projektov so lahko v konfliktu.
Naj bo $S \subseteq 2^\p$ družina množic, ki določa,
da so za vsako množico $H \in S$ projekti iz $H$ med seboj v konfliktu
(tj., hkrati lahko sponzoriramo le enega izmed njih.)
\end{enumerate}
Opiši, kako bi modelirali opisane omejitve.
\end{vprasanje}
\begin{odgovor}
\end{odgovor}
\end{naloga}


\begin{naloga}[problem prevoza in skladiščenja dobrin]{?}{Vaje OR 16.3.2016}
\begin{vprasanje}
V Evropski uniji je na voljo $n$ skladišč,
pri čemer znašajo stroški najema $i$-tega skladišča $f_i$
(ne glede na zasedenost),
vsako skladišče pa lahko hrani enoto dobrine.
Imamo $m$ strank, ki jim dostavljamo dobrine,
pri čemer $c_{ij}$ ($1 \le i \le n$, $1 \le j \le m$)
predstavlja strošek dostave dobrine stranki $j$ iz skladišča $i$.
Predpostavimo tudi, da ima vsaka stranka določeno potrebo $d_j$,
ki ponazarja število enot dobrine, ki jo potrebuje.
V katerih skladiščih naj hranimo dobrine,
da bodo skupni stroški najema in dostave čim manjši?
Na smiseln način modeliraj opisani problem z linearnim programom.
\end{vprasanje}
\begin{odgovor}
\end{odgovor}
\end{naloga}


\begin{naloga}[problem kombinatorične dražbe]%
{Sergio Cabello}{Vaje OR 5.3.2018}
\begin{vprasanje}
Dražitelj ponuja predmete iz množice $A$
in prejme ponudbe $\{(B_i, c_i)\}_{i=1}^k$,
pri čemer je $c_i$ cena,
ki jo udeleženec dražbe ponudi za predmete v množici $B_i \subseteq A$.
Katere ponudbe naj dražitelj sprejme,
da maksimizira dobiček,
če lahko vsak predmet proda največ enkrat?
Modeliraj opisani problem z linearnim programom.
\end{vprasanje}
\begin{odgovor}
\end{odgovor}
\end{naloga}


\begin{naloga}{?}{Vaje OR 16.3.2016}
\begin{vprasanje}
Definiraj problem dominacijske množice v grafu
in zapiši celoštevilski linearni program,
ki rešuje opisani problem.
\end{vprasanje}
\begin{odgovor}
\end{odgovor}
\end{naloga}


\begin{naloga}{Janoš Vidali}{Vaje OR 12.10.2016}
\begin{vprasanje}
Napiši linearni program,
ki modelira iskanje največjega prirejanja v dvodelnem grafu.
\end{vprasanje}
\begin{odgovor}
\end{odgovor}
\end{naloga}


\begin{naloga}{?}{Vaje OR 16.3.2016}
\begin{vprasanje}
Napiši linearni program,
ki modelira iskanje najkrajše poti
med danima vozliščema $u$ in $v$ v usmerjenem grafu $G$.
\end{vprasanje}
\begin{odgovor}
\end{odgovor}
\end{naloga}


\begin{naloga}{?}{Izpit OR 28.8.2013}
\begin{vprasanje}
Kukavica bo izlegla $16$ jajc in jih podtaknila v $12$ gnezd,
ki pripadajo dvema taščicama, štirim vrtnim penicam, trem travniškim cipam,
dvema belima pastiricama in eni sovi.
V vsako gnezdo lahko izleže največ tri jajca,
pri čemer je verjetnost, da mladiči v gnezdu $i$ preživijo,
enaka $p_{ij}$, kjer je $j$ število podtaknjenih jajc v gnezdu $i$
(preživijo bodisi vsi ali noben mladič v posameznem gnezdu).
Pri vsaki od petih vrst ptic želi izleči vsaj eno jajce,
pri taščicah pa želi izleči strogo več jajc kot pri belih pastiricah.
Poleg tega pri drugi beli pastirici ne bo odložila jajca,
če bo pri prvi taščici odložila dve jajci ali več.
Kukavica želi maksimizirati pričakovano število preživelih mladičev.

Zapiši problem kot celoštevilski linearni program.
\end{vprasanje}
\begin{odgovor}
\end{odgovor}
\end{naloga}


\begin{naloga}{?}{Kolokvij OR 31.5.2012}
\begin{vprasanje}
Vinar Janez je pridelal $2000$ litrov rumenega muškata,
$10000$ litrov laškega rizlinga in $5000$ litrov renskega rizlinga.
Njegovi kupci so bara Kocka in Luka ter župnišče Sv.~Martin in občina Duplek.
Vsak od njih je pripravljen kupiti največ določeno količino vina
po fiksni ceni, ne glede na sorto:

\begin{center}
\begin{tabular}{r|cccc}
kupec & Kocka & Luka & župnišče & občina \\ \hline
cena za liter & $1.0$ & $1.1$ & $1.5$ & $1.8$ \\
največja količina v litrih & $15000$ & $5000$ & $500$ & $1000$ \\
\end{tabular}
\end{center}

Janez se je odločil, da bo vsako sorto prodal največ enemu kupcu,
in sicer v maksimalni količini
(če kupec ne kupi vsega vina iste sorte, ga Janez ohrani zase).
Župan pravi, da občina drugega vina kot renskega rizlinga ne bo kupila.
Bar Luka želi rumeni muškat, če bar Kocka dobi laški rizling.
Pri Kocki so se dogovorili, da če občina in župnišče ne kupijo nič,
tudi oni ne bodo kupili ničesar.
Janezova žena pa vztraja,
da če kupec $A$ kupi sorto $s_A$ in kupec $B$ kupi sorto $s_B$,
potem naj sorta $s_C$ ostane doma ali jo kupi kupec $C$ (za neke $A, B, C$).
Kako naj Janez proda vino, da bo čim več zaslužil?

Zapiši problem kot celoštevilski linearni program.
\end{vprasanje}
\begin{odgovor}
\end{odgovor}
\end{naloga}

\begin{naloga}{?}{Vaje OR 16.3.2016}
\begin{vprasanje}
Napiši linearni program,
ki poišče razdalje od danega vozlišča $u$ v grafu $G$.
\end{vprasanje}
\begin{odgovor}
\end{odgovor}
\end{naloga}


\begin{naloga}{?}{Vaje OR 16.3.2016}
\begin{vprasanje}
Napiši linearni program,
ki modelira določanje kromatičnega števila grafa.
\end{vprasanje}
\begin{odgovor}
\end{odgovor}
\end{naloga}


\begin{naloga}[problem trgovskega potnika]{?}{Vaje OR 16.3.2016}
\begin{vprasanje}
Danih je $n$ mest na zemljevidu.
Strošek potovanja iz mesta $i$ v mesto $j$ je $c_{ij}$ ($1 \le i, j \le n$).
Trgovski potnik želi obiskati vseh $n$ mest,
pri tem pa minimizirati strošek potovanja.
Na smiseln način modeliraj opisani problem z linearnim programom.
\end{vprasanje}
\begin{odgovor}
\end{odgovor}
\end{naloga}


\begin{naloga}{?}{Vaje OR 23.3.2016}
\begin{vprasanje}
S celoštevilskim linearnim programom
modeliraj problem iskanja minimalnega vpetega drevesa v grafu.
\end{vprasanje}
\begin{odgovor}
\end{odgovor}
\end{naloga}


\begin{naloga}{Janoš Vidali}{Izpit OR 15.12.2016}
\begin{vprasanje}
Na oddelku za matematiko je zaposlenih $n$ asistentov,
ki jim moramo dodeliti vaje pri $m$ predmetih.
Za asistenta $i$ ($1 \le i \le n$) naj bosta $a_i$ in $b_i$
najmanjše in največje število ur, ki jih lahko izvaja,
ter $N_i \subseteq \{1, 2, \dots, m\}$ množica predmetov,
ki jih ne želi izvajati.
Za predmet $j$ ($1 \le j \le m$)
naj bo $c_j$ število skupin za vaje pri predmetu,
ter $u_j$ število ur vaj na skupino.
Poleg tega vemo, da sta asistenta $p$ in $q$ skregana,
zato pri nobenem predmetu ne smeta oba izvajati vaj.

Predmete želimo asistentom dodeliti tako,
da bomo ob upoštevanju njihovih želja
minimizirali največje število različnih predmetov,
ki smo jih dodelili posamezenmu asistentu.

Zapiši celoštevilski linearni program, ki modelira zgoraj opisani problem.


{\small {\bf Namig:} napiši program s spremenljivko $t$,
ki je dopusten natanko tedaj,
ko vsak asistent dobi največ $t$ različnih predmetov,
in potem minimimiziraj $t$.}
\end{vprasanje}
\begin{odgovor}
\end{odgovor}
\end{naloga}


\begin{naloga}{Janoš Vidali}{Izpit OR 31.1.2017}
\begin{vprasanje}
Imamo $m$ opravil, ki jih želimo razporediti med $n$ strojev.
Vsak stroj lahko hkrati opravlja le eno opravilo,
vsa opravila pa trajajo eno časovno enoto, neodvisno od stroja.
Če stroj $i$ ($1 \le i \le n$) uporabimo za vsaj eno opravilo,
plačamo ceno $c_i$
(cena ostane enaka, če na istem stroju naredimo več opravil).
Skupni stroški ne smejo preseči količine $C$.
Dani sta še množici parov $P$ in $S$,
pri čemer $(j, k) \in P$ pomeni,
da mora biti opravilo $j$ dokončano pred začetkom opravila $k$,
$(j, k) \in S$ pa pomeni, da se opravili $j$ in $k$ ne smeta izvajati hkrati.
Imamo še dodaten pogoj, ki zahteva,
da je lahko v posamezni časovni enoti lahko aktivnih največ $A$ strojev.
Minimizirati želimo čas dokončanja zadnjega stroja.

Zapiši celoštevilski linearni program, ki modelira zgoraj opisani problem.
\end{vprasanje}
\begin{odgovor}
\end{odgovor}
\end{naloga}


\begin{naloga}[Problem polnjenja košev]{Sergio Cabello}{Izpit OR 15.3.2017}
\begin{vprasanje}
Imamo neskončno košev $B_1, B_2, \dots$, od katerih ima vsak velikost $V > 0$,
in $n$ predmetov pozitivnih velikosti $s_1, s_2, \dots, s_n$,
od katerih je vsaka največ $V$.
Predmete želimo postaviti v koše tako, da porabimo čim manj košev,
pri čemer noben koš ne vsebuje predmetov skupne velikosti več kot $V$.
Na primer, če $V = 10$ ter $s_1 = 4$, $s_2 = 5$ in $s_3 = 7$,
lahko to storimo z dvema košema, medtem ko z enim samim košem to ni mogoče.

\begin{enumerate}[(a)]
\item Zapiši celoštevilski linearni program, ki modelira zgornji problem.

\item Denimo, da imamo seznam $L$ s pari predmetov,
ki jih ne smemo postaviti v isti koš.
Če imamo na primer $L = \{(1, 2), (3, 4), (1, 4)\}$,
potem predmetov $1$ in $2$ ne moremo postaviti skupaj,
prav tako ne predmetov $3$ in $4$ oziroma $1$ in $4$.

Dopolni zgornji celoštevilski linearni program tako,
da bo vključeval te omejitve.
\end{enumerate}
\end{vprasanje}
\begin{odgovor}
\end{odgovor}
\end{naloga}


\begin{naloga}{Janoš Vidali}{Izpit OR 10.7.2017}
\begin{vprasanje}
Za hitrejše nalaganje posnetkov na YouTubu se uporabljajo krajevni strežniki,
do katerih lahko uporabniki na določeni lokaciji hitreje dostopajo
kakor do glavnega strežnika, ki vsebuje vse posnetke.
Vendar pa imajo krajevni strežniki omejen prostor,
zato je potrebno ugotoviti,
kateri posnetki naj se naložijo na katere krajevne strežnike.

Denimo, da imamo poleg glavnega strežnika še $k$ krajevnih strežnikov,
$m$ internetnih po\-nud\-ni\-kov in $n$ posnetkov.
Naj bo $c_h$ ($1 \le h \le k$) prostor v megabajtih,
ki je na voljo na krajevnem strežniku $h$.
Z indeksom $0$ označimo glavni strežnik
-- lahko torej predpostavljaš $c_0 = \infty$.
Nadalje naj bo $s_j$ ($1 \le j \le n$) velikost posnetka $j$,
prav tako v megabajtih,
$\ell_{hi}$ ($0 \le h \le k$, $1 \le i \le m$)
latenca (zakasnitev pri prenosu v milisekundah)
ponudnika $i$ do strežnika $h$,
in $r_{ij}$ ($1 \le i \le m$, $1 \le j \le n$)
število zahtevkov za posnetek $j$, ki jih pričakujejo od ponudnika $i$.
Vsi parametri so cela števila.

Pri vsakem zahtevku bo posnetek poslan iz strežnika z najmanjšo latenco,
ki vsebuje želeni posnetek.
Lahko predpostavljaš,
da ima za vsakega ponudnika glavni strežnik največjo latenco
(torej $\ell_{0i} \ge \ell_{hi}$ za vsaka $1 \le h \le k$, $1 \le i \le m$).
Določiti želimo, katere posnetke naj naložimo na posamezen krajevni strežnik,
da minimiziramo vsoto pričakovanih latenc pri vseh zahtevkih.
Posamezen posnetek lahko seveda naložimo tudi na več krajevnih strežnikov,
ali pa na nobenega (v tem primeru bo poslan iz glavnega strežnika).

Zapiši celoštevilski linearni program, ki modelira zgoraj opisani problem. \\
{\small {\bf Namig:}
pri določitvi latence ponudnika za dan posnetek si pomagaj s spremenljivko,
ki šteje strežnike z latenco, ki ne presega izračunane latence.}
\end{vprasanje}
\begin{odgovor}
\end{odgovor}
\end{naloga}


\begin{naloga}{Janoš Vidali}{Izpit OR 29.8.2017}
\begin{vprasanje}
Distributer ima $A$ zabojev avokadov in $B$ zabojev banan,
ki jih bo prodal $n$ trgovcem.
Trgovec $i$ ($1 \le i \le n$)
plača $a_i$ evrov za zaboj avokadov in $b_i$ evrov za zaboj banan,
skupaj pa bo porabil največ $c_i$ evrov.
Distributer bo zaboje dostavil s tovornjaki,
v katerih je lahko največ $K$ zabojev (ne glede na vsebino).
Če nekemu trgovcu dostavi $t$ zabojev,
bo torej opravljenih $\lceil t/K \rceil$ voženj.
Vsaka vožnja do trgovca $i$ (ne glede na to, koliko je poln tovornjak)
ga stane $d_i$ evrov.
Poleg tega bo trgovec $p$ kupil samo banane ali samo avokade,
trgovec $q$ pa bo kupil vsaj en zaboj avokadov,
če bo tudi trgovec $r$ kupil vsaj en zaboj avokadov.
Distributer želi zaboje razdeliti med trgovce tako,
da bo maksimiziral svoj dobiček
-- torej vsoto cen, ki jih plačajo trgovci,
zmanjšano za stroške dostave.
Lahko predpostavljaš, da so vse cene pozitivne.

Zapiši celoštevilski linearni program, ki modelira zgoraj opisani problem.
\end{vprasanje}
\begin{odgovor}
\end{odgovor}
\end{naloga}


\begin{naloga}{Janoš Vidali}{Kolokvij OR 23.4.2018}
\begin{vprasanje}
Pri izvedbi projekta bo potrebno narediti $n$ nalog,
ki jih bomo dodelili delavcem.
Vsako nalogo bo opravil natanko en delavec,
vsak delavec pa lahko hkrati izvaja samo eno nalogo.
Naj bo $t_i \in \N$ število časovnih enot,
ki ga posamezen delavec potrebuje za dokončanje naloge $i$
($1 \le i \le n$).
Vsaka naloga mora biti opravljena v enem kosu
(če torej začnemo z nalogo $i$ v času $s$,
bo ta dokončana v času $s+t_i$,
brez možnosti prekinitve in kasnejšega dokončanja).
Celoten projekt mora biti dokončan v času $T$.
Dana je še množica parov $S$,
kjer $(i, j) \in S$ pomeni,
da se naloga $j$ ne sme začeti, preden se zaključi naloga $i$
(lahko se pa $j$ začne izvajati v trenutku, ko se $i$ konča).

Delavce bomo najeli preko podjetja za posredovanje dela,
to pa nam zaračuna fiksno ceno na najetega delavca
(za ustrezna plačila delavcem bodo tako poskrbeli oni).
Minimizirati želimo torej število najetih delavcev.
Zapiši celoštevilski linearni program, ki modelira zgoraj opisani problem. \\
{\small {\bf Namig:}
rešitev celoštevilskega linearnega programa naj eksplicitno določi vrstni red,
v katerem bo posamezen delavec izvajal naloge.}
\end{vprasanje}
\begin{odgovor}
\end{odgovor}
\end{naloga}


\begin{naloga}{Janoš Vidali}{Izpit OR 11.6.2018}
\begin{vprasanje}
Avtobusno podjetje želi uvesti novo avtobusno linijo.
Dan je neusmerjen enostaven graf $G = (V, E)$,
katerega vozlišča predstavljajo postajališča,
povezave pa ceste med njimi.
Za vsako povezavo $uv \in E$ poznamo še čas $c_{uv} \in \N$,
v katerem avtobus pride od $u$ do $v$.

Dan je še seznam postajališč $p_1, p_2, \dots p_n \in V$,
ki jih linija mora obiskati v tem vrstnem redu
-- začeti mora v $p_1$ in končati v $p_n$,
na poti med dvema postajališčema iz seznama
pa lahko obišče tudi druga postajališča.
Linija lahko vsako postajališče obišče največ enkrat
(ko doseže končno postajališče, se vrne po isti poti do začetnega).
Skupno trajanje vožnje (v eno smer) je lahko največ $T$.
Podjetje želi določiti linijo tako, da bo obiskala čim več postaj.

Zapiši celoštevilski linearni program, ki modelira zgoraj opisani problem. \\
{\small {\bf Namig:} poskrbi, da linija nima ciklov.}
\end{vprasanje}
\begin{odgovor}
\end{odgovor}
\end{naloga}


\begin{naloga}{Janoš Vidali}{Izpit OR 5.7.2018}
\begin{vprasanje}
Nogometni klub pred novo sezono prenavlja svoj igralski kader.
Naj bo $A$ množica trenutnih igralcev kluba,
$B$ pa množica igralcev, za katere se v klubu zanimajo
($A$ in $B$ sta disjunktni množici).
Za vsakega igralca $i \in A$ imajo s strani kluba $j$ ($1 \le j \le n$)
ponudbo v višini $p_{ij} \in \N$
(če se klub $j$ ne zanima za igralca $i$, lahko predpostaviš $p_{ij} = 0$),
za vsakega igralca $i \in B$ pa bodo morali plačati odkupnino $r_i \in \N$,
če ga hočejo pripeljati v klub.
Vsakega igralca $i \in A$ lahko seveda prodamo kvečjemu enemu klubu.
Skupni stroški trgovanja
(tj., razlika stroškov nakupov in dobička od odprodaj)
ne smejo preseči $S$,
število igralcev v klubu pa mora ostati enako kot pred trgovanjem.

Za vsakega igralca $i \in A \cup B$ poznamo njegov količnik kvalitete $q_i$,
vsak pa pripada natanko eni izmed množic
$G$ (vratarji), $D$ (branilci), $M$ (vezni igralci) in $F$ (napadalci).
Število igralcev kluba v vsaki izmed teh množic se lahko spremeni
(poveča ali zmanjša) za največ $1$.
Poleg tega lahko igralca $a, b \in A$ prodamo le istemu klubu
-- ali pa oba igralca ostaneta v klubu.
Doseči želimo, da bo kvaliteta kluba čim večja
-- maksimizirati želimo torej vsoto količnikov $q_i$ za igralce v klubu.

Zapiši celoštevilski linearni program, ki modelira zgoraj opisani problem.
\end{vprasanje}
\begin{odgovor}
\end{odgovor}
\end{naloga}
