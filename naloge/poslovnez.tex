\begin{naloga}{?}{Kolokvij OR 19.11.2009}
\begin{vprasanje}
Janez je prekaljen poslovnež, ki hoče vedno zaslužiti kar največ.
A Janez lahko naenkrat prevzame le en posel.
Če se en posel začne med izvajanjem nekega drugega posla,
novega posla ne more prevzeti.
Poslovne priložnosti so predstavljene z grafom.
Vozlišča predstavljajo stanja, v katerih Janez izbira med posli,
izhodne povezave iz vsakega vozlišča pa predstavljajo posle,
ki se jih lahko loti.
Cene povezav predstavljajo dobiček pri poslu oziroma izgubo,
če je cena negativna
(včasih je potrebno sprejeti tudi kak posel, ki nosi izgubo,
da se lahko prebijemo do dobičkonosnega posla \dots).
V vsakem stanju lahko Janez izbere katerega koli izmed poslov,
ki pripadajo izhodnim povezavam.
Tekom svoje poslovne kariere se lahko Janez
v določenem stanju znajde tudi večkrat.
Janez se trenutno nahaja v izbranem vozlišču grafa.
\begin{enumerate}[(a)]
\item Za poljuben graf poslovnih priložnosti ugotovi,
kakšen problem moraš na njem rešiti, da boš lahko pravilno svetoval Janezu,
da bo uresničil svoje poslovne ambicije.
Predlagaj algoritem, s katerim rešiš problem,
in oceni časovno zahtevnost predlagane rešitve.
Janeza med drugim še posebej zanima,
ali mu graf poslovnih priložnosti zagotavlja stalno pridobivanje poslov
in dobičkonosno poslovanje za celo kariero.
\item Za graf poslovnih priložnosti s slike~\fig
izberi ustrezen algoritem in ga izvedi ter izračunaj,
koliko največ lahko zasluži Janez, če ``vstopi v igro'' v stanju $a$ ali $b$.
Upoštevaj lastnosti spodnjega grafa in navedi, kateri algoritem je to.
Ali graf predstavlja poslovne priložnosti,
ki bodo Janezu zagotovile trajno dobičkonosno poslovanje?
\end{enumerate}

\begin{slika}
\pgfslika
\podnaslov{Graf}
\end{slika}
\end{vprasanje}
\begin{odgovor}
\end{odgovor}
\end{naloga}
