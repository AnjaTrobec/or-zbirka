\begin{naloga}{Jernej Azarija}{Izpit OR 24.5.2016}
\begin{vprasanje}
Organizatorji olimpijskih iger so zgradili dvorano za judo,
ki sprejme $1\,000$ ljudi.
Popularnost juda je na vrhuncu,
zato organizatorji pričakujejo, da bodo vse karte zakupljene.
Zaradi straha pred virusom zika so analitiki
za $0 \le i \le 5$ ($i \in \Z$) izračunali verjetnosti $p_i$,
da se natanko $i$ izmed imetnikov kart ne pojavi na prireditvi.
Verjetnosti so zbrane v spodnji tabeli.
\begin{center}
\begin{tabular}{c|cccccc}
$i$   & $0$    & $1$    & $2$   & $3$   & $4$    & $5$ \\ \hline
$p_i$ & $0.01$ & $0.19$ & $0.3$ & $0.4$ & $0.05$ & $0.05$
\end{tabular}
\end{center}

S prodajo vstopnice imajo organizatorji $1\,000$ denot dobička.
Za vsakega obiskovalca,
ki bo ostal brez sedeža in bo moral v VIP sekcijo dvorane,
imajo $1\,400$ denot stroškov.
Koliko kart naj organizatorji olimpijskih iger prodajo,
če želijo maksimizirati pričakovani dobiček?
\end{vprasanje}
\begin{odgovor}
\end{odgovor}
\end{naloga}
