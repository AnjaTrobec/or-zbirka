\begin{naloga}{Jernej Azarija}{Izpit OR 24.5.2016}
\begin{vprasanje}
Organizatorji olimpijskih iger so zgradili dvorano za judo,
ki sprejme $1\,000$ ljudi.
Popularnost juda je na vrhuncu,
zato organizatorji pričakujejo, da bodo vse karte zakupljene.
Zaradi straha pred virusom zika so analitiki
za $0 \le i \le 5$ ($i \in \Z$) izračunali verjetnosti $p_i$,
da se natanko $i$ izmed imetnikov kart ne pojavi na prireditvi.
Verjetnosti so zbrane v spodnji tabeli.
\begin{center}
\begin{tabular}{c|cccccc}
$i$   & $0$    & $1$    & $2$   & $3$   & $4$    & $5$ \\ \hline
$p_i$ & $0.01$ & $0.19$ & $0.3$ & $0.4$ & $0.05$ & $0.05$
\end{tabular}
\end{center}

S prodajo vstopnice imajo organizatorji $1\,000$ denot dobička.
Za vsakega obiskovalca,
ki bo ostal brez sedeža in bo moral v VIP sekcijo dvorane,
imajo $1\,400$ denot stroškov.
Koliko kart naj organizatorji olimpijskih iger prodajo,
če želijo maksimizirati pričakovani dobiček?
\end{vprasanje}

\begin{odgovor}
Naj bo $X_k$ ($1000 \le k \le 1005$) pričakovani dobiček v denotah,
če organizatorji olimpijskih iger prodajo $k$ kart.
\begin{alignat*}{2}
E(X_{1000}) &= 1\,000 \cdot 1\,000 &&= 1\,000\,000 \\
E(X_{1001}) &= 1\,001 \cdot 1\,000 - 0.01 \cdot 1\,400 &&= 1\,000\,986 \\
E(X_{1002}) &= 1\,002 \cdot 1\,000 - (0.01 \cdot 2 + 0.19) \cdot 1\,400
&&= 1\,001\,706 \\
E(X_{1003}) &= 1\,003 \cdot 1\,000 -
(0.01 \cdot 3 + 0.19 \cdot 2 + 0.3) \cdot 1\,400 &&= 1\,002\,006 \\
E(X_{1004}) &= 1\,004 \cdot 1\,000 -
(0.01 \cdot 4 + 0.19 \cdot 3 + 0.3 \cdot 2 + 0.4) \cdot 1\,400
&&= 1\,001\,746 \\
E(X_{1005}) &= 1\,005 \cdot 1\,000 -
(0.01 \cdot 5 + 0.19 \cdot 4 + 0.3 \cdot 3 + 0.4 \cdot 2 + 0.05) \cdot 1\,400
&&= 1\,001\,416
\end{alignat*}
Podrobneje si poglejmo primer, ko organizatorji prodajo $1\,003$ karte.
Najprej izračunamo, kolikšen dobiček ustvarijo s prodajo $1\,003$ kart,
nato upoštevamo,
da z ve\-rjet\-nost\-jo $0.01$ na tekmo pridejo vsi imetniki kart,
naprej upoštevamo da z ve\-rjet\-nost\-jo $0.19$
na tekmo prideta $1\,002$ imetnika kart,
ter da z ve\-rjet\-nost\-jo $0.3$ na tekmo pride $1\,001$ imetnik kart.
V slednjih treh primerih bodo organizatorji
torej morali zagotoviti sedež v VIP sekciji dvorane
za tri, dva oziroma enega človeka,
za vsakega pa bodo imeli $1\,400$ denot stroškov.

Organizatorji olimpijskih iger naj torej
za maksimizacijo svojega dobička prodajo $1\,003$ karte.
\end{odgovor}

\end{naloga}
