\begin{naloga}{Janoš Vidali}{Izpit OR 15.12.2016}
\begin{vprasanje}
Mudi se ti na izpit, a ravno v trenutku,
ko prideš na postajo Konzorcij, odpelje avtobus številka 1.
Na prikazovalniku se izpiše,
da bo naslednji avtobus številka 1 prispel čez $10$ minut,
naslednji avtobus številka 6 čez $6$ minut,
naslednji avtobus številka 14 pa čez $2$ minuti.

Avtobusa 1 in 6 ob ugodnih semaforjih potrebujeta $6$ minut do postaje pri FE,
pri čemer se lahko čas vož\-nje zaradi rdeče luči na semaforju pri FF
podaljša za $1$ minuto.
Verjetnosti, da bo rdečo luč imel avtobus 1, da bo rdečo luč imel avtobus 6,
ter da bosta oba avtobusa imela zeleno luč, so enake $1/3$
(zaradi majhnega razmaka se ne more zgoditi,
da bi oba avtobusa naletela na rdečo luč).
Avtobus številka 1 nadaljuje pot do postaje pri FMF,
za kar potrebuje še $2$ minuti.

Avtobus številka 14 potrebuje $5$ minut do postaje pri študentskih domovih,
od tam pa greš peš do postaje pri FE, za kar potrebuješ še $4$ minute.
Pri tem prečkaš že\-lez\-ni\-co -- če mimo pripelje vlak
(kar se zgodi z verjetnostjo $0.05$),
se čas hoje podaljša za $3$ minute.
Ko prideš na postajo pri FE
(ne glede na to, ali si prišel z avtobusom 6 ali 14),
te čakajo še $4$ minute hoje do FMF,
vendar moraš najprej prečkati Tržaško cesto.
Če je na semaforju rdeča luč
(kar se zgodi z verjetnostjo 0.9, neodvisno od drugih dogodkov),
se lahko odločiš, da $2$ minuti počakaš na zeleno luč in potem nadaljuješ peš,
ali pa da greš nazaj do postaje in počakaš na avtobus številka $1$
(ki bo, tako kot prej, vozil še $2$ minuti do FMF).

Kakšne bodo tvoje odločitve,
da bo pričakovano trajanje poti do FMF čim krajše?
Nariši od\-lo\-čit\-ve\-no drevo
in odločitve sprejmi na podlagi izračunanih verjetnosti!
Glej sliko~\fig{} za shemo možnih poti.

\begin{slika}
\pgfslika
\podnaslov{Shema možnih poti}
\end{slika}
\end{vprasanje}
\begin{odgovor}
\end{odgovor}
\end{naloga}
