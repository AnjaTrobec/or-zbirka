\begin{naloga}{W.~L.~Winston}{\cite[\S 18, Example~5]{w}}
\begin{vprasanje}
Podjetje ima na voljo $6\,000 €$ za investiranje.
Na voljo so tri investicije, pri katerih je donosnost (v $1\,000 €$) enaka
\begin{align*}
r_1(d_1) &= \begin{cases}
7d_1 + 2, & \text{če $d_1 \ge 1$, in} \\
0 & \text{sicer;}
\end{cases} \\
r_2(d_2) &= \begin{cases}
3d_2 + 7, & \text{če $d_2 \ge 1$, in} \\
0 & \text{sicer;}
\end{cases}
\quad \text{oziroma} \\
r_3(d_3) &= \begin{cases}
4d_3 + 5, & \text{če $d_3 \ge 1$, in} \\
0 & \text{sicer;}
\end{cases}
\end{align*}
kjer so $d_1, d_2, d_3$ vložki v vsako investicijo v $1\,000 €$.
Kako naj podjetje investira svoj denar, da bo zaslužek čim večji?
\end{vprasanje}

\begin{odgovor}
Problem bomo razdelili na tri dele,
pri čemer bomo v $i$-tem delu obravnavali situacijo,
ko vlagamo v prvih $i$ investicij.
Vrednost $p_i(x)$ naj torej označuje največji zaslužek,
če v prvih $i$ investicij vložimo znesek $x$.

Ker vrednost $r_1(x)$ narašča z $x$, velja $p_1(x) = r_1(x)$.
Sedaj določimo $p_2(x)$.
\begin{align*}
p_2(x) &= \max\set{p_1(y) + r_2(x-y)}{0 \le y \le x} \\
&= \begin{cases}
\begin{aligned}
\max&\big(\set{3x - 3y + 7}{0 \le y < 1, \ y \le x-1} \\
&\cup \set{3x + 4y + 9}{1 \le y \le x-1} \\
&\cup \set{0}{x-1 < y < 1} \\
&\cup \set{7y+2}{x-1 < y \le x, \ y \ge 1}\big),
\end{aligned}
& \text{če $x \ge 1$, in} \\
0 & \text{sicer}
\end{cases} \\
&= \begin{cases}
0 & 0 \le x < 1 \\
3x+7 & 1 \le x < {5 \over 4}, \\
7x+2 & {5 \over 4} \le x < 2, \\
7x+5 & x \ge 2
\end{cases}
\end{align*}
Največji zaslužek bomo dobili kot $p^* = p_3(6)$.
\begin{alignat*}{2}
p_3(6) &= \max&&{} \set{p_2(z) + r_3(6-z)}{0 \le z \le 6} \\
&= \max&&{} \big(\set{-4z+29}{0 \le z < 1} \\
&&&\cup \set{-z+36}{1 \le z < {5 \over 4}} \\
&&&\cup \set{3z+31}{{5 \over 4} \le z < 2} \\
&&&\cup \set{3z+34}{2 \le z \le 5} \\
&&&\cup \set{7z+5}{5 < z \le 6}\big) \\
&= 49
\end{alignat*}
Ker velja $p^* = p_3(6) = p_2(5) + r_3(1) = r_1(4) + r_2(1) + r_3(1)$,
največji zaslužek $49\,000 €$ dosežemo,
če vložimo $4\,000 €$ v prvo investicijo
in po $1\,000 €$ v drugi dve investiciji.
\end{odgovor}
\end{naloga}
