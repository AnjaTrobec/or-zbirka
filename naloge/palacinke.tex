\begin{naloga}{?}{Izpit OR 19.5.2015}
\begin{vprasanje}
Dinamika priprave dveh palačink z dvema kuharjema
je opisana v tabeli~\tab.
\begin{enumerate}[(a)]
\item Topološko uredi ustrezni graf in ga nariši.
\item Določi kritična opravila in kritično pot ter trajanje priprave.
\item Katero opravilo je najmanj kritično?
\end{enumerate}

\begin{tabela}
\begin{tabular}{c|l|c|c}
& aktivnost & trajanje & predhodna opravila \\
\hline
$a$ & nakup moke, jajc in mleka & 5 min & $c$ \\
$b$ & rezanje sira & 3 min & / \\
$c$ & vožnja do trgovine & 5 min & / \\
$d$ & čiščenje mešalnika & 2 min & $e$ \\
$e$ & mešanje sestavin & 5 min & $a$ \\
$f$ & pečenje prve palačinke & 2 min & $e$ \\
$g$ & mazanje prve palačinke z marmelado & 1 min & $f, j$ \\
$h$ & pečenje palačinke (s sirom) & 3 min & $b, f$ \\
$i$ & pomivanje posode & 8 min & $g, h$ \\
$j$ & odpiranje marmelade & 1 min & / \\
\end{tabular}
\caption{Podatki za nalogi~\nal in~\nal[palacinke2].}
\end{tabela}
\end{vprasanje}

\begin{odgovor}
\begin{enumerate}[(a)]
\item Projekt lahko predstavimo z uteženim grafom s slike~\fig,
iz katerega je razvidna topološka ureditev
$s, b, c, j, a, e, d, f, g, h, i, t$.

\item V tabeli~\tab[brezzobec-resitev]
so podani najzgodnejši začetki opravil in najpoznejši začetki,
da se celotno trajanje priprave ne podaljša.
Priprava bo torej trajala najmanj $28$ minut,
edina kritična pot pa je $s - c - a - e - f - h - i - t$,
ki tako vsebuje tudi vsa kritična opravila.

\item Iz tabele~\tab[brezzobec-resitev] je razvidno,
da je najmanj kritično opravilo $j$,
saj lahko njegovo trajanje podaljšamo za $18$ minut,
ne da bi vplivali na trajanje celotnega projekta.
\end{enumerate}
%
\begin{slika}
\makebox[\textwidth][c]{
\pgfslika
}
\podnaslov{Graf odvisnosti med opravili in kritična pot}
\end{slika}
%
\begin{tabela}
\setlabel{palacinke-resitev}
$$
\begin{array}{r|cccccccccccc}
& s & b & c & j & a & e & d & f & g & h & i & t \\ \hline
\text{najprej} &
0 & 0_s & 0_s & 0_s & 5_b & 10_a & 15_e & 15_e & 17_f & 17_f & 20_h & 28_i \\
\text{najkasneje} &
0_c & 14_h & 0_a & 18_g & 5_e & 10_f & 26_t & 15_h & 19_i & 17_i & 20_t & 28 \\
\text{razlika} &
0 & 14 & 0^* & 18 & 0^* & 0^* & 11 & 0^* & 2 & 0^* & 0^* & 0
\end{array}
$$
\podnaslov{Razporejanje opravil}
\end{tabela}
\end{odgovor}
\end{naloga}
