\begin{naloga}{?}{Kolokvij OR 26.1.2010}
\begin{vprasanje}
Upravljalec nove moderne visoko tehnološko opremljene kongresne dvorane
v Kongresnem centru Grebka Pohleparja zaračunava za najem dvorane $5 €$/min.
Povpraševanje za najem nove dvorane je ogromno,
a se na žalost ponudbe časovno prekrivajo,
dvorano pa seveda lahko najame le ena stranka naenkrat.
Tako se je upravljalec že prvi dan znašel pod goro ponudb,
med katerimi je napol v paniki izbiral,
bodimo pošteni, tako bolj ``čez palec''.
A za kongresni center, ki nosi ime takega velikana ekonomije, se spodobi,
da si uredi vse segmente poslovanja čim bolj optimalno
in je kot tak zgled drugim.
Predpostavimo,
da ima upravljalec v trenutku odločanja podatke v obliki intervalov
$[z_i, k_i]$, $i = 1, \dots, n$,
kjer $z_i$ in $k_i$ predstavljata začetni in končni čas ponudbe $i$.
Ponudbi se prekrivata, če je njun presek neprazen časovni interval.

\begin{enumerate}[(a)]
\item Denimo, da uredimo intervale ponudb
glede na začetne čase ponudb od najmanjšega do največjega.
Potem izberemo prvi interval, vse intervale, ki ga prekrivajo, pa zavržemo,
ter postopek ponovimo rekurzivno.
Upravljalcu se ta postopek zdi optimalen. Kaj pa meniš ti?

\item Denimo, da so intervali ponudb urejeni kot v prejšnji točki
ter da je interval $I$ zadnji interval v neki optimalni izbiri OPT.
Premisli,
kateri intervali so lahko kandidati za predzadnji interval v izbiri OPT.
Na podlagi tega premisleka sestavi postopek,
ki s pomočjo dinamičnega programiranja najde optimalno rešitev problema.
Postopek utemelji.

\item Kakšna je časovna zahtevnost tvojega algoritma v najslabšem primeru?

\item Z uporabo razvitega algoritma reši problem za podatke
$(15, 60)$, $(60, 180)$, $(30, 90)$, $(120, 210)$,
$(165, 255)$, $(75, 135)$, $(30, 105)$, $(90, 150)$.
\end{enumerate}
\end{vprasanje}
\begin{odgovor}
\end{odgovor}
\end{naloga}
