\begin{naloga}{Janoš Vidali}{Izpit OR 27.8.2019}
\begin{vprasanje}
Glasbeni festival bo trajal $m$ dni,
vsak dan pa bo nastopilo $k$ glasbenih skupin.
Organizator izbira nastopajoče izmed $n$ glasbenih skupin
(kjer je $n > km$).
Na voljo ima $M$ evrov za plačilo glasbenikom,
pri čemer $i$-ta skupina ($1 \le i \le n$)
računa $m_i$ evrov za nastop,
pripeljala pa bo $a_{ij}$ obiskovalcev,
če bo nastopila $j$-ta po vrsti v dnevu ($1 \le j \le k$).
Vsaka skupina lahko seveda nastopi samo enkrat,
prav tako naenkrat nastopa samo ena skupina.
V posameznem dnevu se skupine zvrstijo glede na njihovo ceno
(tj., zadnja nastopa skupina, ki največ računa),
organizator pa želi zagotoviti,
da bo vsak dan festivala prišlo vsaj toliko ljudi kot prejšnji dan.

Skupina $r$ pogojuje svoj nastop s tem,
da skupina $s$ ne nastopa na festivalu.
Skupina $t$ bo nastopila samo kot glavna skupina
(tj., zadnja v posameznem dnevu),
razen če nastopi neposredno pred skupino $u$
(lahko predpostaviš, da velja $m_t \le m_u$)
-- ali pa sploh ne bo nastopila.
Skupini $v$ in $w$ ne smeta nastopati na isti dan,
saj se njuni oboževalci med seboj ne marajo.

Organizator želi določiti spored tako,
da bo na festival prišlo čim več ljudi.
Zapiši celoštevilski linearni program, ki modelira opisani problem.
\end{vprasanje}

\begin{odgovor}
Za $h$-ti dan ($1 \le h \le m$),
$i$-to skupino ($1 \le i \le n$) in $j$-ti termin ($1 \le j \le k$)
bomo uvedli spremenljivko $x_{hij}$,
katere vrednost interpretiramo kot
$$
x_{hij} = \begin{cases}
1, & \text{$i$-ta skupina nastopi $j$-ta v $h$-tem dnevu festivala, in} \\
0  & \text{sicer.}
\end{cases}
$$
Zapišimo celoštevilski linearni program.
\begin{align*}
\max\ \sum_{h=1}^m \sum_{i=1}^n \sum_{j=1}^k a_{ij} x_{hij}
\quad \text{p.p.} \\
\forall h \in \{1, \dots, m\} \ \forall i \in \{1, \dots, n\} \
\forall j \in \{1, \dots, j\} : \\[-1mm]
0 \le x_{hij} &\le 1, \quad x_{hij} \in \Z
\opis{Vsaka skupina nastopi največ enkrat}
\forall i \in \{1, \dots, n\} : \\[-1mm]
\sum_{h=1}^m \sum_{j=1}^k x_{hij} &\le 1
\opis{Naenkrat nastopa ena skupina}
\forall h \in \{1, \dots, m\} \ \forall j \in \{1, \dots, k\} : \\[-1mm]
\sum_{i=1}^n x_{hij} &= 1
\opis{Omejitev plačila}
\sum_{h=1}^m \sum_{i=1}^n \sum_{j=1}^k m_i x_{hij} &\le M
\opis{Vrstni red v dnevu}
\forall h \in \{1, \dots, m\} \ \forall j \in \{2, \dots, k\} : \\[-1mm]
\sum_{i=1}^n m_i (x_{hij} - x_{h,i,j-1}) &\ge 0
\opis{Razporeditev po dnevih}
\forall h \in \{2, \dots, n\} : \\[-1mm]
\sum_{i=1}^n \sum_{j=1}^k a_{ij} (x_{hij} - x_{h-1,i,j}) &\ge 0
\opis{$r$ ne nastopa, če nastopa $s$}
\sum_{h=1}^m \sum_{j=1}^k (x_{hrj} + x_{hsj}) &\le 1
\opis{$t$ ne nastopa, če ni zadnja ali pred $u$}
\forall h \in \{1, \dots, m\} : \\[-1mm]
\sum_{j=1}^{k-1} x_{htj} \le
\sum_{j=2}^k j x_{huj} - \sum_{j=1}^{k-1} j x_{htj}
&\le k - (k-1) \sum_{j=1}^{k-1} x_{htj}
\opis{$v$ in $w$ ne nastopata na isti dan}
\forall h \in \{1, \dots, m\} : \\[-1mm]
\sum_{j=1}^k (x_{hvj} + x_{hwj}) &\le 1
\end{align*}
\end{odgovor}
\end{naloga}
