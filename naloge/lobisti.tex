\begin{naloga}{Janoš Vidali}{Izpit OR 29.8.2017}
\begin{vprasanje}
V veliki multinacionalni korporaciji želijo,
da bi se zakonodaja spremenila v njihov prid.
V ta namen so najeli $m$ lobistov,
ki se bodo pogajali z $n$ političnimi strankami,
da pridobijo njihovo podporo pri spremembi zakonodaje.
Vsak lobist se bo pogajal s samo eno stranko;
k vsaki stranki lahko pošljejo več lobistov.
Naj bo $p_{ij}$ ($0 \le i \le m$, $1 \le j \le n$) verjetnost,
da pridobijo podporo stranke $j$,
če se z njo pogaja $i$ lobistov
(lahko predpostaviš $p_{i-1,j} \le p_{ij}$ za vsaka $i, j$).
Verjetnosti za različne stranke so med seboj neodvisne.
Maksimizirati želijo verjetnost,
da bodo lobisti pridobili podporo vseh $n$ političnih strank.

\begin{enumerate}[(a)]
\item Zapiši rekurzivne enačbe za reševanje danega problema.
\item Naj bo $m = 6$ in $n = 3$.
K vsaki stranki želijo poslati vsaj enega lobista
(tj., $p_{0j} = 0$ za vsak $j$),
vrednosti $p_{ij}$ za $i \ge 1$ pa so podane v spodnji tabeli.
$$
\begin{array}{c|ccc}
p_{ij} & 1 & 2 & 3 \\ \hline
1 & 0.2 & 0.4 & 0.3 \\
2 & 0.5 & 0.5 & 0.4 \\
3 & 0.7 & 0.5 & 0.8 \\
4 & 0.8 & 0.6 & 0.9 \\
\end{array}
$$
Za dane podatke reši problem z zgoraj zapisanimi enačbami.
\end{enumerate}
\end{vprasanje}
\begin{odgovor}
\end{odgovor}
\end{naloga}
