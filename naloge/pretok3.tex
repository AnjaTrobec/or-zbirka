\begin{naloga}{?}{Vaje OM 23.4.2014}
\begin{vprasanje}
Poišči maksimalni pretok in minimalni prerez na grafu s slike~\fig
v odvisnosti od parametra $\alpha$.

\begin{slika}
\pgfslika
\podnaslov{Graf}
\end{slika}
\end{vprasanje}

\begin{odgovor}
Najprej poiščemo maksimalni pretok za $\alpha = 0$
(tj., brez povezave s parametrom), glej sliko~\fig[pretok3a].
Dobimo maksimalni pretok $0+3 = 3+0+0 = 3$, glej sliko~\fig[pretok3b].

Za $a > 0$ poiščemo povečujočo pot.
Če je $0 < \alpha \le 3$,
povečamo pretok po povečujoči poti za $\alpha$
in tako dobimo maksimalen pretok $p^* = \alpha+3$ ($3 < p^* \le 6$),
saj sta obe povezavi iz izvora zasičeni, glej sliko~\fig[pretok3c].
V nasprotnem primeru ($\alpha > 3$) pretok povečamo za $3$
in poiščemo novo povečujočo pot.

Če je $3 < \alpha \le 5$, povečamo pretok po povečujoči poti za $\alpha-3$
in tako dobimo maksimalen pretok $p^* = \alpha+3$ ($6 < p^* \le 8$),
saj sta obe povezavi iz izvora zasičeni, glej sliko~\fig[pretok3d].
V nasprotnem primeru ($\alpha > 5$) pretok povečamo za $2$.

Za $\alpha > 5$ dobimo maksimalni pretok $5+3 = 3+2+3 = 8$,
glej sliko~\fig[pretok3e].

\begin{slika}
\pgfslika[pretok3a]
\podnaslov{Prvi korak}
\end{slika}
\begin{slika}
\pgfslika[pretok3b]
\podnaslov{Maksimalni pretok pri $\alpha = 0$ in drugi korak pri $\alpha > 0$}
\end{slika}
\begin{slika}
\pgfslika[pretok3c]
\podnaslov{Maksimalni pretok pri $0 < \alpha = \beta \le 3$
in tretji korak pri $\alpha > \beta = 3$}
\end{slika}
\begin{slika}
\pgfslika[pretok3d]
\podnaslov{Maksimalni pretok pri $3 < \alpha \le 5$}
\end{slika}
\begin{slika}
\pgfslika[pretok3e]
\podnaslov{Maksimalni pretok pri $\alpha > 5$}
\end{slika}
\end{odgovor}
\end{naloga}
