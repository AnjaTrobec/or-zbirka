\begin{naloga}{?}{Vaje OM 23.4.2014}
\begin{vprasanje}
Poišči maksimalni pretok in minimalni prerez na grafu s slike~\fig{}
v odvisnosti od parametra $a$.

\begin{slika}
\pgfslika
\podnaslov{Graf}
\end{slika}
\end{vprasanje}

\begin{odgovor}
Najprej poiščemo maksimalni pretok za $a = 0$
(tj., brez povezave s parametrom).
Dobimo maksimalni pretok $0+3 = 3+0+0 = 3$, glej sliko~\fig{pretok3a}.

Za $a > 0$ poiščemo povečujočo pot.
Če je $0 < a \le 3$,
povečamo pretok po povečujoči poti za $a$
in tako dobimo maksimalen pretok $p^* = a+3$ ($3 < p^* \le 6$),
saj sta obe povezavi iz izvora zasičeni, glej sliko~\fig{pretok3b}.
V nasprotnem primeru ($a > 3$) pretok povečamo za $3$
in poiščemo novo povečujočo pot.

Če je $3 < a \le 5$, povečamo pretok po povečujoči poti za $a-3$
in tako dobimo maksimalen pretok $p^* = a+3$ ($6 < p^* \le 8$),
saj sta obe povezavi iz izvora zasičeni, glej sliko~\fig{pretok3c}.
V nasprotnem primeru ($a > 5$) pretok povečamo za $2$.

Za $a > 5$ dobimo maksimalni pretok $5+3 = 3+2+3 = 8$,
glej sliko~\fig{pretok3d}.

\begin{slika}
\pgfslika[pretok3a]
\podnaslov{Maksimalni pretok pri $a = 0$}
\end{slika}
\begin{slika}
\pgfslika[pretok3b]
\podnaslov{Maksimalni pretok pri $0 < a \le 3$}
\end{slika}
\begin{slika}
\pgfslika[pretok3c]
\podnaslov{Maksimalni pretok pri $3 < a \le 5$}
\end{slika}
\begin{slika}
\pgfslika[pretok3d]
\podnaslov{Maksimalni pretok pri $a > 5$}
\end{slika}
\end{odgovor}
\end{naloga}
