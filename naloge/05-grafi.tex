\razdelek{Algoritmi na grafih}

\begin{naloga}{?}{Vaje OR 20.4.2016}
\begin{vprasanje}[matgraf]
Zasnuj podatkovno strukturo za grafe,
ki temelji na matrični predstavitvi.
Podatkovna struktura naj ima sledeče metode:
\begin{itemize}
\item {\tt \_\_init\_\_(G)}: ustvarjanje praznega grafa
\item {\tt dodajVozlisce(G, u)}: dodajanje novega vozlišča
\item {\tt dodajPovezavo(G, u, v)}: dodajanje nove povezave
\item {\tt brisiPovezavo(G, u, v)}: brisanje povezave
\item {\tt brisiVozlisce(G, u)}: brisanje vozlišča
\item {\tt sosedi(G, u)}: seznam sosedov danega vozlišča
\end{itemize}
Za vsako od naštetih metod podaj tudi njeno časovno zahtevnost
v odvisnosti od števila vozlišč, števila povezav in stopenj vhodnih vozlišč.
Oceni tudi prostorsko zahtevnost celotne strukture.
\end{vprasanje}
\begin{odgovor}
\end{odgovor}
\end{naloga}


\begin{naloga}{?}{Vaje OR 20.4.2016}
\begin{vprasanje}[sosgraf]
Zasnuj podatkovno strukturo za grafe,
ki temelji na seznamih sosedov.
Zapiši metode kot pri prejšnji strukturi
ter oceni njihovo časovno zahtevnost
in prostorsko zahtevnost celotne strukture.
\end{vprasanje}
\begin{odgovor}
\end{odgovor}
\end{naloga}


\begin{naloga}{?}{Vaje OR 20.4.2016}
\begin{vprasanje}
Kako moramo spremeniti strukturi
iz nalog~\nal{matgraf} in~\nal{sosgraf},
da bosta predstavljali digrafe?
\end{vprasanje}
\begin{odgovor}
\end{odgovor}
\end{naloga}


\begin{naloga}{?}{Vaje OR 20.4.2016}
\begin{vprasanje}
Napiši algoritem, ki za vhodni graf $G$ določi, ali ima trikotnik.
Katero podatkovno strukturo za grafe boš uporabil?
\end{vprasanje}
\begin{odgovor}
\end{odgovor}
\end{naloga}


\begin{naloga}{?}{Vaje OR 4.5.2016}
\begin{vprasanje}
Dan je digraf $D = (V, E)$.
Pravimo, da je vozlišče $v \in V$ {\em zvezda} digrafa $D$,
če ima izhodno povezavo do vseh ostalih vozlišč
in v digrafu $D$ ni drugih povezav.
Napiši algoritem, ki poišče zvezdo danega digrafa, če ta obstaja.
\end{vprasanje}
\begin{odgovor}
\end{odgovor}
\end{naloga}


\begin{naloga}{Janoš Vidali}{Vaje OR 30.11.2016}
\begin{vprasanje}[bfs]
Na grafu s slike~\fig{} izvedi iskanje v širino.
V primerih, ko imaš več ena\-ko\-vred\-nih izbir,
upoštevaj abecedni vrstni red.
Za vsako povezavo določi, ali se nahaja v drevesu iskanja v širino.

\begin{slika}
\pgfslika
\caption{Graf za nalogi~\nal{} in~\nal{dfs}.}
\end{slika}
\end{vprasanje}
\begin{odgovor}
\end{odgovor}
\end{naloga}


\begin{naloga}{?}{Vaje OR 4.5.2016}
\begin{vprasanje}
Zapiši algoritem, ki za vhodni graf $G$ določi njegov premer.

\end{vprasanje}
\begin{odgovor}
\end{odgovor}
\end{naloga}


\begin{naloga}{Janoš Vidali}{Vaje OR 7.12.2016}
\begin{vprasanje}[dijkstra]
S pomočjo Dijkstrovega algoritma
določi razdalje od vozlišča $A$ do ostalih vozlišč
v grafu s slike~\fig{}.

\begin{slika}
\pgfslika
\podnaslov{Graf}
\end{slika}
\end{vprasanje}
\begin{odgovor}
\end{odgovor}
\end{naloga}

\begin{naloga}{?}{Vaje OR 4.5.2016}
\begin{vprasanje}
Naj bo $G = (V, E)$ graf,
za katerega so dolžine povezav določene s funkcijo $\ell : E \to \R$
(tj., dolžine so lahko tudi negativne).
Definirajmo še funkcijo $\ell' : E \to \R$ tako,
da velja $\ell'(e) = \ell(e) - \min\set{\ell(f)}{f \in E}$
(dolžine, določene z $\ell'$, so torej nenegativne).
Dokaži ali ovrzi: drevo najkrajših poti,
ki ga Dijkstrov algoritem ustvari ob vhodu $(G, \ell')$,
je tudi drevo najkrajših poti za graf $G$ z dolžinami povezav,
določenimi z $\ell$.
\end{vprasanje}
\begin{odgovor}
\end{odgovor}
\end{naloga}


\begin{naloga}%
{Dasgupta, Papadimitriou, Vazirani}{\cite[Exercise~4.13]{dpv}}
\begin{vprasanje}
Denimo, da imamo neusmerjen graf $G = (V, E)$,
katerega vozlišča pred\-stav\-lja\-jo mesta,
povezave pa predstavljajo ceste, ki jih povezujejo.
Za vsako povezavo $e \in E$ poznamo njeno dolžino $\ell_e$ (v kilometrih).

Priti želimo iz mesta $s$ v mesto $t$.
V vsakem mestu je bencinska črpalka, ob cestah pa teh ni.
Žal imamo na voljo samo star avto,
ki lahko s polnim rezervoarjem prepelje le $L$ kilometrov.
\begin{enumerate}[(a)]
\item Zapiši algoritem, ki v linearnem času poišče pot,
ki jo lahko prevozimo z našim avtom,
oziroma ugotovi, da ta ne obstaja.
\item Izkaže se, da z našim avtom te poti ne moremo prevoziti,
zato se odločimo za nakup novega.
Zapiši algoritem, ki v času $O(m \log n)$
določi najmanjše število prevoženih kilometrov,
ki naj jih avto zmore z enim polnjenjem,
da bo pot od $s$ do $t$ mogoča.
\end{enumerate}

\end{vprasanje}
\begin{odgovor}
\end{odgovor}
\end{naloga}


\begin{naloga}{Sergio Cabello}{Vaje OR 7.12.2016}
\begin{vprasanje}
Zasnuj različico Dijkstrovega algoritma
za iskanje najkrajše poti med vozliščema $s$ in $t$ v grafu $G$,
ki iskanje hkrati začne v vozliščih $s$ in $t$.
Kdaj naj se iskanje konča in kako naj se poišče rešitev?
\end{vprasanje}
\begin{odgovor}
\end{odgovor}
\end{naloga}


\begin{naloga}%
{Dasgupta, Papadimitriou, Vazirani}{\cite[Exercise~3.1]{dpv}}
\begin{vprasanje}[dfs]
Na grafu s slike~\fig{bfs} izvedi iskanje v globino.
V primerih, ko imaš več ena\-ko\-vred\-nih izbir,
upoštevaj abecedni vrstni red.
Za vsako povezavo določi, ali se nahaja v drevesu iskanja v globino.
\end{vprasanje}
\begin{odgovor}
\end{odgovor}
\end{naloga}


\begin{naloga}{Janoš Vidali}{Vaje OR 21.5.2018}
\begin{vprasanje}[bf]
S pomočjo Bellman-Fordovega algoritma
določi razdalje od vozlišča $A$ do ostalih vozlišč
v grafu s slike~\fig{}.

\begin{slika}
\pgfslika
\podnaslov{Graf}
\end{slika}
\end{vprasanje}
\begin{odgovor}
\end{odgovor}
\end{naloga}


\begin{naloga}{Janoš Vidali}{Vaje OR 7.12.2016}
\begin{vprasanje}[topo]
Dan je usmerjen acikličen graf s slike~\fig{}.

\begin{enumerate}[(a)]
\item Poišči topološko ureditev vozlišč zgornjega grafa.

\item Poišči najkrajšo pot od vozlišča $G$ do vozlišča $E$.

\item Poišči najdaljšo pot od vozlišča $G$ do vozlišča $E$.
\end{enumerate}

\begin{slika}
\pgfslika
\podnaslov{Graf}
\end{slika}
\end{vprasanje}
\begin{odgovor}
\end{odgovor}
\end{naloga}


\begin{naloga}{?}{Kolokvij OR 9.5.2013}
\begin{vprasanje}[oviratlon]
Oviratlon je tekalna preizkušnja
na 8 do 10 kilometrov dolgi poti z različnimi ovirami.
Zanima nas, na koliko različnih načinov lahko pridemo od štarta do cilja.
Dan je utežen usmerjen acikličen graf $G$ ter vozlišči $s$ in $t$,
ki predstavljata štart oziroma cilj.
Uteži na povezavah nam predstavljajo,
na koliko načinov jih lahko prečkamo.

\begin{enumerate}[(a)]
\item Reši nalogo za graf s slike~\fig{}.

\item Zapiši algoritem, ki reši dani problem.
Kakšna je njegova časovna zahtevnost?
\end{enumerate}

\begin{slika}
\pgfslika
\podnaslov{Graf}
\end{slika}
\end{vprasanje}
\begin{odgovor}
\end{odgovor}
\end{naloga}


\begin{naloga}{Sergio Cabello}{Vaje OR 14.12.2016}
\begin{vprasanje}
Zapiši celoštevilski linearni program, ki določi topološko urejanje grafa.
\end{vprasanje}
\begin{odgovor}
\end{odgovor}
\end{naloga}


\begin{naloga}{Sergio Cabello}{Vaje OR 14.12.2016}
\begin{vprasanje}
Zapiši algoritem, ki ugotovi, ali ima graf več kot eno topološko ureditev.
\end{vprasanje}
\begin{odgovor}
\end{odgovor}
\end{naloga}


\begin{naloga}{Janoš Vidali}{Izpit OR 15.12.2016}
\begin{vprasanje}[zaklad]
Lovec na zaklade se z bogatim ulovom
vrača iz Kalifornije nazaj domov v Chicago,
pri čemer mora seveda prečkati Divji zahod.
Potoval bo s kočijo,
pri čemer bo vsak dan potoval med dvema mestoma in nato prespal.
Zaradi varnosti se bo držal samo državnih cest, ki so varne.
Toda mesta, kjer bo prespal, niso povsem varna.
Za vsako mesto pozna verjetnosti,
da ga tam ne bodo oropali (te so med seboj neodvisne).
Tako bi želel načrtovati najvarnejšo pot domov
-- torej pot z največjo verjetnostjo,
da ga pri nobenem postanku ne bodo oropali.

\begin{enumerate}[(a)]
\item Mesta in ceste med njimi lahko predstavimo z vozlišči in povezavami
v ne\-usme\-rje\-nem grafu $G$, verjetnosti pa kot teže vozlišč.
Opiši, kako lahko za dani graf $G$ z uteženimi vozlišči
učinkovito poiščemo ustrezno pot med danima vozliščema $s$ in $t$
z uporabo variante Dijkstrovega algoritma,
ter utemelji njegovo ustreznost.
Lahko predpostaviš, da sta teži začetnega in končnega vozlišča enaki $1$.

\item Reši problem za graf s slike~\fig{},
pri čemer naj se pot začne v LA in konča v CH.
Zadostovalo bo, če verjetnosti računaš na $3$ decimalke natančno.
\end{enumerate}

\begin{slika}
\pgfslika
\podnaslov{Graf}
\end{slika}
\end{vprasanje}
\begin{odgovor}
\end{odgovor}
\end{naloga}


\begin{naloga}{Sergio Cabello}{Izpit OR 15.3.2017}
\begin{vprasanje}
Dan je usmerjen graf $G = (V, E)$ s pozitivnimi dolžinami povezav.
Dano imamo še vozlišče $s \in V$
ter vrednost $\delta_v$ za vsako vozlišče $v \in V$.
Natančno opiši algoritem (z besedami ali psevdokodo),
ki v času $O(m)$ (kjer je $m = |V| + |E|$) preveri,
ali velja $\delta_v = d_G(s, v)$ za vse $v \in V$
-- t.j., v linearnem času preveri,
ali so vrednosti $\delta_v$
enake razdalji med vozliščema $s$ in $v$ v grafu $G$.

Lahko predpostaviš, da so vsa vozlišča grafa $G$ dosegljiva iz vozlišča $s$.
Utemelji pravilnost meje za časovno zahtevnost algoritma.
\end{vprasanje}
\begin{odgovor}
\end{odgovor}
\end{naloga}


\begin{naloga}{Janoš Vidali}{Izpit OR 10.7.2017}
\begin{vprasanje}[pocitnice]
Z električnim vozilom se odpravljamo na počitnice.
Vozilo moramo vsako noč napolniti,
zato smo si pripravili seznam krajev in cestnih povezav med njimi,
ki jih lahko prevozimo v enem dnevu.
Poiskati želimo pot od začetne točke do destinacije,
ki bo imela čim manjše število postankov (tj., bo trajala čim manj dni).

\begin{enumerate}[(a)]
\item Predstavi problem v jeziku grafov
in predlagaj algoritem za njegovo reševanje.

\item Na koncu počitnic razmišljamo o poti nazaj.
Spet bi radi naredili čim manj postankov,
a se pri tem ne želimo ustaviti v nobenem kraju,
kjer smo se ustavili na poti naprej.
Dopolni zgornji algoritem, da bo našel še ustrezno pot nazaj.

\item S pomočjo zgornjih algoritmov poišči najkrajšo pot od LJ do AM
in najkrajšo pot nazaj v grafu s slike~\fig{},
ki ne gre čez kraje iz prejšnje poti.
\end{enumerate}

\begin{slika}
\pgfslika
\caption{Graf za nalogi~\nal{} (brez uteži) in~\nal{pot}.}
\end{slika}
\end{vprasanje}
\begin{odgovor}
\end{odgovor}
\end{naloga}


\begin{naloga}{Janoš Vidali}{Kolokvij OR 11.6.2018}
\begin{vprasanje}[prerezna]
Dan je povezan neusmerjen enostaven graf $G = (V, E)$
(tj., brez zank in večkratnih povezav).
{\em Prerezno vozlišče} v grafu $G$ je tako vozlišče $u \in V$,
da graf $G - u$
(tj., graf $G$ brez vozlišča $u$ in povezav s krajiščem v $u$)
ni več povezan.
Poiskati želimo seznam prereznih vozlišč grafa $G$.

Pri iskanju si bomo pomagali s preiskovanjem v globino.
Ob prvem obisku vozlišča $u$ s predhodnikom $v$
se tako pokliče funkcija \verb|previsit(u, v)|,
ob njegovem zadnjem obisku pa funkcija \verb|postvisit(u, v)|.
Če je $u$ koren preiskovalnega drevesa, potem ima $v$ vrednost \verb|None|.
Predpostavi, da imaš v obeh funkcijah dostop do seznama \verb|izhod|,
kamor bo treba dodati najdena presečna vozlišča.
Prav tako imata lahko obe funkciji dostop do drugih pomožnih spremenljivk.

Naj bo $\ell_u$ globina vozlišča $u$ v drevesu iskanja v globino
(tj., razdalja od korena do $u$ v drevesu iskanja v globino).
Za vsako vozlišče $u$ definiramo vrednost $p_u$ kot najmanjšo globino vozlišč,
ki so v grafu $G$ sosedna vozlišču $u$
ali njegovim potomcem v drevesu iskanja v globino.

\begin{enumerate}[(a)]
\item Za graf na sliki~\fig{} nariši drevo iskanja v globino
(v njem označi tudi povratne povezave, npr.~s črtkano črto)
in določi njegova prerezna vozlišča.
Upoštevaj abecedni vrstni red obiskovanja vozlišč.
Za vsako vozlišče $u$ določi še vrednosti $\ell_u$ in $p_u$.
\namig{vrednosti $p_u$ najprej določi za vozlišča z večjo globino.}

\item Napiši rekurzivno formulo za vrednost $p_u$.
\namig{loči med sosedi $v$ vozlišča $u$ v grafu $G$ (pišeš lahko $u \sim v$)
in njegovimi neposrednimi nasledniki $w$ v preiskovalnem drevesu ($u \to w$).}

\item Natančno opiši funkcijo \verb|previsit(u, v)|
(z besedami ali psevdokodo),
ki poskrbi za izračun vrednosti $\ell_u$.

\item Natančno opiši funkcijo \verb|postvisit(u, v)|
(z besedami ali psevdokodo),
ki naj za vozlišče $u$ izračuna vrednost $p_u$ in ugotovi,
ali je $u$ prerezno vozlišče, in ga v tem primeru doda v \verb|izhod|.
Predpostavi, da imaš globine vozlišč že poračunane
(tudi, če točka (c) ni rešena).
\namig{obravnavaj dve možnosti --
ko je $u$ koren drevesa, in ko $u$ ni koren drevesa.
Kako v vsakem od teh primerov ugotoviš, ali je $u$ prerezno vozlišče?}

\item Oceni časovno zahtevnost celotnega algoritma.
\end{enumerate}

\begin{slika}
\pgfslika
\podnaslov{Graf}
\end{slika}
\end{vprasanje}
\begin{odgovor}
\end{odgovor}
\end{naloga}


\begin{naloga}{Janoš Vidali}{Izpit OR 5.7.2018}
\begin{vprasanje}[dag]
Dan je utežen usmerjen acikličen graf s slike~\fig{}.

\begin{enumerate}[(a)]
\item Poišči topološko ureditev grafa s slike~\fig{}.

\item Poišči najcenejšo pot od vozlišča $S$ do vozlišča $T$
v grafu s slike~\fig{}.

\item Naj bo $G = (V, E)$ usmerjen acikličen graf
z nenegativno uteženimi povezavami
ter $s, t \in V$ njegovi vozlišči.
Algoritem $\A$ se po grafu $G$ sprehaja po naslednjem pravilu:
začne v vozlišču $s$,
v vsakem koraku pa se iz vozlišča $u$ premakne
v njegovega izhodnega soseda $v$ z verjetnostjo
$$
p_{uv} = {\ell_{uv} \over \sum_{u \to w} \ell_{uw}} ,
$$
kjer je $\ell_{uv}$ teža povezave od $u$ do $v$.
Algoritem $\A$ se ustavi, ko doseže vozlišče $t$.

Natančno opiši (z besedami ali psevdokodo),
kako bi v času $O(m)$ (kjer je $m = |V| + |E|$)
za vsako vozlišče $u \in V$ določil verjetnost $q_u$,
da algoritem $\A$ obišče vozlišče $u$.
Verjetnosti za graf s slike~\fig{} ni potrebno računati.
\end{enumerate}

\begin{slika}
\pgfslika
\podnaslov{Graf}
\end{slika}
\end{vprasanje}
\begin{odgovor}
\end{odgovor}
\end{naloga}


\begin{naloga}{Janoš Vidali}{Izpit OR 28.8.2018}
\begin{vprasanje}[pot]
Odpravljamo se na pot, ki bo trajala več dni.
Pripravili smo si seznam krajev in povezav med njimi,
ki jih lahko prevozimo v enem dnevu.
Za vsako povezavo poznamo stroške prevoza,
prav tako pa za vsak kraj poznamo še stroške nočitev.
Poiskati želimo čim cenejšo pot od začetne točke do destinacije
(tj., skupna cena prevozov in nočitev naj bo čim manjša).

\begin{enumerate}[(a)]
\item Predstavi problem v jeziku grafov
in predlagaj čim bolj učinkovit algoritem za njegovo reševanje.

\item S pomočjo zgornjega algoritma poišči najcenejšo pot od LJ do BX
v grafu s slike~\fig{pocitnice}.
Na povezavah so napisani stroški prevozov med krajema
(veljajo za obe smeri),
pri vozliščih pa stroški prenočitve v kraju.
\end{enumerate}
\end{vprasanje}
\begin{odgovor}
\end{odgovor}
\end{naloga}


\begin{naloga}{?}{Kolokvij OR 19.11.2009}
\begin{vprasanje}
Naj bo $G$ graf z uteženimi povezavami, kjer so uteži nenegativne.
Dokaži ali ovrzi naslednje trditve:
\begin{enumerate}[(a)]
\item Če v grafu obstaja enolična najkrajša povezava,
potem je ta povezava vsebovana v vsakem drevesu najkrajših poti.
\item Če v grafu obstaja enolična najkrajša povezava,
potem je ta vključena v vsako minimalno vpeto drevo.
\item Naj bo $e$ povezava v grafu $G$,
ki je izmed povezav na nekem ciklu $C$ najdaljša.
Če iz $G$ odstranimo povezavo $e$, dobimo graf $G'$.
Pokaži, da je poljubno minimalno vpeto drevo v $G'$
tudi minimalno vpeto drevo v $G$.
\end{enumerate}
\end{vprasanje}
\begin{odgovor}
\end{odgovor}
\end{naloga}


\begin{naloga}{?}{Kolokvij OR 19.11.2009}
\begin{vprasanje}
Zaradi naravnih nesreč so reševalci prejeli klice
za $n$ ponesrečenih oseb iz različnih lokacij,
ki jih je potrebno prepeljati v $k$ bolnišnic po danem cestnem omrežju.
Ponesrečence lahko prepeljemo do bolnišnic,
ki so oddaljene za največ pol ure urgentne vožnje
(za dane ponesrečence imamo torej v splošnem več izbir za bolnišnice).
Da pa bolnišnic ne bi preobremenili, zahtevamo,
da v vsako bolnišnico pride največ $\lceil {n \over k} \rceil$ ponesrečencev.
Zanima nas, ali je to izvedljivo.
Modeliraj problem in ga prevedi na kakšen znan problem.
\end{vprasanje}

\begin{odgovor}
Dano cestno omrežje lahko predstavimo kot usmerjen graf $G$,
kjer uteži predstavljajo čas potovanja po cestni povezavi.
Na grafu $G$ lahko uporabimo Floyd-Warshallov algoritem
za izračun razdalje med vsemi pari vozlišč.
Nato zgradimo dvodelen graf $H$,
v katerem imamo po eno vozlišče za vsakega ponesrečenca
in po $\lceil {n \over k} \rceil$ vozlišč za vsako bolnišnico,
vozlišči pa sta povezani,
če je razdalja od ponesrečenca do bolnišnice v grafu $G$ največ pol ure.
Na grafu $H$ nato z madžarsko metodo poiščemo največje prirejanje $M$.
Če ima najdeno prirejanje $n$ povezav,
potem smo našli ustrezno dodelitev ponesrečencev bolnišnicam,
sicer pa ta ne obstaja.
\end{odgovor}
\end{naloga}


\begin{naloga}{?}{Kolokvij OR 19.11.2009}
\begin{vprasanje}[poslovnez]
Janez je prekaljen poslovnež, ki hoče vedno zaslužiti kar največ.
A Janez lahko naenkrat prevzame le en posel.
Če se en posel začne med izvajanjem nekega drugega posla,
novega posla ne more prevzeti.
Poslovne priložnosti so predstavljene z grafom.
Vozlišča predstavljajo stanja, v katerih Janez izbira med posli,
izhodne povezave iz vsakega vozlišča pa predstavljajo posle,
ki se jih lahko loti.
Cene povezav predstavljajo dobiček pri poslu oziroma izgubo,
če je cena negativna
(včasih je potrebno sprejeti tudi kak posel, ki nosi izgubo,
da se lahko prebijemo do dobičkonosnega posla \dots).
V vsakem stanju lahko Janez izbere katerega koli izmed poslov,
ki pripadajo izhodnim povezavam.
Tekom svoje poslovne kariere se lahko Janez
v določenem stanju znajde tudi večkrat.
Janez se trenutno nahaja v izbranem vozlišču grafa.
\begin{enumerate}[(a)]
\item Za poljuben graf poslovnih priložnosti ugotovi,
kakšen problem moraš na njem rešiti, da boš lahko pravilno svetoval Janezu,
da bo uresničil svoje poslovne ambicije.
Predlagaj algoritem, s katerim rešiš problem,
in oceni časovno zahtevnost predlagane rešitve.
Janeza med drugim še posebej zanima,
ali mu graf poslovnih priložnosti zagotavlja stalno pridobivanje poslov
in dobičkonosno poslovanje za celo kariero.
\item Za graf poslovnih priložnosti s slike~\fig{}
izberi ustrezen algoritem in ga izvedi ter izračunaj,
koliko največ lahko zasluži Janez, če ``vstopi v igro'' v stanju $A$ ali $B$.
Upoštevaj lastnosti spodnjega grafa in navedi, kateri algoritem je to.
Ali graf predstavlja poslovne priložnosti,
ki bodo Janezu zagotovile trajno dobičkonosno poslovanje?
\end{enumerate}

\begin{slika}
\pgfslika
\podnaslov{Graf}
\end{slika}
\end{vprasanje}
\begin{odgovor}
\end{odgovor}
\end{naloga}


\begin{naloga}{?}{Izpit OR 5.2.2010}
\begin{vprasanje}
Odgovori na naslednje trditve.
\begin{enumerate}[(a)]
\item Neusmerjeno omrežje $G$ s pozitivnimi razdaljami na povezavah
razdelimo na dve disjunktni množici vozlišč $A$ in $B$,
ki pokrivata vsa vozlišča.
Naj bo $e$ najkrajša povezava z enim krajiščem v $A$ in drugim v $B$.
Pokaži ali ovrzi:
    \begin{itemize}
    \item Vsaka najkrajša pot med kakim vozliščem iz $A$
    in kakim vozliščem iz $B$ vsebuje povezavo $e$.
    \item Obstaja najkrajša pot med nekima vozlščema v grafu $G$,
    ki vsebuje povezavo $e$.
    \item Za vsak par vozlišč $a \in A$ in $b \in B$ obstaja najkrajša pot,
    ki vsebuje povezavo $e$.
    \end{itemize}
\item Naj bo $e$ povezava v grafu $G$,
ki je izmed povezav na nekem ciklu $C$ najdaljša.
Če iz $G$ odstranimo povezavo $e$, dobimo graf $G'$.
Pokaži, da je poljubno minimalno vpeto drevo v $G'$
tudi minimalno vpeto drevo v $G$.
\end{enumerate}
\end{vprasanje}
\begin{odgovor}
\end{odgovor}
\end{naloga}


\begin{naloga}{?}{Izpit OR 28.6.2010}
\begin{vprasanje}
Na univerzi na grškem otoku Lenoritas
so zaradi gospodarske krize uvedli varčevalne ukrepe,
ki vključujejo tudi nov režim plačevanja za izpite,
ki jih pazijo asistenti.
Asistentom sta ostali le dve boniteti:
nadomestilo za izvedbo strnjenega zaporedja izpitov
ter nadomestilo za vsako ``luknjo'', daljšo od pol ure,
med strnjenimi zaporedji izpitov.
Ostale bonitete,
kot so trinajsta in štirinajsta plača
ter dodatek za točnost na delovnem mestu,
je univerza pred kratkim ukinila.
Tako se univerzi splača, da uporabi čim manj asistentov,
ki pazijo strnjena zaporedja izpitov brez ``lukenj''.
Za vsak dan so termini izpitov podani v naprej.
Podatki za ponedeljek so $(7, 8)$, $(8, 9)$, $(9, 10)$, $(8, 12)$,
$(10, 13)$, $(12, 15)$, $(13, 15)$, $(9, 12)$, $(7, 10)$.

Ne glede na ekonomičnost se je dekan odločil,
da bo v ponedeljek vodji sindikata asistentov dr.~Dusanikusu Semolitakisu
zagodel in mu dal kar najdaljše možno strnjeno zaporedje izpitov.
Izberi ustrezen algoritem za izvedbo te naloge
s kar se da najboljšo časovno zahtevnostjo (in jo tudi navedi),
ter s pomočjo le-tega reši dekanov ``problem'' za ponedeljek.

\nadaljevanje{lenority}
\end{vprasanje}
\begin{odgovor}
\end{odgovor}
\end{naloga}


\begin{naloga}{?}{Izpit OR 15.9.2010}
\begin{vprasanje}
Z usmerjenim grafom je podano omrežje enosmernih prehodov
med letališkimi terminali (glej npr.~sliko~\fig{terminali}).
Načrtovalci niso čisto gotovi, ali je možno prehajati med vsemi terminali.
Predlagaj algoritem, ki bi ugotovil,
ali je vedno možen prehod med poljubnima izbranima terminaloma.
Kakšna je časovna zahtevnost predlaganega algoritma?

\nadaljevanje{terminali}
\end{vprasanje}
\begin{odgovor}
\end{odgovor}
\end{naloga}


\begin{naloga}{?}{Kolokvij OR 25.11.2010}
\begin{vprasanje}[stpoti]
Napiši psevdokodo algoritma,
ki v acikličnem usmerjenem grafu prešteje število poti
od danega vozlišča $s$ do vseh vozlišč grafa.
Algoritem nato uporabi na grafu s slike~\fig{}.

\begin{slika}
\pgfslika
\podnaslov{Graf}
\end{slika}
\end{vprasanje}
\begin{odgovor}
\end{odgovor}
\end{naloga}


\begin{naloga}{?}{Kolokvij OR 25.11.2010}
\begin{vprasanje}[otoki]
Otoke z grafa na sliki~\fig{} želimo povezati z mostovi,
tako da bo skupna cena gradnje čim manjša.
Cena gradnje mostu je milijon evrov na kilometer
(razdalje v kilometrih so označene na sliki;
na povezavah, ki niso narisane,
zaradi tehničnih preprek ne moremo zgraditi mostov).
Med katerimi otoki naj zgradimo mostove?
Pri iskanju odgovora uporabi primeren algoritem
in zapiši vse vmesne rezultate.
Kakšen je odgovor,
če je za vsak zgrajeni most potrebno dobiti potrdilo inšpektorjev,
ti pa za vsak pregled zaračunajo milijon evrov?

\begin{slika}
\pgfslika
\podnaslov{Graf}
\end{slika}
\end{vprasanje}
\begin{odgovor}
\end{odgovor}
\end{naloga}


\begin{naloga}{?}{Izpit OR 28.6.2011}
\begin{vprasanje}[cesta]
Radi bi zgradili cesto med točkama $s$ in $t$.
Stroški vsakega možnega cestnega odseka so prikazani na grafu na sliki~\fig{}.
Kje naj poteka cesta, da bodo stroški gradnje najmanjši možni?

\begin{slika}
\pgfslika
\podnaslov{Graf}
\end{slika}
\end{vprasanje}
\begin{odgovor}
\end{odgovor}
\end{naloga}


\begin{naloga}{?}{Kolokvij OR 4.5.2012}
\begin{vprasanje}
Na kongresu se je zbralo $n$ delegatov.
Problem je v tem, da ne govorijo vsi skupnega jezika.
Za vsakega delegata poznamo seznam jezikov, ki jih govori,
skupaj s stopnjo znanja za vsak posamični jezik.
Stopnja znanja je število med $1$ in $100$.
Stopnja $1$ pomeni, da oseba jezik popolnoma obvlada,
stopnja $100$ pa pomeni, da pozna zgolj nekaj osnovnih fraz.
Recimo, da bi oseba $A$ rada osebi $B$ posredovala neko sporočilo.
Če znata skupni jezik, se lahko pogovorita neposredno.
Lahko pa oseba $A$ pošlje sporočilo preko enega ali več posrednikov.

\begin{enumerate}[(a)]
\item Radi bi, da oseba $B$ prejme sporočilo v najkrajšem možnem času
(pri čemer lahko sporočilo potuje preko enega ali več posrednikov).
Če osebi $X$ in $Y$ govorita skupen jezik
ter sta stopnji obvladovanja tega jezika
$s_X$ za osebo $X$ in $s_Y$ za osebo $Y$,
prenos sporočila traja $\max\{s_X, s_Y\}$ časovnih enot.
(Če dve osebi ne obvladata dovolj dobro skupnega jezika,
si lahko pomagata z opisovanjem pojmov, mahanjem rok, risanjem ipd.)
Formuliraj zgornjo nalogo
kot problem iskanja najkrajše poti v ustreznem grafu.

\item Dan je naslednji seznam delegatov:

\smallskip
\makebox[\textwidth][r]{
\begin{small}
\begin{tabular}{c|c}
Delegat & Jeziki \\ \hline
Frank & (angleščina, $5$), (španščina, $10$), (ruščina, $80$) \\
Ivan & (ruščina, $5$), (španščina, $20$), (angleščina, $95$) \\
Paul-Henri & (francoščina, $5$), (nemščina, $85$), (angleščina, $95$) \\
Brigitte & (nizozemščina, $10$), (nemščina, $15$) \\
Andrej & (slovenščina, $5$), (nemščina, $10$), (latinščina, $90$) \\
Wolfgang & (nemščina, $5$), (angleščina, $90$) \\
Jafar &
(arabščina, $5$), (ruščina, $10$), (nizozemščina, $30$), (francoščina, $80$)
\end{tabular}
\end{small}
}

\bigskip
Kako lahko Frank najhitreje posreduje informacijo Wolfgangu?
\end{enumerate}
\end{vprasanje}
\begin{odgovor}
\end{odgovor}
\end{naloga}


\begin{naloga}{?}{Izpit OR 9.7.2012}
\begin{vprasanje}[cmrlji]
Čmrlja Gaber in Bor slovita kot najhitrejša letalca med čmrlji.
Nedavno je Gaber izzval Bora na dirko čez travnik,
kjer pot ni vnaprej določena.
Bor je izziv sprejel in dogovorila sta se za pravili:
\begin{itemize}
\item določila sta točki $s$ in $t$:
začneta v $s$, in kdor prvi pride v $t$, zmaga;
\item določila sta tudi graf poti $G$, po katerih lahko letita.
Povezave v tem grafu sta določila tako,
da je čas letenja (če ni rožic na povezavi)
od enega do drugega krajišča za vse povezave enak (čmrlja sta enako hitra).
Vozlišča v tem grafu pa sta določila tako, da na njih ne raste nobena rožica.
\end{itemize}
Znano je, da če katerikoli čmrlj prileti do kake rožice,
se na njej zadrži enako dolgo,
kot potrebuje za prelet ene povezave brez rožic.
Rožice izven poti bosta čmrlja z lahkoto spregledala,
saj gre za prestižno tekmo.

Tvoja naloga je, da Gabru pomagaš do zmage.
Če torej poznaš graf $G$, vozlišči $s$ in $t$
ter število rožic na vsaki povezavi, katero pot naj izbere?

\begin{enumerate}[(a)]
\item Prevedi zgornji problem na kak znan problem
in predlagaj znani algoritem, s katerim ga lahko rešiš.

\item Reši problem za graf s slike~\fig{},
kjer oznake na povezavah povedo, koliko rožic se nahaja vzdolž povezave.
\end{enumerate}

\begin{slika}
\pgfslika
\podnaslov{Graf}
\end{slika}
\end{vprasanje}
\begin{odgovor}
\end{odgovor}
\end{naloga}


\begin{naloga}{?}{Izpit OR 4.9.2012}
\begin{vprasanje}
Žabec Rok in žabica Neli živita v mlaki,
na kateri plavajo lokvanjevi listi.
Na enem listu sedi Rok,
na drugi strani mlake pa je list, na katerem počiva Neli.
Rok bi rad po listih priskakljal k Neli.
Z enim skokom lahko premosti razdaljo največ $d$.
Čisto vseeno mu je, koliko skokov naredi
in koliko je skupna preskakana razdalja.
Ali jo lahko obišče?
Lokvanjevi listi so podani kot seznam koordinat v $\R^2$.
Listi so majhni v primerjavi z razdaljami med njimi,
zato jih lahko obravnavaš kot točke.

\begin{enumerate}[(a)]
\item Formuliraj zgornji problem kot problem na grafih in predlagaj algoritem,
s katerim ga lahko rešiš.

\item Reši problem za $d = 3$ in lokvanje na koordinatah
$$
(0, 0), \ (2, 1), \ (4, 1), \ (2, 4), \ (7, 4),
\ (1, 6), \ (5, 6), \ (3, 8), \ (8, 8),
$$
pri čemer Rok sedi na lokvanju $(0, 0)$, Neli pa na $(8, 8)$.
\end{enumerate}
\end{vprasanje}
\begin{odgovor}
\end{odgovor}
\end{naloga}
