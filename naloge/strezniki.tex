\begin{naloga}{Blaž Jelenc}{Izpit OR 25.8.2014}
\begin{vprasanje}
Dana sta računalnika $s$ in $t$ ter omrežje strežnikov
in usmerjenih povezav.
S pomočjo algoritma za maksimalni pretok v grafu
opiši in utemelji splošen algoritem,
ki določi minimalno število strežnikov, ki jih moramo onemogočiti,
da prekinemo komunikacijo od $s$ do $t$,
in ga uporabi na grafu s slike~\fig.

\begin{slika}
\pgfslika
\podnaslov{Graf}
\end{slika}
\end{vprasanje}

\begin{odgovor}
Naj bo $G = (V, E)$ graf, ki predstavlja naše omrežje,
pri čemer velja $s, t \in V$.
Definirajmo omrežje $G' = (V', E')$, kjer je
\begin{align*}
V' &= \{s = v_s, t = u_t\} \cup \set{u_x, v_x}{x \in V \setminus \{s, t\}}
\quad \text{in} \\
E' &= \set{v_x \to u_y}{x \to y \in E}
 \cup \set{u_x \to v_x}{x \in V \setminus \{s, t\}},
\end{align*}
povezave $v_x \to u_y$ ($x \to y \in E$) imajo kapaciteto $\infty$,
povezave $u_x \to v_x$ ($x \in V \setminus \{s, t\}$) pa imajo kapaciteto $1$.
Maksimalni pretok v omrežju ustreza številu strežnikov,
ki jih moramo onemogočiti, da prekinemo komunikacijo od $s$ do $t$,
minimalni prerez pa vsebuje povezave $u_x \to v_x$ za tiste strežnike $x$,
ki jih moramo onemogočiti.

Omrežje za graf s slike~\fig je prikazano na sliki~\fig[strezniki-resitev1],
kjer sta prikazani tudi povečujoči poti,
s katerima pridemo do maksimalnega pretoka $2$ in minimalnega prereza,
prikazanega na sliki~\fig[strezniki-resitev2].
Iz slike je razvidno, da moramo za prekinitev komunikacije od $s$ do $t$
onemogočiti strežnika $d$ in $i$.

\begin{slika}
\pgfslika[strezniki-resitev1]
\podnaslov{Omrežje in povečuječe poti}
\end{slika}
\begin{slika}
\pgfslika[strezniki-resitev2]
\podnaslov{Maksimalni pretok in minimalni prerez}
\end{slika}
\end{odgovor}
\end{naloga}
