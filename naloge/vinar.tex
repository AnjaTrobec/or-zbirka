\begin{naloga}{David Gajser}{Kolokvij OR 31.5.2012}
\begin{vprasanje}
Vinar Janez je pridelal $2\,000$ litrov rumenega muškata,
$10\,000$ litrov laškega rizlinga in $5\,000$ litrov renskega rizlinga.
Njegovi kupci so bara Kocka in Luka ter župnišče Sv.~Martin in občina Duplek.
Vsak od njih je pripravljen kupiti največ določeno količino vina
po fiksni ceni, ne glede na sorto:

\begin{center}
\begin{tabular}{r|cccc}
kupec & Kocka & Luka & župnišče & občina \\ \hline
cena za liter & $1.0$ & $1.1$ & $1.5$ & $1.8$ \\
največja količina v litrih & $15\,000$ & $5\,000$ & $500$ & $1\,000$ \\
\end{tabular}
\end{center}

Janez se je odločil, da bo vsako sorto prodal največ enemu kupcu,
in sicer v maksimalni količini
(če kupec ne kupi vsega vina iste sorte, ga Janez ohrani zase).
Župan pravi, da občina drugega vina kot renskega rizlinga ne bo kupila.
Bar Luka želi rumeni muškat, če bar Kocka dobi laški rizling.
Pri Kocki so se dogovorili, da če občina in župnišče ne kupijo nič,
tudi oni ne bodo kupili ničesar.
Janezova žena pa vztraja,
da če kupec $A$ kupi sorto $s_A$ in kupec $B$ kupi sorto $s_B$,
potem naj sorta $s_C$ ostane doma ali jo kupi kupec $C$ (za neke $A, B, C$).
Kako naj Janez proda vino, da bo čim več zaslužil?

Zapiši problem kot celoštevilski linearni program.
\end{vprasanje}

\begin{odgovor}
Za vsako sorto $i \in I = \{RM, LR, RR\}$
(rumeni muškat, laški rizling, renski rizling)
in kupca $j \in J = \{K, L, Z, O\}$
(bar Kocka, bar Luka, župnišče Sv. Martin, občina Duplek)
bomo uvedli spremenljivko $x_{ij}$,
katere vrednost interpretiramo kot
$$
x_{ij} = \begin{cases}
1, & \text{če naj Janez proda sorto $i$ kupcu $j$, in} \\
0  & \text{sicer.}
\end{cases}
$$
Zapišimo celoštevilski linearni program.
\begin{gather*}
\begin{alignedat}{4}
\max &\  2\,000 x_{RM, K} &&+ 2\,200 x_{RM, L}
    &&+     750 x_{RM, Z} &&+ 1\,800 x_{RM, O} \\
{}+  &\ 10\,000 x_{LR, K} &&+ 5\,500 x_{LR, L}
    &&+     750 x_{LR, Z} &&+ 1\,800 x_{LR, O} \\
{}+  &\  5\,000 x_{RR, K} &&+ 5\,500 x_{RR, L} &&+
            750 x_{RR, Z} &&+ 1\,800 x_{RR, O}
\quad \text{p.p.}
\end{alignedat} \\
\begin{aligned}
\forall i \in I \ \forall j \in J: \ 0 \le x_{ij} &\le 1, \quad x_{ij} \in \Z
\opis{Vsako sorto proda največ enemu kupcu}
\forall i \in I: \sum_{j \in J} x_{ij} &\le 1
\opis{Vsak kupec dobi največ eno sorto}
\forall j \in J: \sum_{i \in I} x_{ij} &\le 1
\opis{Občina ne bo kupila rumenega muškata ali laškega rizlinga}
x_{RM, O} + x_{LR, O} &= 0
\opis{Če Kocka dobi laški rizling, potem Luka dobi rumeni muškat}
x_{LR, K} &\le x_{RM, L}
\opis{Če občina in župnišče ne kupita ničesar, tudi Kocka ne kupi ničesar}
\sum_{i \in I} (x_{i, O} + x_{i, Z} - x_{i, K}) &\ge 0
\opis{Ženin pogoj}
x_{s_A, A} + x_{s_B, B} + \sum_{j \in J \setminus \{C\}} x_{s_C, j} &\le 2
\end{aligned}
\end{gather*}
\end{odgovor}
\end{naloga}
