\begin{naloga}{Batagelj, Kaufman}{\cite[Naloga~10.7]{bk}}
\begin{vprasanje}
Tovarna Mitsuzuki načrtuje,
da bo naslednje leto izdelala $10\,000$ stereo televizorjev.
Odločijo se, da zvočnikov ne bodo izdelovali sami,
ampak jih bodo (po dva za vsak televizor) naročili
pri ponudniku najkvalitetnejših zvočnikov,
ki je predložil naslednji cenik:
\begin{center}
\begin{tabular}{c|c}
število zvočnikov v enem naročilu & cena zvočnika \\
\hline
      1 --  1\,999 & $150 €$ \\
 2\,000 --  4\,999 & $135 €$ \\
 5\,000 --  7\,999 & $125 €$ \\
 8\,000 -- 19\,999 & $120 €$ \\
20\,000 ali več    & $115 €$ \\
\end{tabular}
\end{center}
Poleg tega naročilo pošiljke stane Mitsuzuki $500 €$,
letni stroški skladiščenja vsakega zvočnika pa znašajo $20\%$ njegove cene.
Koliko zvočnikov naj vsakič naročijo
in kakšni so skupni stroški naročil za naslednje leto?

\end{vprasanje}
\begin{odgovor}
\end{odgovor}
\end{naloga}
