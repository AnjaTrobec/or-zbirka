\begin{naloga}{Blaž Jelenc}{Izpit OR 5.6.2014}
\begin{vprasanje}
Dana je množica števil $A = \{a_1, a_2, \dots, a_n\}$.
Iščemo podmnožico $L \subseteq A$,
za katero doseže izraz
$$
\sum_{a_i \in L} (a_i)^2
$$
maksimalno vrednost,
pri čemer morajo biti izpolnjeni pogoji:
\begin{itemize}
\item če $a_1 \in L$, potem $a_2 \not\in L$,
\item če $a_3 \in L$ in $a_5 \in L$, potem $a_7 \in L$,
\item če $|L| \ge 5$, potem $a_4 \not\in L$, ter
\item v $L$ je več pozitivnih kot negativnih števil.
\end{itemize}
Problem formuliraj kot celoštevilski linearni program.
\end{vprasanje}
\begin{odgovor}
\end{odgovor}
\end{naloga}
