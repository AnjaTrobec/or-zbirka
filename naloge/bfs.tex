
\begin{naloga}{Janoš Vidali}{Vaje OR 30.11.2016}
\begin{vprasanje}
Na grafu s slike~\fig izvedi iskanje v širino.
V primerih, ko imaš več ena\-ko\-vred\-nih izbir,
upoštevaj abecedni vrstni red.
Za vsako povezavo določi, ali se nahaja v drevesu iskanja v širino.

\begin{slika}
\pgfslika
\caption{Graf za nalogi~\nal in~\nal[dfs].}
\end{slika}
\end{vprasanje}

\begin{odgovor}

\begin{small}
\begin{algorithmic}
\Require{graf G, vozlišče $s \in V$}
	\For{$v \in V(G)$}:
		\State oznaceni[v] = \False
	\EndFor
	\For{$s \in V(G)$}:
		\If{oznaceni[s] = \False}:
			\State oznaceni[s] = \True
			\State Q = [s]
			\While{Q ni prazna}:
				\State u = odstrani-prvega(Q)
				\For{$(u, v) \in E(G)$}:
					\If{oznaceni[v] = \False}:
						\State oznaceni[v] = \True
						\State vstavi(Q, v)
					\EndIf
				\EndFor
			\EndWhile
		\EndIf
	\EndFor
		
\end{algorithmic}
\end{small}
%
Če sledimo zgornjemu algoritmu in predpostavki, da upoštevamo vrstni red, začnemo v vozlišču A ter je potek sledeč: \\
%
\underline{Predpriprava:}\\
oznaceni = $[\False, \False, \False, \False, \False, \False, \False, \False, \False]$\\
oznaceni[A] = \True\\
oznaceni = $[\True, \False, \False, \False, \False, \False, \False, \False, \False]$\\
Q = [A]\\
\\
\underline{1. korak:}\\
Odstranimo A iz vrste in pregledamo vse njegove sosede.\\
oznaceni[B] = \True\\
oznaceni = $[\True, \True, \False, \False, \False, \False, \False, \False, \False]$\\
Q = [B]\\
oznaceni[E] = \True\\
oznaceni = $[\True, \True, \False, \False, \True, \False, \False, \False, \False]$\\
Q = [B, E]\\
\\
\underline{2. korak:}\\
Odstranimo B iz vrste in pogledamo vse njegove sosede, ki so še vedno neoznačeni.\\
oznaceni[C] = \True\\
oznaceni = $[\True, \True, \True, \False, \True, \False, \False, \False, \False]$\\
Q = [E, C]\\
\\
\underline{3.korak:}\\
Odstranimo E iz vrste.\\
oznaceni[F] = \True\\
oznaceni = $[\True, \True, \True, \False, \True, \True, \False, \False, \False]$\\
Q = [C, F]\\
\\
\underline{4. korak:}\\
Odstranimo C iz vrste. Ker nima sosedov, ki bi bili še vedno neoznačeni, gremo naprej.\\
Q = [F]\\
\\
\underline{5. korak:}\\
Odstranimo F iz vrste.\\
oznaceni[I] = \True\\
oznaceni = $[\True, \True, \True, \False, \True, \True, \False, \False, \True]$\\
Q = [I]\\
\\
\underline{6. korak:}\\
Odstranimo I iz vrste. Ker nima neoznačenih sosedov, gremo naprej.\\
\\
\underline{7. korak:}\\
Ker je na tej točki vrsta prazna, se vrnemo v zanko for, kjer preišče vsa vozlišča, dokler ne pride do enega, ki je še vedno neoznačeno. V našem primeru, ker vozlišča jemljemo po abecednem vrstenm redu, bo to vozlišče D.\\
\\
\underline{8. korak:}\\
oznaceni[D] = \True\\
oznaceni = $[\True, \True, \True, \True, \True, \True, \False, \False, \True]$\\
Q = [D]\\
\\
\underline{9. korak:}\\
Iz vrste odstranimo D.\\
oznaceni[G] = \True\\
oznaceni = $[\True, \True, \True, \True, \True, \True, \True, \False, \True]$\\
Q = [G]\\
oznaceni[H] = \True\\
oznaceni = $[\True, \True, \True, \True, \True, \True, \True, \True, \True]$\\
Q = [G, H]\\
\\
\underline{10. korak:}\\
Odstranimo G iz vrste. Ker nima neoznačenih sosedov, gremo naprej.\\
\\
\underline{11. korak:}\\
Odstranimo H iz vrste. Ker nima neoznačenih sosedov, gremo naprej.\\
\\
\underline{12. korak:}\\
Ker je na tej točki vrsta prazna, se algoritem ustavi.
Vidimo, da so vrednosti za vsa vozlišča nastavljene na \True, zato vemo, da smo obiskali vsa. \\
%
Drevo iskanja v širino: 
%
\begin{slika}
\pgfslika[bfs-resitev]
\podnaslov{Gozd iskanja v širino}
\end{slika}
\\
\\
%
Torej iz drevesa \fig[bfs-resitev] je jasno razvidno, da so drevesne povezave: (A, B), (B, C), (A, E), (E, F), (F, I), (D, G), (D, H).
%

\end{odgovor}
\end{naloga}
