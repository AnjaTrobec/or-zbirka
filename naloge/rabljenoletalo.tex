\begin{naloga}{?}{Izpit OR 15.9.2010}
\begin{vprasanje}
Letalska družba namerava nabaviti že rabljeno letalo, ki stane $170.000 €$.
Ocenjujejo, da bodo z njim imeli,
če je odlično ohranjeno, $1.000.000 €$ dobička,
če je zadovoljivo ohranjeno, $340.000 €$ dobička,
in če je slabo ohranjeno, le $10.000 €$ dobička.
Verjetnosti, da je letalo odlično, zadovoljivo ali slabo ohranjeno,
so zaporedoma $0.2$, $0.3$ in $0.5$.
\begin{enumerate}[(a)]
\item Modeliraj problem v okviru teorije odločanja (stanja, odločitve).
Kakšno odločitev svetuješ vodstvu družbe?

\item Družba lahko naroči oceno letala pri izvedenski firmi,
ki zahteva za svoje poročilo $10.000 €$.
Vodstvo družbe takole ocenjuje pogojne verjetnosti:
\begin{center}
\begin{tabular}{c|ccc}
$P(\text{rezultat poročila} \;|\;\ \text{kakovost letala})$
& odlično & zadovoljivo & slabo \\ \hline
ugodno & 0.9 & 0.6 & 0.1 \\
neugodno & 0.1 & 0.4 & 0.9
\end{tabular}
\end{center}
Kako naj se vodstvo družbe odloči?
\end{enumerate}

\end{vprasanje}
\begin{odgovor}
\end{odgovor}
\end{naloga}
