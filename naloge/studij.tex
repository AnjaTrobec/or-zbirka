\begin{naloga}{Janoš Vidali}{Izpit OR 29.8.2017}
\begin{vprasanje}
Peter zaključuje študij na Fakulteti za alternativno znanost.
Opravil je že vse obvezne predmete,
za pristop k zaključnemu izpitu pa mora opraviti še nekaj izbirnih predmetov.
To bi rad storil čim hitreje.
V tabeli~\tab{} so našteti izbirni predmeti
skupaj s trajanjem (v tednih, od pristopa do uspešnega opravljanja)
in predmeti, h katerim lahko pristopi po uspešnem opravljanju.
K predmetom A, C in F lahko pristopi že takoj,
za pristop k ostalim predmetom pa zadostuje, če je opravljen eden od pogojev.

\begin{enumerate}[(a)]
\item Topološko uredi ustrezni graf in ga nariši.
\item Katere predmete naj Peter opravi,
da bo lahko čim prej pristopil k zaključnemu izpitu?
Koliko časa bo za to potreboval?
Natančno opiši postopek iskanja odgovora.
\end{enumerate}

\begin{tabela}
\makebox[\textwidth][c]{
\begin{tabular}{c|lcc}
Oznaka & Predmet & Trajanje & Zadosten pogoj za \\ \hline
A & Alternativna zgodovina   &  5 & I, J \\
B & Astrološki praktikum     &  7 & D, G \\
C & Diskretna numerologija   &  9 & B    \\
D & Filozofija magije        &  4 & Z    \\
E & Kvantno pravo            &  8 & B, H \\
F & Postmoderna ekonomija    &  4 & E    \\
G & Telepatija in telekineza &  5 & Z    \\
H & Teorija antigravitacije  &  4 & G    \\
I & Teorije zarote           &  5 & H    \\
J & Ufologija II             & 10 & K    \\
K & Uvod v kriptozoologijo   &  6 & Z    \\ \hline
Z & Zaključni izpit          &  / & /
\end{tabular}
}
\podnaslov{Podatki}
\end{tabela}
\end{vprasanje}
\begin{odgovor}
\end{odgovor}
\end{naloga}
