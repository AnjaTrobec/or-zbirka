\begin{naloga}{David Gajser}{Izpit OR 28.8.2013}
\begin{vprasanje}
Za šestimi gorami poteka tekmovanje v plezanju na goro Lep Trikotnik.
Vsak odsek te gore ima prirejeno težavnost od $-4$ do $4$.
Goro lahko ponazorimo s številskim trikotnikom
$$
\begin{array}{ccccccccc}
                &&&& a_{1,1} \\
            &&& a_{2,1} && a_{2,2} \\
        && a_{3,1} && a_{3,2} && a_{3,3} \\
     & \iddots && \vdots && \vdots && \ddots \\
a_{n,1} && a_{n,2} && \cdots && a_{n,n-1} && a_{n,n}
\end{array} ,
$$
kjer so $a_{i,j} \in \{-4, -3, \dots, 3, 4\}$ težavnosti odsekov.
{\em Veljavna pot} v takem trikotniku je pot od vznožja do vrha,
ki gre zmeraj desno navzgor ali levo navzgor
-- t.j., zaporedje $(a_{n-i, j_i})_{i=0}^{n-1}$,
kjer je $1 \le j_i \le n-i$ in $j_i \in \{j_{i-1}-1, j_{i-1}\}$
za vsak $i = 0, 1, \dots, n-1$.
{\em Teža} veljavne poti je vsota težavnosti odsekov,
po katerih poteka pot (torej $\sum_{i=0}^{n-1} a_{n-i, j_i}$).

Na tekmovanje sta se med drugimi prijavila tudi gornika Marin in Mirko.
V okviru svojih sposobnosti želita izbrati kar najtežjo pot do vrha.
Oba zmoreta plezati po odsekih težavnosti $3$ ali manj.

\begin{enumerate}[(a)]
\item Marin ni dovolj izkušen, da bi lahko plezal po odsekih težavnosti $4$.
Opiši, kako lahko najdemo najtežjo veljavno pot za Marina,
oz.~mu povemo, da s svojim znanjem ne more priti do vrha po veljavni poti!

\item Gornik Mirko je izkušenejši od Marina.
Odseke težavnosti $4$ lahko prepleza,
a ne more preplezati dveh zaporednih odsekov te težavnosti.
Opiši, kako lahko najdemo najtežjo veljavno pot za Mirka,
oz.~mu povemo, da s svojim znanjem ne more priti do vrha po veljavni poti!
\end{enumerate}
\end{vprasanje}
\begin{odgovor}
\end{odgovor}
\end{naloga}
