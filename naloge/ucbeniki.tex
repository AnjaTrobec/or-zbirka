\begin{naloga}{Hillier, Lieberman}{\cite[Problem~11.2-2]{hl}}
\begin{vprasanje}
Vodja prodaje pri založniku učbenikov za fakulteto
ima na voljo $6$ trgovskih potnikov,
ki jim želi dodeliti eno od treh regij, v kateri bodo delovali.
V vsaki regiji mora delovati vsaj en trgovski potnik.
Naj bo $p_{ij}$ pričakovana porast v prodaji v regiji $j$,
če bo tam delovalo $i$ trgovskih potnikov:
$$
\begin{array}{c|ccc}
p_{ij} & 1 & 2 & 3 \\
\hline
1 & 35 & 21 & 28 \\
2 & 48 & 42 & 41 \\
3 & 70 & 56 & 63 \\
4 & 89 & 70 & 75 \\
\end{array}
$$
Reši problem s pomočjo dinamičnega programiranja.
\end{vprasanje}

\begin{odgovor}
Naj bo $r_{ij}$ največja pričakovana porast v prodaji,
če prvim $j$ regijam dodelimo $i$ trgovskih potnikov.
Določimo začetne pogoje in rekurzivne enačbe.
\begin{align*}
r_{i1} &= p_{i1} && (1 \le i \le 4) \\
r_{ij} &= \max\set{r_{h,j-1} + p_{i-h,j}}{1 \le h < i}
&& (2 \le i \le 3, \ i \le j \le i+3)
\end{align*}
Vrednosti $r_{ij}$ računamo v leksikografskem vrstnem redu glede na $(j, i)$.
Največji pričakovani porast dobimo kot $r^* = r_{63}$.

Izračunajmo vrednosti $r_{ij}$ ($2 \le i \le 3$, $i \le j \le i+3$).
\begin{alignat*}{2}
r_{22} &= 35 + 21 &&= 56 \\
r_{32} &= \max\{\underline{35+42}, 48+21\} &&= 77 \\
r_{42} &= \max\{\underline{35+56}, 48+42, 70+21\} &&= 91 \\
r_{52} &= \max\{35+70, 48+56, \underline{70+42}, 89+21\} &&= 112 \\
r_{33} &= 56 + 28 &&= 84 \\
r_{43} &= \max\{56+41, \underline{77+28}\} &&= 105 \\
r_{53} &= \max\{\underline{56+63}, 77+41, 91+28\} &&= 119 \\
r_{63} &= \max\{56+75, \underline{77+63}, 91+41, 112+28\} &&= 140
\end{alignat*}
Največji pričakovani porast je torej $r^* = r_{63} = 140$.
Poiščimo še optimalno razporeditev trgovskih potnikov.
\begin{align*}
r_{63} &= r_{32} + p_{33} && \text{$3$ trgovski potniki v tretjo regijo} \\
r_{32} &= r_{11} + p_{22} && \text{$2$ trgovska potnika v drugo regijo} \\
r_{11} &= p_{11}          && \text{$1$ trgovski potnik v prvo regijo}
\end{align*}
Druga optimalna razporeditev je,
da razporedimo $3$ trgovske potnike v prvo regijo,
$2$ v drugo regijo in $1$ v tretjo regijo.
\end{odgovor}
\end{naloga}
