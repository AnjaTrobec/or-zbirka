\begin{naloga}{Bašić, Gajser}{Kolokvij OR 31.5.2012}
\begin{vprasanje}
Samostojna umetnica z nadimkom Stekelce se preživlja z barvanjem na steklo.
Tokrat je dobila naročilo
za izdelavo dveh zelo posebnih pobarvanih steklenih kroglic.
Plačilo je dobila vnaprej, stroške pa bo imela le z nabavo stekla,
saj ima barv in čopičev že dovolj na zalogi.
Če ne uspe izdelati obeh kroglic v dogovorjenem času,
bo morala plačati $800 €$ kazni.
Steklene kroglice lahko naroči v steklarni, in sicer jo vsaka stane $100 €$.
Steklarna ima stroške vsakič, ko zažene proizvodnjo teh kroglic,
zato ji za vsako naročilo (ne glede na število naročenih kroglic)
zaračuna dodatnih $300 €$.
Kljub temu, da mora Stekelce zmeraj plačati vse dobljene kroglice,
je za vsako le $50 \%$ verjetnost, da bo zanjo uporabna.
Neuporabne kroglice lahko vrže kar v smeti,
odvečne uporabne kroglice pa zanjo prav tako niso vredne nič
in jih bo brez stroškov zavrgla.
\begin{enumerate}[(a)]
\item Ker je naročilo časovno omejeno,
ima Stekelce čas za največ dve naročili v steklarni.
Po koliko kroglic naj vsakič naroči, da bodo pričakovani stroški čim manjši?
(Število naročenih kroglic pri drugem naročilu
bo seveda odvisno od števila dobrih kroglic pri prvem naročilu.)

\item Kaj pa,
če steklarna v prvi seriji za naročilo namesto $300 €$ zahteva le $150 €$,
v drugi seriji pa stroški naročila ostanejo $300 €$?
\end{enumerate}
\end{vprasanje}
\begin{odgovor}
\end{odgovor}
\end{naloga}
