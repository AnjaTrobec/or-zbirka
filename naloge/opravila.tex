\begin{naloga}{Janoš Vidali}{Izpit OR 31.1.2017}
\begin{vprasanje}
Imamo $m$ opravil, ki jih želimo razporediti med $n$ strojev.
Vsak stroj lahko hkrati opravlja le eno opravilo,
vsa opravila pa trajajo eno časovno enoto, neodvisno od stroja.
Če stroj $i$ ($1 \le i \le n$) uporabimo za vsaj eno opravilo,
plačamo ceno $c_i$
(cena ostane enaka, če na istem stroju naredimo več opravil).
Skupni stroški ne smejo preseči količine $C$.
Dani sta še množici parov $P$ in $S$,
pri čemer $(j, k) \in P$ pomeni,
da mora biti opravilo $j$ dokončano pred začetkom opravila $k$,
$(j, k) \in S$ pa pomeni, da se opravili $j$ in $k$ ne smeta izvajati hkrati.
Imamo še dodaten pogoj, ki zahteva,
da je lahko v posamezni časovni enoti lahko aktivnih največ $A$ strojev.
Minimizirati želimo čas dokončanja zadnjega stroja.

Zapiši celoštevilski linearni program, ki modelira zgoraj opisani problem.
\end{vprasanje}
\begin{odgovor}
\end{odgovor}
\end{naloga}
