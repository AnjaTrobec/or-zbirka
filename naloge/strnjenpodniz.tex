\begin{naloga}{?}{Vaje OR 6.4.2016}
\begin{vprasanje}
Dan je niz $S = a_1 a_2 \dots a_n$,
kjer so $a_i$ ($1 \le i \le n$) elementi neke končne abecede.
Nizu $a_j a_{j+1} \dots a_k$, kjer je $1 \le j \le k \le n$,
pravimo {\em strnjen podniz} niza $S$.
S pomočjo dinamičnega programiranja napiši algoritem,
ki določi najdaljši palindromski strnjen podniz v $S$.
\end{vprasanje}

\begin{odgovor}
Naj $b_{ij}$ pove,
ali je strnjen podniz $a_i a_{i+1} \dots a_{j-2} a_{j-1}$ palindromski,
torej, ali velja $a_{i+k-1} = a_{j-k}$ za vsak $k$ ($1 \le k \le j-i$).
Ker so vsi podnizi dolžin $0$ in $1$ palindromski,
lahko določimo začetne pogoje in rekurzivno enačbo.
\begin{align*}
b_{i,i} &= \top && (1 \le i \le n+1) \\
b_{i,i+1} &= \top && (1 \le i \le n) \\
b_{i,j} &= b_{i+1,j-1} \land (a_i = a_{j-1}) && (1 \le i < j-1 \le n)
\end{align*}
Vrednosti $b_{ij}$ računamo v leksikografskem vrstnem redu
glede na $(j-i, i)$.
Dolžino $p^*$ najdaljšega palindromskega strnjenega podniza
dobimo kot največjo razliko indeksov,
za katere je $b_{ij}$ resničen, torej
$$
p^* = \max\set{j-i}{1 \le i \le j \le n+1 \ \land \ b_{ij}} .
$$
Hkrati lahko seveda dobimo tudi začetek in konec
najdaljšega palindromskega strnjenega podniza.

Zapišimo algoritem,
ki vrne dolžino najdaljšega palindromskega strnjenega podniza
$a_i a_{i+1} \dots a_{j-2} a_{j-1}$ skupaj z indeksoma $i$ in $j$.
Vrednost $p^*$ ter pripadajoča indeksa bomo računali sproti.
\begin{small}
\begin{algorithmic}
\Function{Palindrom}{$(a_i)_{i=1}^n$}
    \If{$n = 0$} \hfill če je niz prazen, imamo podniz dolžine $0$
        \State \Return $(0, 1, 1)$
    \EndIf
    \State $(p^*, i^*, j^*) \gets (1, n, n+1)$ \hfill podniz dolžine $1$
    \For{$i = 1, \dots, n$} \hfill začetni pogoji
        \State $b_{ii} \gets \True$
        \State $b_{i,i+1} \gets \True$
    \EndFor
    \State $b_{n+1,n+1} \gets \True$
    \For{$h = 2, \dots, n$} \hfill rekurzivna enačba
        \For{$i = 1, \dots, n-h+1$}
            \State $b_{i,i+h} \gets b_{i+1,i+h-1} \land (a_i = a_{i+h-1})$
            \If{$b_{i,i+h}$} \hfill če najdemo palindrom, posodobimo izhod
                \State $(p^*, i^*, j^*) \gets (h, i, i+h)$
            \EndIf
        \EndFor
    \EndFor
    \State \Return $(p^*, i^*, j^*)$ \hfill vrnemo dolžino in meje
\EndFunction
\end{algorithmic}
\end{small}
V prvi zanki {\bf for} naredimo $n$ korakov,
v dvojni zanki pa
$$
\sum_{h=2}^n (n-h+1) = \sum_{j=1}^{n-1} j = {n(n-1) \over 2} = O(n^2)
$$
korakov.
Časovna zahtevnost algoritma je torej $O(n^2)$.

Problem je mogoče reševati tudi v linearnem času z
\href{https://en.wikipedia.org/wiki/Longest_palindromic_substring#%
Manacher's_algorithm}{Manacherjevim algoritmom}.
\end{odgovor}
\end{naloga}
