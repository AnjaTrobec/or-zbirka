\begin{naloga}{?}{Kolokvij OR 26.1.2010}
\begin{vprasanje}
Jatifi Bajrami se ukvarja s preprodajo pomaranč.
Naslednja sezona je razdeljena na $n$ mesecev.
Za vsak mesec so znana predvidena povpraševanja $r_1, \dots, r_n$.
Nabavni stroški v mesecu $i$ so sestavljeni iz fiksnih stroškov $K_i$
ter stroškov za nabavo sadja.
Na začetku meseca $i$ stane škatla pomaranč $c_i$.
Če Jatifi takrat kupi $m$ škatel,
je za nabavo sadja torej moral odšteti $mc_i$ denarnih enot.
Če se Jatifi zaradi nižje cene v nekem mesecu
odloči nakupiti vnaprej za več mesecev,
mora plačati skladiščenje za škatle pomaranč za preostale mesece,
pri čemer je cena skladiščenja za škatlo v mesecu $i$ enaka $h_i$.
Jatifi kupuje izključno, ko ve, da so mu zaloge popolnoma pošle,
in od tega ne odstopa, saj se mu zdi škoda,
da bi v skladišču hranil stare pomaranče.
Pomaranče prodaja ves čas po isti ceni.
Jatifi bi rad maksimiziral dobiček in zadovoljil celotno povpraševanje.

\begin{enumerate}[(a)]
\item Predlagaj rešitev problema s pomočjo dinamičnega programiranja.
Pri tem so vhodni podatki $n$ mesecev, $r = (r_i)_{i=1}^n$,
$K = (K_i)_{i=1}^n$, $c = (c_i)_{i=1}^n$ ter $h = (h_i)_{i=1}^n$.
Kakšno časovno zahtevnost ima tvoj algoritem?

\item Naj bo $n = 4$, $r = (20, 40, 15, 10)$, $K = (2, 3, 4, 2)$,
$c = (3, 4, 3, 4)$ ter $h = (1, 1, 1, 2)$.
Poišči optimalno strategijo kupovanja pomaranč.
\end{enumerate}
\end{vprasanje}
\begin{odgovor}
\end{odgovor}
\end{naloga}
