\begin{naloga}{?}{Izpit OR 24.5.2016}
\begin{vprasanje}
Dan je seznam pozitivnih celih števil $S = [a_1, a_2, \dots, a_n]$,
ki ga interpretiramo na naslednji način.
Element $a_i$ pomeni,
da se lahko iz $i$-te pozicije v seznamu
premaknemo na pozicije $a_{i+1}, a_{i+2}, \dots, a_{i+a_i}$.
Naj bo $f(S)$ minimalno število korakov,
ki so potrebni, da se iz elementa $a_1$ premaknemo v element $a_n$.
Na primer, če je $S = [1, 3, 5, 8, 9, 2, 6, 7, 6, 8, 9]$,
potem je $f(S) = 3$,
saj lahko opravimo skoke $a_1 = 1 \to a_2 = 3 \to a_4 = 8 \to a_{11} = 9$.

\begin{enumerate}[(a)]
\item Napiši rekurzivne enačbe za računanje funkcije $f(S)$.

\item S pomočjo dinamičnega programiranja napiši algoritem,
ki rešuje dani problem.
Kakšna je njegova časovna zahtevnost?
\end{enumerate}
\end{vprasanje}
\begin{odgovor}

\begin{enumerate}[(a)]

\item Definirajmo usmerjen graf $G_a$,
ki ima za vozlišča števila $1, 2, \dots n$
in povezavo $i \rightarrow j$ natanko tedaj, ko velja $i < j \leq i + a_i$.
Skratka, vsakemu členu (indeksu) zaporedja dodelimo vozlišče
in ga povežemo s tistimi, ki so iz njega dosegljiva v enem koraku.

Problem smo prevedli na osnovno nalogo dinamičnega programiranja, saj moramo poiskati najkrajšo pot med začetno in končno točko v 
usmerjenem, acikličnem grafu $G_a$.

Naj bo $d_i$ najkrajša pot od vozlišča $i$ do vozlišča $j$ v grafu $G_a$. 
Rekurzivno zvezo potem lahko zapišemo kot
\begin{align*}
d_n &= 0 \\
d_i &= 1 + \min\set{d_k}{i+1 \le k \le \min(i+a_i, n)}
\qquad (1 \le i < n)
\end{align*}
Da bomo imeli vse prejšnje probleme rešene pred trenutnim, bomo vrednosti $d_i$ računali padajoče po $i$ za $1 \leq i \leq n$, 
saj začnemo v zadnjem vozlišču $n$ in se nato pomikamo nazaj.
Po razrešitvi vseh problemov optimum $d^*$ dobimo kot vrednost $d_1$,
torej $f(S) = d^* = d_1$, kar sledi iz definicije vrednosti $d_i$.

\item Algoritem bo sledil rekurzivni zvezi.

\begin{small}
\begin{algorithmic}
\Function{Skoki}{$(a_i)_{i=1}^n$}
	\State $d_n \gets 0$
	\For{$i = n - 1 , n - 2, \dots, 1$}
		\State $d_i, p_i \gets
            \min\set{(1+d_k, k)}{i+1 \le k \le \min(i+a_i, n)}$
	\EndFor
	\State {\sl pot} $\gets$ prazen seznam
	\State $i \gets 1$
	\While{$i < n$}
		\State {\sl pot}.$\append(i)$
		\State $i \gets p_i$
	\EndWhile
    \State {\sl pot}.$\append(n)$
	\State \Return ($d_1$, {\sl pot})
\EndFunction
\end{algorithmic}
\end{small}

Algoritem ima časovno zahtevnost enako $O(|V| + |E|)$,
kjer sta $V$ in $E$ množici vozlišč in povezav v grafu $G_a$,
saj za vsako vozlišče pregledamo njegove vhod\-ne povezave.
Število vozlišč v našem grafu je $n$, število povezav pa je odvisno od elementov v vhodnem seznamu $S$.
V najslabšem primeru lahko za vhod velja $a_i \ge n-i$ za vsak $i$.
Časovna zahtevnost našega algoritma bo potem $O(n^2)$,
saj je število vozlišč enako $n$,
število povezav pa lahko izračunamo kot
$$
(n - 1) + (n - 2) + \dots + 1 + 0 = \frac{n (n - 1)}{2}
= \frac{n ^2 - n}{2} = O(n^2) .
$$

Poglejmo si delovanje algoritma na primeru $S = [1,3,5,1,3,10,6,13]$.
Ustrezni graf $G_a$ je prikazan na sliki~\fig.
V tabeli~\tab je prikazan potek izvajanja algoritma.
Algoritem nam vrne optimalno pot $[1, 2, 3, 8]$, torej je $f(S) = 3$.
\end{enumerate}

\begin{slika}
\pgfslika
\podnaslov[\res{}(b)]{Predstavitev skokov z grafom}
\end{slika}

\begin{tabela}
\begin{tabular}{cc|cc}
$i$ & $a_i$ & $d_i$ & $p_i$ \\ \hline
$8$ & $13$ & $0$       \\
$7$ & $6$  & $1$ & $8$ \\
$6$ & $10$ & $1$ & $8$ \\
$5$ & $3$  & $1$ & $8$ \\
$4$ & $1$  & $2$ & $5$ \\
$3$ & $5$  & $1$ & $8$ \\
$2$ & $3$  & $2$ & $3$ \\
$1$ & $1$  & $3$ & $2$
\end{tabular}
\podnaslov[\res{}(b)]{Potek izvajanja algoritma}
\end{tabela}

\end{odgovor}
\end{naloga}
