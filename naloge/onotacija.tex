\begin{naloga}{?}{Kolokvij OR 25.11.2010}
\begin{vprasanje}
\begin{enumerate}[(a)]
\item Dokaži, da za poljubni konstanti $a, b \in \R$, kjer je $b > 0$,
velja ${(n + a)}^b = O(n^b)$.

\item Naj bo $f$ naraščajoča funkcija.
Ali velja $g(n) = O(f(g(n)))$?

\item Dokaži,
da če $T$ zadošča pogoju $T(n) = 2T(\lceil n/2 \rceil) - 13$ za $n \ge 2$,
potem je $T(n) = O(n)$.
\end{enumerate}
\end{vprasanje}

\begin{odgovor}
\begin{enumerate}[(a)]
\item Dokažimo, da velja $(n + a)^b = \Theta(n^b)$.
Dokazati moramo torej,
da obstajata konstanti $c, C \in \R_+$ ter konstanta $n_0 \in \mathbb{N}$,
tako da velja
$$
c \cdot n^b \leq (n + a)^b \leq C \cdot n^b
$$
za vsak $n > n_0$.
Ker je $b > 0$,
funkcija $x \mapsto x^{\frac{1}{b}}$ monotono narašča pri $x>0$,
torej jo lahko brez škode za splošnost uporabimo na zgornji neenakosti.
Tako dobimo
$$
c^{1 \over b} \cdot n \leq n + a \leq C^{1 \over b} \cdot n.
$$
Naj bosta $c = \frac{1}{2^b}$ in $C = 2^b$.
Dobimo pogoja
$$
\frac{n}{2} \leq n + a \leq 2n .
$$
Ker moramo paziti, da so vsi členi znotraj neenakosti pozitivni,
dodamo še pogoj $n + a > 0$.
Če izberemo $n_0 = \max(-2a, a)$,
bodo ti pogoji veljali pri vseh $n > n_0$.

\item Po premisleku hitro ugotovimo, da za poljubno funkcijo $f$ to ne velja,
saj lahko ta konkretno spremeni hitrost naraščanja funkcije $g$.

Za protiprimer vzemimo $g(n) = n$ in $f(n) = \log(n)$
ter poglejmo $\lim_{n \rightarrow \infty} (c \cdot \log(n) / n)$
pri poljubni izbiri $c > 0$.
Z razširitvijo funkcij na $\R$ in L'Hôpitalovim pravilom ugotovimo,
da bo limita enaka $0$ ne glede na izbrano konstanto $c$.
Tako $n$ ne moremo asimptotično omejiti z $O(\log(n))$.

\item Rešitev naloge je klasična uporaba krovnega izreka,
a jo enostavno rešimo tudi brez tega.

Ob upoštevanju tega, da velja $T(n) < 2T(\lfloor (n+1)/2 \rfloor)$,
razvijajmo rekurzivno zvezo do $k$-tega koraka.
$$
T(n) < 2T\left(\left\lfloor{n+1 \over 2}\right\rfloor\right)
< 4T\left(\left\lfloor{n+3 \over 4}\right\rfloor\right)
< \dots < 2^k T\left(\left\lfloor{n + 2^k - 1 \over 2^k}\right\rfloor\right)
$$
Ker velja $T(1) = O(1)$, naj bo $k$ tak,
da bo veljalo $(n-1) / 2^k < 1$, torej $k = \log_2(n)$.
Tako dobimo
$$
T(n) = 2^{\log_2(n)} \cdot O(1) = n \cdot O(1) = O(n).
$$
\end{enumerate}
\end{odgovor}
\end{naloga}
