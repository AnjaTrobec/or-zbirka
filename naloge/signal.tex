\begin{naloga}{Janoš Vidali}{Kolokvij OR 19.4.2019}
\begin{vprasanje}
Gradimo avtocesto skozi puščavo in želimo zagovoviti,
da bo v celoti pokrita z mobilnim signalom.
Cesta je ravna in dolga $n$ milj ($n \in \Z$),
na vsako miljo pa imamo možnost postaviti
bodisi manjšo bazno postajo z dosegom $1$ miljo,
bodisi večjo bazno postajo z dosegom $2$ milji.
Cena postavitve manjše bazne postaje na $i$-ti milji
je podana s parametrom $a_i$,
cena postavitve večje bazne postaje pa s parametrom $b_i$ ($0 \le i \le n$).
Predpostaviš lahko, da so vse cene pozitivne.
Pri obstoječi infrastrukturi je pokrita le začetna točka avtoceste
(tj., milja $0$).
Poiskati želimo torej čim cenejšo postavitev baznih postaj,
da je vsaka točka avtoceste pokrita s signalom.

\medskip
\noindent
{\bf Primer:} če postavimo manjši postaji na milji $0$ in $3$,
potem je interval $(1, 2)$ nepokrit.
Če pa npr.~postajo namesto na milji $0$ postavimo na milji $1$,
smo tako v celoti pokrili interval $[0, 4]$.
Enak učinek bi dosegli,
če bi namesto dveh manjših postaj postavili večjo postajo na miljo $2$.

\begin{enumerate}[(a)]
\item Zapiši rekurzivne enačbe za reševanje danega problema.
Razloži, kaj pred\-stav\-lja\-jo spremenljivke,
v kakšnem vrstnem redu jih računamo,
ter kako dobimo optimalno rešitev.
\namig{uporabi spremenljivko za pokritost do vsake milje
s predhodnimi postajami,
pri čemer ni potrebno, da postaja stoji na zadnji milji.}

\item Oceni časovno zahtevnost algoritma, ki sledi iz zgoraj zapisanih enačb.

\item S svojim algoritmom poišči optimalno postavitev postaj za $n = 10$ ter
\begin{align*}
(a_i)_{i=0}^{10} &= (4,  6,  1, 10, 14, 21, 15,  6,  9,  3,  2)
\quad \text{in} \\
(b_i)_{i=0}^{10} &= (10, 12, 3, 18, 24, 25, 20, 11, 16,  7,  4) .
\end{align*}
\end{enumerate}
\end{vprasanje}

\begin{odgovor}
\begin{enumerate}[(a)]
\item Naj bo $p_i$ najmanjša cena postavitve baznih postaj
do $i$-te milje tako,
da se v celoti pokrije interval $[0, i]$.
Določimo začetne pogoje in rekurzivne enačbe.
\begin{alignat*}{3}
&& b_{-1} &= \infty \\
p_{-3} = p_{-2} &=& p_{-1} = p_0 &= 0 \\
p_i &=&{} \min\{b_{i-2} &+ \min\{p_{i-4},\ p_{i-3}\}, \\
    & &         b_{i-1} &+ \min\{p_{i-3},\ p_{i-2}\}, \\
    & &         b_i &+ \min\{p_{i-2},\ p_{i-1}\}, \\
    & &         a_{i-1} &+ p_{i-2}, \\
    & &         a_i &+ p_{i-1}\} & (1 \le i \le n)
\end{alignat*}
Vrednosti $p_i$ računamo naraščajoče po indeksu $i$ ($0 \le i \le n$).
Najmanjšo ceno postavitve baznih postaj dobimo s $p^* = p_n$.

\item Za izračun vrednosti $p_i$ za posamezen $i$
potrebujemo konstantno mnogo časa.
Ker ta izračun opravimo $(n+1)$-krat,
je torej časovna zahtevnost ustreznega algoritma $O(n)$.

\item Izračunajmo vrednosti $p_i$ ($1 \le i \le 10$).
\begin{alignat*}{2}
p_1    &= \min\{\infty+0, 10+0, 12+0, \underline{4+0}, 6+0\}    &&= 4  \\
p_2    &= \min\{10+0, 12+0, \underline{3+0}, 6+0, 1+4\}         &&= 3  \\
p_3    &= \min\{12+0, \underline{3+0}, 18+3, 1+4, 10+3\}        &&= 3  \\
p_4    &= \min\{\underline{3+0}, 18+3, 24+3, 10+3, 14+3\}       &&= 3  \\
p_5    &= \min\{18+3, 24+3, 25+3, \underline{14+3}, 21+3\}      &&= 17 \\
p_6    &= \min\{24+3, 25+3, \underline{20+3}, 21+3, 15+17\}     &&= 23 \\
p_7    &= \min\{25+3, \underline{20+3}, 11+17, 15+17, 6+23\}    &&= 23 \\
p_8    &= \min\{\underline{20+3}, 11+17, 16+23, 6+23, 9+23\}    &&= 23 \\
p_9    &= \min\{11+17, 16+23, 7+23, 9+23, \underline{3+23}\}    &&= 26 \\
p_{10} &= \min\{16+23, 7+23, 4+23, \underline{3+23}, 2+26\}     &&= 26
\end{alignat*}
Cena postavitve baznih postaj je torej $p^* = p_{10} = 26$.
Poglejmo, kam moramo postaviti bazne postaje.
\begin{align*}
p_{10} &= a_9 + p_8 && \text{manjša postaja na $9$} \\
p_8    &= b_6 + p_4 && \text{večja postaja na $6$}  \\
p_4    &= b_2 + p_0 && \text{večja postaja na $2$}
\end{align*}
\end{enumerate}
\end{odgovor}
\end{naloga}
