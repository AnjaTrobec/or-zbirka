\begin{naloga}{David Gajser}{Izpit OR 19.5.2015}
\begin{vprasanje}
Podjetje Nijke proizvaja moške copate.
Po svetu ima več tovarn,
vsaka je na stopnji razvoja $1$, $2$, $3$ ali $4$.
Koliko dobička prinese
tovarna na posamezni stopnji razvoja v posamezni državi
ter koliko tovarn na posamezni stopnji razvoja ima trenutno vsaka država,
je razvidno iz tabele~\tab.

V naslednjem letu lahko nadgradimo do $15$ tovarn,
v vsaki državi največ $3$.
Vsako tovarno lahko nadgradimo največ za eno stopnjo.
Dodatno moramo upoštevati še naslednje omejitve.
\begin{itemize}
\item Slovenija mora imeti več tovarn stopnje $4$ kot Hrvaška
in več tovarn stopnje $3$ kot Italija.
\item Število nadgradenj tovarn v EU (države nad črto tabeli~\tab)
mora biti vsaj tolikšno kot število nadgradenj tovarn izven EU.
\item Če Azerbajdžan nadgradi svojo tovarno
in Armenija ne nadgradi tovarne stopnje $3$,
potem mora Armenija nadgraditi vse tri tovarne stopnje $1$.
\item Iran mora imeti vsaj toliko tovarn stopnje $4$
kot vse države EU skupaj.
\end{itemize}

Napiši celoštevilski linearni program,
ki upošteva opisane omejitve in pove,
kje in koliko tovarn moramo nadgraditi, da bomo imeli kar največji dobiček.

\begin{tabela}
\begin{tabular}{c|cccc|cccc}
& \multicolumn{4}{|c|}{dobiček}
& \multicolumn{4}{c}{število tovarn} \\
država & st.~$1$ & st.~$2$ & st.~$3$ & st.~$4$
& st.~$1$ & st.~$2$ & st.~$3$ & st.~$4$ \\ \hline
Avstrija     & 3 &  4 &  7 & 10 &  0 & 0 & 0 &  5 \\
Slovenija    & 3 &  3 &  5 &  8 &  0 & 1 & 2 &  1 \\
Italija      & 4 &  4 &  6 &  8 &  1 & 1 & 1 &  1 \\
Madžarska    & 5 &  7 &  8 & 10 &  0 & 1 & 2 &  1 \\
Hrvaška      & 3 &  4 &  5 &  7 &  1 & 2 & 1 &  0 \\ \hline
Iran         & 6 & 10 & 10 & 10 &  3 & 0 & 4 & 10 \\
Irak         & 2 &  4 &  4 &  6 &  5 & 3 & 0 &  0 \\
Afganistan   & 6 &  6 &  5 &  2 & 20 & 0 & 0 &  0 \\
Turkmenistan & 6 &  7 &  7 &  6 & 10 & 5 & 5 &  0 \\
Pakistan     & 6 &  7 &  7 &  7 & 10 & 7 & 2 &  0 \\
Armenija     & 6 &  7 &  9 & 10 &  3 & 0 & 1 &  0 \\
Azerbajdžan  & 6 &  7 &  9 &  8 &  0 & 1 & 0 &  0 \\
\end{tabular}
\podnaslov{Podatki}
\end{tabela}
\end{vprasanje}

\begin{odgovor}
Naj bo $d_{ij}$ dobiček, ki ga tovarna na stopnji $j$ prinese v državi $i$,
ter $n_{ij}$ trenutno število tovarn na stopnji $j$ v državi $i$.
Definirajmo še množice
\begin{align*}
\text{EU} &= \{\text{AT}, \text{SI}, \text{IT}, \text{HU}, \text{HR}\}, \\
\overline{\text{EU}} &= \{\text{IR}, \text{IQ}, \text{AF}, \text{TM},
                          \text{PK}, \text{AM}, \text{AZ}\} \quad \text{in} \\
D &= \text{EU} \cup \overline{\text{EU}}
\end{align*}
(tj., države označujemo s kodami ISO 3166-1 alpha-2).
Za državo $i \in D$ in stopnjo razvoja $j$ ($1 \le j \le 3$)
bomo uvedli spremenljivko $x_{ij}$,
katere vrednost bo enaka številu tovarn na stopnji $j$ v državi $i$,
ki jih nadgradimo do stopnje $j+1$.

Zapišimo celoštevilski linearni program.
\begin{alignat*}{3}
&& \max & \sum_{i \in D} \sum_{j=1}^3 (d_{i,j+1} - d_{ij}) x_{ij}
\quad \text{p.p.} \\
\forall i \in D \ \forall j \in \{1, 2, 3\} : &\ & 0 \le x_{ij} &\le n_{ij},
\quad x_{ij} \in \Z
\opis{Nadgradimo največ $15$ tovarn}
&& \sum_{i \in D} \sum_{j=1}^3 x_{ij} &\le 15
\opis{V vsaki državi nadgradimo največ $3$ tovarne}
\forall i \in D : &\ & \sum_{j=1}^3 x_{ij} &\le 3
\opis{Slovenija ima več tovarn stopnje $4$ kot Hrvaška}
&& n_{\text{SI},4} + x_{\text{SI},3}
&\ge n_{\text{HR},4} + x_{\text{HR},3} + 1
\opis{Slovenija ima več tovarn stopnje $3$ kot Italija}
&& n_{\text{SI},3} + x_{\text{SI},2} - x_{\text{SI},3}
&\ge n_{\text{IT},3} + x_{\text{IT},2} - x_{\text{IT},3} + 1
\opis{Nadgradenj v EU vsaj toliko kot izven EU}
&& \sum_{i \in \text{EU}} \sum_{j=1}^3 x_{ij}
&\ge \sum_{i \in \overline{\text{EU}}} \sum_{j=1}^3 x_{ij}
\opis{Pogoj za Azerbajdžan in Armenijo}
&& x_{\text{AM},1} &\ge n_{\text{AM},1} (x_{\text{AZ},2} - x_{\text{AM},3})
\opis{Iran ima vsaj toliko tovarn stopnje $4$ kot EU}
&& \sum_{i \in \text{EU}} (n_{i,4} + x_{i,3})
&\le n_{\text{IR},4} + x_{\text{IR},3}
\end{alignat*}
\end{odgovor}
\end{naloga}
