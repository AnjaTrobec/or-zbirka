\begin{naloga}{?}{Izpit OR 19.5.2015}
\begin{vprasanje}
Podjetje Nijke proizvaja moške copate.
Po svetu ima več tovarn,
vsaka je na stopnji razvoja $1$, $2$, $3$ ali $4$.
Koliko dobička prinese
tovarna na posamezni stopnji razvoja v posamezni državi
ter koliko tovarn na posamezni stopnji razvoja ima trenutno vsaka država,
je razvidno iz tabele~\tab{}.

V naslednjem letu lahko nadgradimo do $15$ tovarn,
v vsaki državi največ $3$.
Vsako tovarno lahko nadgradimo največ za eno stopnjo.
Dodatno moramo upoštevati še naslednje omejitve.
\begin{itemize}
\item Slovenija mora imeti več tovarn stopnje $4$ kot Hrvaška
in več tovarn stopnje $3$ kot Italija.
\item Število nadgradenj tovarn v EU (države nad črto tabeli~\tab{})
mora biti vsaj tolikšno kot število nadgradenj tovarn izven EU.
\item Če Azerbajdžan nadgradi svojo tovarno
in Armenija ne nadgradi tovarne stopnje $3$,
potem mora Armenija nadgraditi vse tri tovarne stopnje $1$.
\item Iran mora imeti vsaj toliko tovarn stopnje $4$
kot vse države EU skupaj.
\end{itemize}

Napiši celoštevilski linearni program,
ki upošteva opisane omejitve in pove,
kje in koliko tovarn moramo nadgraditi, da bomo imeli kar največji dobiček.

\begin{tabela}
\begin{tabular}{c|cccc|cccc}
& \multicolumn{4}{|c|}{dobiček}
& \multicolumn{4}{c}{število tovarn} \\
država & st.~$1$ & st.~$2$ & st.~$3$ & st.~$4$
& st.~$1$ & st.~$2$ & st.~$3$ & st.~$4$ \\ \hline
Avstrija     & 3 &  4 &  7 & 10 &  0 & 0 & 0 &  5 \\
Slovenija    & 3 &  3 &  5 &  8 &  0 & 1 & 2 &  1 \\
Italija      & 4 &  4 &  6 &  8 &  1 & 1 & 1 &  1 \\
Madžarska    & 5 &  7 &  8 & 10 &  0 & 1 & 2 &  1 \\
Hrvaška      & 3 &  4 &  5 &  7 &  1 & 2 & 1 &  0 \\ \hline
Iran         & 6 & 10 & 10 & 10 &  3 & 0 & 4 & 10 \\
Irak         & 2 &  4 &  4 &  6 &  5 & 3 & 0 &  0 \\
Afganistan   & 6 &  6 &  5 &  2 & 20 & 0 & 0 &  0 \\
Turkmenistan & 6 &  7 &  7 &  6 & 10 & 5 & 5 &  0 \\
Pakistan     & 6 &  7 &  7 &  7 & 10 & 7 & 2 &  0 \\
Armenija     & 6 &  7 &  9 & 10 &  3 & 0 & 1 &  0 \\
Azerbajdžan  & 6 &  7 &  9 &  8 &  0 & 1 & 0 &  0 \\
\end{tabular}
\podnaslov{Podatki}
\end{tabela}
\end{vprasanje}
\begin{odgovor}
\end{odgovor}
\end{naloga}
