\begin{naloga}{Blaž Jelenc}{Izpit OR 5.6.2014}
\begin{vprasanje}
Pacient ima na voljo operacijo.
Brez operacije bo živel natanko $3$ mesece.
Z uspeš\-no opravljeno operacijo bo živel natanko $12$ mesecev.
Operacija je neuspešna z verjetnostjo $0.3$
(v tem primeru pacient dočaka $0$ mesecev).
Cilj pacienta je maksimiranje pričakovane življenjske dobe.
\begin{enumerate}[(a)]
\item Ali naj pacient sprejme operacijo?
\item Pacient lahko opravi predhodni test,
ki z zanesljivostjo $0.9$ napove uspeš\-nost operacije,
vendar z verjetnostjo $0.005$ pacient zaradi komplikacij med testom umre.
Ali naj pacient opravi test?
\end{enumerate}
Nariši odločitveno drevo
in odločitve sprejmi na podlagi izračunanih ve\-rjet\-no\-sti!
\end{vprasanje}

\begin{odgovor}
\begin{enumerate}[(a)]
\item Ob opravljeni operaciji
je pričakovana življenjska doba (v mesecih) enaka
$$
0.7 \cdot 12 + 0.3 \cdot 0 = 8.4 .
$$
Ker je to več od $3$ mesecev brez operacije, se zanjo pacient odloči.

\item Zanesljivost $0.9$ razumemo tako,
da test z verjetnostjo $0.9$ pravilno napove vsak izid operacije,
torej
\begin{align*}
P(\text{test napove uspeh} \;|\; \text{operacija je uspešna}) &= 0.9
\quad \text{in} \\
P(\text{test napove uspeh} \;|\; \text{operacija je neuspešna}) &= 0.1 .
\end{align*}
Izračunajmo še verjetnosti napovedi testa
ter uspešnosti operacije ob vsaki napovedi.
\begin{small}
\begin{align*}
P(\text{test napove uspeh}) &= 0.7 \cdot 0.9 + 0.3 \cdot 0.1 = 0.66 \\
P(\text{test napove neuspeh}) &= 0.7 \cdot 0.1 + 0.3 \cdot 0.9 = 0.34 \\
P(\text{operacija je uspešna} \;|\; \text{test napove uspeh})
&= {0.7 \cdot 0.9 \over 0.66} = {21 \over 22} \approx 0.9545 \\
P(\text{operacija je neuspešna} \;|\; \text{test napove uspeh})
&= {0.3 \cdot 0.1 \over 0.66} = {1 \over 22} \approx 0.0455 \\
P(\text{operacija je uspešna} \;|\; \text{test napove neuspeh})
&= {0.7 \cdot 0.1 \over 0.34} = {7 \over 34} \approx 0.2059 \\
P(\text{operacija je uspešna} \;|\; \text{test napove neuspeh})
&= {0.3 \cdot 0.9 \over 0.34} = {27 \over 34} \approx 0.7941
\end{align*}
\end{small}
Ob upoštevanju, da se bo test končal brez komplikacij z verjetnostjo $0.995$,
sta verjetnosti pozitivnega in negativnega izida testa brez komplikacij
enaki $0.6567$ oziroma $0.3383$.

S pomočjo zgoraj izračunanih verjetnosti
lahko narišemo odločitveno drevo s slike~\fig.
Izračunamo lahko, da je ob opravljanju testa
pričakovana življenjska doba $8.537$ mesecev,
kar je več kot $8.4$ mesecev brez testa,
zato se odločimo za test.
Če test napove uspeh, se odločimo za operacijo, sicer pa ne.
\end{enumerate}

\begin{slika}
\makebox[\textwidth][c]{
\pgfslika
}
\podnaslov{Odločitveno drevo}
\end{slika}
\end{odgovor}
\end{naloga}
