\begin{naloga}{Bašić, Gajser}{Kolokvij OR 5.4.2012}
\begin{vprasanje}
Iz zajetja $O$ želimo preko treh čistilnih postaj
(vsako predstavljajo $4$ vozlišča)
napeljati vodo do tovarne $T$.
Opravka imamo z vodovodnim omrežjem,
pred\-stav\-lje\-nim z grafom na sliki~\fig.
Na povezavah so kapacitete cevi (v litrih na sekundo),
cevi z oznakami $a_1$, $a_2$ in $a_3$ pa še niso zgrajene.

\begin{enumerate}[(a)]
\item Koliko vode lahko po danem omrežju dobi tovarna na sekundo?

\item Trenutni dotok vode v tovarno za potrebe proizvodnje ni dovolj velik,
zato razmišljajo o izgradnji novih cevi.
Možne lokacije novih cevi so $a_1$, $a_2$ in $a_3$.
Uprava tovarne se je odločila,
da bodo s postavitvijo novih cevi na nekaterih možnih lokacijah
povečali pretok kar največ, kot je mogoče.
Cev kapacitete $k$ stane $k \cdot 100 €$.
Za inštalacijo ene cevi morajo plačati dodatnih $10\,000 €$.
Koliko najmanj jih bo stal projekt?
\end{enumerate}

\begin{slika}
\makebox[\textwidth][c]{
\pgfslika
}
\podnaslov{Omrežje}
\end{slika}
\end{vprasanje}
\begin{odgovor}
\end{odgovor}
\end{naloga}
