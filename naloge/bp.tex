\begin{naloga}{?}{Izpit OR 28.6.2010}
\begin{vprasanje}
Na naftni ploščadi podjetja BP
je prišlo do nekontroliranih izpustov nafte iz vrtine v zalivu.
Po vseh mogočih podvigih, da bi zamašili vrtino,
se je ob\-upa\-no vodstvo podjetja pod pritiski predsednika bližnje države
začelo dobivati z ruskimi strokovnjaki
za mašenje vrtin s pomočjo jedrskih eksplozij pod vodstvom
prof.~dr.~Mikhaila Razturoviča Totalkova.
Vodstvo BP ocenjuje, da bo ekipa dr.~Totalkova
z verjetnostjo $p = 3/7$ uspešno zamašila vrtino.
Tako bo podjetje imelo sicer samo $2.8$ milijarde USD dobička,
sicer pa bi zaradi upada dobička in povzročene škode
utrpeli izgubo $700$ milijonov USD.

\begin{enumerate}[(a)]
\item Modeliraj problem v okviru teorije odločanja (stanja, odločitve).
Kakšno odločitev svetuješ vodstvu BP?
Kako je odločitev odvisna od verjetnosti $p$?
Nariši graf, ki prikazuje optimalni pričakovani dobiček v odvisnosti od $p$.

\item Za dodatnih $90.000$ USD lahko podjetje BP naroči študijo
pri kitajskem ljudskem inštitutu za naftne vrtine iz mesta Lanzhou,
ki ocenjuje uspešnost rizičnih projektov,
in bi ocenilo, ali bo ekipa dr.~Totalkov uspešna.
Statistični povzetek uspešnosti raziskav inštituta v preteklosti je naslednji:
\begin{center}
\begin{tabular}{l|cc}
$P(\text{rez.~raziskave} \;|\; \text{rez.~prov})$ &
\multicolumn{2}{c}{Rezultat raziskave inštituta} \\
Rezultat projekta & pozitiven & negativen \\ \hline
uspešen   &  $7/9$ & $2/9$ \\
neuspešen &  $1/3$ & $2/3$
\end{tabular}
\end{center}
Izračunaj $\EVPI$ in $\EVE$ ter komentiraj,
ali je smiselno izvesti dodatno raz\-iska\-vo.

\item Nariši drevo odločitev in ugotovi,
ali naj podjetje naroči raziskavo in če jo,
kako naj se odloči glede mašenja vrtine.
\end{enumerate}
\end{vprasanje}
\begin{odgovor}
\end{odgovor}
\end{naloga}
