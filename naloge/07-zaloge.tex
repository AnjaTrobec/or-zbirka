\subsection{Upravljanje zalog}

\begin{naloga}{Hillier, Lieberman}{\cite[Problem~19.3-2]{hl}}
\begin{vprasanje}
V trgovini vsak teden prodajo $600$ škatel toaletnega papirja.
Vsako naročilo stane $25 €$, za posamezno škatlo pa trgovina plača $3 €$.
Cena skladiščenja posamezne škatle je $0.05 €$ na teden.
\begin{enumerate}[(a)]
\item Denimo, da primanjkljaj ni dovoljen.
Kako pogosto naj trgovina naroča toaletni papir?
Kako velika naj bodo naročila?
\item Kaj pa, če dovolimo primanjkljaj,
ki nas stane $2 €$ na teden za vsako škatlo?
\end{enumerate}

\end{vprasanje}
\begin{odgovor}
\end{odgovor}
\end{naloga}


\begin{naloga}{Batagelj, Kaufman}{\cite[Naloga~10.7]{bk}}
\begin{vprasanje}
Tovarna Mitsuzuki načrtuje,
da bo naslednje leto izdelala $10000$ stereo televizorjev.
Odločijo se, da zvočnikov ne bodo izdelovali sami,
ampak jih bodo (po dva za vsak televizor) naročili
pri ponudniku najkvalitetnejših zvočnikov,
ki je predložil naslednji cenik:
\begin{center}
\begin{tabular}{c|c}
število zvočnikov v enem naročilu & cena zvočnika \\
\hline
1 -- 1999 & $150 €$ \\
2000 -- 4999 & $135 €$ \\
5000 -- 7999 & $125 €$ \\
8000 -- 19999 & $120 €$ \\
20000 ali več & $115 €$ \\
\end{tabular}
\end{center}
Poleg tega naročilo pošiljke stane Mitsuzuki $500 €$,
letni stroški skladiščenja vsakega zvočnika pa znašajo $20\%$ njegove cene.
Koliko zvočnikov naj vsakič naročijo
in kakšni so skupni stroški naročil za naslednje leto?

\end{vprasanje}
\begin{odgovor}
\end{odgovor}
\end{naloga}


\begin{naloga}{W.~L.~Winston}{\cite[\S15, Example~5]{w}}
\begin{vprasanje}
Tovarna avtomobilov za svoje potrebe izdeluje akumulatorje.
Vsako leto proizvedejo $10000$ avtomobilov,
izdelajo pa lahko $25000$ akumulatorjev letno.
Stroški zagona proizvodnje so $200 €$,
stroški zaloge pa $0.25 €$ letno za vsak akumulator.
Primanjkljaja ne dovolimo.
Kolikokrat na leto naj zaženejo proizvodnjo akumulatorjev?
Koliko časa naj traja proizvodnja?
Koliko akumulatorjev naj takrat izdelajo?
Kako veliko skladišče potrebujejo?
\end{vprasanje}
\begin{odgovor}
\end{odgovor}
\end{naloga}


\begin{naloga}{Janoš Vidali}{Vaje OR 6.6.2018}
\begin{vprasanje}
Marta izdeluje nakit iz školjk v delavnici,
ki jo najema v bližini ljubljanske tržnice,
in ga prodaja po vnaprej dogovorjeni ceni.
Povpraševanje je 10 kosov na teden,
stroški skladiščenja so $0.2 €$ za kos na teden.
Zagonski stroški izdelovanja nakita so $150 €$.
Marta lahko na teden izdela $12.5$ kosa nakita.
Dovoli si, da pride do primanjkljaja,
pri čemer jo ta stane $0.8 €$ za kos nakita na teden.
Kako naj Marta organizira proizvodnjo in skladiščenje,
da bo imela čim manj stroškov?
\end{vprasanje}
\begin{odgovor}
\end{odgovor}
\end{naloga}


\begin{naloga}{Janoš Vidali}{Kolokvij OR 11.6.2018}
\begin{vprasanje}
Vulkanizer v svoji delavnici izdeluje in prodaja avtomobilske pnevmatike.
Glede na trenutno povpraševanje vsak teden proda $60$ pnevmatik,
v istem času pa jih lahko izdela $90$.
Cena zagona proizvodnje je $180 €$,
cena skladiščenja posamezne pnevmatike pa je $0.5 €$ na teden.

\begin{enumerate}[(a)]
\item Denimo, da primanjkljaja ne dovolimo.
Izračunaj dolžino cikla pro\-iz\-vod\-nje in prodaje pnevmatik,
pri kateri so stroški najmanjši.
Kako veliko skladišče mora vulkanizer imeti?
Izračunaj tudi enotske stroške.

\item Kako dolgo naj pri zgornji rešitvi traja proizvodnja?
Koliko pnevmatik naj vulkanizer naredi v vsakem ciklu?

\item Vulkanizer je ocenil,
da bi ga primanjkljaj ene pnevmatike stal $2 €$ na teden.
Kakšna naj bosta dolžina cikla in velikost skladišča,
da bodo stroški čim manjši?
Izračunaj tudi enotske stroške in določi največji primanjkljaj.
\end{enumerate}
\end{vprasanje}
\begin{odgovor}
\end{odgovor}
\end{naloga}


\begin{naloga}{Janoš Vidali}{Izpit OR 5.7.2018}
\begin{vprasanje}
V trgovini s pohištvom vsak mesec prodajo $20$ sedežnih garnitur.
Z vsakim naročilom imajo $750 €$ stroškov,
pri tem pa vse naročeno blago dobijo hkrati.
Skladiščenje posamezne sedežne garniture jih stane $10 €$ na mesec,
primanjkljaj za en kos pa jih stane $50 €$ na mesec.

\begin{enumerate}[(a)]
\item Kako pogosto naj v trgovini naročajo sedežne garniture,
da bodo stroški čim manjši.
Kako veliko skladišče morajo imeti?
Izračunaj tudi enotske stroške.

\item Izračunaj največji dovoljeni primanjkljaj.
Koliko sedežnih garnitur naj vsakič naročijo?

\item V trgovini dobijo ponudbo za uporabo drugega skladišča,
v katerem imajo za po\-sa\-mez\-no se\-dež\-no garnituro
le $6 €$ stroškov na mesec,
vendar lahko hrani največ $40$ sedežnih garnitur.
Kako naj organizirajo naročanje,
če namesto prvotnega uporabljajo to skladišče?
Ali se jim ga splača uporabiti?
\namig{izpelji formulo za enotske stroške
in poišči optimalen interval naročanja pri fiksni velikosti skladišča.}
\end{enumerate}
\end{vprasanje}
\begin{odgovor}
\end{odgovor}
\end{naloga}
