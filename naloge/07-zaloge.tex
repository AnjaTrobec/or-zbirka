\subsection{Upravljanje zalog}

\begin{naloga}{Hillier, Lieberman}{\cite[Problem~19.3-2]{hl}}
\begin{vprasanje}
V trgovini vsak teden prodajo $600$ škatel toaletnega papirja.
Vsako naročilo stane $25 €$, za posamezno škatlo pa trgovina plača $3 €$.
Cena skladiščenja posamezne škatle je $0.05 €$ na teden.
\begin{enumerate}[(a)]
\item Denimo, da primanjkljaj ni dovoljen.
Kako pogosto naj trgovina naroča toaletni papir?
Kako velika naj bodo naročila?
\item Kaj pa, če dovolimo primanjkljaj,
ki nas stane $2 €$ na teden za vsako škatlo?
\end{enumerate}

\end{vprasanje}
\begin{odgovor}
\end{odgovor}
\end{naloga}


\begin{naloga}{Batagelj, Kaufman}{\cite[Naloga~10.7]{bk}}
\begin{vprasanje}
Tovarna Mitsuzuki načrtuje,
da bo naslednje leto izdelala $10000$ stereo televizorjev.
Odločijo se, da zvočnikov ne bodo izdelovali sami,
ampak jih bodo (po dva za vsak televizor) naročili
pri ponudniku najkvalitetnejših zvočnikov,
ki je predložil naslednji cenik:
\begin{center}
\begin{tabular}{c|c}
število zvočnikov v enem naročilu & cena zvočnika \\
\hline
1 -- 1999 & $150 €$ \\
2000 -- 4999 & $135 €$ \\
5000 -- 7999 & $125 €$ \\
8000 -- 19999 & $120 €$ \\
20000 ali več & $115 €$ \\
\end{tabular}
\end{center}
Poleg tega naročilo pošiljke stane Mitsuzuki $500 €$,
letni stroški skladiščenja vsakega zvočnika pa znašajo $20\%$ njegove cene.
Koliko zvočnikov naj vsakič naročijo
in kakšni so skupni stroški naročil za naslednje leto?

\end{vprasanje}
\begin{odgovor}
\end{odgovor}
\end{naloga}


\begin{naloga}{W.~L.~Winston}{\cite[\S15, Example~5]{w}}
\begin{vprasanje}
Tovarna avtomobilov za svoje potrebe izdeluje akumulatorje.
Vsako leto proizvedejo $10000$ avtomobilov,
izdelajo pa lahko $25000$ akumulatorjev letno.
Stroški zagona proizvodnje so $200 €$,
stroški zaloge pa $0.25 €$ letno za vsak akumulator.
Primanjkljaja ne dovolimo.
Kolikokrat na leto naj zaženejo proizvodnjo akumulatorjev?
Koliko časa naj traja proizvodnja?
Koliko akumulatorjev naj takrat izdelajo?
Kako veliko skladišče potrebujejo?
\end{vprasanje}
\begin{odgovor}
\end{odgovor}
\end{naloga}


\begin{naloga}{Janoš Vidali}{Vaje OR 6.6.2018}
\begin{vprasanje}
Marta izdeluje nakit iz školjk v delavnici,
ki jo najema v bližini ljubljanske tržnice,
in ga prodaja po vnaprej dogovorjeni ceni.
Povpraševanje je 10 kosov na teden,
stroški skladiščenja so $0.2 €$ za kos na teden.
Zagonski stroški izdelovanja nakita so $150 €$.
Marta lahko na teden izdela $12.5$ kosa nakita.
Dovoli si, da pride do primanjkljaja,
pri čemer jo ta stane $0.8 €$ za kos nakita na teden.
Kako naj Marta organizira proizvodnjo in skladiščenje,
da bo imela čim manj stroškov?
\end{vprasanje}
\begin{odgovor}
\end{odgovor}
\end{naloga}
