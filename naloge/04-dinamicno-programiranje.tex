\razdelek{Dinamično programiranje}

\begin{naloga}{?}{Izpit OR 15.9.2010}
\begin{vprasanje}
Na avtocestni odsek dolžine $M$ kilometrov
želimo postaviti oglasne plakate.
Dovoljene lokacije plakatov določa urad za oglaševanje
in so predstavljene s števili $x_1, x_2, \dots x_n$,
kjer $x_i$ ($1 \le i \le n$)
predstavlja oddaljenost od začetka odseka v kilometrih.
Profitabilnost oglasa na lokaciji $x_i$ določa vrednost $v_i$
($1 \le i \le n$).
Urad za oglaševanje podaja tudi omejitev,
da mora biti razdalja med oglasi vsaj $d$ kilometrov.
Oglase želimo postaviti tako, da bodo čim bolj profitabilni.
\begin{enumerate}[(a)]
\item Reši problem za parametre $M = 20$, $d = 5$, $n = 8$,
$(x_i)_{i=1}^n = (1, 2, 8, 10, 12,$ $14, 17, 20)$ in
$(v_i)_{i=1}^n = (8, 8, 12, 10, 7, 5, 6, 10)$.
\item Napiši rekurzivne enačbe za opisani problem.
\item Napiši algoritem,
ki poišče najbolj profitabilno postavitev oglasov za dane parametre.
Kakšna je njegova časovna zahtevnost?
\end{enumerate}

\end{vprasanje}
\begin{odgovor}
\end{odgovor}
\end{naloga}


\begin{naloga}{?}{Vaje OR 6.4.2016}
\begin{vprasanje}
Imamo nahrbtnik nosilnosti $M$ kilogramov.
Danih je $n$ objektov z vrednostmi $v_i$ in težami $t_i$ ($1 \le i \le n$).
Problem nahrbtnika sprašuje po izbiri predmetov
$I \subseteq \{1, 2, \dots, n\}$,
ki maksimizira njihovo skupno vrednost pri omejitvi $\sum_{i \in I} t_i \le M$.
\begin{enumerate}[(a)]
\item Napiši rekurzivne enačbe za opisani problem.
\item Z uporabo rekurzivnih enačb reši problem za parametre $M = 8$, $n = 8$,
$(v_i)_{i=1}^n = (9, 9, 8, 11, 10, 15,$ $3, 12)$ in
$(t_i)_{i=1}^n = (3, 5, 1, 4, 3, 8, 2, 7)$.
\end{enumerate}

\end{vprasanje}
\begin{odgovor}
\end{odgovor}
\end{naloga}


\begin{naloga}{?}{Vaje OR 6.4.2016}
\begin{vprasanje}
Dana je matrika $A = (a_{ij})_{i,j=1}^{m,n}$.
Poiskati želimo pot minimalne vsote,
ki se začne v levem zgornjem kotu (pri $a_{11}$)
in konča v desnem spodnjem kotu (pri $a_{mn}$).
Dovoljeni so zgolj premiki v desno in navzdol.
\begin{enumerate}[(a)]
\item Reši problem za matriko
$$
A = \begin{pmatrix}
131 & 673 & 234 & 103 &  18 \\
201 &  96 & 342 & 965 & 150 \\
630 & 803 & 746 & 422 & 111 \\
537 & 699 & 497 & 121 & 956 \\
805 & 732 & 524 &  37 & 332
\end{pmatrix} .
$$
\item Napiši rekurzivne enačbe za opisani problem.
\item Na osnovi rekurzivnih enačb napiši algoritem, ki reši opisani problem.
Oceni tudi njegovo časovno zahtevnost v odvisnosti od $m$ in $n$.
\end{enumerate}

\end{vprasanje}
\begin{odgovor}
\end{odgovor}
\end{naloga}


\begin{naloga}{?}{Vaje OR 6.4.2016}
\begin{vprasanje}
Dan je niz $S = a_1 a_2 \dots a_n$,
kjer so $a_i$ ($1 \le i \le n$) elementi neke končne abecede.
Nizu $a_j a_{j+1} \dots a_k$, kjer je $1 \le j \le k \le n$,
pravimo {\em strnjen podniz} niza $S$.
S pomočjo dinamičnega programiranja napiši algoritem,
ki določi najdaljši palindromski strnjen podniz v $S$.

\end{vprasanje}
\begin{odgovor}
\end{odgovor}
\end{naloga}


\begin{naloga}{?}{Vaje OR 6.4.2016}
\begin{vprasanje}
Dana je matrika $A = (a_{ij})_{i,j=1}^{m,n}$.
Poiskati želimo strnjeno podmatriko matrike $A$
z največjo vsoto komponent.
\begin{enumerate}[(a)]
\item Reši problem za matriko
$$
A = \begin{pmatrix}
 1 & -1 &  2 &  4 \\
-3 & -2 &  8 &  2 \\
-3 &  2 & -2 &  4 \\
 1 & -5 & -1 & -2
\end{pmatrix} .
$$
\item Napiši rekurzivne enačbe za opisani problem.
\item Napiši algoritem, ki reši opisani problem.
Oceni tudi njegovo časovno zahtevnost v odvisnosti od $m$ in $n$.
\end{enumerate}

\end{vprasanje}
\begin{odgovor}
\end{odgovor}
\end{naloga}


\begin{naloga}{?}{Vaje OR 6.4.2016}
\begin{vprasanje}
Na voljo imamo kovance z vrednostmi $1 = v_1 < v_2 < \cdots < v_n$
in vsoto $C$, ki jo želimo izplačati s kovanci.
Predpostavljamo, da imamo dovolj velik nabor kovancev.
\begin{enumerate}[(a)]
\item Poišči izplačilo z najmanjšim številom kovancev
za $C = 25$, $n = 4$ in $(v_i)_{i=1}^n = (1, 2, 5, 7)$.
\item S pomočjo dinamičnega programiranja reši problem v splošnem.
\end{enumerate}

\end{vprasanje}
\begin{odgovor}
\end{odgovor}
\end{naloga}


\begin{naloga}{?}{Vaje OR 6.4.2016}
\begin{vprasanje}
Na ulici je $n$ vrstnih hiš,
pri čemer je v $i$-ti hiši $c_i$ denarja.
Tat se odloča, katere izmed hiš naj oropa.
Vsak oropan stanovalec to sporoči svojim sosedom,
zato tat ne sme oropati dveh sosednjih hiš.
Ker je tat poslušal predmet Operacijske raziskave,
pozna dinamično programiranje.
Pokaži, kako naj tat določi, katere hiše naj oropa.

\end{vprasanje}
\begin{odgovor}
\end{odgovor}
\end{naloga}


\begin{naloga}{?}{Kolokvij OR 31.5.2012}
\begin{vprasanje}
Imamo hlod dolžine $\ell$,
ki bi ga radi razžagali na $n$ označenih mestih
$0 < x_1 < x_2 < \dots < x_n < \ell$.
Eno rezanje stane toliko, kolikor je dolžina hloda, ki ga režemo.
Ko hlod prerežemo, dobimo dva manjša hloda, ki ju režemo naprej.
Poiskati želimo zaporedje rezanj z najmanjšo ceno.
\begin{enumerate}[(a)]
\item Reši problem pri podatkih $\ell = 10$ in $(x)_{i=1}^4 = (3, 5, 7, 8)$.
\item S pomočjo dinamičnega programiranja reši problem v splošnem.
Oceni tudi njegovo časovno zahtevnost.
\end{enumerate}

\end{vprasanje}
\begin{odgovor}
\end{odgovor}
\end{naloga}


\begin{naloga}{Hillier, Lieberman}{\cite[Problem~11.3-1]{hl}}
\begin{vprasanje}
Lastnik verige $n$ trgovin z živili je kupil $m$ zabojev svežih jagod.
Naj bo $p_{ij}$ pričakovan dobiček v trgovini $j$,
če tja dostavimo $i$ zabojev.
Zanima nas, koliko zabojev naj gre v vsako trgovino,
da bomo imeli čim večji zaslužek.
Zaradi logističnih razlogov zabojev ne želimo deliti.
\begin{enumerate}[(a)]
\item Z dinamičnim programiranjem reši problem za podatke $m = 5$, $n = 3$
in $p_{ij}$ iz sledeče tabele:
$$
\begin{array}{c|ccc}
p_{ij} & 1 & 2 & 3 \\
\hline
0 &  0 &  0 &  0 \\
1 &  5 &  6 &  4 \\
2 &  9 & 11 &  9 \\
3 & 14 & 15 & 13 \\
4 & 17 & 19 & 18 \\
5 & 21 & 22 & 20 \\
\end{array}
$$
\item Napiši algoritem, ki reši opisani problem v splošnem.
\end{enumerate}

\end{vprasanje}
\begin{odgovor}
\end{odgovor}
\end{naloga}


\begin{naloga}{Hillier, Lieberman}{\cite[Problem~11.3-8]{hl}}
\begin{vprasanje}
Podjetje bo kmalu uvedlo nov izdelek na zelo konkurenčen trg,
zato trenutno pripravlja marketinško strategijo.
Odločili so se, da bodo izdelek uvedli v treh fazah.
V prvi fazi bodo pripravili posebno začetno ponudbo z močno znižano ceno,
da bi privabili zgodnje kupce.
Druga faza bo vključevala intenzivno oglaševalsko kampanjo,
da bi zgodnje kupce prepričali, naj izdelek še vedno kupujejo po redni ceni.
Znano je, da bo ob koncu druge faze
konkurenčno podjetje predstavilo svoj izdelek.
Zato bo v tretji fazi okrepljeno oglaševanje z namenom,
da bi preprečili beg strank h konkurenci.

Podjetje ima za oglaševanje na voljo $4$ milijone evrov,
ki jih želimo čim bolj učinkovito porabiti.
Naj bo $m$ tržni delež v procentih, pridobljen v prvi fazi,
$f_2$ delež, ohranjen po drugi fazi,
in $f_3$ delež, ohranjen po tretji fazi.
Maksimizirati želimo končni tržni delež, torej količino $m f_2 f_3$.

\begin{enumerate}[(a)]
\item Denimo, da želimo v vsaki fazi porabiti nek večkratnik milijona evrov,
pri čemer bomo pri prvi fazi porabili vsaj milijon evrov.
V spodnji tabeli so zbrani vplivi porabljenih količin
na vrednosti $m$, $f_2$ in $f_3$.
$$
\begin{array}{c|ccc}
M€ & m & f_2 & f_3 \\
\hline
0 &  - & 0.2 & 0.3 \\
1 & 20 & 0.4 & 0.5 \\
2 & 30 & 0.5 & 0.6 \\
3 & 40 & 0.6 & 0.7 \\
4 & 50 &   - &   - \\
\end{array}
$$
Kako naj razdelimo sredstva?
\item Denimo sedaj,
da lahko v vsaki fazi porabimo poljubno pozitivno količino denarja
(seveda glede na omejitev skupne porabe).
Naj bodo torej $x_1$, $x_2$ in $x_3$ količine denarja v milijonih evrov,
ki jih porabimo v prvi, drugi in tretji fazi.
Vpliv na tržni delež je podan s formulami
$$
m = x_1 (10 - x_1), \quad
f_2 = 0.4 + 0.1 x_2, \quad \text{in} \quad
f_3 = 0.6 + 0.07 x_3 .
$$
Kako naj sedaj razdelimo sredstva?
\end{enumerate}

\end{vprasanje}
\begin{odgovor}
\end{odgovor}
\end{naloga}


\begin{naloga}{?}{Izpit OR 26.6.2012}
\begin{vprasanje}
Nori profesor Boltežar stanuje v stolpnici z $n$ nadstropji,
oštevilčenimi od $1$ do $n$.
Nori stanovalci tega bloka radi mečejo cvetlične lončke z balkonov.
Boltežar bi rad ugotovil, katero je najvišje nadstropje,
s katerega lahko pade cvetlični lonček, ne da bi se razbil.
Jasno je, da če se lonček razbije pri padcu iz $k$-tega nadstropja,
potem se razbije tudi pri padcu s $(k+1)$-tega nadstropja.
Če bi Boltežar imel le en cvetlični lonček,
bi ga lahko metal po vrsti od najnižjega nadstropja navzgor,
dokler se ne bi razbil.
V najslabšem primeru bi lonček torej vrgel $n$ krat
(možno je, da bi lonček preživel tudi padec iz najvišjega nadstropja).

Ker ima Boltežar doma $k$ cvetličnih lončkov,
lahko do rezultata pride tudi z manjšim številom metov.
S pomočjo dinamičnega programiranja bi rad po\-is\-kal strategijo metanja,
ki bi minimizirala število potrebnih metov v najslabšem primeru.
\begin{enumerate}[(a)]
\item Napiši rekurzivne enačbe za opisani problem.
\item Napiši algoritem, ki reši opisani problem.
Oceni tudi njegovo časovno zahtevnost v odvisnosti od $n$ in $k$.
\end{enumerate}

\end{vprasanje}
\begin{odgovor}
\end{odgovor}
\end{naloga}


\begin{naloga}{Hillier, Lieberman}{\cite[Problem~11.2-2]{hl}}
\begin{vprasanje}
Vodja prodaje pri založniku učbenikov za fakulteto
ima na voljo $6$ trgovskih potnikov,
ki jim želi dodeliti eno od treh regij, v kateri bodo delovali.
V vsaki regiji mora delovati vsaj en trgovski potnik.
Naj bo $p_{ij}$ pričakovana porast v prodaji v regiji $j$,
če bo tam delovalo $i$ trgovskih potnikov:
$$
\begin{array}{c|ccc}
p_{ij} & 1 & 2 & 3 \\
\hline
1 & 35 & 21 & 28 \\
2 & 48 & 42 & 41 \\
3 & 70 & 56 & 63 \\
4 & 89 & 70 & 75 \\
\end{array}
$$
Reši problem s pomočjo dinamičnega programiranja.

\end{vprasanje}
\begin{odgovor}
\end{odgovor}
\end{naloga}


\begin{naloga}{Hillier, Lieberman}{\cite[Problem~11.3-16]{hl}}
\begin{vprasanje}
Dan je sledeči nelinearni program.
\begin{align*}
\max &\quad 2x_1^2 + 2x_2 + 4x_3 - x_3^2 \\[1ex]
2x_1 + x_2 + x_3 &\le 4 \\
x_1, x_2, x_3 &\ge 0
\end{align*}
Reši ga s pomočjo dinamičnega programiranja.

\end{vprasanje}
\begin{odgovor}
\end{odgovor}
\end{naloga}


\begin{naloga}{Hillier, Lieberman}{\cite[Problem~11.4-1]{hl}}
\begin{vprasanje}
Igralec na srečo bo odigral tri partije s svojimi prijatelji,
pri čemer lahko vsakič stavi na svojo zmago.
Stavi lahko katerokoli vsoto denarja, ki jo ima na voljo
-- če izgubi partijo, zastavljeno vsoto izgubi, sicer pa tako vsoto pridobi.
Pri vsaki partiji sta verjetnosti zmage in poraza enaki $1/2$.
Na začetku ima $75 €$, na koncu pa želi imeti $100 €$
(ker igra s prijatelji, noče imeti več kot toliko).

Z dinamičnim programiranjem poišči strategijo stavljenja,
ki maksimizira verjetnost, da bo na koncu imel natanko $100 €$.
\end{vprasanje}
\begin{odgovor}
\end{odgovor}
\end{naloga}


\begin{naloga}%
{Dasgupta, Papadimitriou, Vazirani}{\cite[Exercise~6.14]{dpv}}
\begin{vprasanje}[blago]
Imamo pravokoten kos blaga dimenzij $m \times n$,
kjer sta $m$ in $n$ pozitivni celi števili,
ter seznam $k$ izdelkov,
pri čemer potrebujemo za izdelek $i$
pravokoten kos blaga dimenzij $a_i \times b_i$
($a_i, b_i$ sta pozitivni celi števili),
ki ga prodamo za ceno $c_i > 0$.
Imamo stroj, ki lahko poljuben kos blaga razreže na dva dela
bodisi vodoravno, bodisi navpično.
Začetni kos blaga želimo razrezati tako,
da bomo lahko naredili izdelke,
ki nam bodo prinašali čim večji dobiček.
Pri tem smemo izdelati poljubno število kosov posameznega izdelka.
Kose blaga lahko seveda tudi obračamo
(tj., za izdelek $i$ lahko narežemo kos velikosti
$a_i \times b_i$ ali $b_i \times a_i$).

Zapiši rekurzivne enačbe za reševanje danega problema.
Razloži, kaj predstavljajo spremenljivke,
v kakšnem vrstnem redu jih računamo,
ter kako dobimo optimalno rešitev.
\end{vprasanje}
\begin{odgovor}
\end{odgovor}
\end{naloga}


\begin{naloga}{Janoš Vidali}{Izpit OR 15.12.2016}
\begin{vprasanje}
Oceni časovno zahtevnost algoritma,
ki sledi iz rekurzivnih enačb za nalogo~\nal{blago}.
Reši problem za podatke $m = 5, n = 3, k = 4$,
$(a_i)_{i=1}^k = (2, 3, 1, 2)$, $(b_i)_{i=1}^k = (2, 1, 4, 3)$
in $(c_i)_{i=1}^k = (6, 3, 5, 7)$.
\end{vprasanje}
\begin{odgovor}
\end{odgovor}
\end{naloga}


\begin{naloga}{Janoš Vidali}{Izpit OR 31.1.2017}
\begin{vprasanje}
Dano je zaporedje $n$ realnih števil $a_1, a_2, \dots, a_n$.
Želimo poiskati strnjeno podzaporedje z največjim produktom
-- t.j., taka indeksa $i, j$ ($1 \le i \le j \le n$),
da je produkt $a_i a_{i+1} \cdots a_{j-1} a_j$ čim večji.

\begin{enumerate}[(a)]
\item Zapiši rekurzivne enačbe za reševanje danega problema.
Razloži, kaj pred\-stav\-lja\-jo spremenljivke,
v kakšnem vrstnem redu jih računamo,
ter kako dobimo optimalno rešitev.
\namig{posebej obravnavaj pozitivne in negativne delne produkte.}

\item Oceni časovno zahtevnost algoritma, ki sledi iz zgoraj zapisanih enačb.

\item S svojim algoritmom reši problem za zaporedje
$$
0.9, \ -2, \ -0.6, \ -0.5, \ -2, \ 5, \ 0.1, \ 3, \ 0.5, \ -3 \ .
$$
\end{enumerate}
\end{vprasanje}
\begin{odgovor}
\end{odgovor}
\end{naloga}


\begin{naloga}{Sergio Cabello}{Izpit OR 15.3.2017}
\begin{vprasanje}
Za zaporedje števil $x_1, x_2, \dots x_m$ pravimo, da je {\em oscilirajoče},
če velja $x_i < x_{i+1}$ za vse sode $i$ in $x_i > x_{i+1}$ za vse lihe $i$.
S pomočjo dinamičnega programiranja zasnuj algoritem,
ki v polinomskem času izračuna dolžino najdaljšega oscilirajočega podzaporedja
zaporedja celih števil $a_1, a_2, \dots a_n$.
\end{vprasanje}
\begin{odgovor}
\end{odgovor}
\end{naloga}


\begin{naloga}{Janoš Vidali}{Izpit OR 10.7.2017}
\begin{vprasanje}
V podjetju imajo na voljo $m$ milijonov evrov sredstev,
ki jih bodo vložili v razvoj nove aplikacije.
Denar bodo porazdelili med tri skupine.
Naj bodo $x_1$, $x_2$ in $x_3$ količine denarja (v milijonih evrov),
ki jih bodo dodelili razvijalcem, oblikovalcem in marketingu.
Vrednosti $x_1, x_2, x_3$ niso nujno cela števila.
Razvijalci morajo dobiti vsaj $a_1$ milijonov evrov,
potencial, ki ga ustvarijo, pa je $p_1 = n_1 + k_1 x_1$.
Oblikovalci morajo dobiti vsaj $a_2$ milijonov evrov,
potencial, ki ga ustvarijo, pa je $p_2 = n_2 + k_2 x_2$.
Marketing mora dobiti vsaj $a_3$ milijonov evrov,
ustvari pa faktor $p_3 = n_3 + k_3 x_3$.
Pričakovani dobiček v milijonih evrov
se izračuna po formuli $d = (p_1 + p_2) p_3$.
V podjetju bi radi sredstva porazdelili med skupine tako,
da bo pričakovani dobiček čim večji.

\begin{enumerate}[(a)]
\item Zapiši rekurzivne enačbe za reševanje danega problema.
\item Z zgoraj zapisanimi enačbami reši problem
pri podatkih $m = 15$, $a_1 = 4$, $n_1 = 3$, $k_1 = 1.5$,
$a_2 = 3$, $n_2 = 4$, $k_2 = 2$, $a_3 = 2$, $n_3 = 0.4$ in $k_3 = 0.3$.
\end{enumerate}
\end{vprasanje}
\begin{odgovor}
\end{odgovor}
\end{naloga}


\begin{naloga}{Janoš Vidali}{Izpit OR 29.8.2017}
\begin{vprasanje}
V veliki multinacionalni korporaciji želijo,
da bi se zakonodaja spremenila v njihov prid.
V ta namen so najeli $m$ lobistov,
ki se bodo pogajali z $n$ političnimi strankami,
da pridobijo njihovo podporo pri spremembi zakonodaje.
Vsak lobist se bo pogajal s samo eno stranko;
k vsaki stranki lahko pošljejo več lobistov.
Naj bo $p_{ij}$ ($0 \le i \le m$, $1 \le j \le n$) verjetnost,
da pridobijo podporo stranke $j$,
če se z njo pogaja $i$ lobistov
(lahko predpostaviš $p_{i-1,j} \le p_{ij}$ za vsaka $i, j$).
Verjetnosti za različne stranke so med seboj neodvisne.
Maksimizirati želijo verjetnost,
da bodo lobisti pridobili podporo vseh $n$ političnih strank.

\begin{enumerate}[(a)]
\item Zapiši rekurzivne enačbe za reševanje danega problema.
\item Naj bo $m = 6$ in $n = 3$.
K vsaki stranki želijo poslati vsaj enega lobista
(tj., $p_{0j} = 0$ za vsak $j$),
vrednosti $p_{ij}$ za $i \ge 1$ pa so podane v spodnji tabeli.
$$
\begin{array}{c|ccc}
p_{ij} & 1 & 2 & 3 \\ \hline
1 & 0.2 & 0.4 & 0.3 \\
2 & 0.5 & 0.5 & 0.4 \\
3 & 0.7 & 0.5 & 0.8 \\
4 & 0.8 & 0.6 & 0.9 \\
\end{array}
$$
Za dane podatke reši problem z zgoraj zapisanimi enačbami.
\end{enumerate}
\end{vprasanje}
\begin{odgovor}
\end{odgovor}
\end{naloga}


\begin{naloga}{Janoš Vidali}{Kolokvij OR 23.4.2018}
\begin{vprasanje}[domine]
Imamo zaporedje $n$ polj,
pri čemer je na $i$-tem polju zapisano število $a_i$.
Na voljo imamo še $\lfloor n/2 \rfloor$ domin,
z vsako od katerih lahko pokrijemo dve sosednji polji.
Vsaka domina je sestavljena iz dveh delov:
na enem je znak $+$, na drugem pa znak $-$.
Posamezno polje lahko pokrijemo z le eno domino;
če sta pokriti dve sosednji polji, morata biti pokriti z različnima znakoma
(bodisi z iste, bodisi z druge domine).
Iščemo tako postavitev domin,
ki maksimizira vsoto pokritih števil,
pomnoženih z znakom na delu domine, ki pokriva število.
Pri tem ni potrebno, da uporabimo vse domine.
Primer je podan na sliki~\fig{}.

\begin{enumerate}[(a)]
\item Zapiši rekurzivne enačbe za reševanje danega problema.
Razloži, kaj pred\-stav\-lja\-jo spremenljivke,
v kakšnem vrstnem redu jih računamo,
ter kako dobimo optimalno rešitev.
\namig{posebej obravnavaj dva primera glede na postavitev zadnje domine.}

\item Oceni časovno zahtevnost algoritma, ki sledi iz zgoraj zapisanih enačb.

\item S svojim algoritmom poišči optimalno pokritje
za primer s slike~\fig{}.
\end{enumerate}

\begin{slika}
\pgfslika
\caption{Primer dopustnega (ne nujno optimalnega) pokritja
za nalogo~\nal{}.
Vsota tega pokritja je $3 - (-4) - 5 + 9 - 1 + 2 = 12$.
Če bi eno od zadnjih dveh domin obrnili (zamenjala bi se znaka),
dobljeno pokritje ne bi bilo dopustno,
saj bi dve zaporedni polji bili pokriti z enakima znakoma.}
\end{slika}
\end{vprasanje}
\begin{odgovor}
\end{odgovor}
\end{naloga}


\begin{naloga}{Janoš Vidali}{Izpit OR 5.7.2018}
\begin{vprasanje}
Vlagatelj ima na voljo $50$ milijonov evrov sredstev,
ki jih lahko porabi za donosno, a tvegano naložbo.
Ocenjuje, da bi se mu ob uspehu naložbe vložek povrnil petkratno,
verjetnost uspeha pa ocenjuje na $0.6$.
Zaradi tveganja se lahko odloči za zavarovanje naložbe,
pri čemer ima ponudbi dveh zavarovalnic,
ki mu proti plačilu ustrezne premije ponujata povračilo dela vložka,
če bo naložba neuspešna.
Vlagatelj lahko del sredstev obdrži tudi zase
(tj., ga ne porabi za naložbo ali premijo).

Naj bodo torej $x_1, x_2, x_3$ vrednosti v milijonih evrov,
ki zaporedoma pred\-stav\-lja\-jo količine,
ki jih vlagatelj obrži zase, porabi za naložbo,
in plača za zavarovalniško premijo.
Pričakovana vrednost naložbene strategije vlagatelja
(tj., količina denarja, ki jo ima na koncu)
je potem
$$
x_1 + x_2 (0.6 \cdot 5 + 0.4 q(x_3)) ,
$$
kjer $q(x_3)$ predstavlja delež vložka,
ki ga glede na vloženo premijo zavarovalnica povrne ob ne\-uspe\-hu naložbe.

Vlagatelj ima dve ponudbi konkurenčnih zavarovalnic.
Zavarovalnica Zvezna d.z.z.~za premijo v višini $x_3$ milijonov evrov
ponuja povračilo deleža $0.15 x_3$ celotne naložbe v primeru neuspeha,
pri čemer je največja možna premija $4$ milijone evrov.
Zavarovalnica Diskretna d.d.z.~pa ponuja le tri možne premije:
\begin{center}
\begin{tabular}{c|c}
premija & delež povračila ob neuspešni naložbi \\ \hline
$1$ milijon evrov & $0.1$ \\
$2$ milijona evrov & $0.35$ \\
$3$ milijoni evrov & $0.5$
\end{tabular}
\end{center}
Pogodbo smemo skleniti samo pri eni zavarovalnici.

\begin{enumerate}[(a)]
\item Zapiši definicijo funkcije $q(x)$
skupaj z izbiro najugodnejše zavarovalnice pri vsakem $x$.

\item Zapiši rekurzivne formule za določitev strategije vlaganja,
ki nam bo prinesla največji pričakovani dobiček.

\item S pomočjo zgornjih rekurzivnih enačb ugotovi,
kako naj ravna vlagatelj, da bo imel čim večji dobiček.
\end{enumerate}
\end{vprasanje}
\begin{odgovor}
\end{odgovor}
\end{naloga}


\begin{naloga}{Janoš Vidali}{Izpit OR 28.8.2018}
\begin{vprasanje}
Pri direkciji za ceste načrtujejo nov avtocestni odsek dolžine $M$ kilometrov.
Ob cesti želijo zgraditi počivališča tako,
da je razdalja med dvema zaporednima počivališčema največ $K$ kilometrov.
Prav tako mora biti prvo počivališče največ $K$ kilometrov od začetka,
zadnje pa največ $K$ kilometrov od konca avtocestnega odseka.
Naj bodo $x_1 < x_2 < \dots < x_n$ možne lokacije počivališč
(v kilometrih od začetka avtocestnega odseka),
in $c_i$ ($1 \le i \le n$) cena izgradnje počivališča na lokaciji $x_i$.
Postavitev počivališč želijo izbrati tako,
da bo skupna cena izgradnje čim manjša.

\begin{enumerate}[(a)]
\item Zapiši rekurzivne enačbe za reševanje danega problema.
Razloži, kaj pred\-stav\-lja\-jo spremenljivke,
v kakšnem vrstnem redu jih računamo, ter kako dobimo optimalno rešitev.

\item Oceni časovno zahtevnost algoritma, ki sledi iz zgoraj zapisanih enačb.

\item S pomočjo rekurzivnih enačb reši zgornji problem za podatke
\begin{align*}
M &= 100, & (x_i)_{i=1}^8 &= ( 5, 12, 22, 34, 49, 65, 83, 91), \\
K &= 30,  & (c_i)_{i=1}^8 &= (18, 11, 21, 16, 23, 15, 19, 13).
\end{align*}
\end{enumerate}
\end{vprasanje}
\begin{odgovor}
\end{odgovor}
\end{naloga}


\begin{naloga}{?}{Kolokvij OR 26.1.2010}
\begin{vprasanje}
Pri neznanem problemu, ki smo se ga lotili z dinamičnim programiranjem,
smo prišli do naslednje rekurzivne zveze:
$$
c_{i,j} = \max\{c_{i-1,j-1}, c_{i+1,j-1} + 2c_{i-1,j+1}\} .
$$
Izračunati želimo vrednost $c_{m,n}$.
Katere vrednosti $c_{i,j}$,
pri čemer je $(i, j)$ zunaj $\{1, \dots, m\} \times \{1, \dots, n\}$,
moramo poznati, da bo $c_{m,n}$ enolično določen?
Poišči čim manjšo takšno množico.
\end{vprasanje}
\begin{odgovor}
\end{odgovor}
\end{naloga}


\begin{naloga}{?}{Kolokvij OR 26.1.2010}
\begin{vprasanje}
Upravljalec nove moderne visoko tehnološko opremljene kongresne dvorane
v Kongresnem centru Grebka Pohleparja zaračunava za najem dvorane $5 €$/min.
Povpraševanje za najem nove dvorane je ogromno,
a se na žalost ponudbe časovno prekrivajo,
dvorano pa seveda lahko najame le ena stranka naenkrat.
Tako se je upravljalec že prvi dan znašel pod goro ponudb,
med katerimi je napol v paniki izbiral,
bodimo pošteni, tako bolj ``čez palec''.
A za kongresni center, ki nosi ime takega velikana ekonomije, se spodobi,
da si uredi vse segmente poslovanja čim bolj optimalno
in je kot tak zgled drugim.
Predpostavimo,
da ima upravljalec v trenutku odločanja podatke v obliki intervalov
$[z_i, k_i]$, $i = 1, \dots, n$,
kjer $z_i$ in $k_i$ predstavljata začetni in končni čas ponudbe $i$.
Ponudbi se prekrivata, če je njun presek neprazen časovni interval.

\begin{enumerate}[(a)]
\item Denimo, da uredimo intervale ponudb
glede na začetne čase ponudb od najmanjšega do največjega.
Potem izberemo prvi interval, vse intervale, ki ga prekrivajo, pa zavržemo,
ter postopek ponovimo rekurzivno.
Upravljalcu se ta postopek zdi optimalen. Kaj pa meniš ti?

\item Denimo, da so intervali ponudb urejeni kot v prejšnji točki
ter da je interval $I$ zadnji interval v neki optimalni izbiri OPT.
Premisli,
kateri intervali so lahko kandidati za predzadnji interval v izbiri OPT.
Na podlagi tega premisleka sestavi postopek,
ki s pomočjo dinamičnega programiranja najde optimalno rešitev problema.
Postopek utemelji.

\item Kakšna je časovna zahtevnost tvojega algoritma v najslabšem primeru?

\item Z uporabo razvitega algoritma reši problem za podatke
$(15, 60)$, $(60, 180)$, $(30, 90)$, $(120, 210)$,
$(165, 255)$, $(75, 135)$, $(30, 105)$, $(90, 150)$.
\end{enumerate}
\end{vprasanje}
\begin{odgovor}
\end{odgovor}
\end{naloga}


\begin{naloga}{?}{Kolokvij OR 26.1.2010}
\begin{vprasanje}
Jatifi Bajrami se ukvarja s preprodajo pomaranč.
Naslednja sezona je razdeljena na $n$ mesecev.
Za vsak mesec so znana predvidena povpraševanja $r_1, \dots, r_n$.
Nabavni stroški v mesecu $i$ so sestavljeni iz fiksnih stroškov $K_i$
ter stroškov za nabavo sadja.
Na začetku meseca $i$ stane škatla pomaranč $c_i$.
Če Jatifi takrat kupi $m$ škatel,
je za nabavo sadja torej moral odšteti $mc_i$ denarnih enot.
Če se Jatifi zaradi nižje cene v nekem mesecu
odloči nakupiti vnaprej za več mesecev,
mora plačati skladiščenje za škatle pomaranč za preostale mesece,
pri čemer je cena skladiščenja za škatlo v mesecu $i$ enaka $h_i$.
Jatifi kupuje izključno, ko ve, da so mu zaloge popolnoma pošle,
in od tega ne odstopa, saj se mu zdi škoda,
da bi v skladišču hranil stare pomaranče.
Pomaranče prodaja ves čas po isti ceni.
Jatifi bi rad maksimiziral dobiček in zadovoljil celotno povpraševanje.

\begin{enumerate}[(a)]
\item Predlagaj rešitev problema s pomočjo dinamičnega programiranja.
Pri tem so vhodni podatki $n$ mesecev, $r = (r_i)_{i=1}^n$,
$K = (K_i)_{i=1}^n$, $c = (c_i)_{i=1}^n$ ter $h = (h_i)_{i=1}^n$.
Kakšno časovno zahtevnost ima tvoj algoritem?

\item Naj bo $n = 4$, $r = (20, 40, 15, 10)$, $K = (2, 3, 4, 2)$,
$c = (3, 4, 3, 4)$ ter $h = (1, 1, 1, 2)$.
Poišči optimalno strategijo kupovanja pomaranč.
\end{enumerate}
\end{vprasanje}
\begin{odgovor}
\end{odgovor}
\end{naloga}


\begin{naloga}{?}{Izpit OR 5.2.2010}
\begin{vprasanje}
Dana je rekurzija
$$
c_{i,j} = \min_{i \le k \le j} \left(c_{i-1,k} + \sum_{u=i}^k a_{i,u}\right),
\quad c_{0,j} = c_{i,0} = 1 ,
$$
kjer je $A = (a_{i,j})_{i,j=1}^n$ matrika realnih števil.
Pokaži in zelo natančno utemelji,
da lahko izračunaš $c_{n,n}$ v času $O(n^3)$.
\end{vprasanje}
\begin{odgovor}
\end{odgovor}
\end{naloga}


\begin{naloga}{?}{Izpit OR 5.2.2010}
\begin{vprasanje}
Na znanem TV kanalu HOP.tv bodo ob sredah ob 17h
predvajali humanitarno igro Prodajalec\texttrademark,
pri kateri igralec prodaja predmete tipa $X$ in predmete tipa $Y$
v določenem zaporedju $P$,
v studio pa po vrsti prihajajo potencialni kupci.
Organizatorji vnaprej določijo zaporedje kupcev
in $i$-temu kupcu določijo tudi tip predmeta $T_i$.
Kupec $i$ lahko sam poda le ceno $c_i$,
ki jo je pripravljen plačati za predmet.
Organizatorji na podlagi tega določijo zaporedje $P$ tako,
da je izvedljiva prodaja vseh predmetov v zaporedju $P$.
Podatkov $P$, $T$ in $c$ igralec seveda ne pozna vnaprej.
Kupci prihajajo v studio in igralec se odloča,
ali jim za ponujeno ceno trenutni predmet proda ali ne.
Pri tem mu svetuje in vpliva na njegov humanitarni čut
znani voditelj Joras Š.
Če igralec trenutnega predmeta ne proda $i$-temu kupcu,
pride naslednji kupec, če pa ga proda,
pa prisluži za humanitarne namene $c_i$ enot denarja
in naslednjemu kupcu poskusi prodati naslednji predmet v zaporedju $P$.
V sezoni zmaga igralec,
katerega prodaja je procentualno največja glede na optimalno prodajo.
Na HOP.tv so na žalost kupili
le okrnjeno licenco za igro Prodajalec\texttrademark,
ki ne vključuje programske opreme za preračun optimalnih rešitev,
tako da te prosijo za pomoč.
Za primer: če so podatki $T = XYXXY$, $c = (2, 2, 7, 1, 1)$ in $P = XY$,
je optimalna rešitev prodaja $X$ tretjemu kupcu ter prodaja $Y$ petemu kupcu,
s katero prislužimo $c_3 + c_5 = 7 + 1 = 8$ enot denarja.

\begin{enumerate}[(a)]
\item S pomočjo dinamičnega programiranja izpelji algoritem,
ki za dane $T$, $c$ in $P$ poišče optimalno rešitev.
Podaj ustrezno rekurzivno zvezo ter opiši,
kako rekonstruiramo optimalno rešitev.
\namig{zgleduj se po algoritmu za najdaljše skupno podzaporedje.}

\item Kakšna je časovna zahtevnost tvojega algoritma v najslabšem primeru?

\item Izvedi algoritem na zgornjih podatkih in tako preveri,
da je opisana rešitev v zgornjem primeru res optimalna.
\end{enumerate}
\end{vprasanje}
\begin{odgovor}
\end{odgovor}
\end{naloga}


\begin{naloga}{?}{Izpit OR 28.6.2010}
\begin{vprasanje}
Podatke (nize) stiskamo na naslednji način.
Podana je tabela $m$ nizov, dolžina vsakega je kvečjemu $k$.
Radi bi zakodirali podatkovni niz $D$ dolžine $n$
z najkrajšim možnim zaporedjem nizov iz tabele.
Npr., če naša tabela vsebuje nize $(a, ba, abab, b)$
in bi radi zakodirali niz $bababbaababa$,
je najboljši način kodiranja zaporedje $(b, abab, ba, abab, a)$,
torej s petimi kodnimi besedami.

\begin{enumerate}[(a)]
\item S pomočjo strategije dinamičnega programiranja
izpelji čim bolj učinkovit algoritem,
ki poišče najkrajše kodiranje danega niza $D$,
če kodiranje za tak niz obstaja.

\item Demonstriraj uporabo algoritma na podanem nizu $bababbaababa$.

\item Kakšna je časovna zahtevnost tvoje rešitve?
\end{enumerate}
\end{vprasanje}
\begin{odgovor}
\end{odgovor}
\end{naloga}


\begin{naloga}{?}{Kolokvij OR 24.1.2011}
\begin{vprasanje}
Danih imamo $n$ kock, ki jih želimo zložiti v čim višji stolp.
Dimenzije $i$-te kocke so $a_i \times b_i \times c_i$,
kjer je $a_i$ dolžina, $b_i$ širina in $c_i$ višina.
Kock ne smemo rotirati,
kocko $i$ pa lahko postavimo na kocko $j$ samo,
če je $a_i < a_j$ in $b_i < b_j$.
Brez škode za splošnost lahko predpostavimo,
da je $a_1 \ge a_2 \ge \dots \ge a_n$.
S pomočjo dinamičnega programiranja poišči najvišji možni stolp.

Kako bi nalogo rešil, če je dovoljena tudi rotacija kock?
\end{vprasanje}
\begin{odgovor}
\end{odgovor}
\end{naloga}


\begin{naloga}{?}{Izpit OR 9.2.2011}
\begin{vprasanje}
Znanstveniki proučujejo novo vrsto bakterije.
V DNK te bakterije so našli na\-sled\-nje zaporedje nukleotidov:
\begin{center}
{\tt AACAGTTA}.
\end{center}
Sumijo, da je to različica gena že znane bakterije, ki ima zaporedje
\begin{center}
{\tt ACCATGTA}.
\end{center}
Če sta zaporedji dovolj podobni,
obstaja verjetnost, da imata gena podobni funkciji.
Dovoljene so naslednje operacije:
\begin{itemize}
\item substitucija ({\tt ACT} $\to$ {\tt AGT}),
\item vstavljanje ({\tt AC} $\to$ {\tt AGC}),
\item izbris ({\tt AGC} $\to$ {\tt AC}) in
\item transpozicija ({\tt AT} $\to$ {\tt TA}).
\end{itemize}
\begin{enumerate}[(a)]
\item Pomagajte znanstvenikom in napišite postopek,
ki bo preštel najmanše število operacij,
s katerim pridemo od enega DNK zaporedja do drugega.
\item Postopek iz prejšnje točke izvedite nad podanima zaporedjema
in ugotovite število razlik.
\end{enumerate}
\end{vprasanje}
\begin{odgovor}
\end{odgovor}
\end{naloga}


\begin{naloga}{?}{Izpit OR 28.6.2011}
\begin{vprasanje}
Dano imamo zaporedje simbolov $\top$ (resnično) in $\bot$ (neresnično).
Med vsakima izrazoma imamo $\land$ (prvi in drugi),
$\lor$ (prvi ali drugi) ali $\oplus$ (prvi ali drugi, ne pa tudi oba),
npr.~$\top \land \top \oplus \bot$.
Izračunaj število postavitev oklepajev, da je rezultat $\top$,
in število postavitev oklepajev, da je rezultat $\bot$.
\namig{definiraj $T_{i,j}$ in $F_{i,j}$
kot število takih postavitev oklepajev med $i$-tim in $j$-tim izrazom.}
\end{vprasanje}
\begin{odgovor}
\end{odgovor}
\end{naloga}


\begin{naloga}{?}{Kolokvij OR 31.5.2012}
\begin{vprasanje}
Samostojna umetnica z nadimkom Stekelce se preživlja z barvanjem na steklo.
Tokrat je dobila naročilo
za izdelavo dveh zelo posebnih pobarvanih steklenih kroglic.
Plačilo je dobila vnaprej, stroške pa bo imela le z nabavo stekla,
saj ima barv in čopičev že dovolj na zalogi.
Če ne uspe izdelati obeh kroglic v dogovorjenem času,
bo morala plačati $800 €$ kazni.
Steklene kroglice lahko naroči v steklarni, in sicer jo vsaka stane $100 €$.
Steklarna ima stroške vsakič, ko zažene proizvodnjo teh kroglic,
zato ji za vsako naročilo (ne glede na število naročenih kroglic)
zaračuna dodatnih $300 €$.
Kljub temu, da mora Stekelce zmeraj plačati vse dobljene kroglice,
je za vsako le $50 \%$ verjetnost, da bo zanjo uporabna.
Neuporabne kroglice lahko vrže kar v smeti,
odvečne uporabne kroglice pa zanjo prav tako niso vredne nič
in jih bo brez stroškov zavrgla.
\begin{enumerate}[(a)]
\item Ker je naročilo časovno omejeno,
ima Stekelce čas za največ dve naročili v steklarni.
Po koliko kroglic naj vsakič naroči, da bodo pričakovani stroški čim manjši?
(Število naročenih kroglic pri drugem naročilu
bo seveda odvisno od števila dobrih kroglic pri prvem naročilu.)

\item Kaj pa,
če steklarna v prvi seriji za naročilo namesto $300 €$ zahteva le $150 €$,
v drugi seriji pa stroški naročila ostanejo $300 €$?
\end{enumerate}
\end{vprasanje}
\begin{odgovor}
\end{odgovor}
\end{naloga}


\begin{naloga}{?}{Izpit OR 9.7.2012}
\begin{vprasanje}
Dan je številski trikotnik
$$
\begin{array}{ccccccccc}
                &&&& a_{1,1} \\
            &&& a_{2,1} && a_{2,2} \\
        && a_{3,1} && a_{3,2} && a_{3,3} \\
     & \iddots && \vdots && \vdots && \ddots \\
a_{n,1} && a_{n,2} && \cdots && a_{n,n-1} && a_{n,n}
\end{array} ,
$$
kjer so $a_{i,j} \in \Z$.
{\em Spust} v takem trikotniku je pot od vrha do dna
-- t.j., zaporedje $(a_{i, j_i})_{i=1}^n$,
kjer je $j_1 = 1$ in $j_i \in \{j_{i-1}, j_{i-1}+1\}$
za vsak $i = 2, \dots, n$.
Teža spusta je vsota števil, po katerih poteka spust
(torej $\sum_{i=1}^n a_{i, j_i}$).

S pomočjo dinamičnega programiranja opiši postopek,
ki poišče najtežji spust v danem številskem trikotniku.
\end{vprasanje}
\begin{odgovor}
\end{odgovor}
\end{naloga}


\begin{naloga}{?}{Izpit OR 4.9.2012}
\begin{vprasanje}
Antonio se preživlja z graviranjem na zlato.
Ravnokar je dobil naročilo za graviranje
dveh zelo umetelno izdelanih zaročnih prstanov.
Če ne uspe izdelati prstanov v dogovorjenem času,
bo moral plačati $1600 €$ kazni.
Prstane lahko naroči pri zlatarju, ki ga je izbral naročnik.
Ta zlatar bo izdelavo vsakega prstana zaračunal po $100 €$,
varovana dostava pa (ne glede na število prstanov) stane $200 €$.
Ker prstani niso čisto homogeni,
je verjetnost, da graviranje uspe, le $50 \%$.
Ves ostali material (neuspeli prstani, opilki ipd.)
bo moral na koncu vrniti zlatarju (tako je določeno v pogodbi).
Ker je naročilo časovno omejeno, ima časa za največ 2 naročili pri zlatarju.
Koliko prstanov naj vsakič naroči, da bodo pričakovani stroški čim manjši?
\end{vprasanje}
\begin{odgovor}
\end{odgovor}
\end{naloga}
