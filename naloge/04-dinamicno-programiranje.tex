\subsection{Dinamično programiranje}

\begin{naloga}{?}{Vaje OR 6.4.2016}
\begin{vprasanje}
Na avtocestni odsek dolžine $M$ kilometrov
želimo postaviti oglasne plakate.
Dovoljene lokacije plakatov določa urad za oglaševanje
in so predstavljene s števili $x_1, x_2, \dots x_n$,
kjer $x_i$ ($1 \le i \le n$)
predstavlja oddaljenost od začetka odseka v kilometrih.
Profitabilnost oglasa na lokaciji $x_i$ določa vrednost $v_i$
($1 \le i \le n$).
Urad za oglaševanje podaja tudi omejitev,
da mora biti razdalja med oglasi vsaj $d$ kilometrov.
Oglase želimo postaviti tako, da bodo čim bolj profitabilni.
\begin{enumerate}[(a)]
\item Reši problem za parametre $M = 20$, $d = 5$, $n = 8$,
$(x_i)_{i=1}^n = (1, 2, 8, 10, 12,$ $14, 17, 20)$ in
$(v_i)_{i=1}^n = (8, 8, 12, 10, 7, 5, 6, 10)$.
\item Napiši rekurzivne enačbe za opisani problem.
\item Napiši algoritem,
ki poišče najbolj profitabilno postavitev oglasov za dane parametre.
Kakšna je njegova časovna zahtevnost?
\end{enumerate}

\end{vprasanje}
\begin{odgovor}
\end{odgovor}
\end{naloga}


\begin{naloga}{?}{Vaje OR 6.4.2016}
\begin{vprasanje}
Imamo nahrbtnik nosilnosti $M$ kilogramov.
Danih je $n$ objektov z vrednostmi $v_i$ in težami $t_i$ ($1 \le i \le n$).
Problem nahrbtnika sprašuje po izbiri predmetov
$I \subseteq \{1, 2, \dots, n\}$,
ki maksimizira njihovo skupno vrednost pri omejitvi $\sum_{i \in I} t_i \le M$.
\begin{enumerate}[(a)]
\item Napiši rekurzivne enačbe za opisani problem.
\item Z uporabo rekurzivnih enačb reši problem za parametre $M = 8$, $n = 8$,
$(v_i)_{i=1}^n = (9, 9, 8, 11, 10, 15,$ $3, 12)$ in
$(t_i)_{i=1}^n = (3, 5, 1, 4, 3, 8, 2, 7)$.
\end{enumerate}

\end{vprasanje}
\begin{odgovor}
\end{odgovor}
\end{naloga}


\begin{naloga}{?}{Vaje OR 6.4.2016}
\begin{vprasanje}
Dana je matrika $A = (a_{ij})_{i,j=1}^{m,n}$.
Poiskati želimo pot minimalne vsote,
ki se začne v levem zgornjem kotu (pri $a_{11}$)
in konča v desnem spodnjem kotu (pri $a_{mn}$).
Dovoljeni so zgolj premiki v desno in navzdol.
\begin{enumerate}[(a)]
\item Reši problem za matriko
$$
A = \begin{pmatrix}
131 & 673 & 234 & 103 &  18 \\
201 &  96 & 342 & 965 & 150 \\
630 & 803 & 746 & 422 & 111 \\
537 & 699 & 497 & 121 & 956 \\
805 & 732 & 524 &  37 & 332
\end{pmatrix} .
$$
\item Napiši rekurzivne enačbe za opisani problem.
\item Na osnovi rekurzivnih enačb napiši algoritem, ki reši opisani problem.
Oceni tudi njegovo časovno zahtevnost v odvisnosti od $m$ in $n$.
\end{enumerate}

\end{vprasanje}
\begin{odgovor}
\end{odgovor}
\end{naloga}


\begin{naloga}{?}{Vaje OR 6.4.2016}
\begin{vprasanje}
Dan je niz $S = a_1 a_2 \dots a_n$,
kjer so $a_i$ ($1 \le i \le n$) elementi neke končne abecede.
Nizu $a_j a_{j+1} \dots a_k$, kjer je $1 \le j \le k \le n$,
pravimo {\em strnjen podniz} niza $S$.
S pomočjo dinamičnega programiranja napiši algoritem,
ki določi najdaljši palindromski strnjen podniz v $S$.

\end{vprasanje}
\begin{odgovor}
\end{odgovor}
\end{naloga}


\begin{naloga}{?}{Vaje OR 6.4.2016}
\begin{vprasanje}
Dana je matrika $A = (a_{ij})_{i,j=1}^{m,n}$.
Poiskati želimo strnjeno podmatriko matrike $A$
z največjo vsoto komponent.
\begin{enumerate}[(a)]
\item Reši problem za matriko
$$
A = \begin{pmatrix}
 1 & -1 &  2 &  4 \\
-3 & -2 &  8 &  2 \\
-3 &  2 & -2 &  4 \\
 1 & -5 & -1 & -2
\end{pmatrix} .
$$
\item Napiši rekurzivne enačbe za opisani problem.
\item Napiši algoritem, ki reši opisani problem.
Oceni tudi njegovo časovno zahtevnost v odvisnosti od $m$ in $n$.
\end{enumerate}

\end{vprasanje}
\begin{odgovor}
\end{odgovor}
\end{naloga}


\begin{naloga}{?}{Vaje OR 6.4.2016}
\begin{vprasanje}
Na voljo imamo kovance z vrednostmi $1 = v_1 < v_2 < \cdots < v_n$
in vsoto $C$, ki jo želimo izplačati s kovanci.
Predpostavljamo, da imamo dovolj velik nabor kovancev.
\begin{enumerate}[(a)]
\item Poišči izplačilo z najmanjšim številom kovancev
za $C = 25$, $n = 4$ in $(v_i)_{i=1}^n = (1, 2, 5, 7)$.
\item S pomočjo dinamičnega programiranja reši problem v splošnem.
\end{enumerate}

\end{vprasanje}
\begin{odgovor}
\end{odgovor}
\end{naloga}


\begin{naloga}{?}{Vaje OR 6.4.2016}
\begin{vprasanje}
Na ulici je $n$ vrstnih hiš,
pri čemer je v $i$-ti hiši $c_i$ denarja.
Tat se odloča, katere izmed hiš naj oropa.
Vsak oropan stanovalec to sporoči svojim sosedom,
zato tat ne sme oropati dveh sosednjih hiš.
Ker je tat poslušal predmet Operacijske raziskave,
pozna dinamično programiranje.
Pokaži, kako naj tat določi, katere hiše naj oropa.

\end{vprasanje}
\begin{odgovor}
\end{odgovor}
\end{naloga}


\begin{naloga}{?}{Kolokvij OR 31.5.2012}
\begin{vprasanje}
Imamo hlod dolžine $\ell$,
ki bi ga radi razžagali na $n$ označenih mestih
$0 < x_1 < x_2 < \dots < x_n < \ell$.
Eno rezanje stane toliko, kolikor je dolžina hloda, ki ga režemo.
Ko hlod prerežemo, dobimo dva manjša hloda, ki ju režemo naprej.
Poiskati želimo zaporedje rezanj z najmanjšo ceno.
\begin{enumerate}[(a)]
\item Reši problem pri podatkih $\ell = 10$ in $(x)_{i=1}^4 = (3, 5, 7, 8)$.
\item S pomočjo dinamičnega programiranja reši problem v splošnem.
Oceni tudi njegovo časovno zahtevnost.
\end{enumerate}

\end{vprasanje}
\begin{odgovor}
\end{odgovor}
\end{naloga}


\begin{naloga}{Hillier, Lieberman}{\cite[Problem~11.3-1]{hl}}
\begin{vprasanje}
Lastnik verige $n$ trgovin z živili je kupil $m$ zabojev svežih jagod.
Naj bo $p_{ij}$ pričakovan dobiček v trgovini $j$,
če tja dostavimo $i$ zabojev.
Zanima nas, koliko zabojev naj gre v vsako trgovino,
da bomo imeli čim večji zaslužek.
Zaradi logističnih razlogov zabojev ne želimo deliti.
\begin{enumerate}[(a)]
\item Z dinamičnim programiranjem reši problem za podatke $m = 5$, $n = 3$
in $p_{ij}$ iz sledeče tabele:
$$
\begin{array}{c|ccc}
p_{ij} & 1 & 2 & 3 \\
\hline
0 &  0 &  0 &  0 \\
1 &  5 &  6 &  4 \\
2 &  9 & 11 &  9 \\
3 & 14 & 15 & 13 \\
4 & 17 & 19 & 18 \\
5 & 21 & 22 & 20 \\
\end{array}
$$
\item Napiši algoritem, ki reši opisani problem v splošnem.
\end{enumerate}

\end{vprasanje}
\begin{odgovor}
\end{odgovor}
\end{naloga}


\begin{naloga}{Hillier, Lieberman}{\cite[Problem~11.3-8]{hl}}
\begin{vprasanje}
Podjetje bo kmalu uvedlo nov izdelek na zelo konkurenčen trg,
zato trenutno pripravlja marketinško strategijo.
Odločili so se, da bodo izdelek uvedli v treh fazah.
V prvi fazi bodo pripravili posebno začetno ponudbo z močno znižano ceno,
da bi privabili zgodnje kupce.
Druga faza bo vključevala intenzivno oglaševalsko kampanjo,
da bi zgodnje kupce prepričali, naj izdelek še vedno kupujejo po redni ceni.
Znano je, da bo ob koncu druge faze
konkurenčno podjetje predstavilo svoj izdelek.
Zato bo v tretji fazi okrepljeno oglaševanje z namenom,
da bi preprečili beg strank h konkurenci.

Podjetje ima za oglaševanje na voljo $4$ milijone evrov,
ki jih želimo čim bolj učinkovito porabiti.
Naj bo $m$ tržni delež v procentih, pridobljen v prvi fazi,
$f_2$ delež, ohranjen po drugi fazi,
in $f_3$ delež, ohranjen po tretji fazi.
Maksimizirati želimo končni tržni delež, torej količino $m f_2 f_3$.

\begin{enumerate}[(a)]
\item Denimo, da želimo v vsaki fazi porabiti nek večkratnik milijona evrov,
pri čemer bomo pri prvi fazi porabili vsaj milijon evrov.
V spodnji tabeli so zbrani vplivi porabljenih količin
na vrednosti $m$, $f_2$ in $f_3$.
$$
\begin{array}{c|ccc}
M€ & m & f_2 & f_3 \\
\hline
0 &  - & 0.2 & 0.3 \\
1 & 20 & 0.4 & 0.5 \\
2 & 30 & 0.5 & 0.6 \\
3 & 40 & 0.6 & 0.7 \\
4 & 50 &   - &   - \\
\end{array}
$$
Kako naj razdelimo sredstva?
\item Denimo sedaj,
da lahko v vsaki fazi porabimo poljubno pozitivno količino denarja
(seveda glede na omejitev skupne porabe).
Naj bodo torej $x_1$, $x_2$ in $x_3$ količine denarja v milijonih evrov,
ki jih porabimo v prvi, drugi in tretji fazi.
Vpliv na tržni delež je podan s formulami
$$
m = x_1 (10 - x_1), \quad
f_2 = 0.4 + 0.1 x_2, \quad \text{in} \quad
f_3 = 0.6 + 0.07 x_3 .
$$
Kako naj sedaj razdelimo sredstva?
\end{enumerate}

\end{vprasanje}
\begin{odgovor}
\end{odgovor}
\end{naloga}


\begin{naloga}{?}{Izpit OR 26.6.2012}
\begin{vprasanje}
Nori profesor Boltežar stanuje v stolpnici z $n$ nadstropji,
oštevilčenimi od $1$ do $n$.
Nori stanovalci tega bloka radi mečejo cvetlične lončke z balkonov.
Boltežar bi rad ugotovil, katero je najvišje nadstropje,
s katerega lahko pade cvetlični lonček, ne da bi se razbil.
Jasno je, da če se lonček razbije pri padcu iz $k$-tega nadstropja,
potem se razbije tudi pri padcu s $(k+1)$-tega nadstropja.
Če bi Boltežar imel le en cvetlični lonček,
bi ga lahko metal po vrsti od najnižjega nadstropja navzgor,
dokler se ne bi razbil.
V najslabšem primeru bi lonček torej vrgel $n$ krat
(možno je, da bi lonček preživel tudi padec iz najvišjega nadstropja).

Ker ima Boltežar doma $k$ cvetličnih lončkov,
lahko do rezultata pride tudi z manjšim številom metov.
S pomočjo dinamičnega programiranja bi rad poiskal strategijo metanja,
ki bi minimizirala število potrebnih metov v najslabšem primeru.
\begin{enumerate}[(a)]
\item Napiši rekurzivne enačbe za opisani problem.
\item Napiši algoritem, ki reši opisani problem.
Oceni tudi njegovo časovno zahtevnost v odvisnosti od $n$ in $k$.
\end{enumerate}

\end{vprasanje}
\begin{odgovor}
\end{odgovor}
\end{naloga}


\begin{naloga}{Hillier, Lieberman}{\cite[Problem~11.2-2]{hl}}
\begin{vprasanje}
Vodja prodaje pri založniku učbenikov za fakulteto
ima na voljo $6$ trgovskih potnikov,
ki jim želi dodeliti eno od treh regij, v kateri bodo delovali.
V vsaki regiji mora delovati vsaj en trgovski potnik.
Naj bo $p_{ij}$ pričakovana porast v prodaji v regiji $j$,
če bo tam delovalo $i$ trgovskih potnikov:
$$
\begin{array}{c|ccc}
p_{ij} & 1 & 2 & 3 \\
\hline
1 & 35 & 21 & 28 \\
2 & 48 & 42 & 41 \\
3 & 70 & 56 & 63 \\
4 & 89 & 70 & 75 \\
\end{array}
$$
Reši problem s pomočjo dinamičnega programiranja.

\end{vprasanje}
\begin{odgovor}
\end{odgovor}
\end{naloga}


\begin{naloga}{Hillier, Lieberman}{\cite[Problem~11.3-16]{hl}}
\begin{vprasanje}
Dan je sledeči nelinearni program.
\begin{align*}
\max &\quad 2x_1^2 + 2x_2 + 4x_3 - x_3^2 \\[1ex]
2x_1 + x_2 + x_3 &\le 4 \\
x_1, x_2, x_3 &\ge 0
\end{align*}
Reši ga s pomočjo dinamičnega programiranja.

\end{vprasanje}
\begin{odgovor}
\end{odgovor}
\end{naloga}


\begin{naloga}{Hillier, Lieberman}{\cite[Problem~11.4-1]{hl}}
\begin{vprasanje}
Igralec na srečo bo odigral tri partije s svojimi prijatelji,
pri čemer lahko vsakič stavi na svojo zmago.
Stavi lahko katerokoli vsoto denarja, ki jo ima na voljo
-- če izgubi partijo, zastavljeno vsoto izgubi, sicer pa tako vsoto pridobi.
Pri vsaki partiji sta verjetnosti zmage in poraza enaki $1/2$.
Na začetku ima $75 €$, na koncu pa želi imeti $100 €$
(ker igra s prijatelji, noče imeti več kot toliko).

Z dinamičnim programiranjem poišči strategijo stavljenja,
ki maksimizira verjetnost, da bo na koncu imel natanko $100 €$.
\end{vprasanje}
\begin{odgovor}
\end{odgovor}
\end{naloga}
