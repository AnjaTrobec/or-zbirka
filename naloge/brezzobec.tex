\begin{naloga}{?}{Kolokvij OR 9.5.2013}
\begin{vprasanje}
Gradbinec in samooklicani arhitekt Brezzobec se je odločil,
da bo postavil zelo posebno hišo.
Gradnja bo imela sedem glavnih faz,
ki so opisane v tabeli~\tab.
\begin{enumerate}[(a)]
\item Kdaj je lahko hiša najhitreje zgrajena?
Katere faze so kritične?
\item Koliko je kritičnih poti in katere so?
\item Katero opravilo je najmanj kritično?
Najmanj kritično je opravilo, katerega trajanje lahko največ podaljšamo,
ne da bi vplivali na trajanje gradnje.
\item Brezzobčev brat je ponudil pomoč pri največ eni fazi gradnje.
Slovi po tem, da pri fazi, pri kateri pomaga, zmanjša čas izvajanja za $10\%$.
Pri kateri fazi naj pomaga, da bo čas gradnje čim krajši?
\end{enumerate}

\begin{tabela}
\makebox[\textwidth][c]{
\begin{tabular}{c|l|c|c||c|c}
faza & opis & trajanje & pogoj &
min.~trajanje & cena za dan manj \\
\hline
$a$ & gradnja kleti & 10 dni & / & 7 dni & 200 \\
$b$ & gradnja pritličja & 6 dni & $a$ & 5 dni & 100 \\
$c$ & gradnja prvega nadstropja & 7 dni & $b, f$ & 5 dni & 150 \\
$d$ & gradnja strehe & 8 dni & $c, e$ & 6 dni & 160 \\
$e$ & gradnja desnega podpornega stebra & 13 dni & $a$ & 9 dni & 250 \\
$f$ & gradnja glavnega podpornega stebra & 14 dni & / & 11 dni & 240 \\
$g$ & gradnja baročnega stolpa pred hišo & 30 dni & / & 25 dni & 300 \\
\end{tabular}
}
\caption{Podatki za nalogi~\nal (prvi štirje stolpci)
in~\nal[brezzobec-pert].}
\end{tabela}
\end{vprasanje}

\begin{odgovor}
\begin{enumerate}[(a)]
\item Projekt predstavimo z usmerjenim acikličnim grafom $G = (V, E)$,
kjer vozlišča predstavljajo opravila,
posebej pa dodamo še začetno vozlišče $s$ in končno vozlišče $t$.
Opravilo $u$ povežemo z opravilom $v$,
če je $u$ pogoj za začetek opravila $v$.
Poleg tega vozlišče $s$ povežemo z opravili, ki nimajo predpogojev;
opravila, ki niso predpogoj za katero drugo opravilo,
pa povežemo z vozliščem $t$.
Povezave iz opravila $u$ utežimo s trajanjem opravila $u$,
na povezave iz $s$ pa postavimo utež $0$.

Na grafu $G$ uporabimo algoritem {\sc NajdaljšaPot} iz naloge~\res[topo]{}(c)
z začetkom v $s$ in tako za vsako opravilo dobimo najzgodnejši čas,
ko ga lahko začnemo.
Dolžina najdaljše poti do vozlišča $t$
predstavlja najkrajše možno trajanje ce\-lot\-ne\-ga projekta.
Nato na obratnem grafu $G' = (V, E')$,
kjer je $E' = \set{vu}{uv \in E}$ množica obratnih povezav z enakimi utežmi,
še enkrat uporabimo algoritem {\sc NajdaljšaPot}, tokrat z začetkom v $t$,
in dobljene razdalje odštejemo od najkrajšega možnega trajanja projekta.
Tako za vsako opravilo dobimo še najpoznejši možen čas začetka,
da se celotno trajanje projekta ne poveča.
Pri kritičnih opravilih sta oba časa enaka.

Projekt lahko predstavimo z uteženim grafom s slike~\fig,
iz katerega je razvidna topološka ureditev $s, a, f, g, b, e, c, d, e$.
V tabeli~\tab[brezzobec-resitev] so podani rezultati,
dobljeni z zgornjim postopkom.
Vidimo, da je najkrajše trajanje projekta $31$ dni,
kritična opravila pa so $a, e, d$.

\item Opazimo, da vsa kritična opravila ležijo na isti poti od $s$ do $t$.
Imamo torej natanko eno kritično pot $s - a - e - d - t$.

\item Iz tabele~\tab[brezzobec-resitev] je razvidno,
da je najmanj kritično opravilo $f$,
saj ga lahko brez podaljševanja celotnega trajanja podaljšamo za $4$ dni,
kar je več kot katerokoli drugo opravilo.

\item Skrajšati želimo katero od kritičnih opravil,
saj sicer ne bomo skrajšali celotnega trajanja projekta.
Ker ima pot $s - g - t$ dolžino $30$ dni,
bomo lahko celotno trajanje skrajšali za največ $1$ dan.
Izplača se nam torej skrajšati katerega od kritičnih opravil,
ki traja vsaj $10$ dni -- taki sta $a$ in $e$.
\end{enumerate}
%
\begin{slika}
\pgfslika
\podnaslov{Graf odvisnosti med opravili}
\end{slika}
%
\begin{tabela}
\setlabel{brezzobec-resitev}
$$
\begin{array}{r|ccccccccc}
& s & a & f & g & b & e & c & d & t \\ \hline
\text{najzgodnejši začetek} &
0 & 0_s & 0_s & 0_s & 10_a & 10_a & 16_b & 23_e & 31_d \\
\text{najpoznejši začetek} &
0_a & 0_e & 4_c & 1_t & 12_c & 10_d & 18_d & 23_t & 31 \\
\text{razlika} &
0 & 0^* & 4 & 1 & 2 & 0^* & 2 & 0^* & 0
\end{array}
$$
\podnaslov{Razporejanje opravil}
\end{tabela}
\end{odgovor}
\end{naloga}
