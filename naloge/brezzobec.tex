\begin{naloga}{?}{Kolokvij OR 9.5.2013}
\begin{vprasanje}
Gradbinec in samooklicani arhitekt Brezzobec se je odločil,
da bo postavil zelo posebno hišo.
Gradnja bo imela sedem glavnih faz,
ki so opisane v tabeli~\tab{}.
\begin{enumerate}[(a)]
\item Kdaj je lahko hiša najhitreje zgrajena?
Katere faze so kritične?
\item Koliko je kritičnih poti in katere so?
\item Katero opravilo je najmanj kritično?
Najmanj kritično je opravilo, katerega trajanje lahko največ podaljšamo,
ne da bi vplivali na trajanje gradnje.
\item Brezzobčev brat je ponudil pomoč pri največ eni fazi gradnje.
Slovi po tem, da pri fazi, pri kateri pomaga, zmanjša čas izvajanja za $10\%$.
Pri kateri fazi naj pomaga, da bo čas gradnje čim krajši?
\end{enumerate}

\begin{tabela}
\makebox[\textwidth][c]{
\begin{tabular}{c|l|c|c||c|c}
faza & opis & trajanje & pogoj &
min.~trajanje & cena za dan manj \\
\hline
A & gradnja kleti & 10 dni & / & 7 dni & 200 \\
B & gradnja pritličja & 6 dni & A & 5 dni & 100 \\
C & gradnja prvega nadstropja & 7 dni & B, F & 5 dni & 150 \\
D & gradnja strehe & 8 dni & C, E & 6 dni & 160 \\
E & gradnja desnega podpornega stebra & 13 dni & A & 9 dni & 250 \\
F & gradnja glavnega podpornega stebra & 14 dni & / & 11 dni & 240 \\
G & gradnja baročnega stolpa pred hišo & 30 dni & / & 25 dni & 300 \\
\end{tabular}
}
\caption{Podatki za nalogi~\nal{} (prvi štirje stolpci)
in~\nal{brezzobec-pert}.}
\end{tabela}
\end{vprasanje}
\begin{odgovor}
\end{odgovor}
\end{naloga}
