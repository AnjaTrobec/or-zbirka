\begin{naloga}{?}{Vaje OR 6.4.2016}
\begin{vprasanje}
Na ulici je $n$ vrstnih hiš,
pri čemer je v $i$-ti hiši $c_i$ denarja.
Tat se odloča, katere izmed hiš naj oropa.
Vsak oropan stanovalec to sporoči svojim sosedom,
zato tat ne sme oropati dveh sosednjih hiš.
Ker je tat poslušal predmet Operacijske raziskave,
pozna dinamično programiranje.
Pokaži, kako naj tat določi, katere hiše naj oropa.
\end{vprasanje}

\begin{odgovor}
Naj bo $p_i$ največja količina denarja,
ki jo lahko tat pridobi, če oropa $i$-to hišo.
Določimo začetna pogoja in rekurzivne enačbe.
\begin{align*}
p_0 &= 0, \qquad p_1 = c_1 \\
p_i &= \max\{p_{i-1}, p_{i-2} + c_i\} \qquad (2 \le i \le n)
\end{align*}
Vrednosti $p_i$ ($0 \le i \le n$) računamo naraščajoče po indeksu $n$.
Največjo količino denarja dobimo s $p^* = p_n$.
Da poiščemo, katere hiše se tatu izplača oropati,
pregledamo vrednosti $p_i$ v padajočem vrstnem redu ($n \ge i \ge 1$):
če velja $p_i > p_{i-1}$,
naj $i$-to hišo oropa in pregled nadaljujemo pri $p_{i-2}$,
sicer pa naj $i$-te hiše ne oropa in pregled nadaljujemo pri $p_{i-1}$.
\end{odgovor}
\end{naloga}
