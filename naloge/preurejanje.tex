\begin{naloga}{?}{Izpit OR 24.6.2015}
\begin{vprasanje}
Dan je algoritem, ki na vhod sprejme seznam števil
$A = [a_1, a_2, \dots, a_n]$
in vrne preurejen seznam $A$,
pri čemer preurejanje poteka tako:
\begin{itemize}
\item Algoritem najprej primerja in uredi prvi dve števili v tabeli,
nato drugo in tretje število, nato tretje in četrto število, \dots,
in nazadnje še $(n-1)$-to in $n$-to število.
\item Celoten opisan postopek iz prejšnje alineje
algoritem ponavlja tako dolgo,
dokler se med postopkom vsaj dvakrat dva elementa tabele zamenjata
(tj., algoritem se ustavi,
če se v ponovitvi prejšnje alineje zgodi največ ena zamenjava).
\end{itemize}

Ponazorimo delovanje algoritma na primeru vhoda $A = [\pi, 7, 3, \sqrt{2}]$.
Potem se med izvajanjem algoritma seznam spreminja tako:
$$
\begin{array}{cccccccc}
A &\to& [\pi, 7, 3, \sqrt{2}]
  &\to& [\pi, 3, 7, \sqrt{2}]
  &\to& [\pi, 3, \sqrt{2}, 7] &\to \\
  &   & [3, \pi, \sqrt{2}, 7]
  &\to& [3, \sqrt{2}, \pi, 7]
  &\to& [3, \sqrt{2}, \pi, 7] &\to \\
  &   & [\sqrt{2}, 3, \pi, 7]
  &\to& [\sqrt{2}, 3, \pi, 7]
  &\to& [\sqrt{2}, 3, \pi, 7]
\end{array}
$$

\begin{enumerate}[(a)]
\item Podaj primer vhoda, kjer algoritem ne vrne urejene tabele.
Na tem primeru zapiši zaporedje tabel med izvajanjem algoritma.

\item Gornji algoritem zapiši v psevdokodi (ne prirejaj/izboljšuj algoritma).
Kakšna je njegova časovna zahtevnost?
\namig{ali je po tem, ko se prvič izvede prva alineja v opisu algoritma,
katero število že na istem mestu, kot bi bilo v urejeni tabeli?}

\item Recimo, da na vhod dobimo tabelo dolžine $n$.
Kakšni morajo biti vhodni podatki,
da algoritem naredi najmanjše število primerjav?
Kolikšno je v tem primeru to število?
\end{enumerate}
\end{vprasanje}
\begin{odgovor}
\end{odgovor}
\end{naloga}
