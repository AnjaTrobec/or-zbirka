\begin{naloga}{David Gajser}{Izpit OR 24.6.2015}
\begin{vprasanje}
Dan je algoritem, ki na vhod sprejme seznam števil
$A = [a_1, a_2, \dots, a_n]$
in vrne preurejen seznam $A$,
pri čemer preurejanje poteka tako:
\begin{itemize}
\item Algoritem najprej primerja in uredi prvi dve števili v tabeli,
nato drugo in tretje število, nato tretje in četrto število, \dots,
in nazadnje še $(n-1)$-to in $n$-to število.
\item Celoten opisan postopek iz prejšnje alineje
algoritem ponavlja tako dolgo,
dokler se med postopkom vsaj dvakrat dva elementa tabele zamenjata
(tj., algoritem se ustavi,
če se v ponovitvi prejšnje alineje zgodi največ ena zamenjava).
\end{itemize}

Ponazorimo delovanje algoritma na primeru vhoda $A = [\pi, 7, 3, \sqrt{2}]$.
Potem se med izvajanjem algoritma seznam spreminja tako:
$$
\begin{array}{cccccccc}
A &\to& [\pi, 7, 3, \sqrt{2}]
  &\to& [\pi, 3, 7, \sqrt{2}]
  &\to& [\pi, 3, \sqrt{2}, 7] &\to \\
  &   & [3, \pi, \sqrt{2}, 7]
  &\to& [3, \sqrt{2}, \pi, 7]
  &\to& [3, \sqrt{2}, \pi, 7] &\to \\
  &   & [\sqrt{2}, 3, \pi, 7]
  &\to& [\sqrt{2}, 3, \pi, 7]
  &\to& [\sqrt{2}, 3, \pi, 7]
\end{array}
$$

\begin{enumerate}[(a)]
\item Podaj primer vhoda, kjer algoritem ne vrne urejene tabele.
Na tem primeru zapiši zaporedje tabel med izvajanjem algoritma.

\item Gornji algoritem zapiši v psevdokodi (ne prirejaj/izboljšuj algoritma).
Kakšna je njegova časovna zahtevnost?
\namig{ali je po tem, ko se prvič izvede prva alineja v opisu algoritma,
katero število že na istem mestu, kot bi bilo v urejeni tabeli?}

\item Recimo, da na vhod dobimo tabelo dolžine $n$.
Kakšni morajo biti vhodni podatki,
da algoritem naredi najmanjše število primerjav?
Kolikšno je v tem primeru to število?
\end{enumerate}
\end{vprasanje}

\begin{odgovor}
\begin{enumerate}[(a)]
\item Primer vhoda, ki ga algoritem ne uredi pravilno, je $[2, 3, 1]$.
V prvem obhodu algoritem najprej primerja $2$ in $3$ ter ju ne zamenja,
nato pa še zamenja $3$ in $1$.
Ker je v obhodu opravil samo eno zamenjavo,
se algoritem ustavi in vrne seznam $[2, 1, 3]$.

Primer lahko posplošimo na sezname $A[1 \dots n]$ poljubne dolžine,
za katere velja $A[i-1] \le A[i]$ ($2 \le i \le n-1$) in $A[n] < A[n-2]$.
Tako kot v prejšnjem primeru algoritem zamenja le zadnja dva elementa,
kar pa ne zadostuje za pravilno ureditev.

\item Zapišimo opisani algoritem.
\begin{small}
\begin{algorithmic}
\Require{$A[1 \dots n]$ seznam števil}
\State $k \gets 2$
\While{$k \ge 2$}
    \State $k \gets 0$
    \For{$i = 2, \dots, n$}
        \If{$A[i-1] > A[i]$}
            \State $A[i-1], A[i] \gets A[i], A[i-1]$
            \State $k \gets k+1$
        \EndIf
    \EndFor
\EndWhile
\State \Return $A$
\end{algorithmic}
\end{small}
Prepričamo se lahko,
da po $j$-tem obhodu zunanje zanke zadnjih $j$ elementov na mestu.
V najslabšem primeru algoritem torej opravi $O(n^2)$ korakov,
kar tudi dosežemo, če na vhod dobi padajoče urejen seznam.
Časovna zahtevnost algoritma je torej $O(n^2)$.

\item Število primerjav je odvisno od števila obhodov zanke {\bf while}.
En sam obhod dosežemo pri že urejenih seznamih in pri primerih iz točke (a).
V tem primeru algoritem opravi $n$ primerjav elementov seznama $A$
in dve primerjavi pri vstopu v zanko {\bf while}
(pri zadnji se zanka konča).
\end{enumerate}
\end{odgovor}
\end{naloga}
