\begin{naloga}{Janoš Vidali}{Izpit OR 5.7.2018}
\begin{vprasanje}
V trgovini s pohištvom vsak mesec prodajo $20$ sedežnih garnitur.
Z vsakim naročilom imajo $750 €$ stroškov,
pri tem pa vse naročeno blago dobijo hkrati.
Skladiščenje posamezne sedežne garniture jih stane $10 €$ na mesec,
primanjkljaj za en kos pa jih stane $50 €$ na mesec.

\begin{enumerate}[(a)]
\item Kako pogosto naj v trgovini naročajo sedežne garniture,
da bodo stroški čim manjši.
Kako veliko skladišče morajo imeti?
Izračunaj tudi enotske stroške.

\item Izračunaj največji dovoljeni primanjkljaj.
Koliko sedežnih garnitur naj vsakič naročijo?

\item V trgovini dobijo ponudbo za uporabo drugega skladišča,
v katerem imajo za po\-sa\-mez\-no se\-dež\-no garnituro
le $6 €$ stroškov na mesec,
vendar lahko hrani največ $40$ sedežnih garnitur.
Kako naj organizirajo naročanje,
če namesto prvotnega uporabljajo to skladišče?
Ali se jim ga splača uporabiti?
\namig{izpelji formulo za enotske stroške
in poišči optimalen interval naročanja pri fiksni velikosti skladišča.}
\end{enumerate}
\end{vprasanje}
\begin{odgovor}
\end{odgovor}
\end{naloga}
