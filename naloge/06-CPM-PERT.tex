\subsection{CPM/PERT}

\begin{naloga}{?}{Kolokvij OR 9.5.2013}
\begin{vprasanje}[nal:brezzobec]
Gradbinec in samooklicani arhitekt Brezzobec se je odločil,
da bo postavil zelo posebno hišo.
Gradnja bo imela sedem glavnih faz,
ki so opisane v tabeli~\ref{tab:brezzobec}.
\begin{enumerate}[(a)]
\item Kdaj je lahko hiša najhitreje zgrajena?
Katere faze so kritične?
\item Koliko je kritičnih poti in katere so?
\item Katero opravilo je najmanj kritično?
Najmanj kritično je opravilo, katerega trajanje lahko največ podaljšamo,
ne da bi vplivali na trajanje gradnje.
\item Brezzobčev brat je ponudil pomoč pri največ eni fazi gradnje.
Slovi po tem, da pri fazi, pri kateri pomaga, zmanjša čas izvajanja za $10\%$.
Pri kateri fazi naj pomaga, da bo čas gradnje čim krajši?
\end{enumerate}

\begin{table}[t]
\makebox[\textwidth][c]{
\begin{tabular}{c|l|c|c||c|c}
faza & opis & trajanje & pogoj &
min.~trajanje & cena za dan manj \\
\hline
A & gradnja kleti & 10 dni & / & 7 dni & 200 \\
B & gradnja pritličja & 6 dni & A & 5 dni & 100 \\
C & gradnja prvega nadstropja & 7 dni & B, F & 5 dni & 150 \\
D & gradnja strehe & 8 dni & C, E & 6 dni & 160 \\
E & gradnja desnega podpornega stebra & 13 dni & A & 9 dni & 250 \\
F & gradnja glavnega podpornega stebra & 14 dni & / & 11 dni & 240 \\
G & gradnja baročnega stolpa pred hišo & 30 dni & / & 25 dni & 300 \\
\end{tabular}
}
\caption{Podatki za nalogi~\ref{nal:brezzobec} in~\ref{nal:brezzobec-pert}.}
\label{tab:brezzobec}
\end{table}
\end{vprasanje}
\begin{odgovor}
\end{odgovor}
\end{naloga}


\begin{naloga}{Janoš Vidali}{Vaje OR 28.5.2018}
\begin{vprasanje}[nal:brezzobec-pert]
Brezzobčev bratranec ima podjetje, ki lahko pomaga pri gradnji,
vendar za vsak dan krajšanja posamezne faze zahteva ustrezno plačilo
(glej zadnja dva stolpca tabele~\ref{tab:brezzobec}).
Brezzobca zanima način,
kako bi s čim manjšimi stroški čas gradnje zmanjšal na $27$ dni.
Zapiši linearni program za ta problem.
\end{vprasanje}
\begin{odgovor}
\end{odgovor}
\end{naloga}


\begin{naloga}{?}{Izpit OR 19.5.2015}
\begin{vprasanje}[nal:palacinke]
Dinamika priprave dveh palačink z dvema kuharjema
je opisana v tabeli~\ref{tab:palacinke}.
\begin{enumerate}[(a)]
\item Topološko uredi ustrezni graf in ga nariši.
\item Določi kritična opravila in kritično pot ter trajanje priprave.
\item Katero opravilo je najmanj kritično?
\item Določi razpored opravil,
pri čemer en kuhar prevzame opravila na kritični poti,
drugi pa naj čim kasneje začne in čim prej konča.
\end{enumerate}

\begin{table}[t]
\centering
\begin{tabular}{c|l|c|c}
& aktivnost & trajanje & predhodna opravila \\
\hline
A & nakup moke, jajc in mleka & 5 min & C \\
B & rezanje sira & 3 min & / \\
C & vožnja do trgovine & 5 min & / \\
D & čiščenje mešalnika & 2 min & E \\
E & mešanje sestavin & 5 min & A \\
F & pečenje prve palačinke & 2 min & E \\
G & mazanje prve palačinke z marmelado & 1 min & F, J \\
H & pečenje palačinke (s sirom) & 3 min & B, F \\
I & pomivanje posode & 8 min & G, H \\
J & odpiranje marmelade & 1 min & / \\
\end{tabular}
\caption{Podatki za nalogo~\ref{nal:palacinke}.}
\label{tab:palacinke}
\end{table}
\end{vprasanje}
\begin{odgovor}
\end{odgovor}
\end{naloga}


\begin{naloga}{Hillier, Lieberman}{\cite[Problem~10.4-4]{hl}}
\begin{vprasanje}[nal:viadukt]
Pri gradbenem podjetju razmišljajo,
da bi se prijavili na razpis za prenovo avtocestnega viadukta.
Identificirali so pet nalog, ki so opisane v tabeli~\ref{tab:viadukt}.

Če bodo izbrani za izvedbo del, si obetajo zaslužek v višini $250.000 €$.
Če del ne bodo končali v roku $11$ tednov,
bodo morali plačati pogodbeno kazen v višini $500.000 €$.
\begin{enumerate}[(a)]
\item Topološko uredi ustrezni graf in ga nariši.
\item Za vsako opravilo določi pričakovano trajanje in varianco.
\item Določi pričakovano kritično pot ter trajanje izvedbe.
\item Oceni verjetnost, da se bo projekt zaključil v $11$ tednih.
Naj se podjetje prijavi na razpis?
\end{enumerate}

\begin{table}[t]
\makebox[\textwidth][c]{
\begin{tabular}{c|c|c|c|c}
naloga & najkrajše trajanje & najbolj verjetno trajanje
& najdaljše trajanje & predhodna opravila \\
\hline
A & $3$ tedni & $4$ tedni  & $5$ tednov & / \\
B & $2$ tedna & $2$ tedna  & $2$ tedna  & A \\
C & $3$ tedni & $5$ tednov & $6$ tednov & B \\
D & $1$ teden & $3$ tedni  & $5$ tednov & A \\
E & $2$ tedna & $3$ tedni  & $5$ tednov & B, D
\end{tabular}
}
\caption{Podatki za nalogo~\ref{nal:viadukt}.}
\label{tab:viadukt}
\end{table}
\end{vprasanje}
\begin{odgovor}
\end{odgovor}
\end{naloga}


\begin{naloga}{Sergio Cabello}{Vaje OR 14.12.2016}
\begin{vprasanje}
Zapiši celoštevilski linearni program, ki določi topološko urejanje grafa.
\end{vprasanje}
\begin{odgovor}
\end{odgovor}
\end{naloga}


\begin{naloga}{Sergio Cabello}{Vaje OR 14.12.2016}
\begin{vprasanje}
Zapiši algoritem, ki ugotovi, ali ima graf več kot eno topološko ureditev.
\end{vprasanje}
\begin{odgovor}
\end{odgovor}
\end{naloga}
