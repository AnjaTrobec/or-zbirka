\razdelek{CPM/PERT}

\begin{naloga}{?}{Kolokvij OR 9.5.2013}
\begin{vprasanje}[brezzobec]
Gradbinec in samooklicani arhitekt Brezzobec se je odločil,
da bo postavil zelo posebno hišo.
Gradnja bo imela sedem glavnih faz,
ki so opisane v tabeli~\tab{}.
\begin{enumerate}[(a)]
\item Kdaj je lahko hiša najhitreje zgrajena?
Katere faze so kritične?
\item Koliko je kritičnih poti in katere so?
\item Katero opravilo je najmanj kritično?
Najmanj kritično je opravilo, katerega trajanje lahko največ podaljšamo,
ne da bi vplivali na trajanje gradnje.
\item Brezzobčev brat je ponudil pomoč pri največ eni fazi gradnje.
Slovi po tem, da pri fazi, pri kateri pomaga, zmanjša čas izvajanja za $10\%$.
Pri kateri fazi naj pomaga, da bo čas gradnje čim krajši?
\end{enumerate}

\begin{tabela}
\makebox[\textwidth][c]{
\begin{tabular}{c|l|c|c||c|c}
faza & opis & trajanje & pogoj &
min.~trajanje & cena za dan manj \\
\hline
A & gradnja kleti & 10 dni & / & 7 dni & 200 \\
B & gradnja pritličja & 6 dni & A & 5 dni & 100 \\
C & gradnja prvega nadstropja & 7 dni & B, F & 5 dni & 150 \\
D & gradnja strehe & 8 dni & C, E & 6 dni & 160 \\
E & gradnja desnega podpornega stebra & 13 dni & A & 9 dni & 250 \\
F & gradnja glavnega podpornega stebra & 14 dni & / & 11 dni & 240 \\
G & gradnja baročnega stolpa pred hišo & 30 dni & / & 25 dni & 300 \\
\end{tabular}
}
\caption{Podatki za nalogi~\nal{} (prvi štirje stolpci)
in~\nal{brezzobec-pert}.}
\end{tabela}
\end{vprasanje}
\begin{odgovor}
\end{odgovor}
\end{naloga}


\begin{naloga}{Janoš Vidali}{Vaje OR 28.5.2018}
\begin{vprasanje}[brezzobec-pert]
Brezzobčev bratranec ima podjetje, ki lahko pomaga pri gradnji,
vendar za vsak dan krajšanja posamezne faze zahteva ustrezno plačilo
(glej zadnja dva stolpca tabele~\tab{brezzobec}).
Brezzobca zanima način,
kako bi s čim manjšimi stroški čas gradnje zmanjšal na $27$ dni.
Zapiši linearni program za ta problem.
\end{vprasanje}
\begin{odgovor}
\end{odgovor}
\end{naloga}


\begin{naloga}{?}{Izpit OR 19.5.2015}
\begin{vprasanje}[palacinke]
Dinamika priprave dveh palačink z dvema kuharjema
je opisana v tabeli~\tab{}.
\begin{enumerate}[(a)]
\item Topološko uredi ustrezni graf in ga nariši.
\item Določi kritična opravila in kritično pot ter trajanje priprave.
\item Katero opravilo je najmanj kritično?
\item Določi razpored opravil,
pri čemer en kuhar prevzame opravila na kritični poti,
drugi pa naj čim kasneje začne in čim prej konča.
\end{enumerate}

\begin{tabela}
\begin{tabular}{c|l|c|c}
& aktivnost & trajanje & predhodna opravila \\
\hline
A & nakup moke, jajc in mleka & 5 min & C \\
B & rezanje sira & 3 min & / \\
C & vožnja do trgovine & 5 min & / \\
D & čiščenje mešalnika & 2 min & E \\
E & mešanje sestavin & 5 min & A \\
F & pečenje prve palačinke & 2 min & E \\
G & mazanje prve palačinke z marmelado & 1 min & F, J \\
H & pečenje palačinke (s sirom) & 3 min & B, F \\
I & pomivanje posode & 8 min & G, H \\
J & odpiranje marmelade & 1 min & / \\
\end{tabular}
\podnaslov{Podatki}
\end{tabela}
\end{vprasanje}
\begin{odgovor}
\end{odgovor}
\end{naloga}


\begin{naloga}{Hillier, Lieberman}{\cite[Problem~10.4-4]{hl}}
\begin{vprasanje}[viadukt]
Pri gradbenem podjetju razmišljajo,
da bi se prijavili na razpis za prenovo avtocestnega viadukta.
Identificirali so pet nalog, ki so opisane v tabeli~\tab{}.

Če bodo izbrani za izvedbo del, si obetajo zaslužek v višini $250.000 €$.
Če del ne bodo končali v roku $11$ tednov,
bodo morali plačati pogodbeno kazen v višini $500.000 €$.
\begin{enumerate}[(a)]
\item Topološko uredi ustrezni graf in ga nariši.
\item Za vsako opravilo določi pričakovano trajanje in varianco.
\item Določi pričakovano kritično pot ter trajanje izvedbe.
\item Oceni verjetnost, da se bo projekt zaključil v $11$ tednih.
Naj se podjetje prijavi na razpis?
\end{enumerate}

\begin{tabela}
\makebox[\textwidth][c]{
\begin{tabular}{c|c|c|c|c}
naloga & najkrajše trajanje & najbolj verjetno trajanje
& najdaljše trajanje & predhodna opravila \\
\hline
A & $3$ tedni & $4$ tedni  & $5$ tednov & / \\
B & $2$ tedna & $2$ tedna  & $2$ tedna  & A \\
C & $3$ tedni & $5$ tednov & $6$ tednov & B \\
D & $1$ teden & $3$ tedni  & $5$ tednov & A \\
E & $2$ tedna & $3$ tedni  & $5$ tednov & B, D
\end{tabular}
}
\podnaslov{Podatki}
\end{tabela}
\end{vprasanje}
\begin{odgovor}
\end{odgovor}
\end{naloga}


\begin{naloga}{Janoš Vidali}{Izpit OR 31.1.2017}
\begin{vprasanje}[splet]
Izdelati želimo terminski plan za izdelavo spletne aplikacije.
V tabeli~\tab{} so zbrana opravila pri izdelavi.

\begin{enumerate}[(a)]
\item Topološko uredi ustrezni graf in ga nariši.
\item Določi kritična opravila in kritično pot ter čas izdelave.
\item Katero opravilo je najmanj kritično?
Najmanj kritično jo opravilo, katerega trajanje lahko najbolj podaljšamo,
ne da bi vplivali na trajanje izdelave.
\end{enumerate}

\begin{tabela}
\makebox[\textwidth][c]{
\begin{tabular}{c|l|c|c}
opravilo & opis & trajanje & pogoji \\
\hline
A & natančna opredelitev funkcionalnosti & 15 dni & / \\
B & programiranje uporabniškega vmesnika & 40 dni & K \\
C & programiranje skrbniškega vmesnika & 25 dni & A, M \\
D & programiranje strežniškega dela & 30 dni & A, M \\
E & integracija uporabniškega vmesnika s strežnikom & 20 dni & B, D \\
F & alfa testiranje & 20 dni & C, E \\
G & beta testiranje & 30 dni & F, H \\
H & pridobivanje testnih uporabnikov & 45 dni & A \\
I & vnos zadnjih popravkov & 10 dni & G \\
J & izdelava uporabniške dokumentacije & 35 dni & B \\
K & dizajniranje uporabniškega vmesnika & 15 dni & A \\
L & nabava računalniške opreme & 20 dni & / \\
M & postavitev strežnikov & 10 dni & L \\
\end{tabular}
}
\podnaslov{Podatki}
\end{tabela}
\end{vprasanje}
\begin{odgovor}
\end{odgovor}
\end{naloga}


\begin{naloga}{Janoš Vidali}{Izpit OR 29.8.2017}
\begin{vprasanje}[studij]
Peter zaključuje študij na Fakulteti za alternativno znanost.
Opravil je že vse obvezne predmete,
za pristop k zaključnemu izpitu pa mora opraviti še nekaj izbirnih predmetov.
To bi rad storil čim hitreje.
V tabeli~\tab{} so našteti izbirni predmeti
skupaj s trajanjem (v tednih, od pristopa do uspešnega opravljanja)
in predmeti, h katerim lahko pristopi po uspešnem opravljanju.
K predmetom A, C in F lahko pristopi že takoj,
za pristop k ostalim predmetom pa zadostuje, če je opravljen eden od pogojev.

\begin{enumerate}[(a)]
\item Topološko uredi ustrezni graf in ga nariši.
\item Katere predmete naj Peter opravi,
da bo lahko čim prej pristopil k zaključnemu izpitu?
Koliko časa bo za to potreboval?
Natančno opiši postopek iskanja odgovora.
\end{enumerate}

\begin{tabela}
\makebox[\textwidth][c]{
\begin{tabular}{c|lcc}
Oznaka & Predmet & Trajanje & Zadosten pogoj za \\ \hline
A & Alternativna zgodovina   &  5 & I, J \\
B & Astrološki praktikum     &  7 & D, G \\
C & Diskretna numerologija   &  9 & B    \\
D & Filozofija magije        &  4 & Z    \\
E & Kvantno pravo            &  8 & B, H \\
F & Postmoderna ekonomija    &  4 & E    \\
G & Telepatija in telekineza &  5 & Z    \\
H & Teorija antigravitacije  &  4 & G    \\
I & Teorije zarote           &  5 & H    \\
J & Ufologija II             & 10 & K    \\
K & Uvod v kriptozoologijo   &  6 & Z    \\ \hline
Z & Zaključni izpit          &  / & /
\end{tabular}
}
\podnaslov{Podatki}
\end{tabela}
\end{vprasanje}
\begin{odgovor}
\end{odgovor}
\end{naloga}


\begin{naloga}{Janoš Vidali}{Kolokvij OR 11.6.2018}
\begin{vprasanje}[konferenca]
Izdelati želimo terminski plan za organizacijo konference.
V tabeli~\tab{} so zbrana opravila pri organizaciji.

\begin{enumerate}[(a)]
\item Topološko uredi ustrezni graf in ga nariši.
Za trajanja opravil vzemi pričakovana trajanja po modelu PERT.

\item Določi pričakovano kritično pot in čas izdelave.

\item Katero opravilo je (ob zgornjih predpostavkah) najmanj kritično?
Najmanj kritično je opravilo, katerega trajanje lahko najbolj podaljšamo,
ne da bi vplivali na celotno trajanje izvedbe.

\item Določi variance trajanj opravil in oceni ve\-rjet\-nost,
da bo izvedba trajala manj kot $55$ dni.
\end{enumerate}

\begin{tabela}
\makebox[\textwidth][c]{
\begin{tabular}{c|cc|b{2cm}b{2.7cm}b{2cm}}
& Opravilo & Pogoji & Minimalno trajanje
& Najbolj verjetno trajanje & Maksimalno trajanje \\ \hline
A & Izbira lokacije & / & 10 dni & 13 dni & 22 dni \\
B & Rezervacija sob za goste & F & 13 dni & 22 dni & 25 dni \\
C & Dogovarjanje za cene hotelskih sob & A & 3 dni & 6 dni & 9 dni \\
D & Naročilo hrane in pijače & A & 6 dni & 15 dni & 21 dni \\
E & Priprava letakov & C, J & 5 dni & 8 dni & 11 dni \\
F & Pošiljanje letakov & E & 4 dni & 4 dni & 4 dni \\
G & Priprava zbornika s povzetki & D, J & 22 dni & 28 dni & 31 dni \\
H & Določitev glavnega govorca & / & 5 dni & 8 dni & 14 dni \\
I & Planiranje poti za glavnega govorca & A, H & 11 dni & 14 dni & 17 dni \\
J & Določitev ostalih govorcev & H & 12 dni & 15 dni & 21 dni \\
K & Planiranje poti za ostale govorce & A, J & 9 dni & 12 dni & 18 dni \\
\end{tabular}
}
\podnaslov{Podatki}
\end{tabela}
\end{vprasanje}
\begin{odgovor}
\end{odgovor}
\end{naloga}


\begin{naloga}{Janoš Vidali}{Izpit OR 28.8.2018}
\begin{vprasanje}[letalo]
Pri izdelavi letala imamo faze, opisane v tabeli~\tab{}.

\begin{enumerate}[(a)]
\item S pomočjo topološke ureditve ustreznega grafa
določi kritična opravila in čas izdelave.
Uporabi podatke iz stolpca ``trajanje''.

\item Katero opravilo je najmanj kritično?
Najmanj kritično jo opravilo,
katerega trajanje lahko najbolj podaljšamo,
ne da bi vplivali na trajanje izdelave.

\item Naročnik bi rad, da letalo izdelamo v $75$ dneh.
Posamezna opravila lahko skrajšamo tako, da zanje zadolžimo več delavcev.
Seveda prinese tako krajšanje dodatne stroške,
poleg tega pa za vsako opravilo poznamo najmanjše možno trajanje
(glej zadnja dva stolpca v zgornji tabeli).
Zapiši linearni program,
s katerim modeliramo iskanje razporeda opravil,
ki nam prinese čim manjše stroške.
Linearnega programa ne rešuj.
\end{enumerate}

\begin{tabela}
\makebox[\textwidth][c]{
\begin{tabular}{c|l|c|c||c|c}
opravilo & opis & trajanje & pogoji & najmanjše trajanje & cena za dan manj \\
\hline
A & izgradnja nosu              & 40 dni & -    & 36 dni & 1000 \\
B & izgradnja trupa s krili     & 50 dni & -    & 48 dni & 1500 \\
C & izgradnja repa              & 35 dni & -    & 31 dni &  800 \\
D & vgraditev pilotske kabine   & 30 dni & A    & 28 dni & 1200 \\
E & opremljanje potniške kabine & 18 dni & F, G & 16 dni &  500 \\
F & povezovanje nosu s trupom   & 10 dni & A, B &  9 dni & 1100 \\
G & povezovanje repa s trupom   & 12 dni & B, C & 10 dni & 1300 \\
H & vgraditev motorjev          & 20 dni & B    & 18 dni & 1400 \\
\end{tabular}
}
\podnaslov{Podatki}
\end{tabela}
\end{vprasanje}
\begin{odgovor}
\end{odgovor}
\end{naloga}
