\begin{naloga}%
{Dasgupta, Papadimitriou, Vazirani}{\cite[Exercise~3.1]{dpv}}
\begin{vprasanje}
Na grafu s slike~\fig[bfs] izvedi iskanje v globino.
V primerih, ko imaš več ena\-ko\-vred\-nih izbir,
upoštevaj abecedni vrstni red.
Za vsako povezavo določi, ali se nahaja v drevesu iskanja v globino.
\end{vprasanje}

\begin{odgovor}
Sledili bomo naslednjemu algoritmu.
\begin{small}
\begin{algorithmic}
\Function{dfs}{$G = (V, E), \previsit, \postvisit$}
	\For{$v \in V$}
		\State oznaceni$[v] \gets \False$
	\EndFor
    \Function{Razišči}{$v, w$}
        \If{oznaceni$[v]$}
            \State \Return
        \EndIf
        \State oznaceni$[v] \gets \True$
        \State $\previsit(v, w)$
		\For{$u \in \Adj(v)$}:
            \State {\sc Razišči}$(u, v)$
		\EndFor
        \State $\postvisit(v, w)$
    \EndFunction
	\For{$v \in V$}
		\State {\sc Razišči}$(v, \Null)$
	\EndFor
\EndFunction
\end{algorithmic}
\end{small}
%
Potek zgornjega algoritma na grafu s slike~\fig[bfs]
(pri čemer za $\previsit$ in $\postvisit$ vzamemo $\NOp$,
torej ne naredimo ničesar),
pri čemer sledimo abecednemu vrstnemu redu vozlišč,
je prikazan v tabeli~\tab.
Gozd iskanja v širino je prikazan na sliki~\fig,
iz katere je razvidno, da so drevesne povezave
$$
(A, B), (B, C), (C, F), (F, E), (F, I), (D, G), (G, H).
$$
\begin{tabela}
\begin{tabular}{c|c|c}
$v$ & $u$ & množica označenih vozlišč \\ \hline
A &   & \{A\} \\
A &B & \{A, B\} \\
B & A &  \{A, B\} \\
B & C &  \{A, B, C\} \\
C & B & \{A, B, C\} \\
C & F & \{A, B, C, F\} \\
F & C & \{A, B, C, F\} \\
F & E & \{A, B, C, E, F\} \\
E & A & \{A, B, C, E, F\} \\
E & B & \{A, B, C, E, F\} \\
E & F & \{A, B, C, E, F\} \\
F & I & \{A, B, C, E, F, I\} \\
I & F & \{A, B, C, E, F, I\} \\
B & E & \{A, B, C, E, F, I\} \\
A & E & \{A, B, C, E, F, I\} \\
B & & \{A, B, C, E, F, I\} \\
C & & \{A, B, C, E, F, I\} \\
D & & \{A, B, C, D, E, F, I\} \\
D & G &\{A, B, C, D, E, F, G, I\} \\
G & D & \{A, B, C, D, E, F, G, I\} \\
G & H & \{A, B, C, D, E, F, G, H, I\} \\
H & D & \{A, B, C, D, E, F, G, H, I\} \\
H & G & \{A, B, C, D, E, F, G, H, I\} \\
D & H & \{A, B, C, D, E, F, G, H, I\} \\
\end{tabular}
\podnaslov{Potek izvajanja algoritma}
\end{tabela}

\begin{slika}
\pgfslika
\podnaslov{Gozd iskanja v globino}
\end{slika}

\end{odgovor}
\end{naloga}
