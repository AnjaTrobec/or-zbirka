\begin{naloga}{Janoš Vidali}{Vaje OR 21.2.2018}
\begin{vprasanje}
Zapiši rekurziven algoritem,
ki na vhod dobi celo število $n$ in teče v času $O(\sqrt{n})$.
Uporaba korenjenja ni dovoljena.
\end{vprasanje}

\begin{odgovor}
Po krovnem izreku ima časovno zahtevnost $O(\sqrt{n})$ algoritem,
katerega čas izvajanja $T(n)$ je opisan z rekurzivno zvezo
$$
T(n) = O(1) + 2T\left({n \over 4}\right) .
$$
Zapišimo tak algoritem:
\begin{small}
\begin{algorithmic}
\Function{Korenski}{$\ell[1 \dots n]$}
    \If{$n \ge 4$}
        \State $m \gets \lfloor {n \over 4} \rfloor$
        \State $\text{\sc Korenski}(\ell[1 \dots m])$
        \State $\text{\sc Korenski}(\ell[n-m+1 \dots n])$
    \EndIf
\EndFunction
\end{algorithmic}
\end{small}
\end{odgovor}
\end{naloga}
