\begin{naloga}{?}{Kolokvij OR 24.1.2011}
\begin{vprasanje}
Študent tretjega letnika finančne matematike se mora odločiti,
ali bi nadaljeval študij na drugi stopnji.
Ocenjuje, da bo, če študij uspešno zaključi,
v življenju zaslužil $200\,000 €$ več,
kot če študij zaključi že po prvi stopnji.
Če pa študija ne zaključi uspešno,
bo imel zaradi stroškov študija in izgubljenega dohodka $40\,000 €$ izgube.
Verjetnost, da bo študij na drugi stopnji uspešno zaključil, je $80 \%$.
\begin{enumerate}[(a)]
\item Modeliraj problem v okviru teorije odločanja (stanja, odločitve).
Kakšno odločitev svetuješ študentu?

\item Matematični oddelek ponuja dodatno testiranje,
ki študentom pomaga pri odločitvi, ali naj nadaljujejo študij.
Test stane $500$ evrov,
iz izkušenj kolegov pa študent ocenjuje,
da so pogojne verjetnosti naslednje:
\begin{center}
\begin{tabular}{c|cc}
$P(\text{rezultat testa} \;|\; \text{uspešnost študija})$
& uspešen & neuspešen \\ \hline
pozitiven & $19/20$ & $1/10$ \\
negativen & $1/20$ & $9/10$
\end{tabular}
\end{center}
Ali naj se študent prijavi na dodatno testiranje?
\end{enumerate}
\end{vprasanje}
\begin{odgovor}
\end{odgovor}
\end{naloga}
