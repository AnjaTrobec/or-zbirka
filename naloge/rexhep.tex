\begin{naloga}{Bašić, Gajser}{Kolokvij OR 31.5.2012}
\begin{vprasanje}
Rexhep Bajrami bi se rad naslednja štiri leta
ukvarjal s prodajo sadja in zelenjave
(po štirih letih mu poteče delovna viza).
Rad bi najel parcelo za stojnico, ki bo stala $6\,000 €$.
Če je lokacija dobra, bo imel $12\,000 €$ dobička,
če pa je lokacija slaba, bo imel le $3\,000 €$ dobička.
Ocenjuje, da je z verjetnostjo $2/3$ lokacija dobra,
z verjetnostjo $1/3$ pa slaba.
\begin{enumerate}[(a)]
\item Z odločitvenim drevesom opiši njegove možnosti in ugotovi,
kako naj se odloči ter kakšen dobiček naj pričakuje.
\item Za nasvet lahko vpraša znanca Seada, ki ``ima nos'' za tovrstne posle.
Sead mu lahko da nasvet, a zanj zahteva $1\,200 €$.
Dobro je znano, da ima Sead naslednje pogojne verjetnosti
$P(\text{Seadovo mnenje} \; | \; \text{kakovost parcele})$:
\begin{center}
\begin{tabular}{c|cc}
& dobra & slaba \\
\hline
priporoča & $2/3$ & $1/2$ \\
odsvetuje & $1/3$ & $1/2$
\end{tabular}
\end{center}
Ali naj vpraša Seada za nasvet?
Kakšen je pričakovani dobiček?
\end{enumerate}
\end{vprasanje}

\begin{odgovor}
\begin{enumerate}[(a)]
\item Če se Rexhep odloči za najem,
bo ob uspehu imel $12\,000 € - 6\,000 € = 6\,000 €$ dobička,
ob neuspehu pa bo ta enak $3\,000 € - 6\,000 € = -3\,000 €$
(tj., $3\,000 €$ izgube).
Ker je pričakovani dobiček enak
$$
{2 \over 3} \cdot 6\,000 € - {1 \over 3} \cdot 3\,000 € = 3\,000 € ,
$$
se torej odloči za najem.
Proces odločanja je prikazan
v spodnji veji od\-lo\-čit\-ve\-ne\-ga drevesa na sliki~\fig.

\item Izračunajmo verjetnosti za Seadovo mnenje
ter kakovost parcele v odvisnosti od njega.
\begin{align*}
P(\text{Sead priporoča}) &=
{2 \over 3} \cdot {2 \over 3} + {1 \over 3} \cdot {1 \over 2}
= {11 \over 18} \\
P(\text{Sead odsvetuje}) &=
{2 \over 3} \cdot {1 \over 3} + {1 \over 3} \cdot {1 \over 2}
= {7 \over 18} \\
P(\text{dobra parcela} \;|\; \text{Sead priporoča})
&= {2/3 \cdot 2/3 \over 11/18} = {8 \over 11} \\
P(\text{slaba parcela} \;|\; \text{Sead priporoča})
&= {1/3 \cdot 1/2 \over 11/18} = {3 \over 11} \\
P(\text{dobra parcela} \;|\; \text{Sead odsvetuje})
&= {2/3 \cdot 1/3 \over 7/18} = {4 \over 7} \\
P(\text{slaba parcela} \;|\; \text{Sead odsvetuje})
&= {1/3 \cdot 1/2 \over 7/18} = {3 \over 7}
\end{align*}
\end{enumerate}
S pomočjo zgoraj izračunanih verjetnosti
lahko narišemo odločitveno drevo s slike~\fig.
Opazimo, da Seadovo mnenje ne vpliva na našo odločitev o prodaji,
zato naj ga Rexhep ne vpraša in tako pričakuje dobiček $3\,000 €$.

\begin{slika}
\makebox[\textwidth][c]{
\pgfslika
}
\podnaslov{Odločitveno drevo}
\end{slika}
\end{odgovor}
\end{naloga}
