\begin{naloga}{Bašić, Gajser}{Izpit OR 9.7.2012}
\begin{vprasanje}
Dan je številski trikotnik
$$
\begin{array}{ccccccccc}
                &&&& a_{1,1} \\
            &&& a_{2,1} && a_{2,2} \\
        && a_{3,1} && a_{3,2} && a_{3,3} \\
     & \iddots && \vdots && \vdots && \ddots \\
a_{n,1} && a_{n,2} && \cdots && a_{n,n-1} && a_{n,n}
\end{array} ,
$$
kjer so $a_{i,j} \in \Z$.
{\em Spust} v takem trikotniku je pot od vrha do dna
-- t.j., zaporedje $(a_{i, j_i})_{i=1}^n$,
kjer je $j_1 = 1$ in $j_i \in \{j_{i-1}, j_{i-1}+1\}$
za vsak $i = 2, \dots, n$.
Teža spusta je vsota števil, po katerih poteka spust
(torej $\sum_{i=1}^n a_{i, j_i}$).

S pomočjo dinamičnega programiranja opiši postopek,
ki poišče najtežji spust v danem številskem trikotniku.
\end{vprasanje}
\begin{odgovor}
\end{odgovor}
\end{naloga}
