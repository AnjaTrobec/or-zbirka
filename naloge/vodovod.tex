\begin{naloga}{Janoš Vidali}{Izpit OR 19.6.2019}
\begin{vprasanje}
Za novo postavljeno obrtno cono želimo napeljati vodovodno omrežje.
Imamo $n$ delavnic v ravni vrsti,
za $i$-to delavnico pa imamo podano porabo vode $p_i$.
Iz javnega vodovoda bodo napeljali cev,
preko katere bo priteklo dovolj vode za vseh $n$ delavnic,
mi pa želimo postaviti razdelilnike,
ki bodo poskrbeli za ustrezno razdelitev vode med delavnice.
Vsak razdelilnik ima eno vhodno cev in dve izhodni,
vsaka od njiju pa bo pripeljala vodo do zaporednih delavnic
(po potrebi preko nadaljnjih razdelilnikov).
Cena postavitve razdelilnika je sorazmerna porabi delavnic, ki jim služi.
Razdelilnike želimo postaviti tako, da bo skupna cena čim manjša.
\primer{na sliki~\fig
je prikazana možna postavitev razdelilnikov za tri delavnice.
Odločimo se lahko,
ali bomo najprej razdelili vodo delavnici $1$ in delavnicama $2, 3$,
ali pa bomo vodo peljali naprej do delavnic $1, 2$ in do delavnice $3$
(ne moremo pa deliti na delavnici $1, 3$ in delavnico $2$).
Če se odločimo za prvi primer,
potem do delavnice $1$ ne potrebujemo drugih razdelilnikov,
še enega pa moramo postaviti, da razdelimo vodo do delavnic $2$ in $3$.
Cena postavitve prvega razdelilnika je tako $p_1 + p_2 + p_3$
(ne glede na odločitev),
za drugega pa je v tem primeru cena $p_2 + p_3$.}

\begin{enumerate}[(a)]
\item Naj bo $c_i = \sum_{h=1}^i p_h$ ($0 \le i \le n$).
Zapiši rekurzivne enačbe za čim učinkovitejši izračun teh vrednosti.

\item Zapiši rekurzivne enačbe za reševanje danega problema.
Razloži, kaj pred\-stav\-lja\-jo spremenljivke,
v kakšnem vrstnem redu jih računamo, ter kako dobimo optimalno rešitev.
Kakšna je časovna zahtevnost algoritma?
\namig{pomagaj si z vrednostmi $c_i$ iz prejšnje točke.}

\item S pomočjo rekurzivnih enačb reši zgornji problem za $n = 6$ in
$$
(p_i)_{i=1}^6 = (4, 19, 17, 7, 5, 9).
$$
\end{enumerate}
%
\begin{slika}
\pgfslika
\podnaslov{Primer postavitve razdelilnikov}
\end{slika}
\end{vprasanje}

\begin{odgovor}
\begin{enumerate}[(a)]
\item Zapišimo začetni pogoj in rekurzivne enačbe.
\begin{align*}
c_0 &= 0 \\
c_i &= c_{i-1} + p_i \quad (1 \le i \le n)
\end{align*}
Vrednosti $c_i$ ($0 \le i \le n$) računamo naraščajoče po indeksu $i$,
za izračun pa porabimo $O(n)$ časa.

\item Naj bo $x_{ij}$ ($1 \le i \le j \le n$) cena postavitve razdelilnikov,
ki razdelijo vodo za delavnice od $i$ do $j$.
Zapišimo začetne pogoje in rekurzivne enačbe.
\begin{align*}
x_{ii} &= 0 && (1 \le i \le n) \\
x_{ij} &= c_j - c_{i-1} + \min\set{x_{ik} + x_{k+1,j}}{i \le k < j}
&& (1 \le i < j \le n)
\end{align*}
Vrednosti $x_{ij}$ ($1 \le i \le j \le n$)
računamo naraščajoče po razliki $j-i$,
nato pa še naraščajoče po indeksu $i$.
Optimalno rešitev dobimo kot $x^* = x_{1n}$.
Časovna zahtevnost izračuna je $O(n^3)$.

\item Najprej izračunajmo vrednosti $c_i$ ($1 \le i \le 6$).
\begin{alignat*}{3}
c_1 &=\ &  0 &+  4 &&=  4 \\
c_2 &=\ &  4 &+ 19 &&= 23 \\
c_3 &=\ & 23 &+ 17 &&= 40 \\
c_4 &=\ & 40 &+  7 &&= 47 \\
c_5 &=\ & 47 &+  5 &&= 52 \\
c_6 &=\ & 52 &+  9 &&= 61
\end{alignat*}
Sedaj izračunajmo še vrednosti $x_{ij}$ ($1 \le i < j \le 6$).
\begin{alignat*}{4}
x_{12} &=\ &{} 23 &-  0 &&+ 0 + 0 &&= 23 \\
x_{23} &=\ &{} 40 &-  4 &&+ 0 + 0 &&= 36 \\
x_{34} &=\ &{} 47 &- 23 &&+ 0 + 0 &&= 24 \\
x_{45} &=\ &{} 52 &- 40 &&+ 0 + 0 &&= 12 \\
x_{56} &=\ &{} 61 &- 47 &&+ 0 + 0 &&= 14 \\
x_{13} &=\ &{} 40 &-  0 &&+ \min\{0+36, \underline{23+0}\} &&= 63 \\
x_{24} &=\ &{} 47 &-  4 &&+ \min\{\underline{0+24}, 36+0\} &&= 67 \\
x_{35} &=\ &{} 52 &- 23 &&+ \min\{\underline{0+12}, 24+0\} &&= 41 \\
x_{46} &=\ &{} 61 &- 40 &&+ \min\{0+14, \underline{12+0}\} &&= 33 \\
x_{14} &=\ &{} 47 &-  0 &&+ \min\{0+67, \underline{23+24}, 63+0\} &&= 94 \\
x_{25} &=\ &{} 52 &-  4 &&+ \min\{\underline{0+41}, 36+12, 67+0\} &&= 89 \\
x_{36} &=\ &{} 61 &- 23 &&+ \min\{\underline{0+33}, 24+14, 41+0\} &&= 71 \\
x_{15} &=\ &{} 52 &-  0 &&+ \min\{0+89, \underline{23+41}, 63+12, 94+0\} &&= 116 \\
x_{26} &=\ &{} 61 &-  4 &&+ \min\{0+71, \underline{36+33}, 67+14, 89+0\} &&= 126 \\
x_{16} &=\ &{} 61 &-  0 &&+ \min\{0+126, \underline{23+71}, 63+33, 94+14, 116+0\} &&= 155
\end{alignat*}
Cena postavitve razdelilnikov je torej $x^* = x_{16} = 155$.
Poglejmo, kako naj jih postavimo.
\begin{align*}
x_{16} &= c_6 - c_0 + x_{12} + x_{36} & \text{delimo na $1, 2$ in $3$ do $6$} \\
x_{12} &= c_2 - c_0 + x_{11} + x_{22} & \text{delimo na $1$ in $2$} \\
x_{36} &= c_6 - c_2 + x_{33} + x_{46} & \text{delimo na $3$ in $4$ do $6$} \\
x_{46} &= c_6 - c_3 + x_{45} + x_{66} & \text{delimo na $4, 5$ in $6$} \\
x_{45} &= c_5 - c_3 + x_{44} + x_{55} & \text{delimo na $4$ in $5$}
\end{align*}
%
Postavitev je prikazana na sliki~\fig[vodovod-resitev].
\begin{slika}
\pgfslika[vodovod-resitev]
\podnaslov[\res{}(c)]{Postavitev razdelilnikov}
\end{slika}
\end{enumerate}
\end{odgovor}
\end{naloga}
