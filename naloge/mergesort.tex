\begin{naloga}{Janoš Vidali}{Vaje OR 12.10.2016}
\begin{vprasanje}
Algoritem {\sc MergeSort} uredi vhodni seznam tako,
da ga najprej razdeli na dva dela,
nato vsakega rekurzivno uredi,
nazadnje pa zlije dobljena urejena seznama.
\begin{enumerate}[(a)]
\item S psevdokodo zapiši algoritem {\sc MergeSort}.
\item Izvedi algoritem na seznamu $[7, 11, 16, 7, 5, 0, 14, 1, 19, 13]$.
\item Določi časovno zahtevnost algoritma.
\end{enumerate}
\end{vprasanje}

\begin{odgovor}
\begin{enumerate}[(a)]
\item
\begin{small}
\begin{algorithmic}
\Function{MergeSort}{$\ell[1 \dots n]$}
    \If{$n \le 1$}
        \State \Return $\ell$
    \EndIf
    \State $m \gets \lceil {n \over 2} \rceil$
    \State $\ell_1 \gets \text{\sc MergeSort}(\ell[1 \dots m])$
    \State $\ell_2 \gets \text{\sc MergeSort}(\ell[m+1 \dots n])$
    \State $i, j \gets 1, 1$
    \State $\ell' \gets []$
    \While{$i \le m \land j \le n-m$}
        \If{$\ell_1[i] \le \ell_2[j]$}
            \State dodaj $\ell_1[i]$ na konec $\ell'$
            \State $i \gets i+1$
        \Else
            \State dodaj $\ell_2[j]$ na konec $\ell'$
            \State $j \gets j+1$
        \EndIf
    \EndWhile
    \State pripni $\ell_1[i \dots m]$ na konec $\ell'$
    \State pripni $\ell_2[j \dots n-m]$ na konec $\ell'$
    \State \Return $\ell'$
\EndFunction
\end{algorithmic}
\end{small}

\item Izvajanje algoritma je prikazano na sliki~\fig.
Nad črtkano črto je prikazano rekurzivno razbijanje seznamov,
pod njo pa zlivanje dobljenih urejenih podseznamov.

\item Funkcija obsega dva rekurzivna klica na seznamih polovične dolžine
ter združevanje obeh dobljenih seznamov v enega.
Naj bo $T(n)$ čas izvajanja algoritma pri vhodnem seznamu dolžine $n$.
Ker združevanje poteka v linearnem času, velja rekurzivna zveza
$$
T(n) = O(n) + 2T\left({n \over 2}\right) .
$$
Po krovnem izreku lahko izpeljemo,
da je časovna zahtevnost algoritma $O(n \log n)$.
\end{enumerate}

\begin{slika}
\pgfslika
\podnaslov[\res(b)]{Diagram izvajanja algoritma}
\end{slika}
\end{odgovor}
\end{naloga}
