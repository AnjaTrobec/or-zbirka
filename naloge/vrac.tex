\begin{naloga}{Janoš Vidali}{Izpit OR 11.6.2018}
\begin{vprasanje}
Podajamo se na pot v mesto na drugi strani gorovja.
Lahko se odločimo za pot po cesti okoli gorovja,
za kar bomo porabili $12$ ur.
Vendar pa nas najkrajša pot vodi čez prelaz,
ki je eno uro stran od začetne lokacije.
Žal pa so razmere v gorah nepredvidljive:
ocenjujemo, da bo z verjetnostjo $0.2$ prelaz čist
in ga bomo lahko prevozili v $5$ urah,
z ve\-rjet\-nost\-jo $0.1$ bo delno zasnežen in ga bomo prevozili v $9$ urah,
z verjetnostjo $0.7$ pa bo neprevozen,
zaradi česar se bomo morali vrniti na začetek in iti okoli gorovja
(skupno trajanje poti bo v tem primeru torej $14$ ur).

Edini, ki nam lahko kakorkoli pomaga pri oceni vremenskih razmer na prelazu,
je lokalni vrač,
ki pa živi v dolini pod goro in ne uporablja sodobne tehnologije,
s katero bi ga lahko kontaktirali.
Rade volje pa nam bo povedal, ali na gori vlada mir,
če se na poti do prelaza ustavimo pri njem.
Pogojne verjetnosti njegovega odgovora v odvisnosti od razmer na prelazu
so podane v spodnji tabeli.
\begin{center}
\makebox[\textwidth][c]{
\begin{tabular}{c|ccc}
$P\left(\text{vračev odgovor} \;\middle|\; \text{razmere na prelazu}\right)$
& čist & delno zasnežen & neprevozen \\ \hline
na gori vlada mir & $0.9$ & $0.5$ & $0.1$ \\
na gori divja vojna & $0.1$ & $0.5$ & $0.9$
\end{tabular}
}
\end{center}
Do vrača imamo eno uro vožnje, do prelaza pa potem še eno uro.
Če se po obisku vrača odločimo za pot okoli gorovja,
bomo za nadaljnjo pot porabili $12$ ur.

Kako se bomo odločili?
Nariši odločitveno drevo in ga uporabi pri sprejemanju odločitev.
Izračunaj tudi pričakovano trajanje poti.
Glej sliko~\fig{} za shemo možnih poti.

\begin{slika}
\pgfslika
\podnaslov{Shema možnih poti}
\end{slika}
\end{vprasanje}
\begin{odgovor}
\end{odgovor}
\end{naloga}
