\begin{naloga}{?}{Izpit OR 4.9.2012}
\begin{vprasanje}
Antonio se preživlja z graviranjem na zlato.
Ravnokar je dobil naročilo za graviranje
dveh zelo umetelno izdelanih zaročnih prstanov.
Če ne uspe izdelati prstanov v dogovorjenem času,
bo moral plačati $1\,600 €$ kazni.
Prstane lahko naroči pri zlatarju, ki ga je izbral naročnik.
Ta zlatar bo izdelavo vsakega prstana zaračunal po $100 €$,
varovana dostava pa (ne glede na število prstanov) stane $200 €$.
Ker prstani niso čisto homogeni,
je verjetnost, da graviranje uspe, le $50 \%$.
Ves ostali material (neuspeli prstani, opilki ipd.)
bo moral na koncu vrniti zlatarju (tako je določeno v pogodbi).
Ker je naročilo časovno omejeno, ima časa za največ 2 naročili pri zlatarju.
Koliko prstanov naj vsakič naroči, da bodo pričakovani stroški čim manjši?
\end{vprasanje}
\begin{odgovor}
\end{odgovor}
\end{naloga}
