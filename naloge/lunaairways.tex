\begin{naloga}{Jernej Azarija}{Izpit OR 12.5.2016}
\begin{vprasanje}
Smo v letu 2500 in turistične rakete že potujejo na Luno.
LunaAirways oglašuje let, na katerem je lahko $100$ potnikov.
LunaAirways računa, da bodo vsa mesta zasedena.
Z leti izkušenj vedo,
da nekatere potnike zagrabi strah in tako ne pridejo na potovanje.
Ocenjujejo, da je verjetnost $p_i$,
da natanko $i$ potnikov ($0 \le i \le 5$) ne pride na potovanje,
naslednje:
\begin{center}
\begin{tabular}{c|cc}
$i$ & od $0$ do $3$ & od $4$ do $5$ \\ \hline
$p_i$ & $0.2$ & $0.1$
\end{tabular}
\end{center}

S prodajo karte ima družba $300$ denot dobička.
Za vsakega potnika, ki bo moral menjati let,
ima LunaAirways $400$ denot stroškov.
Koliko kart naj proda LunaAirways, če želi maksimizirati pričakovani dobiček?
\end{vprasanje}

\begin{odgovor}
Naj bo $X_k$ ($100 \le k \le 105$) pričakovani dobiček v denotah,
če družba LunaAirways proda $k$ kart.
\begin{alignat*}{2}
E(X_{100}) &= 100 \cdot 300 &&= 30\,000 \\
E(X_{101}) &= 101 \cdot 300 - 0.2 \cdot 400 &&= 30\,220\\
E(X_{102}) &= 102 \cdot 300 - 0.2 \cdot (2+1) \cdot 400 &&= 30\,360\\
E(X_{103}) &= 103 \cdot 300 - 0.2 \cdot (3+2+1) \cdot 400 &&= 30\,420\\
E(X_{104}) &= 104 \cdot 300 - 0.2 \cdot (4+3+2+1) \cdot 400 &&= 30\,400\\
E(X_{105}) &= 105 \cdot 300 - (0.2 \cdot (5+4+3+2) + 0.1) \cdot 400 &&= 30\,340
\end{alignat*}
Podrobneje si poglejmo primer, ko družba proda $103$ karte.
Najprej izračunamo, kolikšen dobiček ustvarimo s prodajo $103$ kart,
nato pa upoštevamo, da z ve\-rjet\-nost\-jo $0.2$ nikogar ne zagrabi strah,
torej pridejo na potovanje vsi $103$ ljudje,
in bodo morali sedaj trije ljudje zamenjati let.
Podoben razmislek naredimo tudi za primere,
če strah zagrabi enega ali dva človeka
-- takrat bosta morala let zamenjati dva oziroma en človek.
Ker se vsi trije primeri zgodijo z isto verjetnostjo,
smo to verjetnost lahko izpostavili.

Družba naj torej za maksimizacijo svojega dobička proda $103$ karte.
\end{odgovor}
\end{naloga}
