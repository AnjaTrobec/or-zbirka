\begin{naloga}{?}{Izpit OR 9.7.2012}
\begin{vprasanje}
Dan je algoritem $H$,
ki na vhod sprejme enostaven neusmerjen graf $G = (V, E)$
(tj., brez zank in vzporednih povezav)
ter različni vozlišči $u, v \in V$:
\begin{small}
\begin{algorithmic}
\Function{$H$}{$G = (V, E), u, v$}
    \If{$|V| = 2 \land u \in \Adj(G, v)$}
        \State \Return \True
    \EndIf
    \For{$w \in \Adj(G, u) \setminus \{v\}$}
        \If{$H(G - u , w, v)$}
            \State \Return \True
        \EndIf
    \EndFor
    \State \Return \False
\EndFunction
\end{algorithmic}
\end{small}
Oznaka $G - u$ predstavlja graf,
ki ga dobimo iz $G$ tako, da odstranimo vozlišče $u$ in vse njegove povezave.

\begin{enumerate}[(a)]
\item Za katere vhode $(G, u, v)$ algoritem $H$ vrne \True?
\namig{poglej, kako je z grafi z $2, 3, 4, 5, \dots$ vozlišči,
in poskusi posplošiti.}

\item Denimo, da je $G = (V, E)$ graf z $n$ vozlišči,
pri čemer je vozlišče $v$ izolirano (tj., nima nobene povezave),
med vsakima drugima dvema vozliščema pa imamo povezavo.
Pokaži, da se pri klicu $H(G, u, v)$ (kjer je $u \in V \setminus \{v\}$)
zadnja vrstica funkcije $H$ izvede $\Theta((n-2)!)$-krat.
\end{enumerate}
\end{vprasanje}
\begin{odgovor}
\end{odgovor}
\end{naloga}
