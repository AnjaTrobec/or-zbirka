\begin{naloga}{Bašić, Gajser}{Izpit OR 9.7.2012}
\begin{vprasanje}
Dan je algoritem $H$,
ki na vhod sprejme enostaven neusmerjen graf $G = (V, E)$
(tj., brez zank in vzporednih povezav)
ter različni vozlišči $u, v \in V$:
\begin{small}
\begin{algorithmic}
\Function{$H$}{$G = (V, E), u, v$}
    \If{$|V| = 2 \land u \in \Adj(G, v)$}
        \State \Return \True
    \EndIf
    \For{$w \in \Adj(G, u) \setminus \{v\}$}
        \If{$H(G - u , w, v)$}
            \State \Return \True
        \EndIf
    \EndFor
    \State \Return \False
\EndFunction
\end{algorithmic}
\end{small}
Oznaka $G - u$ predstavlja graf,
ki ga dobimo iz $G$ tako, da odstranimo vozlišče $u$ in vse njegove povezave.

\begin{enumerate}[(a)]
\item Za katere vhode $(G, u, v)$ algoritem $H$ vrne \True?
\namig{poglej, kako je z grafi z $2, 3, 4, 5, \dots$ vozlišči,
in poskusi posplošiti.}

\item Denimo, da je $G = (V, E)$ graf z $n$ vozlišči,
pri čemer je vozlišče $v$ izolirano (tj., nima nobene povezave),
med vsakima drugima dvema vozliščema pa imamo povezavo.
Pokaži, da se pri klicu $H(G, u, v)$ (kjer je $u \in V \setminus \{v\}$)
zadnja vrstica funkcije $H$ izvede $\Theta((n-2)!)$-krat.
\end{enumerate}
\end{vprasanje}

\begin{odgovor}
\begin{enumerate}[(a)]
\item Algoritem vrne $\True$ natanko tedaj,
ko v gafu $G$ obstaja Hamiltonova pot med vozliščema $u$ in $v$.
Trditev bomo dokazali z indukcijo.

Za graf z dvema vozliščema algoritem vrne \True natanko tedaj,
ko sta vozlišči povezani,
torej, ko obstaja Hamiltonova pot.
Denimo, da je zgornja trditev pravilna za grafe z $n \ge 2$ vozlišči.
Če algoritem na vhod dobi graf z $n+1$ vozlišči, potem vrne \True,
če obstaja vozlišče $w$,
ki je sosednje vozlišču $u$ in različno od vozlišča $v$,
tako da v grafu $G - u$ (z $n$ vozlišči)
obstaja Hamiltonova pot od $w$ do $v$.
Če je to res, potem v grafu $G$ obstaja Hamiltonova pot od $u$ do $v$.
Po indukciji trditev velja za vse grafe.

\item Spet bomo uporabili indukcijo.
Pri $n = 2$ prvi pogoj ni resničen,
v zanko pa ne vstopimo, tako da se zadnja vrstica izvede enkrat.
Denimo, da se pri grafih z $n-1$ vozlišči ($n \ge 3$)
zadnja vrstica izvede $\Theta((n-3)!)$-krat,
in obravnavajmo primer, ko ima graf $G$ natanko $n$ vozlišč.
Ker ima vozlišče $u$ natanko $n-2$ sosedov
in ker vozlišče $v$ ni krajišče nobene Hamiltonove poti,
se v zanki izvede $n-2$ rekurzivnih klicev
na grafu $G-u$ z $n-1$ vozlišči in isto lastnostjo kot $G$.
Po izteku zanke se nato izvede še zadnja vrstica.
Skupno število izvedb zadnje vrstice je tako
$$
1 + (n-2) \Theta((n-3)!) = \Theta((n-2)!) .
$$
\end{enumerate}
\end{odgovor}
\end{naloga}
