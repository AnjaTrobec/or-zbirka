\begin{naloga}%
{Dasgupta, Papadimitriou, Vazirani}{\cite[Exercise~6.14]{dpv}}
\begin{vprasanje}
Imamo pravokoten kos blaga dimenzij $m \times n$,
kjer sta $m$ in $n$ pozitivni celi števili,
ter seznam $k$ izdelkov,
pri čemer potrebujemo za izdelek $i$
pravokoten kos blaga dimenzij $a_i \times b_i$
($a_i, b_i$ sta pozitivni celi števili),
ki ga prodamo za ceno $c_i > 0$.
Imamo stroj, ki lahko poljuben kos blaga razreže na dva dela
bodisi vodoravno, bodisi navpično.
Začetni kos blaga želimo razrezati tako,
da bomo lahko naredili izdelke,
ki nam bodo prinašali čim večji dobiček.
Pri tem smemo izdelati poljubno število kosov posameznega izdelka.
Kose blaga lahko seveda tudi obračamo
(tj., za izdelek $i$ lahko narežemo kos velikosti
$a_i \times b_i$ ali $b_i \times a_i$).

Zapiši rekurzivne enačbe za reševanje danega problema.
Razloži, kaj predstavljajo spremenljivke,
v kakšnem vrstnem redu jih računamo,
ter kako dobimo optimalno rešitev.
\end{vprasanje}
\begin{odgovor}
\end{odgovor}
\end{naloga}
