\begin{naloga}{Bašić, Gajser}{Izpit OR 4.9.2012}
\begin{vprasanje}
Dr.~Kaczyński se ukvarja s podatkovno strukturo $A$,
ki je zelo podobna tabeli (angl.~{\em array}) celih števil.
Vrednosti v posameznih celicah ``tabele'' $A$ lahko beremo,
ne moremo pa jih spreminjati.
Elementi ``tabele'' $A$ so indeksirani s celimi števili od $1$ do $n$,
kjer je $n = A.\length()$ indeks zadnjega elementa ``tabele'' $A$.
Edina operacija (poleg dostopanja do posameznih elementov),
ki jo lahko izvajamo nad $A$, je $A.\reverseStart(i)$.
Če ima na začetku $A$ vrednosti
$$
A = [a_1, a_2, \dots, a_{i-1}, a_i, a_{i+1}, \dots, a_{n-1}, a_n],
$$
po klicu ukaza $A.\reverseStart(i)$ izgleda takole:
$$
A = [a_i, a_{i-1}, \dots, a_2, a_1, a_{i+1}, \dots, a_{n-1}, a_n] .
$$
Dr.~Kaczyński se je lotil implementacije algoritmov
nad to podatkovno strukturo.
Najprej je seveda implementiral algoritem za urejanje:
\begin{small}
\begin{algorithmic}
\State $n \gets A.\length()$
\For{$i = n, \dots, 2$}
    \For{$j = 1, \dots, i-1$}
        \If{$A[j] > A[i]$}
            \State $A.\reverseStart(j)$
            \State $A.\reverseStart(i)$
        \EndIf
    \EndFor
\EndFor
\end{algorithmic}
\end{small}

\begin{enumerate}[(a)]
\item Oceni časovno zahtevnost zgornjega algoritma.
Upoštevaj, da se operacija $A.\reverseStart(i)$ izvede v konstantnem času.

\item Algoritem izvedi na ``tabeli'' $A = [5, 9, 12, 7, 15]$.
Ali deluje pravilno?

\item Napiši algoritem za urejanje, ki bo deloval pravilno.
Njegova časovna zahtevnost ne sme biti slabša
od časovne zahtevnosti algoritma dr.~Kaczyńskega.
\end{enumerate}
\end{vprasanje}

\begin{odgovor}
\begin{enumerate}[(a)]
\item Algoritem kliče ukaz $A.\reverseStart$ $O(n^2)$-krat.
Ker se ta izvede v kon\-stant\-nem času,
je torej časovna zahtevnost algoritma $O(n^2)$.

\item Izpišimo stanje ``tabele'' po vsakem klicu ukaza $A.\reverseStart$.
$$
\begin{array}{cc|cc|c|c}
i & j & A[i] & A[j] & A.\reverseStart(j) & A.\reverseStart(i) \\ \hline
5 & 1 & 15 &  5 && \\
  & 2 &    &  9 && \\
  & 3 &    & 12 && \\
  & 4 &    &  7 && \\ \hline
4 & 1 &  7 &  5 && \\
  & 2 &    &  9 & [9, 5, 12, 7, 15] & [7, 12, 5, 9, 15] \\
  & 3 &  9 &  5 && \\ \hline
3 & 1 &  5 &  7 & [7, 12, 5, 9, 15] & [5, 12, 7, 9, 15] \\
  & 2 &  7 & 12 & [12, 7, 5, 9, 15] & [5, 7, 12, 9, 15] \\ \hline
2 & 1 &  7 &  5 &&
\end{array}
$$
Vidimo, da po koncu algoritma ``tabela'' ni urejena,
torej algoritem ne deluje pravilno.

\item Da bomo dobili algoritem, ki bo pravilno urejal, bomo poskrbeli,
da po obhodu notranje zanke pri indeksih $i$ in $j$
velja $A[k] \le A[i]$ za vse $k \le j$.
Tako bomo dosegli,
da po obhodu zunanje zanke vrednost $A[i]$ ni manjša od svojih predhodnikov,
kar bo privedlo do urejenosti ob izteku zanke.
V notranji zanki bo torej treba poskrbeti,
da vrednost $A[j]$ pride na $i$-to mesto,
elementi med $j$-tim in $i$-tim pa ostanejo v tem intervalu
(ne nujno na istem mestu).
\begin{small}
\begin{algorithmic}
\State $n \gets A.\length()$
\For{$i = n, \dots, 2$}
    \For{$j = 1, \dots, i-1$}
        \If{$A[j] > A[i]$}
            \State $A.\reverseStart(i-1)$
            \State $A.\reverseStart(i-j)$
            \State $A.\reverseStart(i)$
        \EndIf
    \EndFor
\EndFor
\end{algorithmic}
\end{small}
Algoritem še vedno teče v času $O(n^2)$.

Da se prepričamo, da algoritem deluje pravilno,
dokažimo, da velja zgoraj omenjena invarianta za notranjo zanko.
Predpostavimo torej, da velja $A[k] \le A[i]$ za vse $k < j$.
Če velja tudi $A[j] \le A[i]$, potem se ne zgodi nič in invarianta drži.
V nasprotnem primeru označimo z $a_h$ ($1 \le h \le n$)
vrednost $A[i]$ ob vstopu v obhod zanke,
in z $b_h$ ($1 \le h \le n$) vrednost $A[i]$ ob izstopu iz obhoda zanke.
Po predpostavki velja $a_k \le a_i$ ($1 \le k \le j$) in $a_i > a_j$.
Poglejmo, kaj se dogaja ob klicih ukazov $A.\reverseStart$.
$$
\begin{array}{rrrrrrrrrrr}
& a_1, & \dots, & a_{j-1}, & a_j, & a_{j+1}, & \dots, & a_{i-1}, & a_i, & a_{i+1}, & \dots
\shortintertext{\small $A.\reverseStart(i-1)$:}
& a_{i-1}, & \dots, & a_{j+1}, & a_j, & a_{j-1}, & \dots, & a_1, & a_i, & a_{i+1}, & \dots
\shortintertext{\small $A.\reverseStart(i-j)$:}
& a_j, & a_{j+1}, & \dots, & a_{i-1}, & a_{j-1}, & \dots, & a_1, & a_i, & a_{i+1}, & \dots
\shortintertext{\small $A.\reverseStart(i)$:}
& a_i, & a_1, & \dots, & a_{j-1}, & a_{i-1}, & \dots, & a_{j+1}, & a_j, & a_{i+1}, & \dots \\
= & b_1, & b_2, & \dots, & b_j, & b_{j+1}, & \dots, & b_{i-1}, & b_i, & b_{i+1}, & \dots
\end{array}
$$
Opazimo,
da velja $b_i = a_j > a_i = b_1$
in $b_i = a_j \ge a_{k-1} = b_k$ ($2 \le k \le j$).
Zančna invarianta torej drži.

Preizkusimo algoritem še na primeru iz točke (b).

\makebox[\textwidth][c]{
\begin{tabular}{cc|cc|c|c|c}
$i$ & $j$ & $A[i]$ & $A[j]$ & $A.\reverseStart(i-1)$ & $A.\reverseStart(i-j)$ & $A.\reverseStart(i)$ \\ \hline
5 & 1 & 15 &  5 &&& \\
  & 2 &    &  9 &&& \\
  & 3 &    & 12 &&& \\
  & 4 &    &  7 &&& \\ \hline
4 & 1 &  7 &  5 &&& \\
  & 2 &    &  9 & [12, 9, 5, 7, 15] & [9, 12, 5, 7, 15] & [7, 5, 12, 9, 15] \\
  & 3 &  9 & 12 & [12, 5, 7, 9, 15] & [12, 5, 7, 9, 15] & [9, 7, 5, 12, 15] \\ \hline
3 & 1 &  5 &  9 & [7, 9, 5, 12, 15] & [9, 7, 5, 12, 15] & [5, 7, 9, 12, 15] \\
  & 2 &  9 &  7 &&& \\ \hline
2 & 1 &  7 &  5 &&&
\end{tabular}
}

Vidimo, da algoritem na tem primeru res deluje pravilno.
\end{enumerate}
\end{odgovor}
\end{naloga}
