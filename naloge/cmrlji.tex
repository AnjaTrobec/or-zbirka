\begin{naloga}{Bašić, Gajser}{Izpit OR 9.7.2012}
\begin{vprasanje}
Čmrlja Gaber in Bor slovita kot najhitrejša letalca med čmrlji.
Nedavno je Gaber izzval Bora na dirko čez travnik,
kjer pot ni vnaprej določena.
Bor je izziv sprejel in dogovorila sta se za pravili:
\begin{itemize}
\item določila sta točki $s$ in $t$:
začneta v $s$, in kdor prvi pride v $t$, zmaga;
\item določila sta tudi graf poti $G$, po katerih lahko letita.
Povezave v tem grafu sta določila tako,
da je čas letenja (če ni rožic na povezavi)
od enega do drugega krajišča za vse povezave enak (čmrlja sta enako hitra).
Vozlišča v tem grafu pa sta določila tako, da na njih ne raste nobena rožica.
\end{itemize}
Znano je, da če katerikoli čmrlj prileti do kake rožice,
se na njej zadrži enako dolgo,
kot potrebuje za prelet ene povezave brez rožic.
Rožice izven poti bosta čmrlja z lahkoto spregledala,
saj gre za prestižno tekmo.

Tvoja naloga je, da Gabru pomagaš do zmage.
Če torej poznaš graf $G$, vozlišči $s$ in $t$
ter število rožic na vsaki povezavi, katero pot naj izbere?

\begin{enumerate}[(a)]
\item Prevedi zgornji problem na kak znan problem
in predlagaj znani algoritem, s katerim ga lahko rešiš.

\item Reši problem za graf s slike~\fig,
kjer oznake na povezavah povedo, koliko rožic se nahaja vzdolž povezave.
\end{enumerate}

\begin{slika}
\pgfslika
\podnaslov{Graf}
\end{slika}
\end{vprasanje}
\begin{odgovor}
\end{odgovor}
\end{naloga}
