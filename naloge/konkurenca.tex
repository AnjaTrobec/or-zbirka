\begin{naloga}{Janoš Vidali}{Izpit OR 31.1.2017}
\begin{vprasanje}
Dve podjetji bosta predstavili konkurenčna izdelka.
Možnost imaš kupiti delnice prvega podjetja za ceno $10\,000 €$
ali delnice drugega podjetja po ceni $5\,000 €$,
lahko pa se seveda odločiš tudi, da delnic ne kupiš.
Ocenjuješ, da bo z ve\-rjet\-nost\-jo $0.4$ uspelo prvo podjetje,
z verjetnostjo $0.1$ bo uspelo drugo podjetje,
z verjetnostjo $0.5$ pa ne bo uspelo nobeno izmed njiju
(ne more se zgoditi, da bi obe uspeli).
Ob uspehu prvega podjetja se bo vrednost njihovih delnic potrojila,
ob uspehu drugega podjetja pa se bo vrednost njihovih delnic popeterila
-- če si lastiš delnice uspešnega podjetja, jih torej lahko prodaš,
pri čemer bo torej dobiček enak
dvakratniku oziroma štirikratniku vloženega zneska.
Delnic neuspešnega podjetja ne bo želel nihče kupiti,
tako da je v tem primeru vloženi znesek izgubljen.

Za mnenje lahko povprašaš tržnega izvedenca,
ki bo po opravljeni raziskavi povedal,
katero od dveh podjetij ima večje možnosti za uspeh.
Če bo uspešno prvo podjetje, bo to pravilno napovedal z verjetnostjo $0.8$,
če pa bo uspešno drugo podjetje,
bo to pravilno napovedal z verjetnostjo $0.7$.
V primeru, ko podjetji ne bosta uspeli,
bo z verjetnostjo $0.4$ večje možnosti pripisal prvemu,
z verjetnostjo $0.6$ pa drugemu podjetju.
Za svoje mnenje izvedenec računa $1\,000 €$.

Kakšne bodo tvoje odločitve,
da bo tvoj pričakovani dobiček po odprodaji delnic čim večji?
Nariši od\-lo\-čit\-ve\-no drevo
in odločitve sprejmi na podlagi izračunanih verjetnosti.
Pričakovani dobiček tudi izračunaj.
\end{vprasanje}
\begin{odgovor}
\end{odgovor}
\end{naloga}
