\begin{naloga}{Bašić, Gajser}{Izpit OR 26.6.2012}
\begin{vprasanje}
Dr.~Kaczyński se ukvarja s podatkovno strukturo $A$,
ki je zelo podobna tabeli (angl.~{\em array}) celih števil.
Vrednosti v posameznih celicah ``tabele'' $A$ lahko beremo,
ne moremo pa jih spreminjati.
Elementi ``tabele'' $A$ so indeksirani s celimi števili od $1$ do $n$,
kjer je $n = A.\length()$ indeks zadnjega elementa ``tabele'' $A$.
Edina operacija (poleg dostopanja do posameznih elementov),
ki jo lahko izvajamo nad $A$, je ``obračanje podtabele''.
Če ima na začetku $A$ vrednosti
$$
A = [a_1, a_2, \dots, a_{i-1}, a_i, a_{i+1}, \dots, a_{j-1}, a_j, a_{j+1},
     \dots, a_{n-1}, a_n],
$$
po klicu ukaza $A.\reverse(i, j)$ izgleda takole:
$$
A = [a_1, a_2, \dots, a_{i-1}, a_j, a_{j-1}, \dots, a_{i+1}, a_i, a_{j+1},
     \dots, a_{n-1}, a_n] .
$$
Dr.~Kaczyński se je lotil implementacije algoritmov
nad to podatkovno strukturo.
Najprej je seveda implementiral algoritem za urejanje:
\begin{small}
\begin{algorithmic}
\State $n \gets A.\length()$
\For{$i = 2, \dots, n$}
    \State $j \gets i$
    \While{$j > 1 \land A[j-1] > A[i]$}
        \State $j \gets i-1$
    \EndWhile
    \State $A.\reverse(j, i)$
    \If{$j+1 < i$}
        \State
    \EndIf
\EndFor
\end{algorithmic}
\end{small}

\begin{enumerate}[(a)]
\item V zgornji algoritem se je prikradla manjša tipkarska napaka,
poleg tega pa je ena vrstica celo izginila.
Popravi algoritem, da bo deloval pravilno.

\item Popravljeni algoritem izvedi na ``tabeli'' $A = [5, 3, 12, 9, 1, 15]$.
\end{enumerate}
\end{vprasanje}

\begin{odgovor}
\begin{enumerate}[(a)]
\item Zapišimo popravljen algoritem.
\begin{small}
\begin{algorithmic}
\State $n \gets A.\length()$
\For{$i = 2, \dots, n$}
    \State $j \gets i$
    \While{$j > 1 \land A[j-1] > A[i]$}
        \State $j \gets j-1$ \hfill popravek: $j-1$ namesto $i-1$
    \EndWhile
    \State $A.\reverse(j, i)$
    \If{$j+1 < i$}
        \State $A.\reverse(j+1, i)$ \hfill ponovno obrnemo že urejen del
    \EndIf
\EndFor
\end{algorithmic}
\end{small}

\item Izpišimo stanje ``tabele'' po vsakem klicu ukaza $A.\reverse$.
$$
\begin{array}{cc|c|c}
i & j & A.\reverse(j, i) & A.\reverse(j+1, i) \\ \hline
2 & 1 & [3, 5, 12, 9, 1, 15] & \\
3 & 3 & [3, 5, 12, 9, 1, 15] & \\
4 & 3 & [3, 5, 9, 12, 1, 15] & \\
5 & 1 & [1, 12, 9, 5, 3, 15] & [1, 3, 5, 9, 12, 15] \\
6 & 6 & [1, 3, 5, 9, 12, 15] &
\end{array}
$$
\end{enumerate}
\end{odgovor}
\end{naloga}
