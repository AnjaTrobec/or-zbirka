\begin{naloga}{?}{Vaje OM 23.4.2014}
\begin{vprasanje}
V grafu na sliki~\fig lahko kapaciteto ene povezave povečamo za $1$.
Katera naj bo ta povezava, da bomo povečali maksimalni pretok?

\begin{slika}
\pgfslika
\podnaslov{Graf}
\end{slika}
\end{vprasanje}

\begin{odgovor}
Poiščemo disjunktne povečujoče poti.
Postopek reševanja je prikazan na slikah~%
\fig[pretok4a],~\fig[pretok4b],~\fig[pretok4c] in~\fig[pretok4d].
Dobimo maksimalni pretok $5+10+8 = 8+4+3+8 = 23$.
Če želimo povečati pretok,
lahko povečamo kapaciteto take povezave v minimalnem prerezu,
da je mogoče od njenega končnega vozlišča priti do vozlišča $T$
po nezasičenih (ali obratnih nepraznih) povezavah.
Na grafu s slike~\fig[pretok4d]
so taka vozlišča označena s sivo barvo,
povezave, katerih kapaciteto se nam izplača povečati, pa z vijolično barvo.

\begin{slika}
\pgfslika[pretok4a]
\podnaslov{Prvi korak}
\end{slika}
\begin{slika}
\pgfslika[pretok4b]
\podnaslov{Drugi korak}
\end{slika}
\begin{slika}
\pgfslika[pretok4c]
\podnaslov{Tretji korak}
\end{slika}
\begin{slika}
\pgfslika[pretok4d]
\podnaslov{Maksimalni pretok in minimalni prerez}
\end{slika}
\end{odgovor}
\end{naloga}
