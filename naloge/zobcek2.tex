\begin{naloga}{David Gajser}{Izpit OR 24.6.2013}
\begin{vprasanje}
V turistični agenciji Zobček že vrsto let
organizirajo izlete po Sloveniji in na Madžarskem.
Zaradi velikega povpraševanja so se odločili,
da bodo ponudbo razširili tudi na države območja bivše Jugoslavije.
Ker bodo tako nudili izlete po veliko novih državah,
želijo postaviti štiri najboljše dosedanje vodiče za nove šefe,
tako da bo vsak odgovoren za nekaj držav.
Naša naloga je šefe stabilno razporediti po območjih.
Tabeli preferenc sta naslednji:
\begin{center}
\makebox[\textwidth][c]{
\begin{small}
\begin{tabular}{c|c}
Šefi & preference šefov \\ \hline
Črt  & Srbija in Črna gora, Slovenija in Madžarska,
       Hrvaška in BiH, Makedonija in Kosovo \\
Žan  & Slovenija in Madžarska, Srbija in Črna gora,
       Hrvaška in BiH, Makedonija in Kosovo \\
Šime & Slovenija in Madžarska, Makedonija in Kosovo,
       Hrvaška in BiH, Srbija in Črna gora \\
Mujo & Hrvaška in BiH, Slovenija in Madžarska,
       Makedonija in Kosovo, Srbija in Črna gora
\end{tabular}
\end{small}
}

\bigskip
\begin{tabular}{c|c}
Območje                & preference agencije glede šefov \\ \hline
Slovenija in Madžarska & Črt, Mujo, Žan, Šime \\
Hrvaška in BiH         & Šime, Žan, Črt \\
Srbija in Črna gora    & Šime, Žan, Mujo, Črt \\
Makedonija in Kosovo   & Šime, Mujo, Črt, Žan
\end{tabular}
\end{center}
Opazimo, da agencija za šefa območja Hrvaške in BiH ne želi postaviti Muja.

Poišči stabilno prirejanje, v kateri Mujo ni šef območja Hrvaške in BiH,
ali utemelji, da le-ta ne obstaja\footnote{
Slovenija in Madžarska je eno nedeljivo območje.
Enako velja za ostale pare držav.
}.
\end{vprasanje}
\begin{odgovor}
\end{odgovor}
\end{naloga}
