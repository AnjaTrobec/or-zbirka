\begin{naloga}{?}{Vaje OR 6.4.2016}
\begin{vprasanje}
Na voljo imamo kovance z vrednostmi $1 = v_1 < v_2 < \cdots < v_n$
in vsoto $C$, ki jo želimo izplačati s čim manjšim številom kovancev.
Predpostavljamo, da imamo dovolj velik nabor kovancev.
\begin{enumerate}[(a)]
\item S pomočjo dinamičnega programiranja reši problem v splošnem.
\item Poišči izplačilo z najmanjšim številom kovancev
za $C = 25$, $n = 4$ in $(v_i)_{i=1}^n = (1, 2, 5, 7)$.
\end{enumerate}
\end{vprasanje}

\begin{odgovor}
\begin{enumerate}[(a)]
\item Naj bo $p_j$ število kovancev,
ki jih potrebujemo za izplačilo vsote $j$.
Določimo začetni pogoj in rekurzivne enačbe.
\begin{align*}
p_0 &= 0 \\
p_j &= 1 + \min\set{p_{j-v_i}}{1 \le i \le n, \ v_i \le j}
\qquad (1 \le j \le C)
\end{align*}
Vrednosti $p_j$ ($0 \le j \le C$) računamo naraščajoče po indeksu $j$.
Najmanjše število kovancev dobimo s $p^* = p_C$.

\item Izračunajmo vrednosti $p_j$ pri danih podatkih.
\begin{alignat*}{4}
p_0    &= 0                      &&    &\qquad
p_{13} &= 1 + \min\{2, 3, 2, 2\} &&= 3 \\
p_1    &= 1 + \min\{0\}          &&= 1 &\qquad
p_{14} &= 1 + \min\{3, 3, 2, 1\} &&= 2 \\
p_2    &= 1 + \min\{1, 0\}       &&= 1 &\qquad
p_{15} &= 1 + \min\{2, 3, 2, 2\} &&= 3 \\
p_3    &= 1 + \min\{1, 1\}       &&= 2 &\qquad
p_{16} &= 1 + \min\{3, 2, 3, 2\} &&= 3 \\
p_4    &= 1 + \min\{2, 1\}       &&= 2 &\qquad
p_{17} &= 1 + \min\{3, 3, 2, 2\} &&= 3 \\
p_5    &= 1 + \min\{2, 2, 0\}    &&= 1 &\qquad
p_{18} &= 1 + \min\{3, 3, 3, 3\} &&= 4 \\
p_6    &= 1 + \min\{1, 2, 1\}    &&= 2 &\qquad
p_{19} &= 1 + \min\{4, 3, 2, 2\} &&= 3 \\
p_7    &= 1 + \min\{2, 1, 1, 0\} &&= 1 &\qquad
p_{20} &= 1 + \min\{3, 4, 3, 3\} &&= 4 \\
p_8    &= 1 + \min\{1, 2, 2, 1\} &&= 2 &\qquad
p_{21} &= 1 + \min\{4, 3, 3, 2\} &&= 3 \\
p_9    &= 1 + \min\{2, 1, 2, 1\} &&= 2 &\qquad
p_{22} &= 1 + \min\{3, 4, 3, 3\} &&= 4 \\
p_{10} &= 1 + \min\{2, 2, 1, 2\} &&= 2 &\qquad
p_{23} &= 1 + \min\{4, 3, 4, 3\} &&= 4 \\
p_{11} &= 1 + \min\{2, 2, 2, 2\} &&= 3 &\qquad
p_{24} &= 1 + \min\{4, 4, 3, 3\} &&= 4 \\
p_{12} &= 1 + \min\{3, 2, 1, 1\} &&= 2 &\qquad
p_{25} &= 1 + \min\{4, 4, 4, 4\} &&= 5
\end{alignat*}
Za izplačilo vrednosti $25$ torej potrebujemo vsaj $5$ kovancev.
Iz zgornjega izračuna lahko izplačilo rekosntruiramo tako,
da uporabimo tak kovanec,
da je za izplačilo preostanka potrebnih najmanj kovancev.
Opazimo,
da lahko vsoto $25$ s petimi kovanci izplačamo na tri različne načine:
$$
25 = 1 \cdot 1 + 2 \cdot 5 + 2 \cdot 7
   = 2 \cdot 2 + 3 \cdot 7
   = 5 \cdot 5 .
$$
\end{enumerate}
\end{odgovor}
\end{naloga}
