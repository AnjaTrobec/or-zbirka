\begin{naloga}{?}{Vaje OR 16.3.2016}
\begin{vprasanje}
Napiši linearni program,
ki modelira določanje kromatičnega števila grafa.
\end{vprasanje}

\begin{odgovor}
{\em Pravilno barvanje} (neusmerjenega) grafa $G = (V, E)$
je taka dodelitev barv vozliščem grafa $G$,
da imata krajišči vsake povezave različno barvo.
{\em Kromatično število} grafa $G$ je najmanjše število barv,
ki jih potrebujemo za pravilno barvanje grafa $G$.

Naj bo $n = |V|$ število vozlišč grafa $G$
-- to je tudi največje število barv, ki jih lahko uporabimo.
Za vsako vozlišče $v \in V$ in vsako barvo $i$ ($1 \le i \le n$)
bomo uvedli spremenljivko $x_{vi}$,
katere vrednost interpretiramo kot
$$
x_{vi} = \begin{cases}
1, & \text{vozlišče $v$ pobarvamo z barvo $i$, in} \\
0  & \text{sicer.}
\end{cases}
$$
Poleg tega bomo uvedli še spremenljivko $t$, ki šteje uporabljene barve.

Zapišimo celoštevilski linearni program.
\begin{alignat*}{2}
&& \min &\ t \quad \text{p.p.} \\
\forall v \in V \ \forall i \in \{1, \dots, n\}: &\ &
0 \le x_{vi} &\le 1, \quad x_{vi} \in \Z
\opis{Vsako vozlišče je natanko ene barve}
\forall v \in V: &\ & \sum_{i=1}^n x_{vi} &= 1
\opis{Krajišči povezav sta različnih barv}
\forall uv \in E \ \forall i \in \{1, \dots, n\}: &\ &
x_{ui} + x_{vi} &\le 1
\opis{Omejimo $t$ z indeksi uporabljenih barv}
\forall uv \in E \ \forall i \in \{1, \dots, n\}: &\ &
i x_{vi} &\le t
\end{alignat*}
\end{odgovor}
\end{naloga}
