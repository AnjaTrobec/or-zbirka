\begin{naloga}{Janoš Vidali}{Kolokvij OR 11.6.2018}
\begin{vprasanje}
Vulkanizer v svoji delavnici izdeluje in prodaja avtomobilske pnevmatike.
Glede na trenutno povpraševanje vsak teden proda $60$ pnevmatik,
v istem času pa jih lahko izdela $90$.
Cena zagona proizvodnje je $180 €$,
cena skladiščenja posamezne pnevmatike pa je $0.5 €$ na teden.

\begin{enumerate}[(a)]
\item Denimo, da primanjkljaja ne dovolimo.
Izračunaj dolžino cikla pro\-iz\-vod\-nje in prodaje pnevmatik,
pri kateri so stroški najmanjši.
Kako veliko skladišče mora vulkanizer imeti?
Izračunaj tudi enotske stroške.

\item Kako dolgo naj pri zgornji rešitvi traja proizvodnja?
Koliko pnevmatik naj vulkanizer naredi v vsakem ciklu?

\item Vulkanizer je ocenil,
da bi ga primanjkljaj ene pnevmatike stal $2 €$ na teden.
Kakšna naj bosta dolžina cikla in velikost skladišča,
da bodo stroški čim manjši?
Izračunaj tudi enotske stroške in določi največji primanjkljaj.
\end{enumerate}
\end{vprasanje}
\begin{odgovor}
\end{odgovor}
\end{naloga}
