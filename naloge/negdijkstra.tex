\begin{naloga}{?}{Vaje OR 4.5.2016}
\begin{vprasanje}
Naj bo $G = (V, E)$ graf,
za katerega so dolžine povezav določene s funkcijo $\ell : E \to \R$
(tj., dolžine so lahko tudi negativne).
Definirajmo še funkcijo $\ell' : E \to \R$ tako,
da velja $\ell'(e) = \ell(e) - \min\set{\ell(f)}{f \in E}$
(dolžine, določene z $\ell'$, so torej nenegativne).
Dokaži ali ovrzi: drevo najkrajših poti,
ki ga Dijkstrov algoritem ustvari ob vhodu $(G, \ell')$,
je tudi drevo najkrajših poti za graf $G$ z dolžinami povezav,
določenimi z $\ell$.
\end{vprasanje}
\begin{odgovor}

Trditev je napačna, saj z odštevanem konstantnega števila vsaki povezavi na grafu, 
nekatere pot skrajšamo za več kot druge.
Razlog za to je, da bo konstanta odšteta na vsaki povezavi, 
torej bodo poti, ki vsebujejo več povezav prejele večjo olajšavo.
Da bo to jasno si poglejmo primer, 
za katerega bomo našli dve različni drevesi najkrajših poti pri utežeh $\ell'$ in $\ell$.
Za primer grafa bomo vzeli tega s slike~\fig.
Vidimo, da je na začetnem grafu optimalna pot $a-c$, a ker vsebuje samo eno povezavo, 
je po odštetju števila $2$ celotna pot skrajšana za $2$, pot $a-b-c$, 
pa se tako skrajša za $2 + 2 = 4$, torej postane optimalna.

\begin{slika}
\pgfslika
\podnaslov{Protiprimer trditve}
\end{slika}

\end{odgovor}
\end{naloga}
