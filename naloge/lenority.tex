\begin{naloga}{?}{Kolokvij OR 5.2.2010}
\begin{vprasanje}
Na univerzi v mestu Lenority so asistenti res nekoliko razvajeni
in imajo močen sindikat.
(To ni nič nenavadnega, če pa univerza zahteva,
da se izpiti začnejo že ob sedmih zjutraj!)
Bliža se izpitno obdobje in potrebno je paziti izpite.
Za vsak dan so izpiti dani vnaprej.
Znano je, da asistenti ne pazijo izpitov, če so vmes ``luknje'',
saj si je to pravico izboril njihov sindikat.
Vodstvu univerze tako ne preostane nič drugega, kot da to upošteva.
Asistenti torej pridejo na faks ob neki uri,
pazijo poljubno dolgo zaporedje izpitov brez odmorov, in potem gredo domov.
Izpiti so podani v obliki časovnih intervalov,
ki se začnejo in končajo ob polni uri.
Upoštevajoč te pogoje mora vodstvo univerze venomer prositi asistente,
naj pridejo pazit izpite.
Toda tudi vodstvo si želi minimizirati ``bolečino'' ob moledovanju
in hoče vsak dan prositi kar se da malo asistentov,
naj vendarle pridejo pazit izpite.

\begin{enumerate}[(a)]
\item Modeliraj problem s pomočjo usmerjenih acikličnih grafov,
pri katerih so intervali predstavljeni z vozlišči.
Identificiraj, kaj je tvoja optimizacijska naloga na takem grafu.

\item Opiši, kako problem prevedeš na problem maksimalnih pretokov.
Natančno utemelji,
zakaj so maksimalni pretoki v tvojem modelu v bijektivni korespondenci
z optimalnimi rešitvami tvoje optimizacijske naloge.

\item S pomočjo algoritma iz prejšnje točke reši problem za petkove izpite,
ki so podani z intervali $(7, 8)$, $(8, 9)$, $(9, 10)$, $(8, 12)$,
$(10, 13)$, $(12, 15)$, $(13, 15)$, $(9, 12)$, $(7, 10)$.
\end{enumerate}
\end{vprasanje}
\begin{odgovor}
\end{odgovor}
\end{naloga}
