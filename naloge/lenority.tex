\begin{naloga}{Alen Orbanić}{Izpit OR 5.2.2010}
\begin{vprasanje}
Na univerzi v mestu Lenority so asistenti res nekoliko razvajeni
in imajo močen sindikat.
(To ni nič nenavadnega, če pa univerza zahteva,
da se izpiti začnejo že ob sedmih zjutraj!)
Bliža se izpitno obdobje in potrebno je paziti izpite.
Za vsak dan so izpiti dani vnaprej.
Znano je, da asistenti ne pazijo izpitov, če so vmes ``luknje'',
saj si je to pravico izboril njihov sindikat.
Vodstvu univerze tako ne preostane nič drugega, kot da to upošteva.
Asistenti torej pridejo na faks ob neki uri,
pazijo poljubno dolgo zaporedje izpitov brez odmorov, in potem gredo domov.
Izpiti so podani v obliki časovnih intervalov,
ki se začnejo in končajo ob polni uri.
Upoštevajoč te pogoje mora vodstvo univerze venomer prositi asistente,
naj pridejo pazit izpite.
Toda tudi vodstvo si želi minimizirati ``bolečino'' ob moledovanju
in hoče vsak dan prositi kar se da malo asistentov,
naj vendarle pridejo pazit izpite.

\begin{enumerate}[(a)]
\item Modeliraj problem s pomočjo usmerjenih acikličnih grafov,
pri katerih so intervali predstavljeni z vozlišči.
Identificiraj, kaj je tvoja optimizacijska naloga na takem grafu.

\item Opiši, kako problem prevedeš na problem maksimalnih pretokov.
Natančno utemelji,
zakaj so maksimalni pretoki v tvojem modelu v bijektivni korespondenci
z optimalnimi rešitvami tvoje optimizacijske naloge.

\item S pomočjo algoritma iz prejšnje točke reši problem za petkove izpite,
ki so podani z intervali $(7, 8)$, $(8, 9)$, $(9, 10)$, $(8, 12)$,
$(10, 13)$, $(12, 15)$, $(13, 15)$, $(9, 12)$, $(7, 10)$.
\end{enumerate}
\end{vprasanje}

\begin{odgovor}
Predpostavimo,
da so intervali podani kot pari celih števil $(a_i, b_i)$ z $a_i < b_i$
($1 \le i \le n$).

\begin{enumerate}[(a)]
\item Naj bo $G = (V, E)$ usmerjen graf
z množico vozlišč $V = \{1, 2, \dots, n\}$
in povezavo $i \to j$ ($i, j \in V$) natanko tedaj,
ko velja $b_i = a_j$.
Ker za vsak $i \in V$ velja $a_i < b_i$, je graf $G$ acikličen.
Naša optimizacijska naloga je poiskati čim manjše število usmerjenih poti,
ki pokrijejo vsako vozlišče grafa $G$ natanko enkrat.
Pri tem dovoljujemo tudi poti dolžine $0$
-- ta pokrijejo natanko eno vozlišče.
Vsaka taka pot torej ustreza zaporedju terminov,
ki jih bo pazil posamezen asistent.

\item Definirajmo omrežje $G' = (V', E')$,
kjer je
\begin{align*}
V' &= \{s, t\} \cup \set{u_i, v_i}{i \in V}, \\
E' &= \set{s \to u_i, v_i \to t}{i \in V}
 \cup \set{u_i \to v_j}{i \to j \in E},
\end{align*}
vse povezave pa imajo kapaciteto $1$.

Denimo,
da imamo celoštevilsko rešitev problema maksimalnega pretoka
od vozlišča $s$ do vozlišča $t$ v omrežju $G'$
-- ker so kapacitete cela števila,
nam Ford-Fulkersonov algoritem zagotavlja njen obstoj.
Ker gre vsaka enota pretoka čez natanko eno povezavo $u_i \to v_j$,
ki ustreza povezavi $i \to j$ v grafu $G$,
skupni pretok $f$ ustreza dopustni rešitvi problema iz prejšnje točke
z vsoto dolžin poti enako $f$
-- število poti je torej $n-f$.
Tako maksimalni pretok ustreza optimalni rešitvi.

\item Na sliki~\fig je prikazan maksimalen pretok
za omrežje iz prejšnje točke pri danih podatkih,
pri čemer so podani intervali oštevilčeni od $1$ do $9$.
Najdeni maksimalni pretok $f^* = 5$
torej ustreza dodelitvi terminov štirim asistentom:
prvi pazi v terminih $(7, 8)$, $(8, 12)$ in $(12, 15)$,
drugi v terminih $(8, 9)$ in $(9, 10)$,
tretji v terminih $(7, 10)$, $(10, 13)$ in $(13, 15)$,
četrti pa v terminu $(9, 12)$.
\end{enumerate}

\begin{slika}
\pgfslika
\podnaslov[\res{}(c)]{Maksimalni pretok in minimalni prerez}
\end{slika}
\end{odgovor}
\end{naloga}
