\begin{naloga}{?}{Vaje OR 16.3.2016}
\begin{vprasanje}
Napiši linearni program,
ki modelira iskanje najkrajše poti
med danima vozliščema $u$ in $v$ v usmerjenem grafu $G$.
\end{vprasanje}

\begin{odgovor}
Naj bo $G = (V, E)$ usmerjen graf in $u, v \in V$ njegovi vozlišči.
Poiskati želimo najkrajšo pot $P_{uv}$ od vozlišča $u$ do vozlišča $v$.

Za vsako povezavo $uv \in E$ bomo uvedli spremenljivko $x_{uv}$,
katere vrednost interpretiramo kot
$$
x_{uv} = \begin{cases}
1, & \text{povezava $uv$ je v poti $P_{uv}$, in} \\
0  & \text{sicer.}
\end{cases}
$$
Zapišimo celoštevilski linearni program.
\begin{alignat*}{2}
&& \min \ \sum_{uv \in E} x_{uv} &\quad \text{p.p.} \\
\forall uv \in E: &\ & 0 \le x_{uv} &\le 1, \quad x_{uv} \in \Z
\opis{Iz vozlišča $u$ izstopimo enkrat več kakor vstopimo}
&& \sum_{yu \in E} x_{yu} - \sum_{uz \in E} x_{uz} &= -1
\opis{V vozlišče $v$ vstopimo enkrat več kakor izstopimo}
&& \sum_{yv \in E} x_{yv} - \sum_{vz \in E} x_{vz} &= 1
\opis{V ostala vozlišča vstopimo natanko tolikokrat kakor vstopimo}
\forall w \in V \setminus \{u, v\}: &\ &
\sum_{yw \in E} x_{yw} - \sum_{wz \in E} x_{wz} &= 0
\end{alignat*}
Zgornje omejitve sicer dovolijo, da posamezno vozlišče obiščemo večkrat,
a se je mogoče teh zank znebiti,
pri čemer se vrednost ciljne funkcije zmanjša.
Tako bo optimalna rešitev zagotovo pot od vozlišča $u$ do vozlišča $v$.
\end{odgovor}
\end{naloga}
