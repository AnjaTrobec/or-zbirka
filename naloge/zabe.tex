\begin{naloga}{?}{Izpit OR 4.9.2012}
\begin{vprasanje}
Žabec Rok in žabica Neli živita v mlaki,
na kateri plavajo lokvanjevi listi.
Na enem listu sedi Rok,
na drugi strani mlake pa je list, na katerem počiva Neli.
Rok bi rad po listih priskakljal k Neli.
Z enim skokom lahko premosti razdaljo največ $d$.
Čisto vseeno mu je, koliko skokov naredi
in koliko je skupna preskakana razdalja.
Ali jo lahko obišče?
Lokvanjevi listi so podani kot seznam koordinat v $\R^2$.
Listi so majhni v primerjavi z razdaljami med njimi,
zato jih lahko obravnavaš kot točke.

\begin{enumerate}[(a)]
\item Formuliraj zgornji problem kot problem na grafih in predlagaj algoritem,
s katerim ga lahko rešiš.

\item Reši problem za $d = 3$ in lokvanje na koordinatah
$$
(0, 0), \ (2, 1), \ (4, 1), \ (2, 4), \ (7, 4),
\ (1, 6), \ (5, 6), \ (3, 8), \ (8, 8),
$$
pri čemer Rok sedi na lokvanju $(0, 0)$, Neli pa na $(8, 8)$.
\end{enumerate}
\end{vprasanje}

\begin{odgovor}
\begin{enumerate}[(a)]
\item Definirajmo graf $G = (V, E)$,
katerega množica vozlišč $V \subset \R^2$
je enaka vhod\-nemu seznamu koordinat,
za par različnih vozlišč $v_i, v_j \in V$
pa velja $(v_i, v_j) \in E$ natanko tedaj,
ko $\|v_i - v_j\|  \leq d$.
Vozlišči sta torej povezani, če je razdalja med njima manjša od $d$,
kar pomeni, da Rok lahko skoči iz ene na drugo.
Sedaj moramo ugotoviti, če je točka, na kateri počiva Neli, dosegljiva Roku, 
za kar pa imamo na voljo algoritma {\sc dfs} iz naloge~\res[dfs]
in {\sc bfs} iz naloge~\res[bfs].

Zapišimo kratek algoritem, pri čemer upoštevamo, da je $s$ list, na katerem sedi Rok, in $t$ list, na katerem sedi Neli.
\begin{small}
\begin{algorithmic}
\Function{Žabe}{$G = (V, E), s, t$}
	\State {\sl označen} $\gets$ slovar s ključi $v \in V$
                                 in vrednostmi $\False$
	\Function{Visit}{$u, v$}
		\State {\sl označen}$[u] \gets \True$
	\EndFunction
	\State {\sc bfs}$(G, \visit)$
	\State \Return {\sl označen}$[t]$
\EndFunction
\end{algorithmic}
\end{small}

\item Po zgornjem opisu lahko torej povežemo dane točke v ravnini in dobimo graf $G$, prikazan na sliki~\fig. 
Graf je seveda neusmerjen, saj smo povezave definirali s simetrično relacijo.
Za uteži smo vzeli razdalje med točkami, a jih za rešitev naloge ne bomo potrebovali.
Sedaj s klicem {\sc Žabe}($G, (0,0), (8,8)$) ugotovimo,
da Nelino vozlišče Roku ni dosegljivo.
\begin{slika}
\pgfslika
\podnaslov{Graf dopustnih skokov}
\end{slika}

\end{enumerate}
\end{odgovor}
\end{naloga}
