\begin{naloga}{David Gajser}{Izpit OR 28.8.2015}
\begin{vprasanje}
Poslovnež Kopalec je lastnik velike evropske verige bazenov.
Trenutno ima v lasti $20$ bazenov,
ki jih označimo kar s števili od $1$ do $20$.
Vsi razen bazenov $1$, $2$ in $3$ so v ``osnovnem'' stanju,
tj., okoli njih ni zgrajen noben večji kompleks.
Bazeni $1$, $2$ in $3$ so pa v ``nadgrajenem'' stanju,
kar pomeni, da je okoli njih zgrajen večji kompleks,
seveda v lasti gospoda Kopalca.

V naslednjem $5$-letnem obdobju
bo gospod Kopalec nadgradil nekaj svojih bazenov,
ki so trenutno v osnovnem stanju,
nekaj bazenov pa bo prodal.
Napiši celoštevilski linearni program,
ki bo gospodu Kopalcu pomagal maksimizirati število nadgrajenih bazenov
po koncu 5-letnega obdobja. Pri tem upoštevaj naslednje omejitve:
\begin{itemize}
\item Gospod Kopalec bo nadgradil največ dvakrat toliko bazenov,
kot jih bo prodal.
\item Gospod Kopalec ne bo prodal bazenov,
ki so nadgrajeni ali jih bo nadgradil.
\item Če bo gospod Kopalec nadgradil bazen $8$,
potem bo nadgradil tudi bazena $14$ in $11$, ne bo pa prodal bazena $15$.
\item Gospod Kopalec ne bo nadgradil bazenov $4$, $5$, $6$ in $7$,
če ne bo prodal bazenov $15$, $16$, $17$, $18$, $19$ in $20$
(tj., če ne proda nobenega od bazenov $15$--$20$,
potem ne bo nadgradil nobenega od bazenov $4$--$7$).
\item Za neka $a$ in $b$ ($4 \le a, b \le 20$),
ki ju želi gospod Kopalec določiti kasneje,
bo prodal bazen $a$ in nadgradil bazen $b$.
\item Ženi gospoda Kopalca je všeč bazen $c$ ($1 \le c \le 20$) tak, kot je.
Zato ga gospod Kopalec ne bo prodajal, niti nadgrajeval.
\end{itemize}
\end{vprasanje}
\begin{odgovor}
\end{odgovor}
\end{naloga}
