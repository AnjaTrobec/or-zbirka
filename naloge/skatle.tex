\begin{naloga}{Blaž Jelenc}{Kolokvij OR 17.4.2014}
\begin{vprasanje}
Danih je $n$ praznih škatel z volumni $v_1, v_2, \dots, v_n$.
Dana so naslednja pravila pospravljanja škatel:
\begin{itemize}
\item več manjših škatel lahko zložimo v večjo,
če njihov skupni volumen ne presežemo volumna večje škatle, in
\item škatle, ki ima znotraj sebe že kakšno škatlo,
ne smemo več postaviti znotraj večje škatle.
\end{itemize}
Zanima nas takšno pospravljanje škatel,
da bo skupen volumen škatel, ki ostanejo, čim manjši.
Formuliraj nalogo kot celoštevilski linearni program.

Na primer, če imamo škatle z volumni $2, 3, 6, 8, 9$,
potem je ena možna rešitev (ne nujno optimalna!),
da zložimo škatli z volumnoma $2$ in $3$ znotraj škatle z volumnom $6$
ter škatlo z volumnom $8$ znotraj največje škatle.
Ta razporeditev ima skupni volumen $6 + 9 = 15$.
\end{vprasanje}

\begin{odgovor}
Za škatli $i$ in $j$ ($1 \le i, j \le n$) bomo uvedli spremenljivko $x_{ij}$,
katere vrednost interpretiramo kot
$$
x_{ij} = \begin{cases}
1, & \text{če $i$-to škatlo postavimo v $j$-to škatlo, in} \\
0  & \text{sicer.}
\end{cases}
$$
Ker želimo minimizirati volumen škatel, ki jih ne zložimo v večje škatle,
lahko mak\-si\-mi\-zi\-ra\-mo volumen škatel, ki jih zložimo v večje škatle.
Zapišimo celoštevilski linearni program.
\begin{alignat*}{2}
\max \ \sum_{i=1}^n \sum_{j=1}^n v_i x_{ij} && \text{p.p.} \\
\forall i, j \in \{1, \dots, n\}: &\ & 0 \le x_{ij} &\le 1,
\quad x_{ij} \in \Z
\opis{Škatle ne moremo postaviti vase}
\forall i \in \{1, \dots, n\}: &\ & x_{ii} &= 0
\opis{Vsako škatlo lahko postavimo v največ eno večjo škatlo}
\forall i \in \{1, \dots, n\}: &\ & \sum_{j=1}^n x_{ij} &\le 1
\opis{Omejitev na volumnu škatle}
\forall j \in \{1, \dots, n\}: &\ & \sum_{i=1}^n v_i x_{ij} &\le v_j
\opis{Škatle, ki vsebuje drugo škatlo, ne moremo postaviti v večjo škatlo}
\forall i, j, k \in \{1, \dots, n\}: &\ & x_{ij} + x_{jk} &\le 1
\end{alignat*}
\end{odgovor}
\end{naloga}
