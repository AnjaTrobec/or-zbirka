\begin{naloga}{?}{Kolokvij OR 17.4.2013}
\begin{vprasanje}
Danih je $n$ praznih škatel z volumni $v_1, v_2, \dots, v_n$.
Dana so naslednja pravila pospravljanja škatel:
\begin{itemize}
\item več manjših škatel lahko zložimo v večjo,
če njihov skupni volumen ne presežemo volumna večje škatle, in
\item škatle, ki ima znotraj sebe že kakšno škatlo,
ne smemo več postaviti znotraj večje škatle.
\end{itemize}
Zanima nas takšno pospravljanje škatel,
da bo skupen volumen škatel, ki ostanejo, čim manjši.
Formuliraj nalogo kot celoštevilski linearni program.

Na primer, če imamo škatle z volumni $2, 3, 6, 8, 9$,
potem je ena možna rešitev (ne nujno optimalna!),
da zložimo škatli z volumnoma $2$ in $3$ znotraj škatle z volumnom $6$
ter škatlo z volumnom $8$ znotraj največje škatle.
Ta razporeditev ima skupni volumen $6 + 9 = 15$.
\end{vprasanje}
\begin{odgovor}
\end{odgovor}
\end{naloga}
