\begin{naloga}{Janoš Vidali}{Vaje OR 6.6.2018}
\begin{vprasanje}
Marta izdeluje nakit iz školjk v delavnici,
ki jo najema v bližini ljubljanske tržnice,
in ga prodaja po vnaprej dogovorjeni ceni.
Povpraševanje je $10$ kosov na teden,
stroški skladiščenja so $0.2 €$ za kos na teden.
Zagonski stroški izdelovanja nakita so $150 €$.
Marta lahko na teden izdela $12.5$ kosa nakita.
Dovoli si, da pride do primanjkljaja,
pri čemer jo ta stane $0.8 €$ za kos nakita na teden.
Kako naj Marta organizira proizvodnjo in skladiščenje,
da bo imela čim manj stroškov?
\end{vprasanje}

\begin{odgovor}
Imamo sledeče podatke (pri časovni enoti $1$ teden).
\begin{align*}
\nu &= 10 \\
s &= 0.2 € \\
K &= 150 € \\
\lambda &= 12.5 \\
p &= 0.8 €
\end{align*}

Izračunajmo optimalno dolžino cikla $\tau^*$
in največjo zalogo $M^*$.
\begin{alignat*}{3}
\alpha &= \nu \left(1 - {\nu \over \lambda}\right)
&&= 10 \cdot (1 - 0.8) &&= 2 \\
\beta &= 1 + {s \over p}
&&= 1 + {0.2 \over 0.8} &&= 1.25 \\
\tau^* &= \sqrt{2K \beta \over s \alpha}
&&= \sqrt{375 \over 0.4} &&\approx 30.619 \\
M^* &= \sqrt{2K \alpha \over s \beta}
&&= \sqrt{600 \over 0.25} &&\approx 48.990
\end{alignat*}
Optimalna dolžina cikla je torej približno $30.619$ tednov,
pri tem pa Marta potrebuje skladišče za $49$ kosov nakita.

Izračunajmo še število izdelanih kosov nakita $q^*$,
trajanje proizvodnje $t^*$ in največji primanjkljaj $m^*$ v posameznem ciklu.
\begin{alignat*}{3}
q^* &= \tau^* \nu &&\approx 30.619 \cdot 10 &&\approx 306.186 \\
t^* &= {q^* \over \lambda} &&\approx {306.186 \over 12.5} &&\approx 24.495 \\
m^* &= \tau^* \alpha - M^* &&\approx 30.619 \cdot 2 - 48.990 &&\approx 12.247
\end{alignat*}
V vsakem intervalu torej Marta izdela okoli $306$ kosov nakita,
pri čemer izdelovanje traja približno $24.495$ tednov,
primanjkljaj pa ne preseže $13$ kosov nakita.
\end{odgovor}
\end{naloga}
