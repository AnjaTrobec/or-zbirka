\begin{naloga}{Janoš Vidali}{Vaje OR 7.12.2016}
\begin{vprasanje}
Dan je usmerjen acikličen graf s slike~\fig.

\begin{enumerate}[(a)]
\item Poišči topološko ureditev vozlišč zgornjega grafa.

\item Poišči najkrajšo pot od vozlišča $G$ do vozlišča $E$.

\item Poišči najdaljšo pot od vozlišča $G$ do vozlišča $E$.
\end{enumerate}

\begin{slika}
\pgfslika
\podnaslov{Graf}
\end{slika}
\end{vprasanje}
\begin{odgovor}\\

\noindent a) Za lažjo predstavo vozlišč v sledečih algoritmih
označimo črke A do H, ki označujejo vozlišča s števili od 0 do 7.\\

\noindent Upoštevamo, da je graf acikličen, torej za vsaki vozliščii $u, v \in G$ velja: \\[5px]
$u \rightarrow v \implies $ pooznaka(u)\ \textgreater \ pooznaka(v), \\[5px]
kjer je pooznaka števec s katerim označimo vozlišče, ko ga v algoritmu dfs docela preiščemo.\\

\noindent Trditev lahko dokažemo s premislekom poteka določitev pooznak,
ko pridemo v algoritmu dfs prvo do u in prvo do v vozlišč.
Trditev nam omogoča zapis krajšega algoritma, ki bo implementiral dfs.
Ideja je, da padajoče uredimo vozlišča po njihovih pooznakah,
po čemer dobimo inverzno topološko ureditev.
To pa dosežemo tako, da po tem, ko docela preiščemo vozlišče, tega položimo na sklad. \\

\begin{algorithmic}
\Function{Topo}{V, E}
	\State $z \gets \length(V)$
	\State oznaceni $\gets [\False, \dots, \False]$
	\State toposklad $\gets []$
	\Function{dfs}{$u$}
		\State oznaceni[u] = \True
		\For{$\forall (u, v) \in E$}
			\If{not oznaceni[$v$]}
				\State dfs(v)
			\EndIf
		\EndFor
		\State toposklad.\append(u)
	\EndFunction
	\For{$\forall u =0, 1, 2, \dots, n - 1$}
		\If{not oznaceni[$v$]}
			\State dfs(v)
		\EndIf
	\EndFor
	\State vrni \reverse(toposklad)
\EndFunction \\
\end{algorithmic}

\noindent Po klicu topo(Graf), nam ta vrne ureditev vozlišč $[6, 0, 7, 1, 2, 5, 3, 4]$, 
kar lahko pretvorimo nazaj v začetne oznake $[G, A, H, B, C, F, D, E]$. \pagebreak

\noindent b) Tukaj bomo uporabili rezultat prejšnje naloge in osnoven 
koncept dinamičnega programiranja.\\

\noindent Naj dist(u) predstavlja najkrajšo razdaljo do vozlišča u.
Trivialna rešitev je za dist(G), saj je to vozlišče brez vhodnih povezav, torej je najdaljša pot do njega
enaka 0. Zapišimo rekurzivno formulo: \\
$dist(u) = min_{v \rightarrow u}(dist(v) + l(v, u))$, \
kjer l(v, u) določa utež povezave v $\rightarrow$ u \\

\noindent Da bomo imeli vse potrebovane dist(v) definirane pred klicom dist(u) 
poskrbimo z rezultatom prejšnje naloge, saj pri topološki ureditvi vodijo vse povezave le naprej, 
torej po premiku na naslednje vozlišče ne potrebujemo vozlišč po, a le ta pred njim,
katera pa že imamo.\\

\noindent Torej imamo vse kar potrebujemo za izračun najkrajše poti, 
za beleženje vozlišč skozi katera vodi ta pot pa bomo uporabili seznam predhodnikov, 
ki bo nakazoval katero vozlišče smo izbrali za predhodnika, torej: \\
predhodnik(u) $= argmin_{v \rightarrow u}(dist(v) + l(v, u))$ \\
Na koncu dobimo pot tako da sledimo predhodnikom E, dokler ne pridemo do G.  \\

\begin{algorithmic}
\Function{najkrajšaPot}{$V, E, s, e$}
	\State $n \gets \length(V)$
	\State topo $\gets [6, 0, 7, 1, 2, 5, 3, 4]$
	\State dist $\gets [\infty, \dots, \infty]$
	\State dist[s] $\gets 0$
	\State predhodniki $\gets [None, \dots, None]$
	\For{$\forall u \in \text{topo}$}
		\For{$\forall (u, v) \in E$}
			\If{$\text{dist[v]} \ \textgreater \ \text{dist[u]} + l(u, v)$}
				\State dist[v] $\gets$ dist[u] + $l(u, v)$
				\State predhodniki[v] $\gets u$
			\EndIf
		\EndFor
	\EndFor
	\State pot $\gets [e]$
	\While{\True}
		\State predhodnik $\gets$ predhodniki[pot[-1]]
		\State pot.$\append$(predhodnik)
		\If{$\text{predhodnik = s ali predhodnik = None}$}
			\State $\text{zaključi zanko}$
		\EndIf
	\EndWhile
	\State vrni dist[e], $\reverse$(pot)
\EndFunction  \\
\end{algorithmic}

\noindent Premislimo kaj nam vrne algoritem, ko poklicemo najkrajšaPot(Graf, 6, 4).
Vrne nam dolžino od vozlišča G (6) do E (4), kot drugi argument pa vrne vozlišča, 
skozi katera je algoritem potoval, da je opravil to pot, to pa je [6, 0, 1, 4], kar pretvorimo
v [G, A, B, E]. \\

\noindent Pripomnimo, da algoritem deluje pravilno, ker iščemo razdaljo od G naprej, saj je 
G začetno vozlišče v topološki ureditvi in nima predhodnika, torej se bo algoritem, 
ki išče pot preko predhodnikov ustavil po tem, ko pride do G.
Algoritem lahko posplošimo tako, da gledamo vozlišča v topološki ureditvi le od s naprej, 
prejšnja pa ignoriramo, saj iz s ne moremo priti do njih.\\

\noindent c) Postopamo kot v prejšnjem algoritmu, le da namesto $\infty$ vzamemo v dist 
za zaćetne vrednosti $-\infty$, v zanki pa na vsakem koraku preverjamo pogoj  \\
$dist[v] \ \textgreater \ dist[u] + w$ . \\
Poleg teh sprememb je algoritem identičen.\\
Vrne nam [6, 7, 1, 2, 3, 4], torej [G, H, B, C, D, E].

\end{odgovor}
\end{naloga}
