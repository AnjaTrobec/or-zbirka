\begin{naloga}{Janoš Vidali}{Izpit OR 15.12.2016}
\begin{vprasanje}
Lovec na zaklade se z bogatim ulovom
vrača iz Kalifornije nazaj domov v Chicago,
pri čemer mora seveda prečkati Divji zahod.
Potoval bo s kočijo,
pri čemer bo vsak dan potoval med dvema mestoma in nato prespal.
Zaradi varnosti se bo držal samo državnih cest, ki so varne.
Toda mesta, kjer bo prespal, niso povsem varna.
Za vsako mesto pozna verjetnosti,
da ga tam ne bodo oropali (te so med seboj neodvisne).
Tako bi želel načrtovati najvarnejšo pot domov
-- torej pot z največjo verjetnostjo,
da ga pri nobenem postanku ne bodo oropali.

\begin{enumerate}[(a)]
\item Mesta in ceste med njimi lahko predstavimo z vozlišči in povezavami
v ne\-usme\-rje\-nem grafu $G$, verjetnosti pa kot teže vozlišč.
Opiši, kako lahko za dani graf $G$ z uteženimi vozlišči
učinkovito poiščemo ustrezno pot med danima vozliščema $s$ in $t$
z uporabo variante Dijkstrovega algoritma,
ter utemelji njegovo ustreznost.
Lahko predpostaviš, da sta teži začetnega in končnega vozlišča enaki $1$.

\item Reši problem za graf s slike~\fig,
pri čemer naj se pot začne v LA in konča v CH.
Zadostovalo bo, če verjetnosti računaš na $3$ decimalke natančno.
\end{enumerate}

\begin{slika}
\pgfslika
\podnaslov{Graf}
\end{slika}
\end{vprasanje}
\begin{odgovor}
\end{odgovor}
\end{naloga}
