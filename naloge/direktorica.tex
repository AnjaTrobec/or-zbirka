\begin{naloga}{Gabrovšek, Konvalinka}{Izpit OR 9.2.2011}
\begin{vprasanje}
Direktorica banke se mora odločiti, ali bi stranki, računalniškemu podjetju,
odobrila posojilo v vrednosti $100\,000 €$.
Po izkušnjah banke so računalniška podjetja neuspešna z verjetnostjo $20 \%$,
povprečno uspešna z verjetnostjo $50 \%$ in uspešna z verjetnostjo $30 \%$.
Če damo podjetju kredit in se izkaže za neuspešno,
bomo imeli v povprečju za $15\,000 €$ izgube,
če je povprečno upešno, bomo imeli $10\,000 €$ dobička,
če pa je uspešno, bomo imeli $20\,000 €$ dobička.
\begin{enumerate}[(a)]
\item Modeliraj problem v okviru teorije odločanja (stanja, odločitve).
Kakšno odločitev svetuješ direktorici banke?
Kolikšen je EVPI?

\item Za ceno $5\,000 €$ lahko najamemo podjetje,
ki natančno preuči računalniško podjetje in poda svojo oceno:
negativno, nevtralno ali pozitivno.
Po podatkih banke so pogojne verjetnosti
$P(\text{rezultat testa} \;|\; \text{uspešnost podjetja})$ naslednje:
\begin{center}
\begin{tabular}{c|ccc}
& neuspešno & povprečno uspešno & uspešno \\ \hline
negativno & $1/2$  & $2/5$  & $1/5$ \\
nevtralno & $2/5$  & $1/2$  & $2/5$ \\
pozitivno & $1/10$ & $1/10$ & $2/5$
\end{tabular}
\end{center}
Izračunaj EVE in nariši odločitveno drevo.
Kakšno odločitev svetuješ direktorici banke?
\end{enumerate}
\end{vprasanje}

\begin{odgovor}
\begin{enumerate}[(a)]
\item Imamo eno samo odločitev
-- ali naj podjetju odobrimo posojilo.
Če ga ne odobrimo, imamo $0 €$ dobička,
če pa ga, preidemo v stanje,
v katerem imamo z verjetnostjo $0.3$ dobiček v višini $20\,000 €$,
z verjetnostjo $0.5$ dobiček v višini $10\,000 €$
in z verjetnostjo $0.2$ dobiček v višini $-15\,000 €$.
V tem primeru je pričakovani dobiček enak
$$
0.3 \cdot 20\,000 € + 0.5 \cdot 10\,000 € - 0.2 \cdot 15\,000 € = 8\,000 € .
$$
Direktorici torej svetujemo, naj odobri posojilo,
saj bo tedaj pričakovala dobiček $8\,000 €$.

Izračunajmo še EVPI.
Tukaj upoštevamo, da posojilo odobrimo, če bo to prineslo dobiček.
$$
\EVPI = 0.3 \cdot 20\,000 € + 0.5 \cdot 10\,000 € + 0.2 \cdot 0 € = 11\,000 €
$$

\item Izračunajmo verjetnosti za primer,
da direktorica naroči test.
\begin{align*}
P(\text{test pozitiven}) =
{2 \over 5} \cdot 0.3 + {1 \over 10} \cdot 0.5 + {1 \over 10} \cdot 0.2
&= 0.19 \\
P(\text{test nevtralen}) =
{2 \over 5} \cdot 0.3 + {1 \over 2} \cdot 0.5 + {2 \over 5} \cdot 0.2
&= 0.45 \\
P(\text{test negativen}) =
{1 \over 5} \cdot 0.3 + {2 \over 5} \cdot 0.5 + {1 \over 2} \cdot 0.2
&= 0.36 \\
P(\text{podjetje uspešno} \mid \text{test pozitiven}) =
{2/5 \cdot 0.3 \over 0.19} &\approx 0.632 \\
P(\text{podjetje povprečno uspešno} \mid \text{test pozitiven}) =
{1/10 \cdot 0.5 \over 0.19} &\approx 0.263 \\
P(\text{podjetje neuspešno} \mid \text{test pozitiven}) =
{1/10 \cdot 0.2 \over 0.19} &\approx 0.105 \\
P(\text{podjetje uspešno} \mid \text{test nevtralen}) =
{2/5 \cdot 0.3 \over 0.45} &\approx 0.267 \\
P(\text{podjetje povprečno uspešno} \mid \text{test nevtralen}) =
{1/2 \cdot 0.5 \over 0.45} &\approx 0.556 \\
P(\text{podjetje neuspešno} \mid \text{test nevtralen}) =
{2/5 \cdot 0.2 \over 0.45} &\approx 0.178 \\
P(\text{podjetje uspešno} \mid \text{test negativen}) =
{1/5 \cdot 0.3 \over 0.36} &\approx 0.167 \\
P(\text{podjetje povprečno uspešno} \mid \text{test negativen}) =
{2/5 \cdot 0.5 \over 0.36} &\approx 0.556 \\
P(\text{podjetje neuspešno} \mid \text{test negativen}) =
{1/2 \cdot 0.2 \over 0.36} &\approx 0.278
\end{align*}
S pomočjo zgoraj izračunanih verjetnosti
lahko narišemo odločitveno drevo s slike~\fig.
Opazimo, da se direktorici ne izplača naročiti testa,
saj ji ta ne da zadostne informacije za morebitno spremembo odločitve.
\end{enumerate}
%
\begin{slika}
\makebox[\textwidth][c]{
\pgfslika
}
\podnaslov{Odločitveno drevo}
\end{slika}
\end{odgovor}
\end{naloga}
