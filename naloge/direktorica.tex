\begin{naloga}{?}{Izpit OR 9.2.2011}
\begin{vprasanje}
Direktorica banke se mora odločiti, ali bi stranki, računalniškemu podjetju,
odobrila posojilo v vrednosti $100.000 €$.
Po izkušnjah banke so računalniška podjetja neuspešna z verjetnostjo $20 \%$,
povprečno uspešna z verjetnostjo $50 \%$ in uspešna z verjetnostjo $30 \%$.
Če damo podjetju kredit in se izkaže za neuspešno,
bomo imeli v povprečju za $15.000 €$ izgube,
če je povprečno upešno, bomo imeli $10.000 €$ dobička,
če pa je uspešno, bomo imeli $20.000 €$ dobička.
\begin{enumerate}[(a)]
\item Modeliraj problem v okviru teorije odločanja (stanja, odločitve).
Kakšno odločitev svetuješ direktorici banke?
Kolikšen je EVPI?

\item Za ceno $5.000 €$ lahko najamemo podjetje,
ki natančno preuči računalniško podjetje in poda svojo oceno:
negativno, nevtralno ali pozitivno.
Po podatkih banke so pogojne verjetnosti
$P(\text{rezultat testa} \;|\; \text{uspešnost podjetja})$ naslednje:
\begin{center}
\begin{tabular}{c|ccc}
& neuspešno & povprečno uspešno & uspešno \\ \hline
negativno & $1/2$  & $2/5$  & $1/5$ \\
nevtralno & $2/5$  & $1/2$  & $2/5$ \\
pozitivno & $1/10$ & $1/10$ & $2/5$
\end{tabular}
\end{center}
Izračunaj EVE in nariši odločitveno drevo.
Kakšno odločitev svetuješ direktorici banke?
\end{enumerate}
\end{vprasanje}
\begin{odgovor}
\end{odgovor}
\end{naloga}
