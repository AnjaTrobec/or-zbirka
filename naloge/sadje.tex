\begin{naloga}{Janoš Vidali}{Izpit OR 29.8.2017}
\begin{vprasanje}
Distributer ima $A$ zabojev avokadov in $B$ zabojev banan,
ki jih bo prodal $n$ trgovcem.
Trgovec $i$ ($1 \le i \le n$)
plača $a_i$ evrov za zaboj avokadov in $b_i$ evrov za zaboj banan,
skupaj pa bo porabil največ $c_i$ evrov.
Distributer bo zaboje dostavil s tovornjaki,
v katerih je lahko največ $K$ zabojev (ne glede na vsebino).
Če nekemu trgovcu dostavi $t$ zabojev,
bo torej opravljenih $\lceil t/K \rceil$ voženj.
Vsaka vožnja do trgovca $i$ (ne glede na to, koliko je poln tovornjak)
ga stane $d_i$ evrov.
Poleg tega bo trgovec $p$ kupil samo banane ali samo avokade,
trgovec $q$ pa bo kupil vsaj en zaboj avokadov,
če bo tudi trgovec $r$ kupil vsaj en zaboj avokadov.
Distributer želi zaboje razdeliti med trgovce tako,
da bo maksimiziral svoj dobiček
-- torej vsoto cen, ki jih plačajo trgovci,
zmanjšano za stroške dostave.
Lahko predpostavljaš, da so vse cene pozitivne.

Zapiši celoštevilski linearni program, ki modelira zgoraj opisani problem.
\end{vprasanje}
\begin{odgovor}
\end{odgovor}
\end{naloga}
