\begin{naloga}[problem polnjenja košev]{Sergio Cabello}{Izpit OR 15.3.2017}
\begin{vprasanje}
Imamo neskončno košev $B_1, B_2, \dots$,
od katerih ima vsak velikost $V > 0$,
in $n$ predmetov pozitivnih velikosti $s_1, s_2, \dots, s_n$,
od katerih je vsaka največ $V$.
Predmete želimo postaviti v koše tako, da porabimo čim manj košev,
pri čemer noben koš ne vsebuje predmetov skupne velikosti več kot $V$.
Na primer, če $V = 10$ ter $s_1 = 4$, $s_2 = 5$ in $s_3 = 7$,
lahko to storimo z dvema košema, medtem ko z enim samim košem to ni mogoče.
Zapiši celoštevilski linearni program, ki modelira opisani problem.

\begin{enumerate}[(a)]
\item Zapiši celoštevilski linearni program, ki modelira zgornji problem.

\item Denimo, da imamo seznam $L$ s pari predmetov,
ki jih ne smemo postaviti v isti koš.
Če imamo na primer $L = \{(1, 2), (3, 4), (1, 4)\}$,
potem predmetov $1$ in $2$ ne moremo postaviti skupaj,
prav tako ne predmetov $3$ in $4$ oziroma $1$ in $4$.

Dopolni zgornji celoštevilski linearni program tako,
da bo vključeval te ome\-jit\-ve.
\end{enumerate}
\end{vprasanje}

\begin{odgovor}
\begin{enumerate}[(a)]
\item Očitno je, da bomo uporabili največ $n$ košev,
saj lahko vsak predmet postavimo v svoj koš.
Zato bomo za $i$-ti predmet in $j$-ti koš ($1 \le i, j \le n$)
uvedli spremenljivko $x_{ij}$,
katere vrednost interpretiramo kot
$$
x_{ij} = \begin{cases}
1, & \text{$i$-ti predmet postavimo v $j$-ti koš, in} \\
0  & \text{sicer.}
\end{cases}
$$
Poleg tega bomo uvedli še spremenljivko $t$,
ki bo štela število uporabljenih košev.

Zapišimo celoštevilski linearni program.
\begin{alignat*}{3}
&& \min &\ t &\quad \text{p.p.} \\
\forall i, j \in \{1, \dots, n\}: &\ & 0 \le x_{ij} &\le 1 & x_{ij} &\in \Z \\
&& 0 \le t &\le 1 & t &\in \Z
\opis{Omejitev velikosti košev}
\forall j \in \{1, \dots, n\}: &\ & \sum_{i=1}^n s_i x_{ij} &\le V
\opis{Štetje košev}
\forall i, j \in \{1, \dots, n\}: &\ & j x_{ij} &\le t
\intertext{\item Dodajmo še zahtevano omejitev.}
\forall (h, i) \in L \ \forall j \in \{1, \dots n\}: &\ &
x_{hj} + x_{ij} &\le 1
\end{alignat*}
\end{enumerate}
\end{odgovor}
\end{naloga}
