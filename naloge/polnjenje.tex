\begin{naloga}[problem polnjenja košev]{Sergio Cabello}{Izpit OR 15.3.2017}
\begin{vprasanje}
Imamo neskončno košev $B_1, B_2, \dots$,
od katerih ima vsak velikost $V > 0$,
in $n$ predmetov pozitivnih velikosti $s_1, s_2, \dots, s_n$,
od katerih je vsaka največ $V$.
Predmete želimo postaviti v koše tako, da porabimo čim manj košev,
pri čemer noben koš ne vsebuje predmetov skupne velikosti več kot $V$.
Na primer, če $V = 10$ ter $s_1 = 4$, $s_2 = 5$ in $s_3 = 7$,
lahko to storimo z dvema košema, medtem ko z enim samim košem to ni mogoče.

\begin{enumerate}[(a)]
\item Zapiši celoštevilski linearni program, ki modelira zgornji problem.

\item Denimo, da imamo seznam $L$ s pari predmetov,
ki jih ne smemo postaviti v isti koš.
Če imamo na primer $L = \{(1, 2), (3, 4), (1, 4)\}$,
potem predmetov $1$ in $2$ ne moremo postaviti skupaj,
prav tako ne predmetov $3$ in $4$ oziroma $1$ in $4$.

Dopolni zgornji celoštevilski linearni program tako,
da bo vključeval te ome\-jit\-ve.
\end{enumerate}
\end{vprasanje}
\begin{odgovor}
\end{odgovor}
\end{naloga}
