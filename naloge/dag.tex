\begin{naloga}{Janoš Vidali}{Izpit OR 5.7.2018}
\begin{vprasanje}
Dan je utežen usmerjen acikličen graf s slike~\fig{}.

\begin{enumerate}[(a)]
\item Poišči topološko ureditev grafa s slike~\fig{}.

\item Poišči najcenejšo pot od vozlišča $S$ do vozlišča $T$
v grafu s slike~\fig{}.

\item Naj bo $G = (V, E)$ usmerjen acikličen graf
z nenegativno uteženimi povezavami
ter $s, t \in V$ njegovi vozlišči.
Algoritem $\A$ se po grafu $G$ sprehaja po naslednjem pravilu:
začne v vozlišču $s$,
v vsakem koraku pa se iz vozlišča $u$ premakne
v njegovega izhodnega soseda $v$ z verjetnostjo
$$
p_{uv} = {\ell_{uv} \over \sum_{u \to w} \ell_{uw}} ,
$$
kjer je $\ell_{uv}$ teža povezave od $u$ do $v$.
Algoritem $\A$ se ustavi, ko doseže vozlišče $t$.

Natančno opiši (z besedami ali psevdokodo),
kako bi v času $O(m)$ (kjer je $m = |V| + |E|$)
za vsako vozlišče $u \in V$ določil verjetnost $q_u$,
da algoritem $\A$ obišče vozlišče $u$.
Verjetnosti za graf s slike~\fig{} ni potrebno računati.
\end{enumerate}

\begin{slika}
\pgfslika
\podnaslov{Graf}
\end{slika}
\end{vprasanje}
\begin{odgovor}
\end{odgovor}
\end{naloga}
