\begin{naloga}{Bašić, Gajser}{Izpit OR 26.6.2012}
\begin{vprasanje}
Nori profesor Boltežar stanuje v stolpnici z $n$ nadstropji,
oštevilčenimi od $1$ do $n$.
Nori stanovalci tega bloka radi mečejo cvetlične lončke z balkonov.
Boltežar bi rad ugotovil, katero je najvišje nadstropje,
s katerega lahko pade cvetlični lonček, ne da bi se razbil.
Jasno je, da če se lonček razbije pri padcu iz $k$-tega nadstropja,
potem se razbije tudi pri padcu s $(k+1)$-tega nadstropja.
Če bi Boltežar imel le en cvetlični lonček,
bi ga lahko metal po vrsti od najnižjega nadstropja navzgor,
dokler se ne bi razbil.
V najslabšem primeru bi lonček torej vrgel $n$ krat
(možno je, da bi lonček preživel tudi padec iz najvišjega nadstropja).

Ker ima Boltežar doma $k$ cvetličnih lončkov,
lahko do rezultata pride tudi z manjšim številom metov.
S pomočjo dinamičnega programiranja bi rad po\-iskal strategijo metanja,
ki bi minimizirala število potrebnih metov v najslabšem primeru.
\begin{enumerate}[(a)]
\item Napiši rekurzivne enačbe za opisani problem.
\item Napiši algoritem, ki reši opisani problem.
Oceni tudi njegovo časovno zahtevnost v odvisnosti od $n$ in $k$.
\end{enumerate}
\end{vprasanje}

\begin{odgovor}
\begin{enumerate}[(a)]
\item Naj bo $p_{ij}$ največje število metov za določitev nadstropja,
če moramo določiti izmed $i$ nadstropij in imamo na voljo $j$ lončkov.
Določimo začetne pogoje in rekurzivne enačbe.
\begin{align*}
p_{0j} &= 0 && (1 \le j \le k) \\
p_{i1} &= i && (0 \le i \le n) \\
p_{ij} &= 1 + \min\set{\max\{p_{h-1,j-1}, p_{i-h,j}\}}{1 \le h \le i}
&& (1 \le i \le n, \ 2 \le j \le k)
\end{align*}
Vrednosti $p_{ij}$ ($1 \le i \le n$, $2 \le j \le k$)
računamo v leksikografskem vrstnem redu indeksov.
Največje število metov pri optimalni strategiji dobimo kot $p^* = p_{nk}$.

\item Da bomo lahko rekonstruirali strategijo,
bomo hranili še vrednosti $r_{ij}$ ($1 \le i \le n$, $2 \le j \le k$),
ki nam povedo, iz katerega od $i$-tih nadstropij, med katerimi določamo,
naj vržemo lonček, če imamo na voljo še $j$ lončkov.
Velja torej
$$
p_{ij} = 1 + \max\{p_{r_{ij}-1,j-1}, p_{i-r_{ij},j}\}
\quad (1 \le i \le n, \ 2 \le j \le k) .
$$
Sedaj lahko zapišemo algoritem za določanje optimalne strategije metanja.
\begin{small}
\begin{algorithmic}
\Function{Lončki}{$n, k$}
    \For{$j = 1, \dots, k$}
        \State $p_{0j} \gets 0$ \hfill robni pogoji za $0$ nadstropij
    \EndFor
    \For{$i = 1, \dots, n$}
        \State $p_{i1} \gets i$ \hfill robni pogoji za $1$ lonček
        \State $r_{i1} \gets 1$ \hfill če imamo samo $1$ lonček,
                                       ga vržemo iz najnižjega nadstropja
        \For{$j = 2, \dots, k$} \hfill rekurzivna enačba
            \State $p_{ij}, r_{ij} \gets \min\set{(1 + \max\{p_{h-1,j-1},
                                                             p_{i-h,j}\}, h)}%
                                                 {1 \le h \le i}$
        \EndFor
    \EndFor
    \Function{Strategija}{$i, j, h$}
            \hfill pomožna funkcija za izračun strategije
        \If{$j = 0$} \hfill če smo ostali brez lončkov,
            \State \Return $[h]$
                \hfill potem se ne razbijejo do $h$-tega nadstropja
        \EndIf
        \State $\ell \gets$ prazen seznam
        \While{$i > 0$}
                \hfill dodamo nadstropje za naslednji met in strategijo,
            \State $\ell.\append$(($h + r_{ij}$,
                {\sc Strategija$(r_{ij}-1, j-1, h)$})) \hfill če se razbije
            \State $h \gets h + r_{ij}$
                \hfill če se ne razbije, nadaljujemo z višjimi nadstropji
            \State $i \gets i - r_{ij}$
        \EndWhile
        \State $\ell.\append(h)$
            \hfill če do konca ostane cel, imamo najvišje nadstropje
        \State \Return $\ell$
    \EndFunction
    \State \Return ($p_{nk}$, {\sc Strategija}$(n, k, 0)$)
\EndFunction
\end{algorithmic}
\end{small}
Funkcija {\sc Lončki} poleg največjega potrebnega števila metov
vrne še strategijo metanja.
Ta je podana kot seznam parov,
kjer prvi element pove nadstropje, iz katerega naj vržemo lonček,
drugi element pa nadaljnjo strategijo, če se lonček razbije.
Če lonček preživi padec, nadaljujemo z naslednjim elementom seznama.
Zadnji element seznama je številka najvišjega nadstropja,
iz katerega lahko vržemo lonček, da se ta ne razbije,
če je ta preživel vse predhodne mete v seznamu.

Algoritem najprej izračuna vrednosti spremenljivk z rekurzivnimi enačbami,
pri čemer naredi $O(nk)$ korakov.
Strategijo potem sestavi z rekurzivnimi klici funkcije {\sc Strategija}.
Če bi strategijo zapisali kot drevo,
bi opazili, da ima to drevo natanko $n+1$ listov
(enega za vsak možen rezultat),
saj vsak met razdeli množico možnih rezultatov na dva dela.
Velikost strategije je torej $O(n)$,
tak\-šna pa je tudi časovna zahtevnost njenega sestavljanja,
saj v vsakem obhodu zanke {\bf while} dodamo en element.
Skupna časovna zahtevnost algoritma je torej $O(nk)$.
\end{enumerate}
\end{odgovor}
\end{naloga}
