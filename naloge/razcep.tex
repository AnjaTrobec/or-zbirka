\begin{naloga}{Janoš Vidali}{Vaje OR 21.2.2018}
\begin{vprasanje}
Število $n$ želimo razcepiti
na dva netrivialna celoštevilska faktorja,
kar storimo s sledečim algoritmom:
\begin{small}
\begin{algorithmic}
\Function{Razcep}{$n$}
    \For{$i = 2, \dots, \lfloor \sqrt{n} \rfloor$}
        \If{$n/i$ je celo število}
            \State \Return $(i, n/i)$
        \EndIf
    \EndFor
    \State \Return $n$ je praštevilo
\EndFunction
\end{algorithmic}
\end{small}
Določi časovno zahtevnost algoritma.
Ali je ta algoritem polinomski?
\end{vprasanje}

\begin{odgovor}
Algoritem teče v času $O(\sqrt{n})$.
Ker je vhod algoritma število $n$,
ki je zapisano kot zaporedje $\ell = O(\log n)$ bitov,
vidimo, da algoritem teče v času $O(2^{\ell/2})$
in torej ni polinomski v dolžini vhoda.
\end{odgovor}
\end{naloga}
