\begin{naloga}{?}{Izpit OR 24.6.2015}
\begin{vprasanje}
Veliki koncert skupine FiM
se bo odvijal v dvorani s $100$ neoznačenimi sedeži.
Prireditelj se lahko odloči, da proda $100$, $101$, $102$ ali $103$ karte
(povpraševanja je dovolj).
Znane so verjetnosti $p_0 = 0.2$, $p_1 = 0.3$, $p_2 = 0.4$ in $p_3 = 0.1$,
kjer je $p_i$ verjetnost, da $i$ kupcev kart ne pride na koncert
(ne glede na število prodanih kart).
Vsaka prodana karta prinese $10 €$ dobička,
vsak obiskovalec, ki ne bo mogel v dvorano, pa pomeni $30 €$ stroškov
(ker je že plačal $10 €$ za karto, ima torej organizator $20 €$ izgube).
Koliko kart naj prireditelj proda, da bo pričakovani dobiček čim večji?
\end{vprasanje}

\begin{odgovor}
Naj bo $x_n$ pričakovani dobiček pri prodanih $n$ kartah.
Izračunajmo $x_n$ pri $n \in \{100, 101, 102, 103\}$:
\begin{alignat*}{2}
x_{100} &= 100 \cdot 10 € &&= 1\,000 € \\
x_{101} &= 101 \cdot 10 € - 0.2 \cdot 30 € &&= 1\,004 € \\
x_{102} &= 102 \cdot 10 € - 0.2 \cdot 60 € - 0.3 \cdot 30 € &&= 999 € \\
x_{103} &= 103 \cdot 10 € - 0.2 \cdot 90 € - 0.3 \cdot 60 € - 0.4 \cdot 30 €
&&= 982 €
\end{alignat*}
Organizator bo torej maksimiziral pričakovani dobiček,
če proda $101$ karto.
\end{odgovor}
\end{naloga}
