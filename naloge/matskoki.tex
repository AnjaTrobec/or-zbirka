\begin{naloga}{Janoš Vidali}{Izpit OR 4.6.2020}
\begin{vprasanje}
Dana je matrika $A$ dimenzij $m \times n$.
Iščemo tako zaporedje skokov po matriki,
pri čemer začnemo levo zgoraj in končamo desno spodaj,
v vsakem koraku pa skočimo za vsaj eno mesto desno ali dol,
pri katerem je vsota obiskanih mest čim več\-ja.
Drugače povedano,
iščemo tako zaporedje indeksov $(i_1, j_1), (i_2, j_2), \dots (i_k, j_k)$
(za nek $k \ge 1$),
za katere velja $(i_1, j_1) = (1, 1)$, $(i_k, j_k) = (m, n)$ in
$(i_h < i_{h+1} \land j_h = j_{h+1}) \lor (i_h = i_{h+1} \land j_h < j_{h+1})$
za vse $h$ ($2 \le h \le k$),
ki maksimizira vsoto $\sum_{h=1}^k A_{i_h, j_h}$.
Primer je podan na sliki~\fig.

\begin{enumerate}[(a)]
\item Zapiši začetne pogoje in rekurzivne enačbe
za reševanje opisanega problema
ter določi, v kakšnem vrstnem redu računamo spremenljivke
in kako dobimo maksimalno vsoto.
Kakšna je časovna zahtevnost algoritma, ki sledi iz rekurzivnih enačb?
\item Reši problem za matriko s slike~\fig z uporabo rekurzivnih enačb.
\end{enumerate}

\begin{slika}
\pgfslika
\caption{Primer matrike dimenzij $4 \times 4$ za nalogo~\nal
skupaj z dopustno (ne nujno optimalno!) rešitvijo.
Vsota obiskanih mest je v tem primeru $0 + 7 + 2 + (-2) + 0 = 7$.}
\end{slika}
\end{vprasanje}

\begin{odgovor}
\begin{enumerate}[(a)]
\item Naj bo $p_{ij}$ največja vsota,
ki jo lahko dosežemo z zaporedjem skokov od $A_{11}$ do $A_{ij}$.
Določimo začetni pogoj in rekurzivne enačbe.
\begin{align*}
p_{11} &= A_{11} \\
p_{1j} &= A_{1j} + \max\set{p_{1 \ell}}{1 \le \ell \le j-1}
& (2 &\le j \le n) \\
p_{i1} &= A_{i1} + \max\set{p_{\ell 1}}{1 \le \ell \le i-1}
& (2 &\le i \le m) \\
p_{ij} &= A_{ij} + \max(\set{p_{i \ell}}{1 \le \ell \le j-1}
                   \cup \set{p_{\ell j}}{1 \le \ell \le i-1})
& (2 &\le i \le m, \\[-1mm]
&& 2 &\le j \le n)
\end{align*}
Vrednosti $p_{ij}$ računamo v leksikografskem vrstnem redu indeksov
(npr.~najprej naraščajoče po $i$, nato pa naraščajoče po $j$).
Maksimalno vsoto ob\-iska\-nih mest dobimo kot $p^* = p_{mn}$.
Rekurzivne enačbe lahko rešimo v času $O(mn(m+n))$.

\item Matrika vrednosti $(p_{ij})_{i,j=1}^4$
je prikazana na sliki~\fig[matskoki-resitev],
pri čemer je pri vsaki vred\-no\-sti puščica od tiste vrednosti,
ki je bila uporabljena pri njenem izračunu.
Z rdečo barvo je prikazano zaporedje
$(1, 1), (1, 2), (1, 4), (2, 4), (4, 4)$
z največjo vsoto $p^* = 9$.
\end{enumerate}

\begin{slika}
\pgfslika[matskoki-resitev]
\podnaslov[\res{}(b)]{Reševanje in optimalna rešitev}
\end{slika}
\end{odgovor}
\end{naloga}
