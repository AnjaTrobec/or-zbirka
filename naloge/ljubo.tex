\begin{naloga}{Batagelj, Kaufman}{\cite[Naloga~4.3]{bk}}
\begin{vprasanje}
Ljubo prodaja časopise po centru Ljubljane.
Vsak dan se mora odločiti, koliko časopisov naj naroči.
Za vsak naročen časopis plača $0.8 €$,
za vsak prodan časopis pa iztrži $1 €$.
Za neprodane časopise ne dobi povrnjenega denarja.
Vsak dan je verjetnost, da bo imel $i$ kupcev, enaka $p_i$,
kjer je $p_6 = 0.1$, $p_7 = 0.2$, $p_8 = 0.3$, $p_9 = 0.3$ in $p_{10} = 0.1$.
\begin{enumerate}[(a)]
\item Kakšno odločitev svetuješ Ljubu in zakaj?
\item Kolikšen zaslužek lahko pričakuje v mesecu,
ko časopis izide petindvajsetkrat?
\end{enumerate}
\end{vprasanje}

\begin{odgovor}
\begin{enumerate}[(a)]
\item Naj slučajna spremenljivka $X$ predstavlja dnevni zaslužek,
$n$ pa naj bo število naročenih časopisov.
Izračunajmo pričakovane zaslužke v odvisnosti od $n$.
\begin{alignat*}{2}
E(X \mid n \le 6) &= n \cdot 1 € - n \cdot 0.8 € &&\le 1.2 € \\
E(X \mid n = 7) &= (0.1 \cdot 6 + 0.9 \cdot 7) \cdot 1 €
    - 7 \cdot 0.8 € &&= 1.3 € \\
E(X \mid n = 8) &= (0.1 \cdot 6 + 0.2 \cdot 7 + 0.7 \cdot 8) \cdot 1 €
    - 8 \cdot 0.8 € &&= 1.2 € \\
E(X \mid n = 9) &= (0.1 \cdot 6 + 0.2 \cdot 7 + 0.3 \cdot 8 + 0.4 \cdot 9)
    \cdot 1 € - 9 \cdot 0.8 € &&= 0.8 € \\
E(X \mid n \ge 10) &= (0.1 \cdot 6 + 0.2 \cdot 7 + 0.3 \cdot 8 + 0.3 \cdot 9
    + 0.1 \cdot 10) \cdot 1 € - n \cdot 0.8 € &&\le 0.1 €
\end{alignat*}
Ljubo bo torej imel največji pričakovani zaslužek $1.3 €$ na dan,
če vsakič naroči $7$ časopisov.

\item V mesecu lahko pri naročilu $7$ časopisov vsak dan, ko ta izide,
pričakuje zaslužek $25 \cdot 1.3 € = 32.5 €$.
\end{enumerate}
\end{odgovor}
\end{naloga}
