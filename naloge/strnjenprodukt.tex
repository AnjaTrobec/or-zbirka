\begin{naloga}{Janoš Vidali}{Izpit OR 31.1.2017}
\begin{vprasanje}
Dano je zaporedje $n$ realnih števil $a_1, a_2, \dots, a_n$.
Želimo poiskati strnjeno podzaporedje z največjim produktom
-- t.j., taka indeksa $i, j$ ($1 \le i \le j \le n$),
da je produkt $a_i a_{i+1} \cdots a_{j-1} a_j$ čim večji.

\begin{enumerate}[(a)]
\item Zapiši rekurzivne enačbe za reševanje danega problema.
Razloži, kaj pred\-stav\-lja\-jo spremenljivke,
v kakšnem vrstnem redu jih računamo,
ter kako dobimo optimalno rešitev.
\namig{posebej obravnavaj pozitivne in negativne delne produkte.}

\item Oceni časovno zahtevnost algoritma, ki sledi iz zgoraj zapisanih enačb.

\item S svojim algoritmom reši problem za zaporedje
$$
0.9, \ -2, \ -0.6, \ -0.5, \ -2, \ 5, \ 0.1, \ 3, \ 0.5, \ -3 \ .
$$
\end{enumerate}
\end{vprasanje}

\begin{odgovor}
\begin{enumerate}[(a)]
\item Naj bo $x_i$ največji produkt strnjenega podzaporedja,
ki se konča pri številu $a_i$,
in $y_i$ najmanjši tak produkt.
Določimo začetne pogoje in rekurzivne enačbe.
\begin{align*}
x_1 &= y_1 = a_1 \\
x_i &= \max\{a_i, a_i x_{i-1}, a_i y_{i-1}\} & (2 \le i \le n) \\
y_i &= \min\{a_i, a_i x_{i-1}, a_i y_{i-1}\} & (2 \le i \le n)
\end{align*}
Vrednosti $x_i$ in $y_i$ računamo naraščajoče po indeksu $i$.
Največji produkt strnjenega podzaporedja
dobimo kot $x^* = \max\set{x_i}{1 \le i \le n}$.

\item Za izračun vrednosti $x_i$ in $y_i$
pri vsakem $i$ potrebujemo konstantno časa.
Ker naredimo $n$ takih korakov, je časovna zahtevnost algoritma $O(n)$.

\item Izračunajmo vrednosti $x_i$ in $y_i$ ($2 \le i \le 10$).
\begin{alignat*}{2}
x_2 &= \max\{-2, \underline{-2 \cdot 0.9}, -2 \cdot 0.9\} &&= -1.8 \\
y_2 &= \min\{\underline{-2}, -2 \cdot 0.9, -2 \cdot 0.9\} &&= -2 \\
x_3 &= \max\{-0.6, -0.6 \cdot -1.8, \underline{-0.6 \cdot -2}\} &&= 1.2 \\
y_3 &= \min\{\underline{-0.6}, -0.6 \cdot -1.8, -0.6 \cdot -2\} &&= -0.6 \\
x_4 &= \max\{-0.5, -0.5 \cdot 1.2, \underline{-0.5 \cdot -0.6}\} &&= 0.3 \\
y_4 &= \min\{-0.5, \underline{-0.5 \cdot 1.2}, -0.5 \cdot -0.6\} &&= -0.6 \\
x_5 &= \max\{-2, -2 \cdot 0.3, \underline{-2 \cdot -0.6}\} &&= 1.2 \\
y_5 &= \min\{\underline{-2}, -2 \cdot 0.3, -2 \cdot -0.6\} &&= -2 \\
x_6 &= \max\{5, \underline{5 \cdot 1.2}, 5 \cdot -2\} &&= 6 \\
y_6 &= \min\{5, 5 \cdot 1.2, \underline{5 \cdot -2}\} &&= -10 \\
x_7 &= \max\{0.1, \underline{0.1 \cdot 6}, 0.1 \cdot -10\} &&= 0.6 \\
y_7 &= \min\{0.1, 0.1 \cdot 6, \underline{0.1 \cdot -10}\} &&= -1 \\
x_8 &= \max\{\underline{3}, 3 \cdot 0.6, 3 \cdot -1\} &&= 3 \\
y_8 &= \min\{3, 3 \cdot 0.6, \underline{3 \cdot -1}\} &&= -3 \\
x_9 &= \max\{0.5, \underline{0.5 \cdot 3}, 0.5 \cdot -3\} &&= 1.5 \\
y_9 &= \min\{0.5, 0.5 \cdot 3, \underline{0.5 \cdot -3}\} &&= -1.5 \\
x_{10} &= \max\{-3, -3 \cdot 1.5, \underline{-3 \cdot -1.5}\} &&= 4.5 \\
y_{10} &= \min\{-3, \underline{-3 \cdot 1.5}, -3 \cdot -1.5\} &&= -4.5
\end{alignat*}
Največji produkt strnjenega zaporedja je torej
$x^* = 6 = \prod_{i=2}^6 a_i$.
\end{enumerate}
\end{odgovor}
\end{naloga}
