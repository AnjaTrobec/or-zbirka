\begin{naloga}{Janoš Vidali}{Izpit OR 28.8.2018}
\begin{vprasanje}
Odpravljamo se na pot, ki bo trajala več dni.
Pripravili smo si seznam krajev in povezav med njimi,
ki jih lahko prevozimo v enem dnevu.
Za vsako povezavo poznamo stroške prevoza,
prav tako pa za vsak kraj poznamo še stroške nočitev.
Poiskati želimo čim cenejšo pot od začetne točke do destinacije
(tj., skupna cena prevozov in nočitev naj bo čim manjša).

\begin{enumerate}[(a)]
\item Predstavi problem v jeziku grafov
in predlagaj čim bolj učinkovit algoritem za njegovo reševanje.

\item S pomočjo zgornjega algoritma poišči najcenejšo pot od LJ do BX
v grafu s slike~\fig[pocitnice].
Na povezavah so napisani stroški prevozov med krajema
(veljajo za obe smeri),
pri vozliščih pa stroški prenočitve v kraju.
\end{enumerate}
\end{vprasanje}




\begin{odgovor}
\begin{enumerate}[(a)]
\item Imamo neusmerjen graf $G = (V, E)$
z utežmi na vozliščih $c_u$ ($u \in V$) in povezavah $\ell_{uv}$ ($uv \in E$).
Problem bi bilo najlažje reševati z Dijkstrovim algoritmom
na usmerjenem grafu $D = (V, A)$,
kjer je $A = \set{uv, vu}{uv \in E}$ množica usme\-rje\-nih povezav,
z utežmi $L_{uv} = \ell_{uv} + c_v$ ($uv \in A$).

\item Potek reševanja problema je prikazan v tabeli~\tab.
Najcenejša pot od LJ do BX je LJ -- WI -- BN -- LU -- BX s ceno $150$.
\end{enumerate}
\begin{tabela}
\makebox[\textwidth][c]{
\begin{tiny}
\begin{tabular}{c|c|c|c|c|c|c|c|c|c|c|c|c|c|c|c|c}
$u$ & $v$ & $Q$ & BX &   AM &  BL &  PR & BS &  PA & LU &  BN & WI &  BU &  MO &  VE &  LJ &  ZG \\ \hline
 & & [(LJ, 0)] & $\infty$ & $\infty$& $\infty$ & $\infty$ & $\infty$ & $\infty$ & $\infty$ & $\infty$ & $\infty$ & $\infty$ & $\infty$ & $\infty$ & 0 & $\infty$ \\ \hline
LJ &VE & [(VE, 35)] & & & & & & & & & & & & $35_{\text{LJ}}$ & & \\ \hline
LJ & WI & [(WI, 33), (VE, 35)] & & & & & & & & & $33_{\text{LJ}}$ & & & & & \\ \hline
LJ & ZG & [(ZG, 20), (WI, 33), (VE, 35)] & & & & & & & & & & & & & & $20_{\text{LJ}}$\\ \hline
LJ & BU & [(ZG, 20), (BU, 25),  (WI, 33), (VE, 35)] & & & & & & & & & & $25_{\text{LJ}}$ & & & & \\ \hline
ZG & LJ & [(BU, 25),  (WI, 33), (VE, 35)] & & & & & & & & & & & & & & \\ \hline
ZG & BU & [(BU, 25),  (WI, 33), (VE, 35)] & & & & & & & & & & & & & & \\ \hline
BU & LJ & [(WI, 33), (VE, 35)] & & & & & & & & & & & & & & \\ \hline
BU & ZG & [(WI, 33), (VE, 35)] & & & & & & & & & & & & & & \\ \hline
BU & WI & [(WI, 33), (VE, 35)] & & & & & & & & & & & & & & \\ \hline
BU & BS & [(WI, 33), (VE, 35), (BS, 40)] & & & & & $40_{\text{BU}}$ & & & & & & & & & \\ \hline
WI & LJ & [(VE, 35), (BS, 40)] & & & & & & & & & & & & & & \\ \hline
WI & BU & [(VE, 35), (BS, 40)] & & & & & & & & & & & & & & \\ \hline
WI & BS & [(VE, 35), (BS, 40)] & & & & & & & & & & & & & & \\ \hline
WI & PR & [(VE, 35), (BS, 40), (PR, 53)] & & & & $53_{\text{WI}}$ & & & & & & & & & & \\ \hline
WI & BN & [(VE, 35), (BS, 40), (PR, 53), (BN, 68)] & & & & & & & & $68_{\text{WI}}$ & & & & & & \\ \hline
VE & LJ & [(BS, 40), (PR, 53), (BN, 68)] & & & & & & & & & & & & & & \\ \hline
VE & MO & [(BS, 40), (PR, 53), (BN, 68), (MO, 79)] & & & & & & & & & & & $79_{\text{VE}}$ & & & \\ \hline
BS & BU & [(PR, 53), (BN, 68), (MO, 79)] & & & & & & & & & & & & & & \\ \hline
BS & WI & [(PR, 53), (BN, 68), (MO, 79)] & & & & & & & & & & & & & & \\ \hline
BS & PR & [(PR, 53), (BN, 68), (MO, 79)] & & & & & & & & & & & & & & \\ \hline
PR & BS & [(BN, 68), (MO, 79)] & & & & & & & & & & & & & & \\ \hline
PR & WI & [(BN, 68), (MO, 79)] & & & & & & & & & & & & & & \\ \hline
PR & BL & [(BN, 68), (MO, 79), (BL, 83)] & & & $83_{\text{PR}}$ & & & & & & & & & & & \\ \hline
BN & WI & [(MO, 79), (BL, 83)] & & & & & & & & & & & & & & \\ \hline
BN & MO & [(MO, 79), (BL, 83)] & & & & & & & & & & & & & & \\ \hline
BN & LU & [(MO, 79), (BL, 83), (LU, 101)] & & & & & & & $101_{\text{BN}}$ & & & & & & & \\ \hline
MO & VE & [(BL, 83), (LU, 101)] & & & & & & & & & & & & & & \\ \hline
MO & BN & [(BL, 83), (LU, 101)] & & & & & & & & & & & & & & \\ \hline
MO & PA & [(BL, 83), (LU, 101), (PA, 133)] & & & & & & $133_{\text{MO}}$ & & & & & & & & \\ \hline
BL & PR & [(LU, 101), (PA, 133)] & & & & & & & & & & & & & & \\ \hline
BL & AM & [(LU, 101), (AM, 127), (PA, 133)] & & $127_{\text{BL}}$ & & & & & & & & & & & & \\ \hline
LU & BN & [(AM, 127), (PA, 133)] & & & & & & & & & & & & & & \\ \hline
LU & AM & [(AM, 127), (PA, 133)] & & & & & & & & & & & & & & \\ \hline
LU & BX & [(AM, 127), (PA, 133), (BX, 150)] & $150_{\text{LU}}$ & & & & & & & & & & & & & \\ \hline
LU & PA & [(AM, 127), (PA, 133), (BX, 150)] &  & & & & & & & & & & & & & \\ \hline
AM & BL & [(PA, 133), (BX, 150)] &  & & & & & & & & & & & & & \\ \hline
AM & LU & [(PA, 133), (BX, 150)] &  & & & & & & & & & & & & & \\ \hline
AM & BX & [(PA, 133), (BX, 150)] &  & & & & & & & & & & & & & \\ \hline
PA & MO & [(BX, 150)] &  & & & & & & & & & & & & & \\ \hline
PA & LU & [(BX, 150)] &  & & & & & & & & & & & & & \\ \hline
PA & BX & [(BX, 150)] &  & & & & & & & & & & & & & \\ \hline
BX & AM & [\ ] & & & & & & & & & & & & & & \\ \hline
BX & LU & [\ ] &  & & & & & & & & & & & & & \\ \hline
BX & PA & [\ ] &  & & & & & & & & & & & & &
\end{tabular}
\end{tiny}
}
\podnaslov{Potek izvajanja algoritma}
\end{tabela}




\end{odgovor}
\end{naloga}
