\begin{naloga}{Janoš Vidali}{Kolokvij OR 23.4.2018}
\begin{vprasanje}
Imamo zaporedje $n$ polj,
pri čemer je na $i$-tem polju zapisano število $a_i$.
Na voljo imamo še $\lfloor n/2 \rfloor$ domin,
z vsako od katerih lahko pokrijemo dve sosednji polji.
Vsaka domina je sestavljena iz dveh delov:
na enem je znak $+$, na drugem pa znak $-$.
Posamezno polje lahko pokrijemo z le eno domino;
če sta pokriti dve sosednji polji, morata biti pokriti z različnima znakoma
(bodisi z iste, bodisi z druge domine).
Iščemo tako postavitev domin,
ki maksimizira vsoto pokritih števil,
pomnoženih z znakom na delu domine, ki pokriva število.
Pri tem ni potrebno, da uporabimo vse domine.
Primer je podan na sliki~\fig{}.

\begin{enumerate}[(a)]
\item Zapiši rekurzivne enačbe za reševanje danega problema.
Razloži, kaj pred\-stav\-lja\-jo spremenljivke,
v kakšnem vrstnem redu jih računamo,
ter kako dobimo optimalno rešitev.
\namig{posebej obravnavaj dva primera glede na postavitev zadnje domine.}

\item Oceni časovno zahtevnost algoritma, ki sledi iz zgoraj zapisanih enačb.

\item S svojim algoritmom poišči optimalno pokritje
za primer s slike~\fig{}.
\end{enumerate}

\begin{slika}
\pgfslika
\caption{Primer dopustnega (ne nujno optimalnega) pokritja
za nalogo~\nal{}.
Vsota tega pokritja je $3 - (-4) - 5 + 9 - 1 + 2 = 12$.
Če bi eno od zadnjih dveh domin obrnili (zamenjala bi se znaka),
dobljeno pokritje ne bi bilo dopustno,
saj bi dve zaporedni polji bili pokriti z enakima znakoma.}
\end{slika}
\end{vprasanje}
\begin{odgovor}
\end{odgovor}
\end{naloga}
