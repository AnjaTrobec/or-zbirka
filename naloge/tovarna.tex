\begin{naloga}{Blaž Jelenc}{Izpit OR 24.6.2014}
\begin{vprasanje}
Tovarna od dobavitelja dobi paket $10$ komponent $A$.
Tovarna sestavlja izdelke $B$,
kjer gre v vsak izdelek $B$ natanko ena komponenta $A$.
Dane so verjetnosti
$$
p_i = P(\text{med $10$ komponentami $A$ je natanko $i$ pokvarjenih}),
$$
kjer je $p_1 = 0.7$, $p_2 = 0.2$ in $p_3 = 0.1$.
Stroški pregleda posamezne komponente so $45 €$.
Strošek popravila vseh izdelkov $B$, ki imajo okvarjeno komponento $A$
je enak $350 €$.
\begin{enumerate}[(a)]
\item Denimo, da ima tovarna na izbiro samo dve možnosti.
Prva je, da pregleda vseh $10$ dobavljenih izdelkov.
Druga možnost pa je, da ne pregleda nobenega dobavljenega izdelka.
Katera odločitev tovarni povzroči manj stroškov?

\item Naj ima sedaj tovarna na voljo drugačni izbiri.
Prva je, da naključno izbere eno izmed $10$ komponent $A$,
jo pregleda in po potrebi zamenja,
ter gre nato direktno v proizvajanje izdelkov $B$.
Druga možnost pa je, da pregleda vseh $10$ izdelkov.
Katera odločitev tovarni povzroči manj stroškov?
\end{enumerate}
\end{vprasanje}
\begin{odgovor}
    Definiramo si spremenljivke in izpišemo znane podatke.
    \begin{align*}
    X &\dots \text{strošek}\\
    C &\dots \text{izberemo pokvarjen izdelek}\\
    D&_i \dots \text{imamo $i$ pokvarenih izdelkov}\\
    E &\dots \text{tovarna pregleda vseh } 10 \text{ izdelkov}\\
    F &\dots \text{izberemo naključen izdelek}
    \end{align*}
    \begin{align*}
    p_1 &= 0.7 & p_2 &= 0.2 & p_3 &= 0.1
    \end{align*}
    
    \begin{enumerate}
    \item Izračunajmo, koliko tovarno stane, da pregleda vse izdelke in popravi pokvarjene. Vemo, da podjetje plača $45 €\cdot 10$ za $10$ pregledov.
    Pričakovan strošek, če podjetje pregleda vseh $10$ izdelkov je torej
    $$
    E(X\mid E) = 10 \cdot 45 € = 450 €
    $$
    Sedaj si poglejmo še koliko znaša pričakovan strošek, če tovarna ne pregleda nobenega izdelka. Vemo, da je strošek popravila izdelka $B$ enak $350 €$ ne glede na to, koliko jih ima okvarjen izdelek $A$.
    $$
    E(X\mid \neg E) = 0.7 \cdot 350 € + 0.2 \cdot 350 € + 0.1 \cdot 350 € = 350 €
    $$
    Če dobljena stroška primerjamo, lahko vidimo, da ima podjetje manjši strošek, če ne pregleda nobenega izdelka $B$.
    
    \item Strošek za pregled vseh komponent je enak, kot v primeru (a) in sicer $450 €$. \\
    Izračunajmo še pričakovani strošek, če tovarna pogleda eno naključno komponento. Problem ločimo na tri dele: imamo $1$ pokvarjen izdelek, $2$ pokvarena ali $3$. Vsakega od teh primerov ločimo še na dva dela: ali izberemo pokvarjen izdelek ali ne.
    \begin{align*}
    P(C \mid D_1)  &= \frac{1}{10} &E(X\mid C \text{ in } D_1) &= 45 €\\
    P(\neg C \mid D_1) &= \frac{9}{10} &E(X\mid \neg C \text{ in } D_1) &= 45 € + 350 € = 395 €
    \end{align*}
    \begin{alignat*}{2}
    E(X\mid D_1)&= \frac{1}{10} \cdot 45 € + \frac{9}{10} \cdot 395 € &&=  360 €\\
    E(X \mid D_2) &= \frac{2}{10} \cdot 395 € + \frac{8}{10} \cdot 395 € &&= 395 €\\
    E(X\mid D_3) &= \frac{3}{10} \cdot 395 € + \frac{7}{10} \cdot 395 € &&= 395 €
    \end{alignat*}
    
    
    Sedaj lahko izračunamo pričakovan strošek, če tovarna izbere naključen izdelek, ga pregleda in po potrebi popravi:
    $$
    E(X \mid F) = 0.7 \cdot 360 € + 0.2 \cdot 395 € + 0.1 \cdot 395 € = 486 €
    $$
    
    Vidimo, da velja $E(X\mid F) >  E(X\mid E)$, zato je za tovarno bolje, da pregleda vseh $10$ izdelkov.
    \end{enumerate}
\end{odgovor}
\end{naloga}
