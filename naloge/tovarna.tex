\begin{naloga}{Blaž Jelenc}{Izpit OR 24.6.2014}
\begin{vprasanje}
Tovarna od dobavitelja dobi paket $10$ komponent $A$.
Tovarna sestavlja izdelke $B$,
kjer gre v vsak izdelek $B$ natanko ena komponenta $A$.
Dane so verjetnosti
$$
p_i = P(\text{med $10$ komponentami $A$ je natanko $i$ pokvarjenih}),
$$
kjer je $p_1 = 0.7$, $p_2 = 0.2$ in $p_3 = 0.1$.
Stroški pregleda posamezne komponente so $45 €$.
Stroški popravila izdelka $B$ zaradi vgrajene okvarjene komponente $A$
so enaki $350 €$.
\begin{enumerate}[(a)]
\item Denimo, da ima tovarna na izbiro samo dve možnosti.
Prva je, da pregleda vseh $10$ dobavljenih izdelkov.
Druga možnost pa je, da ne pregleda nobenega dobavljenega izdelka.
Katera odločitev tovarni povzroči manj stroškov?

\item Naj ima sedaj tovarna na voljo drugačni izbiri.
Prva je, da naključno izbere eno izmed $10$ komponent $A$,
jo pregleda in po potrebi zamenja,
ter gre nato direktno v proizvajanje izdelkov $B$.
Druga možnost pa je, da pregleda vseh $10$ izdelkov.
Katera odločitev tovarni povzroči manj stroškov?
\end{enumerate}
\end{vprasanje}

\begin{odgovor}
Definirajmo spremenljivke in izpišimo znane podatke.
\begin{align*}
X &\dots \text{strošek} \\
C &\dots \text{izberemo pokvarjen izdelek} \\
D_i &\dots \text{imamo $i$ pokvarjenih izdelkov} \\
E &\dots \text{tovarna pregleda vseh $10$ izdelkov} \\
F &\dots \text{tovarna ne pregleda nobenega izdelka} \\
G &\dots \text{tovarna pregleda naključen izdelek}
\end{align*}
\begin{align*}
p_1 &= 0.7 & p_2 &= 0.2 & p_3 &= 0.1
\end{align*}

\begin{enumerate}[(a)]
\item Izračunajmo, koliko tovarno stane,
da pregleda vse izdelke in popravi po\-kva\-rje\-ne.
Vemo, da podjetje plača $45 € \cdot 10$ za $10$ pregledov.
\begin{alignat*}{3}
E(X \mid E \cap D_1) &= 450 € + 350 € &&=&\ 800 € \\
E(X \mid E \cap D_2) &= 450 € + 350 € \cdot 2 &&=&\ 1\,150 € \\
E(X \mid E \cap D_3) &= 450 € + 350 € \cdot 3 &&=&\ 1\,500 €
\end{alignat*}
Pričakovani strošek, če podjetje pregleda vseh $10$ izdelkov, je torej
$$
E(X \mid E) = 0.7 \cdot 800 € + 0.2 \cdot 1\,150 € + 0.1 \cdot 1\,500 € = 940 €
$$
Sedaj si poglejmo še, koliko znaša pričakovani strošek,
če tovarna ne pregleda nobenega izdelka.
$$
E(X \mid F) = 0.7 \cdot 350 € + 0.2 \cdot 2 \cdot 350 € + 0.1 \cdot 3 \cdot 350 € = 490 €
$$
Če dobljena stroška primerjamo, lahko vidimo,
da ima podjetje manjši strošek, če ne pregleda nobenega dobavljenega izdelka.

\item Strošek za pregled vseh komponent je enak kot v primeru (a),
in sicer $940 €$.
Izračunajmo še pričakovani strošek,
če tovarna naključno pregleda eno komponento.
Problem ločimo na tri dele:
imamo en, dva ali tri pokvarjene izdelke.
Vsakega od teh primerov ločimo še na dva dela:
ali izberemo pokvarjen izdelek ali ne.
\begin{alignat*}{3}
E(X \mid G \cap C \cap D_1) &&&=&\ 45 € \\
E(X \mid G \cap \overline{C} \cap D_1) &= 45 € + 350 € &&=&\ 395 € \\
E(X \mid G \cap D_1) &= \frac{1}{10} \cdot 45 € + \frac{9}{10} \cdot 395 €
&&=&\ 360 € \\[1ex]
E(X \mid G \cap C \cap D_2) &= 45 € + 350 € &&=&\ 395 € \\
E(X \mid G \cap \overline{C} \cap D_2) &= 45 € + 350 € \cdot 2 &&=&\ 745 € \\
E(X \mid G \cap D_2) &= \frac{2}{10} \cdot 395 € + \frac{8}{10} \cdot 745 €
&&=&\ 675 € \\[1ex]
E(X \mid G \cap C \cap D_3) &= 45 € + 350 € \cdot 2 &&=&\ 745 € \\
E(X \mid G \cap \overline{C} \cap D_3) &= 45 € + 350 € \cdot 3 &&=&\ 1\,095 € \\
E(X \mid G \cap D_3) &= \frac{3}{10} \cdot 745 € + \frac{7}{10} \cdot 1\,095 €
&&=&\ 990 €
\end{alignat*}

Sedaj lahko izračunamo pričakovani strošek,
če tovarna izbere naključen izdelek,
ga pregleda in po potrebi popravi:
$$
E(X \mid G) = 0.7 \cdot 360 € + 0.2 \cdot 675 € + 0.1 \cdot 990 € = 486 €
$$

Vidimo, da velja $E(X \mid G) < E(X \mid E)$,
zato je za tovarno bolje, da izbere eno naključno komponento za pregled,
kot če pregleda vseh $10$ komponent.
\end{enumerate}
\end{odgovor}
\end{naloga}
