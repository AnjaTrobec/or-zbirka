\begin{naloga}{Alen Orbanić}{Kolokvij OR 19.11.2009}
\begin{vprasanje}
Zaradi naravnih nesreč so reševalci prejeli klice
za $n$ ponesrečenih oseb iz različnih lokacij,
ki jih je potrebno prepeljati v $k$ bolnišnic po danem cestnem omrežju.
Ponesrečence lahko prepeljemo do bolnišnic,
ki so oddaljene za največ pol ure urgentne vožnje
(za dane ponesrečence imamo torej v splošnem več izbir za bolnišnice).
Da pa bolnišnic ne bi preobremenili, zahtevamo,
da v vsako bol\-niš\-ni\-co
pride največ $\lceil {n \over k} \rceil$ ponesrečencev.
Zanima nas, ali je to izvedljivo.
Modeliraj problem in ga prevedi na kakšen znan problem.
\end{vprasanje}

\begin{odgovor}
Dano cestno omrežje lahko predstavimo kot usmerjen graf $G$,
kjer uteži predstavljajo čas potovanja po cestni povezavi.
Na grafu $G$ lahko uporabimo Floyd-Warshallov algoritem
za izračun razdalje med vsemi pari vozlišč.
Nato zgradimo dvodelen graf $H$,
v katerem imamo po eno vozlišče za vsakega ponesrečenca
in po $\lceil {n \over k} \rceil$ vozlišč za vsako bolnišnico,
vozlišči pa sta povezani,
če je razdalja od ponesrečenca do bolnišnice v grafu $G$ največ pol ure.
Na grafu $H$ nato z madžarsko metodo poiščemo največje prirejanje $M$.
Če ima najdeno prirejanje $n$ povezav,
potem smo našli ustrezno dodelitev ponesrečencev bolnišnicam,
sicer pa ta ne obstaja.
\end{odgovor}
\end{naloga}
