\begin{naloga}{Hillier, Lieberman}{\cite[Problem~11.3-8]{hl}}
\begin{vprasanje}
Podjetje bo kmalu uvedlo nov izdelek na zelo konkurenčen trg,
zato trenutno pripravlja marketinško strategijo.
Odločili so se, da bodo izdelek uvedli v treh fazah.
V prvi fazi bodo pripravili posebno začetno ponudbo z močno znižano ceno,
da bi privabili zgodnje kupce.
Druga faza bo vključevala intenzivno oglaševalsko kampanjo,
da bi zgodnje kupce prepričali, naj izdelek še vedno kupujejo po redni ceni.
Znano je, da bo ob koncu druge faze
konkurenčno podjetje predstavilo svoj izdelek.
Zato bo v tretji fazi okrepljeno oglaševanje z namenom,
da bi preprečili beg strank h konkurenci.

Podjetje ima za oglaševanje na voljo $4$ milijone evrov,
ki jih želimo čim bolj učinkovito porabiti.
Naj bo $m$ tržni delež v procentih, pridobljen v prvi fazi,
$f_2$ delež, ohranjen po drugi fazi,
in $f_3$ delež, ohranjen po tretji fazi.
Maksimizirati želimo končni tržni delež, torej količino $m f_2 f_3$.

\begin{enumerate}[(a)]
\item Denimo, da želimo v vsaki fazi porabiti nek večkratnik milijona evrov,
pri čemer bomo pri prvi fazi porabili vsaj milijon evrov.
V spodnji tabeli so zbrani vplivi porabljenih količin
na vrednosti $m$, $f_2$ in $f_3$.
$$
\begin{array}{c|ccc}
M€ & m & f_2 & f_3 \\
\hline
0 &  - & 0.2 & 0.3 \\
1 & 20 & 0.4 & 0.5 \\
2 & 30 & 0.5 & 0.6 \\
3 & 40 & 0.6 & 0.7 \\
4 & 50 &   - &   - \\
\end{array}
$$
Kako naj razdelimo sredstva?
\item Denimo sedaj,
da lahko v vsaki fazi porabimo poljubno pozitivno količino denarja
(seveda glede na omejitev skupne porabe).
Naj bodo torej $x_1$, $x_2$ in $x_3$ količine denarja v milijonih evrov,
ki jih porabimo v prvi, drugi in tretji fazi.
Vpliv na tržni delež je podan s formulami
$$
m = x_1 (10 - x_1), \quad
f_2 = 0.4 + 0.1 x_2, \quad \text{in} \quad
f_3 = 0.6 + 0.07 x_3 .
$$
Kako naj sedaj razdelimo sredstva?
\end{enumerate}
\end{vprasanje}

\begin{odgovor}
Naj bodo $m(x)$, $f_2(x)$ in $f_3(x)$ začetni tržni delež oziroma deleža,
ohranjena po drugi in tretji fazi,
ob vložku $x$ milijonov evrov v posamezno fazo.
Uporabili bomo sledeče rekurzivne enačbe.
\begin{align*}
p_3(x) &= f_3(x) \\
p_2(x) &= \max\set{f_2(x-y) p_3(y)}{0 \le y \le x} \\
p_1(x) &= \max\set{m(x-y) p_2(y)}{0 \le y \le x}
\end{align*}
Optimalni tržni delež (v milijonih evrov) pri vložku $4 \ME$
bomo dobili kot $p^* = p_1(4)$.

\begin{enumerate}[(a)]
\item Najprej izračunajmo $p_2(x)$ za $x \in \{0, 1, 2, 3\}$.
Ker je $m(0) = 0$, nas namreč $p_2(4)$ ne zanima.
\begin{alignat*}{2}
p_2(0) &= 0.2 \cdot 0.3 &&= 0.06 \\
p_2(1) &= \max\{\underline{0.4 \cdot 0.3}, 0.2 \cdot 0.5\}  &&= 0.12 \\
p_2(2) &= \max\{0.5 \cdot 0.3, \underline{0.4 \cdot 0.5}, 0.2 \cdot 0.6\}
                                                            &&= 0.2 \\
p_2(3) &= \max\{0.6 \cdot 0.3, \underline{0.5 \cdot 0.5},
                0.4 \cdot 0.6, 0.2 \cdot 0.7\}              &&= 0.25
\end{alignat*}
Izračunajmo še $p^* = p_1(4)$.
$$
p_1(4) = \max\{50 \cdot 0.06, 40 \cdot 0.12, \underline{30 \cdot 0.2},
               20 \cdot 0.25\} = 6
$$
Optimalni končni tržni delež $6 \%$ torej dosežemo tako,
da v prvo vazo vložimo $2 \ME$, v vsako od naslednjih dveh faz pa po $1 \ME$.

\item Najprej izračunajmo $p_2(x)$ za $0 \le x \le 4$.
\begin{align*}
p_2(x) &= \max\set{(0.4 + 0.1x - 0.1y) (0.6 + 0.07 y)}{0 \le y \le x} \\
&= \max\set{-0.007y^2 + (0.007x - 0.032) y + 0.06 x + 0.24}{0 \le y \le x}
\end{align*}
Odvajajmo zgornji izraz po $y$, da poiščemo maksimum na intervalu $[0, x]$.
\begin{align*}
0 &= -0.014y + 0.007x - 0.032 \\
y &= {1 \over 2} x - {16 \over 7}
\end{align*}
Za $x \in [0, 32/7)$ je ta vrednost manjša od $0$
-- ker imamo kvadraten polinom z negativnim vodilnim členom,
je maksimum torej dosežen pri $y = 0$.
Ker je $4 < 32/7$, torej velja
$$
p_2(x) = 0.06 x + 0.24 \quad (0 \le x \le 4)
$$
-- tj., v tretjo fazo se nam ne izplača vlagati.
Sedaj izračunajmo še $p^* = p_1(4)$.
\begin{align*}
p_1(4) &= \max\set{(4-z)(6+z) (0.06 z + 0.24)}{0 \le z \le 4} \\
&= \max\set{-0.06z^3 - 0.36z^2 + 0.96z + 5.76}{0 \le z \le 4}
\end{align*}
Poglejmo, kje ima zgornji izraz maksimum na intervalu $[0, 4]$.
\begin{align*}
0 &= -0.18z^2 - 0.72z + 0.96 \\
z &= {-12 \pm 4\sqrt{21} \over 6}
\end{align*}
Ker imamo kubičen polinom z negativnim vodilnim členom,
je maksimum dosežen pri večji od zgornjih rešitev,
torej $z \approx 1.055$.
Optimalni tržni delež je tako $p^* \approx 6.302$,
dobimo pa ga tako, da $1.055 \ME$ vložimo v drugo fazo,
preostanek pa v prvo fazo.
\end{enumerate}
\end{odgovor}
\end{naloga}
