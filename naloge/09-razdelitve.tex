\razdelek{Razdelitve}

\begin{naloga}{?}{Kolokvij OR 26.1.2010}
\begin{vprasanje}
S pomočjo Austinovega postopka opiši postopek,
ki prvemu igralcu zagotovi točno četrtino celote
in drugemu igralcu točno tri četrtine celote.
\end{vprasanje}
\begin{odgovor}
\end{odgovor}
\end{naloga}


\begin{naloga}{?}{Izpit OR 5.2.2010}
\begin{vprasanje}
Naša naloga je,
da med pet ljudi z uporabo denarnih nadomestil razdelimo pet predmetov,
tako da bo vsak dobil pošten delež in da bo delitev brez zavisti.
V spodnji matriki je podano vrednotenje vsakega izmed 5 predmetov
s strani petih ljudi.
$$
\begin{bmatrix}
11 &  9 & 19 & 14 & 11 \\
 4 & 10 & 15 & 20 & 12 \\
 6 &  5 &  5 & 17 & 9  \\
20 & 23 &  6 &  3 & 18 \\
10 & 18 & 23 &  9 & 10
\end{bmatrix}
$$
Seveda želimo pri tem kot svetovalec zaslužiti čimveč.
Ker ni zunanjega financiranja, lahko naš zaslužek pridobimo le iz nadomestil.
Uporabi kak znan algoritem za delitev pod temi pogoji in razloži,
kako si pridobiš kar največji dobiček.
\end{vprasanje}
\begin{odgovor}
\end{odgovor}
\end{naloga}


\begin{naloga}{?}{Kolokvij OR 9.5.2013}
\begin{vprasanje}
V turistični agenciji Zobček že vrsto let organizirajo izlete po Sloveniji.
Zaradi velikega povpraševanja so se odločili,
da bodo ponudbo razširili tudi na Madžarsko.
Izbrali so destinacije in vodiče,
sedaj pa želijo slednje razporediti po destinacijah tako,
da bi vsako destinacijo pokrivala natanko dva vodiča
in bi vsak vodič pokrival natanko eno destinacijo.
Vsak vodič je podal seznam destinacij, urejenih po priljubljenosti,
začenši s tisto, po kateri bi najraje vodil turiste.
Agencija je za vsako destinacijo naredila urejen seznam vodičev,
tako da je na prvem mestu tisti vodič,
ki bi predvidoma najbolje vodil turiste po destinaciji,
na zadnjem pa tisti, ki bi to predvidoma počel najslabše.

Če bi bilo destinacij in vodičev malo,
bi agencija sama nekako razdelila vodiče med destinacije, a temu ni tako.
Še več, seznam destinacij in vodičev nameravajo z leti spreminjati,
zato nujno potrebujejo postopek,
ki bi jim pomagal razdeliti vodiče po destinacijah.

\begin{enumerate}[(a)]
\item Danih je $n$ destinacij in $2n$ vodičev,
pri čemer za vsako destinacijo poznamo preference agencije glede vodičev
in za vsakega vodiča poznamo njegove preference glede destinacij.
Opiši algoritem časovne zahtevnosti $O(n^2)$,
ki izračuna {\em stabilno razdelitev} vodičev med destinacije oz.~nam pove,
da stabilna razdelitev ne obstaja.

Razdelitev $2n$ vodičev med n destinacij je {\em nestabilna},
če obstajata vodič $v$ in destinacija $d$,
tako da bi $v$ raje vodil po destinaciji $d$ kakor po trenutni destinaciji
in bi agencija raje imela vodiča $v$ na destinaciji $d$
namesto enega izmed trenutnih vodičev na tej destinaciji.
Razdelitev $2n$ vodičev med $n$ destinacij je {\em stabilna},
če sta vsaki destinaciji dodeljena natanko dva vodiča in če ni nestabilna.

\item Reši nalogo za naslednje podatke ($n = 3$):
\begin{center}
\begin{tabular}{c|c}
Destinacija & preference agencije glede vodičev \\ \hline
Budimpešta  & Anisa, Bela, Črt, Ema, Dana, Corina \\
Győr        & Črt, Bela, Corina, Dana, Ema, Anisa \\
Szeged      & Ema, Anisa, Bela, Dana, Črt, Corina
\end{tabular}

\bigskip
\begin{tabular}{c|c}
Vodič  & preference vodičev \\ \hline
Anisa  & Szeged, Budimpešta, Győr \\
Bela   & Budimpešta, Szeged, Győr \\
Corina & Győr, Szeged, Budimpešta \\
Črt    & Budimpešta, Győr, Szeged \\
Dana   & Budimpešta, Szeged, Győr \\
Ema    & Győr, Budimpešta, Szeged
\end{tabular}
\end{center}
\end{enumerate}
\end{vprasanje}
\begin{odgovor}
\end{odgovor}
\end{naloga}


\begin{naloga}{?}{Izpit OR 24.6.2013}
\begin{vprasanje}
V turistični agenciji Zobček že vrsto let
organizirajo izlete po Sloveniji in na Madžarskem.
Zaradi velikega povpraševanja so se odločili,
da bodo ponudbo razširili tudi na države območja bivše Jugoslavije.
Ker bodo tako nudili izlete po veliko novih državah,
želijo postaviti štiri najboljše dosedanje vodiče za nove šefe,
tako da bo vsak odgovoren za nekaj držav.
Naša naloga je šefe stabilno razporediti po območjih.
Tabeli preferenc sta naslednji:
\begin{center}
\makebox[\textwidth][c]{
\begin{small}
\begin{tabular}{c|c}
Šefi & preference šefov \\ \hline
Črt  & Srbija in Črna gora, Slovenija in Madžarska,
       Hrvaška in BiH, Makedonija in Kosovo \\
Žan  & Slovenija in Madžarska, Srbija in Črna gora,
       Hrvaška in BiH, Makedonija in Kosovo \\
Šime & Slovenija in Madžarska, Makedonija in Kosovo,
       Hrvaška in BiH, Srbija in Črna gora \\
Mujo & Hrvaška in BiH, Slovenija in Madžarska,
       Makedonija in Kosovo, Srbija in Črna gora
\end{tabular}
\end{small}
}

\bigskip
\begin{tabular}{c|c}
Območje                & preference agencije glede šefov \\ \hline
Slovenija in Madžarska & Črt, Mujo, Žan, Šime \\
Hrvaška in BiH         & Šime, Žan, Črt \\
Srbija in Črna gora    & Šime, Žan, Mujo, Črt \\
Makedonija in Kosovo   & Šime, Mujo, Črt, Žan
\end{tabular}
\end{center}
Opazimo, da agencija za šefa območja Hrvaške in BiH ne želi postaviti Muja.

Poišči stabilno prirejanje, v kateri Mujo ni šef območja Hrvaške in BiH,
ali utemelji, da le-ta ne obstaja\footnote{
Slovenija in Madžarska je eno nedeljivo območje.
Enako velja za ostale pare držav.
}.
\end{vprasanje}
\begin{odgovor}
\end{odgovor}
\end{naloga}
