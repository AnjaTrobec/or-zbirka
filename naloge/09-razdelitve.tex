\razdelek{Razdelitve}

\begin{naloga}{?}{Kolokvij OR 26.1.2010}
\begin{vprasanje}
S pomočjo Austinovega postopka opiši postopek,
ki prvemu igralcu zagotovi točno četrtino celote
in drugemu igralcu točno tri četrtine celote.
\end{vprasanje}
\begin{odgovor}
\end{odgovor}
\end{naloga}


\begin{naloga}{?}{Izpit OR 5.2.2010}
\begin{vprasanje}
Naša naloga je,
da med pet ljudi z uporabo denarnih nadomestil razdelimo pet predmetov,
tako da bo vsak dobil pošten delež in da bo delitev brez zavisti.
V spodnji matriki je podano vrednotenje vsakega izmed 5 predmetov
s strani petih ljudi.
$$
\begin{bmatrix}
11 &  9 & 19 & 14 & 11 \\
 4 & 10 & 15 & 20 & 12 \\
 6 &  5 &  5 & 17 & 9  \\
20 & 23 &  6 &  3 & 18 \\
10 & 18 & 23 &  9 & 10
\end{bmatrix}
$$
Seveda želimo pri tem kot svetovalec zaslužiti čimveč.
Ker ni zunanjega financiranja, lahko naš zaslužek pridobimo le iz nadomestil.
Uporabi kak znan algoritem za delitev pod temi pogoji in razloži,
kako si pridobiš kar največji dobiček.
\end{vprasanje}
\begin{odgovor}
\end{odgovor}
\end{naloga}
