\begin{naloga}{Janoš Vidali}{Izpit OR 31.1.2017}
\begin{vprasanje}
Izdelati želimo terminski plan za izdelavo spletne aplikacije.
V tabeli~\tab so zbrana opravila pri izdelavi.

\begin{enumerate}[(a)]
\item Topološko uredi ustrezni graf in ga nariši.
\item Določi kritična opravila in kritično pot ter čas izdelave.
\item Katero opravilo je najmanj kritično?
Najmanj kritično jo opravilo, katerega trajanje lahko najbolj podaljšamo,
ne da bi vplivali na trajanje izdelave.
\end{enumerate}

\begin{tabela}
\makebox[\textwidth][c]{
\begin{tabular}{c|l|c|c}
opravilo & opis & trajanje & pogoji \\
\hline
A & natančna opredelitev funkcionalnosti & 15 dni & / \\
B & programiranje uporabniškega vmesnika & 40 dni & K \\
C & programiranje skrbniškega vmesnika & 25 dni & A, M \\
D & programiranje strežniškega dela & 30 dni & A, M \\
E & integracija uporabniškega vmesnika s strežnikom & 20 dni & B, D \\
F & alfa testiranje & 20 dni & C, E \\
G & beta testiranje & 30 dni & F, H \\
H & pridobivanje testnih uporabnikov & 45 dni & A \\
I & vnos zadnjih popravkov & 10 dni & G \\
J & izdelava uporabniške dokumentacije & 35 dni & B \\
K & dizajniranje uporabniškega vmesnika & 15 dni & A \\
L & nabava računalniške opreme & 20 dni & / \\
M & postavitev strežnikov & 10 dni & L \\
\end{tabular}
}
\podnaslov{Podatki}
\end{tabela}
\end{vprasanje}
\begin{odgovor}
\end{odgovor}
\end{naloga}
