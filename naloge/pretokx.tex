\begin{naloga}{?}{Izpit OR 28.6.2010}
\begin{vprasanje}
Na sliki~\fig{} je podano enoparametrično omrežje
za problem največjega pretoka.
Parameter $x$ je poljubno nenegativno število.

\begin{enumerate}[(a)]
\item Kakšno je pridruženo omrežje,
če začnemo z ničelnim $(s, t)$-tokom
in ga najprej povečamo vzdolž povečujoče poti $s-a-c-d-t$
ter nato še vzdolž povečujoče poti $s-c-d-b-t$?

\item Poišči maksimalni $(s, t)$-tok in minimalni $(s, t)$-prerez.

\item Kapaciteto ene povezave vhodnega omrežja lahko povečamo za $1$.
Kateri povezavi naj povečamo kapaciteto,
da bo vrednost maksimalnega $(s, t)$-toka v spremenjenem omrežju čim večja?
Odgovor je načeloma odvisen od vred\-no\-sti parametra $x$.
\end{enumerate}

\begin{slika}
\pgfslika
\podnaslov{Omrežje}
\end{slika}
\end{vprasanje}
\begin{odgovor}
\end{odgovor}
\end{naloga}
